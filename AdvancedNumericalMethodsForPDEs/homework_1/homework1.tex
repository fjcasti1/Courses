% --------------------------------------------------------------
% This is all preamble stuff that you don't have to worry about.
% Head down to where it says "Start here"
% --------------------------------------------------------------
 
\documentclass[12pt,answers]{exam}
 
\usepackage[margin=1in]{geometry} 
\usepackage{amsmath,amsthm,amssymb}
\usepackage{graphicx}
\usepackage{subfigure}
\usepackage{float}
\usepackage{color}
\usepackage{amsmath}
\usepackage{amsthm}
\usepackage{amsfonts}
\usepackage{amssymb}
\usepackage{mathrsfs}
\usepackage{verbatim}
\usepackage{listings}
\usepackage[comma, authoryear]{natbib}
\bibliographystyle{plainnat}
\usepackage{mathtools}
\usepackage[shortlabels]{enumitem}
\usepackage{pythontex} 


\usepackage{listings}
\usepackage{xcolor}

\definecolor{codegreen}{rgb}{0,0.6,0}
\definecolor{codegray}{rgb}{0.5,0.5,0.5}
\definecolor{codepurple}{rgb}{0.58,0,0.82}
\definecolor{backcolour}{rgb}{0.95,0.95,0.92}
\definecolor{darkblue}{rgb}{0.1,0.1,1}

\lstdefinestyle{mystyle}{
    backgroundcolor=\color{backcolour},   
    commentstyle=\color{codegreen},
    keywordstyle=\color{darkblue},
    numberstyle=\tiny\color{codegray},
    stringstyle=\color{codepurple},
    basicstyle=\ttfamily\footnotesize,
    breakatwhitespace=false,         
    breaklines=true,                 
    captionpos=b,                    
    keepspaces=true,                 
    numbers=left,                    
    numbersep=5pt,                  
    showspaces=false,                
    showstringspaces=false,
    showtabs=false,                  
    tabsize=2
}

\lstset{style=mystyle}

\graphicspath{{../figures/}}
\newcommand{\R}{\mathbb{R}}
\newcommand{\C}{\mathbb{C}}
\newcommand{\Z}{\mathbb{Z}}
\newcommand{\N}{\mathbb{N}}
\newcommand\tab[1][1cm]{\hspace*{#1}}
\newcommand{\K}{\mathbb{K}}
\newcommand{\zero}{\mathbb{O}}
 
\newenvironment{theorem}[2][Theorem]{\begin{trivlist}
\item[\hskip \labelsep {\bfseries #1}\hskip \labelsep {\bfseries #2.}]}{\end{trivlist}}
\newenvironment{lemma}[2][Lemma]{\begin{trivlist}
\item[\hskip \labelsep {\bfseries #1}\hskip \labelsep {\bfseries #2.}]}{\end{trivlist}}
\newenvironment{exercise}[2][Exercise]{\begin{trivlist}
\item[\hskip \labelsep {\bfseries #1}\hskip \labelsep {\bfseries #2.}]}{\end{trivlist}}
\newenvironment{reflection}[2][Reflection]{\begin{trivlist}
\item[\hskip \labelsep {\bfseries #1}\hskip \labelsep {\bfseries #2.}]}{\end{trivlist}}
\newenvironment{proposition}[2][Proposition]{\begin{trivlist}
\item[\hskip \labelsep {\bfseries #1}\hskip \labelsep {\bfseries #2.}]}{\end{trivlist}}
\newenvironment{corollary}[2][Corollary]{\begin{trivlist}
\item[\hskip \labelsep {\bfseries #1}\hskip \labelsep {\bfseries #2.}]}{\end{trivlist}}
 
\begin{document}
 
% --------------------------------------------------------------
%                         Start here
% --------------------------------------------------------------
 
%\renewcommand{\qedsymbol}{\filledbox}
 
\title{\textbf{Advanced Numerical Methods for PDEs}\\ \Large{Homework 1}}%replace X with the appropriate number
\author{Francisco Castillo}
 

\maketitle
Consider the initial - boundary value problem for the scalar advection diffusion equation
\begin{align}\label{eq:PDE}
\partial_tu(x,t)+a\partial_xu(x,t)-b\partial_x^2u(x,t)=0,~~~~
u(x,t=0)=u^I(x),
\end{align}
on the interval $x\in[-1, 1]$ with periodic boundary conditions $u(x + 2, t) = u(x, t), \forall x, t$.
Consider the explicit difference method 
\begin{align}\label{eq:discrPDE}
U(x, t+\Delta t) = U(x, t)-\frac{a\Delta t}{ 2\Delta x}(T-T^{-1})U(x,t)+\frac{b}{\Delta x^2}(T-2+T^{-1})U(x, t)
\end{align}
for the problem (1).
\section*{Problem 1}
\begin{questions}

\question{Let $\mu$ be a measure on a ring $\mathcal{B}$. Show that $\mu$ is finitely subadditive and $\sigma$-subadditive. (See Lemma 7.11).}

\begin{solution}
  \begin{proof}
Let $\mu$ be a measure on a ring $\mathcal{B}$. By \textit{Definition 7.10}, $\mu$ is additive. Then, by \textit{Lemma 7.8}, $\mu$ is subbadditive, i.e. $\mu(A_1\cup A_2)\leq\mu(A_1)+\mu(A_2)$. We prove that $\mu$ is finitely subadditive,
\begin{equation}\label{eq:subadditive}
\mu\left(\bigcup_{j=1}^nA_j\right)=\sum_{j=1}^n\mu\left(A_j\right)~,
\end{equation}
by induction. It is trivial for $n=1$ is given by the definition of subadditivity for $n=2$. Assume that (\ref{eq:subadditive}) is true for $n$. Now
\begin{align*}
\mu\left(\bigcup_{j=1}^{n+1}A_j\right)&=\mu\left(\bigcup_{j=1}^nA_j\cup A_{n+1}\right)\\
&\leq \mu\left(\bigcup_{j=1}^nA_j\right)+\mu\left(A_{n+1}\right),~~\text{by subadditivity for two sets,}\\
&\leq \sum_{j=1}^n\mu\left(A_j\right)+\mu\left(A_{n+1}\right),~~ \text{by induction hypothesis (\ref{eq:subadditive}),}\\
&=\sum_{j=1}^{n+1}\mu\left(A_j\right).
\end{align*}
Thus, $\mu$ is finitely subadditive.

Now we prove that $\mu$ is $\sigma$-subadditive. By \textit{Definition 7.10}, $\mu$ is continuous from below and we just proved that it is finitely subadditive. Let $(A_n)$ be a sequence of sets in $\mathcal{B}$ such that $\cup_{n\in\N}A_n\in\mathcal{B}$. Set $B_n=\bigcup_{j=1}^nA_j$. Then $(B_n)$ is an increasing sequence in $\mathcal{B}$ with $\bigcup_{n=1}^{\infty}B_n=\bigcup_{n=1}^{\infty}A_n\in\mathcal{B}$. Since $\mu$ is continuous from below,
\begin{align*}
\mu\left(\bigcup_{j=1}^{\infty}A_j\right)=\mu\left(\bigcup_{n=1}^{\infty}B_n\right)=\lim_{n\rightarrow\infty}\mu(B_n).
\end{align*}
Since $\mu$ is finitely subadditive,
\begin{align*}
\mu\left(\bigcup_{j=1}^{\infty}A_j\right)=\lim_{n\rightarrow\infty}\mu(B_n)&=\lim_{n\rightarrow\infty}\mu\left(\bigcup_{j=1}^nA_j\right)\\
&\leq \lim_{n\rightarrow\infty}\sum_{j=1}^n\mu(A_j)\\
&=\sum_{j=1}^{\infty}\mu(A_j).
\end{align*}
Thus,
\begin{align*}
\mu\left(\bigcup_{j=1}^{\infty}A_j\right)\leq \sum_{j=1}^{\infty}\mu(A_j),
\end{align*}
and $\mu$ is $\sigma$-subadditive.
\end{proof}
\end{solution}
\end{questions}
\newpage
\section*{Problem 2}
%\begin{questions}

\question{Show: For each $n\in\N$, $\R^n$ is separable. Hint: Show $\Q^n$ is dense in $\R^n$.}

\begin{solution}
  \begin{proof}
Let $x=(x^1,...,x^n)\in\R^n$ where the upper indices are not powers. Since $\R\subseteq\overline{\Q}$ (by \textit{example 4.9}), we can define $(x_k)\in\Q^n$ such that each $x_k=(x^1_k,...,x^n_k)\in\Q^n$ and $\lim x_k=x$. Thus, $(x_k)$ is a sequence of rational vectors converging to the vector $x$ which is in $\R^n$. In addition, each $x\in\R^n$ is a limit point of $\Q^n$ and therefore $x\in\overline{Q^n}$.
Thus, $\R^n\subseteq\overline{Q^n}$ and $\Q^n$ is a countable dense subset of $\R^n$. By \textit{definition 4.31}, $\R^n$ is separable.
  \end{proof}
\end{solution}


\end{questions}
\newpage
\section*{Problem 3}
%\begin{questions}

\question{How does the spacing depend on $e$?
}
\begin{solution}
We obtain the spacing by focusing in the last digit $a_{23}$ that is multiplied by $2^e$. Hence,
\begin{align*}
1~ulp=2^{-23}2^e=2^{e-23}.
\end{align*}

\end{solution}
\end{questions}


%\begin{figure}[H]
%\centering     %%% not \center
%\hspace*{\fill}
%\subfigure[$f(x)=|x|^3$.]{\includegraphics[scale=0.5]{IMAGES/problem4a.eps}}
%\hfill
%\subfigure[$f(x)=exp\left(\frac{-1}{\sin{(2x^2)}}\right)$.]{\includegraphics[scale=0.5]{IMAGES/problem4b.eps}}
%\hspace*{\fill}

%\hspace*{\fill}
%\subfigure[$f(x)=\frac{1}{\sin{(1+x^2)}}$.]{\includegraphics[scale=0.5]{IMAGES/problem4c.eps}}
%\hfill
%\subfigure[$f(x)=\sinh{x}$.]{\includegraphics[scale=0.5]{IMAGES/problem4d.eps}}
%\hspace*{\fill}
%\caption{Error of the polynomial interpolation on Chebyshev nodes for the different functions.}
%\end{figure}

\bibliography{hw}

\end{document}