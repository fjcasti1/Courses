\begin{questions}
\question{Derive an analytic expression for the solution $u(x,t)$ of problem $(1)$ for a general initial function $u^I(x)$ and general stepsizes $\Delta x$, $\Delta t$, using Fourier transforms.}
\begin{solution}
We will use the Fourier transform 
\begin{align}\label{eq:FourierTransf}
\hat{u}(w,t)=\frac{1}{\sqrt{2\pi}}\int_{-1}^{1}u(x,t)e^{-i\omega x}dx,
\end{align}
to turn our $1-D$ PDE into an ODE. Taking the Fourier transform of the PDE \eqref{eq:PDE} we obtain
\begin{align*}
\partial_t\hat{u}(w,t)+iaw\hat{u}(w,t)+bw^2\hat{u}(w,t)=0,
\end{align*} 
which can be manipulated into
\begin{align*}
\partial_t\hat{u}(w,t)=-(bw^2+iaw)\hat{u}(w,t).
\end{align*}
The previous equation has a simple analytical solution,
\begin{align*}
\hat{u}(w,t)=e^{-(bw^2+iaw)t}\hat{u}(w,0),
\end{align*}
which we can conveniently rewrite as
\begin{align}\label{eq:freqDomSol}
\hat{u}(w,t)=e^{-b\omega^2t}\hat{u}^I(w)e^{-iawt}
\end{align}
To obtain the solution, we will use the following property of Fourier transforms,
\begin{align}
\mathcal{F}\left[f*g\right] = \mathcal{F}\left[f\right] \mathcal{F}\left[g\right],
\end{align}
where $*$ represents convolution. In our case, we have
\begin{align*}
\mathcal{F}\left[f\right](w,t) &= e^{-b\omega^2t},\\
\mathcal{F}\left[g\right](w,t) &= \hat{u}^I(w)e^{-iawt}.
\end{align*}
By using the inverse Fourier transform on the previous equations we obtain
\begin{align}\label{eq:convPartF}
f(x,t)=\mathcal{F}^{-1}\left[\mathcal{F}\left[f\right](w,t)\right](x,t) &= \frac{e^{-x^2/4bt}}{\sqrt{2bt}},\\
g(x,t)=\mathcal{F}^{-1}\left[\mathcal{F}\left[g\right](w,t)\right](x,t) &= \frac{1}{\sqrt{2\pi}}\int_{-\infty}^{\infty}\hat{u}^I(w)e^{-iw(x-at)}dw\nonumber\\ \nonumber
& = u^I(x-at).
\end{align}
Hence, we have proved so far that
\begin{align}
\hat{u}(w,t)=e^{-b\omega^2t}\hat{u}^I(w)e^{-iawt} = \mathcal{F}\left[f\right](w,t) \mathcal{F}\left[g\right](w,t)=\mathcal{F}\left[f*g\right](w,t),
\end{align}
Then, we can finally obtain the solution to the problem,
\begin{align*}
u(x,t)&=\mathcal{F}^{-1}\left[\hat{u}(w,t)\right](x,t)\\
& =\mathcal{F}^{-1}\left[\mathcal{F}\left[f*g\right](w,t)\right](x,t)\\
& =\left[f*g\right](x,t).
\end{align*}

To conclude,
\begin{align}
u(x,t)&=\left[f*u^I(x-at)\right](x,t),
\end{align}
with $f$ given by \eqref{eq:convPartF}.

\end{solution}
\newpage
\question{Derive an analytic expression for the solution $U(x,t)$ of problem $(2)$ for a general initial function $u^I(x)$ and general stepsizes $\Delta x$, $\Delta t$, using Discrete Fourier transforms.}
\begin{solution}
We will use the Discrete Fourier transform 
\begin{align}\label{eq:DiscrFourierTransf}
\hat{u}(w_{\nu},t_n)=\frac{\Delta x}{\sqrt{2\pi}}\sum_{j=-N}^N u(x_j,t_n)e^{-i\omega_{\nu} x_j},
\end{align}
and its inverse,
\begin{align}\label{eq:DiscrInvFourierTransf}
u(x_j,t_n)=\frac{\Delta w}{\sqrt{2\pi}}\sum_{\nu=-N}^N \hat{u}(w_{\nu},t_n)e^{ix_jw_{\nu}},
\end{align}
where $x_j = j\Delta_x$, $t_n=n\Delta t$, $w_{\nu}=\nu\Delta w$ and $N\Delta x\Delta w=\pi$. It is simple to prove that
\begin{align*}
TU(x_j,t_n)=U(x_{j+1},t_n)&=\frac{\Delta w}{\sqrt{2\pi}}\sum_{\nu=-N}^N \hat{u}(w_{\nu},t_n)e^{-i(x_{j+1})w_{\nu}}\\
&=\frac{\Delta w}{\sqrt{2\pi}}\sum_{\nu=-N}^N e^{-i\Delta xw_{\nu}}\hat{u}(w_{\nu},t_n)e^{-ix_jw_{\nu}},\\
T^{-1}U(x_j,t_n)=U(x_{j-1},t_n)&=\frac{\Delta w}{\sqrt{2\pi}}\sum_{\nu=-N}^N \hat{u}(w_{\nu},t_n)e^{-i(x_{j-1})w_{\nu}}\\
&=\frac{\Delta w}{\sqrt{2\pi}}\sum_{\nu=-N}^N e^{i\Delta xw_{\nu}}\hat{u}(w_{\nu},t_n)e^{-ix_jw_{\nu}},\\
\end{align*}
We can substitute into equation\eqref{eq:discrPDE} and simplify to obtain,
\begin{align}\label{eq:discrPDE}
\hat{u}(w_{\nu}, t_{n+1}) &= \left[1-\frac{a\Delta t}{ 2\Delta x}(e^{-i\Delta xw_{\nu}}-e^{i\Delta xw_{\nu}})+\frac{b}{\Delta x^2}(e^{-i\Delta xw_{\nu}}-2+e^{i\Delta xw_{\nu}})\right]\hat{u}(x_j, t_n)\nonumber\\
&= \left[1+i\frac{a\Delta t}{ \Delta x}\sin\left(\Delta xw_{\nu}\right)-\frac{4b}{\Delta x^2}\sin^2\left(\frac{\Delta xw_{\nu}}{2}\right)\right]\hat{u}(w_{\nu}, t_n).
\end{align}
Define $g(\omega_{\nu})$ as
\begin{align}
g(\omega_{\nu}) = \left[1+i\frac{a\Delta t}{ \Delta x}\sin\left(\Delta xw_{\nu}\right)-\frac{4b}{\Delta x^2}\sin^2\left(\frac{\Delta xw_{\nu}}{2}\right)\right].
\end{align}
Then, $\hat{u}(w_{\nu}, t_{n+1}) = g(\omega_{\nu} )\hat{u}(w_{\nu}, t_n)$. We can now find the solution, in frequency domain, as a function of the initial condition.
\begin{align*}
\hat{u}(w_{\nu}, t_n) &= g(\omega_{\nu} )\hat{u}(w_{\nu}, t_{n-1})= g^2(\omega_{\nu} )\hat{u}(w_{\nu}, t_{n-2})= g^3(\omega_{\nu} )\hat{u}(w_{\nu}, t_{n-3}),\\
&= \dots = g^n(\omega_{\nu} )\hat{u}(w_{\nu}, 0),\\
&= g^n(\omega_{\nu} )\hat{u}^I(w_{\nu}).
\end{align*}
We now use equation \eqref{eq:DiscrInvFourierTransf} to obtain the solution in space domain,
\begin{align}
u(x_j,t_n)&=\frac{\Delta w}{\sqrt{2\pi}}\sum_{\nu=-N}^N \hat{u}(w_{\nu},t_n)e^{ix_jw_{\nu}}\nonumber\\
&= \frac{\Delta w}{\sqrt{2\pi}}\sum_{\nu=-N}^N g^n(\omega_{\nu} )\hat{u}^I(w_{\nu})e^{ix_jw_{\nu}}.
\end{align}
To finish, it must be mentioned that the above solution is stable if and only if $|g(w)|\leq 1, \forall w\in[-\frac{\pi}{\Delta x},\frac{\pi}{\Delta x}]$. This condition will produce the following conditions on the parameters:
\begin{align*}
\Delta t a^2\leq 2b,\text{  and  }2\Delta tb\leq\Delta x^2.
\end{align*}
\end{solution}
\end{questions}