\documentclass[compress]{beamer}
\hypersetup{pdfpagemode=FullScreen}
\input{BEAMERoptions.tex}					                                         \usepackage[english]{babel}
\usepackage{graphics}
\usepackage{amsmath}
\usepackage{multimedia} % for movies and sound
\usepackage{times}
\usepackage{tikz}
\usepackage[font=small]{subcaption}
%\usepackage{subcaption}
\captionsetup[sub]{font=small}
\usepackage{stackengine}
\usepackage{caption}
\captionsetup[figure]{labelformat=empty}% redefines the caption setup of the figures environment in the beamer class.




\title{Advance Numerical Methods for PDEs}
\author[F. Castillo-Carrasco]{\emph{Francisco Castillo-Carrasco} }
\date{\today}
% Show ASU logo in title page
%%%%%%%%%%%%%%%%%%%%%%
\institute[Mathematics and Statistics]{
\includegraphics[height=.9cm]{ASUlogo.pdf} \\
{\color{ASUred} \textbf{SCHOOL OF MATHEMATICAL AND STATISTICAL SCIENCES}}}
%%%%%%%%%%%%%%%%%%%%%

	%%%%%%%%%%%%%%%%%%%%%%%%%%%%%%%%%%%%%%%%%%
%%%%%%%%%%%% Presentation Starts Here %%%%%%%%%%%%%%%%
	%%%%%%%%%%%%%%%%%%%%%%%%%%%%%%%%%%%%%%%%%%

\begin{document}
\setbeamertemplate{caption}{\raggedright\insertcaption\par}

%%% Title frame %%%%%
\begin{frame}[plain]
	\titlepage
\end{frame}

%%%%%%%%%%%%%%%%%%%%%%%%%%%%%%%%%%
\begin{frame}
  \frametitle{Outline}
  \tableofcontents[pausesections]
\end{frame}
%%%%%%%%%%%%%%%%%%%%%%%%%%%%%%%%%%


\section{Discretization}
\begin{frame} \frametitle{Discretization}
\begin{columns}
\column{0.45\textwidth}
\vspace*{-0.6cm}
\begin{block}{PDE}
$\partial_tu(x,t)+a\partial_xu(x,t)=0$
\end{block}
We can discretize using finite differences, forward in time and central in space:
\begin{align*}
u_j^{n+1}=u_j^{n}-a\frac{\Delta t}{\Delta x}\left(u_{j+1}^n-u_{j-1}^n\right)
\end{align*}
\color{red} NOT STABLE FOR THIS PDE!
\newline
\newline
\color{black}
Instead, we substitute $u_j^n$ by the average of its two neighboring grid points.
\column{0.5\textwidth}
\vspace*{-1.5cm}
\footnotesize{
\begin{align*}
u_j^{n+1}=\color{blue}\frac{1}{2}\left(u_{j+1}^{n}+u_{j-1}^{n}\right)\color{black}-a\frac{\Delta t}{\Delta x}\left(u_{j+1}^n-u_{j-1}^n\right)
\end{align*}}
\vspace*{-0.5cm}
\begin{block}{Lax-Friedrichs Discretization}
\footnotesize{
\begin{align*}
u_j^{n+1}=\frac{1}{2}\left(u_{j+1}^{n}+u_{j-1}^{n}\right)-ac\left(u_{j+1}^n-u_{j-1}^n\right)
\end{align*}}
\end{block}
This change introduces an artificial viscosity.
\begin{columns}
\column{0.4\columnwidth}
\begin{block}{Artificial Viscosity}
$$2b = \frac{\Delta x^2}{\Delta t}$$
\end{block}
\column{0.4\columnwidth}
\begin{block}{Courant Number}
$$c = \frac{\Delta t}{\Delta x}$$
\end{block}
%\begin{block}{CFL Condition}
%$$\Delta t\leq\frac{\Delta x}{|a|}$$
%\end{block}
\end{columns}
\end{columns}
\end{frame}


\section{Changing to Matrix Form}
\begin{frame} \frametitle{Changing to Matrix Form}
Rearranging terms,
\begin{align*}
u_j^{n+1} &= \frac{1}{2}\left(1+ac\right)u_{j-1}^n+\frac{1}{2}\left(1-ac\right)u_{j+1}^n,\nonumber\\
&= Au_{j-1}^n + Bu_j^n + Cu_{j+1}^n,
\end{align*}
where $A = \frac{1}{2}\left(1+ac\right)$, $B = 0$ and $C = \frac{1}{2}\left(1-ac\right)$. 
\begin{columns}
\column{0.4\columnwidth}
\begin{block}{Lax-Friedrichs Matrix Form}
$$\vec{u}^{n+1} = M\vec{u}^n,$$
\end{block}
\column{0.4\columnwidth}
\begin{block}{Tridiagonal Matrix}
\begin{align*}
M =
  \begin{pmatrix}
    B & C & \\
    A & B & C\\
      & \ddots & \ddots & \ddots\\
      &  & A & B & C\\    
      &  &  & A & B
  \end{pmatrix}
\end{align*}
\end{block}
\end{columns}
\end{frame}


\section{Boundary Conditions}
\begin{frame} \frametitle{Boundary Conditions}
\begin{columns}
\column{0.45\columnwidth}
We will solve the matrix system \textbf{only for the interior}.
\begin{block}{Solve for the interior}
$$\vec{u}^{n+1} = \tilde{M}\vec{u}^n,$$
$\tilde{M}$ is $M$ where we have removed the first and last rows.
\end{block}
The algorithm updates the solution for the interior, then updates the end points using the periodic boundary conditions.
\begin{block}{Periodic Boundary Conditions}
$$u_{j} = u_{j\pm N}$$
\end{block}
\column{0.4\columnwidth}
More specifically,
\begin{align*}
u^{n+1}_{0} &= \color{red}Au^n_{-1}\color{black} + Bu^n_{0} + Cu^n_{1},\\
 &= \color{blue}Au^n_{N-1}\color{black} + Bu^n_{0} + Cu^n_{1},\\
u^{n+1}_{N} &= Au^n_{N-1} + Bu^n_{N} + \color{red}Cu^n_{N+1}\\
 &= Au^n_{N-1} + Bu^n_{N} + \color{blue}Cu^n_{1}\\
\end{align*}
\vspace*{-1.2cm}
\begin{block}{Applied Boundary Conditions}
$u^{n+1}_{0} = Au^n_{-1} + Bu^n_{0} + Cu^n_{1}$
$u^{n+1}_{N} = Au^n_{N-1} + Bu^n_{N} + Cu^n_{N+1}$
\end{block}
*NOTE: in MATLAB the subindeces are shifted +1 since the arrays start at 1, not at 0.
\end{columns}
\end{frame}

\section{Results}
\begin{frame}
\begin{columns}
\hspace{-0.75cm}
\column{0.3\columnwidth}
\begin{figure}
\includegraphics[scale=0.3]{../figures/sol_b0_dx1e-1.png}
\vspace{-0.75cm}
\caption{\footnotesize{$\Delta x = 0.1$}}
\end{figure}
\column{0.3\columnwidth}
\begin{figure}
\includegraphics[scale=0.3]{../figures/sol_b0_dx1e-2.png}
\vspace{-0.75cm}
\caption{\footnotesize{$\Delta x = 0.01$}}
\end{figure}
\column{0.3\columnwidth}
\begin{figure}
\includegraphics[scale=0.3]{../figures/sol_b0_dx1e-3.png}
\vspace{-0.75cm}
\caption{\footnotesize{$\Delta x = 0.001$}}
\end{figure}
\end{columns}
\end{frame}



\begin{frame}
\centering
\Huge
Thank you
\end{frame}

\end{document}
