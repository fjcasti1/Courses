\begin{questions}
\question{Write a program to solve \eqref{eq:PDE} for $0<t<T$ using the discretization \eqref{eq:discrPDE} for general values of $a,b,\Delta x,\Delta t$ and a general initial function $u^I(x)$.}

\begin{solution}
Let $u_j^n = u(x_0+j\Delta x,n\Delta t)$, with $x_0 = -1$. Then, we can rewrite equation \eqref{eq:discrPDE} into
\begin{align*}
u_j^{n+1} &= u_j^n - \frac{a\Delta t}{2\Delta x}\left(u_{j+1}^n-u_{j-1}^n\right) + \frac{b\Delta t}{\Delta x^2}\left(u_{j+1}^n-2u_{j}^n+u_{j-1}^n\right),\\
&= u_j^n - c\frac{a\Delta x}{2}\left(u_{j+1}^n-u_{j-1}^n\right) + cb\left(u_{j+1}^n-2u_{j}^n+u_{j-1}^n\right),
\end{align*}
where $c = \frac{\Delta t}{\Delta x^2}$. Regrouping terms we obtain
\begin{align}\label{eq:reducedForm}
u_j^{n+1} &= c\left(b+\frac{a\Delta x}{2}\right)u_{j-1}^n + \left(1-2bc\right)u_j^n + c\left(b-\frac{a\Delta x}{2}\right)u_{j+1}^n,\nonumber\\
&= Au_{j-1}^n +Bu_j^n + Cu_{j+1}^n,
\end{align}
where $A = c\left(b+\frac{a\Delta x}{2}\right)$, $B = 1-2bc$ and $C = c\left(b-\frac{a\Delta x}{2}\right)$. The previous equation can be represented as a tridiagonal system,
\begin{align}\label{eq:matrixForm}
\vec{u}^{n+1} = M\vec{u}^n,
\end{align}
where $M$ is a tridiagonal matrix with $A$, $B$ and $C$ being its lower, main, and upper diagonal, respectively. Note that 
\begin{align*}
\vec{u} &= \left(\begin{matrix}
           u_{0} \\
           \vdots \\
           u_{j} \\
           \vdots \\
           u_{N}
         \end{matrix}\right)
\end{align*}

We can advance in time by simply using \eqref{eq:matrixForm}. To implement the periodic boundary conditions we consider the end points $x_0$ and $x_N$ in equation \eqref{eq:reducedForm}.
At $x_0$:
\begin{align*}
u_0^{n+1} &= Au_{-1}^n +Bu_0^n + Cu_{1}^n,\\
&= Au_{N-1}^n +Bu_0^n + Cu_{1}^n,
\end{align*}
since $u_{-1} = u_{N-1}$.
At $x_0$:
\begin{align*}
u_N^{n+1} &= Au_{N-1}^n +Bu_N^n + Cu_{N+1}^n,\\
&= Au_{N-1}^n +Bu_N^n + Cu_{2}^n,
\end{align*}
since $u_{N+1} = u_{2}$. This method was coded in Matlab (code at the end of this problem) and used to solve the next question.
\end{solution}

\question{Solve the discretized problem \eqref{eq:discrPDE} for $0<t<1$, using the values $a=1,b=0.5$ and
\begin{equation}\label{eq:uI_step}
u^I(x) = \begin{dcases}
        1 & x\leq 0 \\
        0 & x > 0 \\
\end{dcases}
\end{equation}
Use stepsizes $\Delta x = 0.1$, $\Delta x = 0.01$ and $\Delta x = 0.001$. Use the analysis of Problem 1 to determine an appropriate stepsize $\Delta t$.}
\begin{solution}
The method above was implemented. See figure 1 for the solution profiles at different times and using different mesh sizes. We can barely see any difference between the solutions for $\Delta x=0.01$ (b) and $\Delta x=0.001$ (c). Since the compute time is considerably larger for $\Delta x=0.001$, with very little gain in accuracy, we don't see the need for such a fine mesh. At the same time, it can be observed that $\Delta x = 0.1$ is too big.
\begin{figure}[H]
\centering     %%% not \center
\subfigure[$\Delta x=0.1$.]{\includegraphics[scale=0.52]{sol_b5e-1_dx1e-1.png}}
\hspace{-0.9cm}
\subfigure[$\Delta x=0.01$.]{\includegraphics[scale=0.52]{sol_b5e-1_dx1e-2.png}}
\hspace{-0.9cm}
\subfigure[$\Delta x=0.001$.]{\includegraphics[scale=0.52]{sol_b5e-1_dx1e-3.png}}
\caption{Solution $u(x,t)$ against $x$ for the PDE in \eqref{eq:PDE} for different values of $t$ and $a=1$, $b=0.5$.}
\end{figure}
Find the code that produced the plots in figure 1 below. The reader is welcome to set \textsf{enableVideo = true}, to see the evolution of the signal in real time. Further, if $b$ were to be set to a value very close to zero, we would see a travelling wave that doesn't suffer any diffusion, as expected.To conclude, $\Delta t$ has been chosen following the \textit{CFL} condition given by \eqref{eq:CFL}, implemented in the function \textsf{calculate\_dt}.

Matlab code:
\lstinputlisting[language=Matlab]{./src/problem2.m}
\end{solution}
\end{questions}