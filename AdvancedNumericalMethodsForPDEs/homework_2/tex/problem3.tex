\begin{questions}
\question{Consider the Lighthill - Whitham - Richardson (LWR) mean field model for this problem 
\begin{align*}
& \partial_tu(x,t) + \partial_x[v(u, x)u(x,t)] = 0,\\
& v(u, x) = v_0(x)\left(1 - \frac{u}{c}\right),
\end{align*}
with the initial and periodic boundary conditions
\begin{align*}
&u(x, t = 0) = u^I(x),\\
&u(x + L, t) = u(x).
\end{align*}
Write a code to solve the problem with the Lax-Friedrichs
method for $t\in[0, T]$ with the parameters $L = 50(km)$, $c = 100(cars/km)$, $T = 1(h)$,
\begin{align*}
v_0(x) = \begin{dcases}
         20(km/h) & 30<x<40 \\
        100(km/h) & else
\end{dcases}
\end{align*}
and 
\begin{align*}
u^I(x) = \begin{dcases}
        50(cars/km) & 0<x<10 \\
         0 & else \\
\end{dcases}
\end{align*}
Use $N = 100$ gridpoints ($\Delta x = L/N$) and compute the time
step $\Delta t$ adaptively according to the CFL condition.}
\begin{solution}
We start with the Lax-Friedrichs scheme,
\begin{align}\label{eq:Lax-F}
u_j^{n+1} &= \frac{1}{2}\left(u_{j+1}^n+u_{j-1}^n\right)-\frac{\Delta t}{2\Delta x}\left(f_{j+1}^n-f_{j-1}^n\right),
\end{align}
where the flux function 
\begin{align}
f_j^n = v_j^nu_j^n,
\end{align}
with
\begin{align}
v_j^n = v_{0j}\left(1-\frac{u_j^n}{c}\right),~~
\end{align}
Note that we use \eqref{eq:Lax-F} only for the interior $j=2:N$. We need to calculate $\Delta t$. To do so, since $v(u,x)$ is clearly not constant, we use the most restrictive of cases, i.e., the maximum value of $v$.
\begin{align}
\Delta t = \frac{\Delta x}{\max_j\{v_j\}},~~~~j=1,\dots,N+1
\end{align}
To close the problem, we enforce periodic boundary conditions:
\begin{align*}
u_0 = u_{N+1},\\
u_{N+2} = u_1.
\end{align*}
Which, applied to \eqref{eq:Lax-F}, gives us
\begin{align}\label{eq:Lax-F}
u_1^{n+1} &= \frac{1}{2}\left(u_{2}^n+u_{N+1}^n\right)-\frac{\Delta t}{2\Delta x}\left(f_{2}^n-f_{N+1}^n\right),\\
u_{N+1}^{n+1} &= \frac{1}{2}\left(u_{1}^n+u_{N}^n\right)-\frac{\Delta t}{2\Delta x}\left(f_{1}^n-f_{N}^n\right).
\end{align}
To finish, we wish to see the behaviour of the phase velocity $v_P$, defined as
\begin{align}
v_P(x,t) &= \frac{\partial}{\partial u}\left[v(u,x)u(x,t)\right],\\
&= \frac{\partial}{\partial u}\left[v_0(x)\left(1 - \frac{u(x,t)}{c}\right)
u(x,t)\right],\\
&= v_0(x)\frac{\partial}{\partial u}\left[u(x,t) - \frac{u^2(x,t)}{c}\right],\\
&= v_0(x)\left(1 - \frac{2}{c}u(x,t)\right).
\end{align}
See the matlab code at the end of the problem after the results. In figures 4 and 5 I show the solutions $u(x,t)$ and $v_P(x,t)$, respectively, every $0.2$ seconds.


\newpage
\begin{figure}[H]
\centering     %%% not \center
\subfigure[$t=0$ s.]{\includegraphics[scale=0.52]{p3_u_t0.png}}
\hspace{-0.9cm}
\subfigure[$t=0.2$ s.]{\includegraphics[scale=0.52]{p3_u_t2e-1.png}}
\hspace{-0.9cm}
\subfigure[$t=0.4$ s.]{\includegraphics[scale=0.52]{p3_u_t4e-1.png}}
\hspace{-0.9cm}
\subfigure[$t=0.6$ s.]{\includegraphics[scale=0.52]{p3_u_t6e-1.png}}
\hspace{-0.9cm}
\subfigure[$t=0.8$ s.]{\includegraphics[scale=0.52]{p3_u_t8e-1.png}}
\hspace{-0.9cm}
\subfigure[$t=1$ s.]{\includegraphics[scale=0.52]{p3_u_t1e0.png}}
\hspace{-0.9cm}
\caption{Density $u(x,t)$ against $x$ for different values of time $t$. The red shaded area corresponds to the zone where the cars have to go slower.} 
\end{figure}

\begin{figure}[H]\label{fig:p3_vp}
\centering     %%% not \center
\subfigure[$t=0$ s.]{\includegraphics[scale=0.52]{p3_vp_t0.png}}
\hspace{-0.9cm}
\subfigure[$t=0.2$ s.]{\includegraphics[scale=0.52]{p3_vp_t2e-1.png}}
\hspace{-0.9cm}
\subfigure[$t=0.4$ s.]{\includegraphics[scale=0.52]{p3_vp_t4e-1.png}}
\hspace{-0.9cm}
\subfigure[$t=0.6$ s.]{\includegraphics[scale=0.52]{p3_vp_t6e-1.png}}
\hspace{-0.9cm}
\subfigure[$t=0.8$ s.]{\includegraphics[scale=0.52]{p3_vp_t8e-1.png}}
\hspace{-0.9cm}
\subfigure[$t=1$ s.]{\includegraphics[scale=0.52]{p3_vp_t1e0.png}}
\hspace{-0.9cm}
\caption{Phase velocity $v_P(x,t)$ against $x$ for different values of time $t$. The red shaded area corresponds to the zone where the cars have to go slower.} 
\end{figure}

Matlab code:
\lstinputlisting[language=Matlab]{../src/problem3.m}
\end{solution}
\end{questions}