% --------------------------------------------------------------
% This is all preamble stuff that you don't have to worry about.
% Head down to where it says "Start here"
% --------------------------------------------------------------

\documentclass[12pt,answers]{exam}

\usepackage[margin=1in]{geometry}
\usepackage{amsmath,amsthm,amssymb}
\usepackage{graphicx}
\usepackage{subfigure}
\usepackage{float}
\usepackage{color}
\usepackage{amsmath}
\usepackage{amsthm}
\usepackage{amsfonts}
\usepackage{amssymb}
\usepackage{mathrsfs}
\usepackage{verbatim}
\usepackage{listings}
\usepackage[comma, authoryear]{natbib}
\bibliographystyle{plainnat}
\usepackage{mathtools}
\usepackage[shortlabels]{enumitem}
\usepackage{pythontex}
\usepackage[makeroom]{cancel}


\usepackage{listings}
\usepackage{xcolor}


\definecolor{codegreen}{rgb}{0,0.6,0}
\definecolor{codegray}{rgb}{0.5,0.5,0.5}
\definecolor{codepurple}{rgb}{0.58,0,0.82}
\definecolor{backcolour}{rgb}{0.95,0.95,0.92}
\definecolor{darkblue}{rgb}{0.1,0.1,1}

\lstdefinestyle{mystyle}{
    backgroundcolor=\color{backcolour},
    commentstyle=\color{codegreen},
    keywordstyle=\color{darkblue},
    numberstyle=\tiny\color{codegray},
    stringstyle=\color{codepurple},
    basicstyle=\ttfamily\footnotesize,
    breakatwhitespace=false,
    breaklines=true,
    captionpos=b,
    keepspaces=true,
    numbers=left,
    numbersep=1pt,
    showspaces=false,
    showstringspaces=false,
    showtabs=false,
    tabsize=2
}

\lstset{style=mystyle}

\graphicspath{{../figures/}}
\newcommand{\R}{\mathbb{R}}
\newcommand{\C}{\mathbb{C}}
\newcommand{\Z}{\mathbb{Z}}
\newcommand{\N}{\mathbb{N}}
\newcommand\tab[1][1cm]{\hspace*{#1}}
\newcommand{\K}{\mathbb{K}}
\newcommand{\zero}{\mathbb{O}}

\newenvironment{theorem}[2][Theorem]{\begin{trivlist}
\item[\hskip \labelsep {\bfseries #1}\hskip \labelsep {\bfseries #2.}]}{\end{trivlist}}
\newenvironment{lemma}[2][Lemma]{\begin{trivlist}
\item[\hskip \labelsep {\bfseries #1}\hskip \labelsep {\bfseries #2.}]}{\end{trivlist}}
\newenvironment{exercise}[2][Exercise]{\begin{trivlist}
\item[\hskip \labelsep {\bfseries #1}\hskip \labelsep {\bfseries #2.}]}{\end{trivlist}}
\newenvironment{reflection}[2][Reflection]{\begin{trivlist}
\item[\hskip \labelsep {\bfseries #1}\hskip \labelsep {\bfseries #2.}]}{\end{trivlist}}
\newenvironment{proposition}[2][Proposition]{\begin{trivlist}
\item[\hskip \labelsep {\bfseries #1}\hskip \labelsep {\bfseries #2.}]}{\end{trivlist}}
\newenvironment{corollary}[2][Corollary]{\begin{trivlist}
\item[\hskip \labelsep {\bfseries #1}\hskip \labelsep {\bfseries #2.}]}{\end{trivlist}}

\begin{document}

% --------------------------------------------------------------
%                         Start here
% --------------------------------------------------------------

%\renewcommand{\qedsymbol}{\filledbox}

\title{\textbf{Advanced Numerical Methods for PDEs}\\ \Large{Homework 3}}%replace X with the appropriate number
\author{Francisco Castillo}


\maketitle
Consider the 2D problem
\begin{align*}
\partial_tu(x,y,t) + \partial_xf^x &+ \partial_yf^y = 0,~~ x,y\in[0,1]\\
f^x = u - \partial_xu,&~~ f^y = - \partial_yu
\end{align*}
with boundary conditions
\begin{align*}
f^x(x=0,y,t) = 0,~~f^x(x=1,y,t) = 0,\\
f^y(x,y=0,t) = 0,~~f^y(x,y=1,t) = 0,
\end{align*}
and initial conditions
\begin{align*}
u(x,y,t=0) = u^I(x,y) = \delta(x-0.5)\delta(y-0.5)
\end{align*}
\section*{Problem 1}
\textbf{Based on the theorems of chapter 8 of Approximation Theory and Approximation Practice by Trefethen, what can you say about the convergence as $n\rightarrow\infty$ of the Chebyshev interpolants to the following functions? Which is the case that converges much faster than the theorems predict? Can you speculate why?}
\newline

The convergence, predicted by \textsc{Theorem 8.2}, is 
\begin{align*}
\|f-p_n\|\leq\frac{4M\rho^{-n}}{\rho-1},
\end{align*}
with 
\begin{align*}
\rho = \alpha+\sqrt{\alpha^2-1}~~~~\text{(real singularity at $x=\pm\alpha$)},
\end{align*}
or
\begin{align*}
\rho = \beta+\sqrt{\beta^2+1}~~~~\text{(imaginary singularity at $x=\pm i\beta$)}.
\end{align*}

Applying this to the following functions we obtain:
\begin{enumerate}[label=\alph*)]
\item $f(x) = \tan(x)$. This function has a real singularity at $x=\pm \pi/2$. Hence, 
\begin{align*}
\rho = \frac{\pi}{2}+\sqrt{\frac{\pi^2}{4}-1},
\end{align*}
and we have obtained the rate of convergence showed in the next figure.
\begin{figure}[H]
\centering
\includegraphics[scale=0.75]{tan(x).png}\caption{Convergence as $n\rightarrow\infty$ of the Chebyshev interpolant to $f(x)=\tan(x)$.}
\end{figure}


\item $f(x) = \tanh(x)$. This function has a imaginary singularity at $x=\pm i\pi/2$. Hence, 
\begin{align*}
\rho = \frac{\pi}{2}+\sqrt{\frac{\pi^2}{4}+1},
\end{align*}
and we have obtained the rate of convergence showed in the next figure.
\begin{figure}[H]
\centering
\includegraphics[scale=0.75]{tanh(x).png}\caption{Convergence as $n\rightarrow\infty$ of the Chebyshev interpolant to $f(x)=\tanh(x)$.}
\end{figure}


\item $f(x) = \frac{\log(\frac{x+3}{4})}{x-1}$. This function has a real singularity at $x=-3$. Hence, 
\begin{align*}
\rho = 3+\sqrt{8},
\end{align*}
and we have obtained the rate of convergence showed in the next figure.
\begin{figure}[H]
\centering
\includegraphics[scale=0.75]{log1.png}\caption{Convergence as $n\rightarrow\infty$ of the Chebyshev interpolant to $f(x)= \frac{\log(\frac{x+3}{4})}{x-1}$.}
\end{figure}

\item $f(x) = \int_{-1}^{x}\cos(t^2)dt$. This function is entire, analytic at all finite points of the complex plane. Hence its convergence is much faster as is shown in the following figure.
\begin{figure}[H]
\centering
\includegraphics[scale=0.75]{integral.png}\caption{Convergence as $n\rightarrow\infty$ of the Chebyshev interpolant to $f(x)= \frac{\log(\frac{x+3}{4})}{x-1}$.}
\end{figure}



\item $f(x) = \tan(\tan(x))$. This function has a real singularity at $x=\pm \arctan(\pi/2)=\pm k$. Hence, 
\begin{align*}
\rho = k+\sqrt{k^2-1},
\end{align*}
and we have obtained the rate of convergence showed in the next figure.
\begin{figure}[H]
\centering
\includegraphics[scale=0.75]{tan(tan(x)).png}\caption{Convergence as $n\rightarrow\infty$ of the Chebyshev interpolant to $f(x)= \tan(\tan(x))$.}
\end{figure}

\item $f(x) = (1+x)\log(1+x)$. This function has a real singularity at $x=-1$. Hence, 
\begin{align*}
\rho = -1,
\end{align*}
and we have obtained the rate of convergence showed in the next figure.
\begin{figure}[H]
\centering
\includegraphics[scale=0.75]{(1+x)log(1+x)wrong.png}\caption{Convergence as $n\rightarrow\infty$ of the Chebyshev interpolant to $f(x)= (1+x)\log(1+x)$.}
\end{figure}
This figure is not right since the function does not satisfies the assumptions of \newline \textsc{Theorem 8.1}, it is not analytic in $[-1,1]$. The rate of convergence found is algebraic (much worse than the previous cases), as shown in the following figure.

\begin{figure}[H]
\centering
\includegraphics[scale=0.75]{(1+x)log(1+x)right.png}\caption{Algebraic convergence as $n\rightarrow\infty$ of the Chebyshev interpolant to \newline $f(x)= (1+x)\log(1+x)$.}
\end{figure}
\end{enumerate}

Overall, it has been tested that the larger the ellipse with foci $\{-1,1\}$ within which the function is analytic, the faster the convergence.
\subsection*{Matlab code for this problem}
\begin{verbatim}
rho = 0.5*pi+sqrt(0.25*pi^2-1);
orderAcuracy('tan(x)',50,2,rho)

rho = (0.5*pi+sqrt(0.25*pi^2+1));
orderAcuracy('tanh(x)',40,2,rho)

rho = (3+sqrt(8));
orderAcuracy('log((x+3)/4)/(x-1)',30,2,rho,'log1')

k = atan(pi/2);
rho = k+sqrt(k^2-1);
orderAcuracy('tan(tan(x))',460,20,rho)

rho = -1;
orderAcuracy('(1+x)*log(1+x)',460,20,rho,'(1+x)log(1+x)wrong')
orderAcuracy('(1+x)*log(1+x)',460,20,rho,'(1+x)log(1+x)right')

function orderAcuracy(func,Nmax,Nstep,rho,namefig)
    labelfontsize = 14;
    figformat = 'png';
    if nargin < 5
        namefig = func;
    end
    f = chebfun(func);
    nn = 0:Nstep:Nmax; ee = 0*nn;
    for j=1:length(nn)
        n = nn(j);
        fn = chebfun(f,n+1);
        ee(j) = norm(f-fn);
    end
    figure
    if strcmp(namefig,'(1+x)log(1+x)right')
        semilogy(nn,2000*nn.^(-3.8),'-b')
    else 
        semilogy(nn,rho.^(-nn),'-b')
    end
    hold on
    semilogy(nn,ee,'r*')
    grid on
    xlabel('$N$','interpreter','latex')
    ylabel('$\|f-p_N\|$','interpreter','latex')
    set(gca,'fontsize',labelfontsize)
    txt=['Latex/FIGURES\' namefig];
    saveas(gcf,txt,figformat)
end

% For the integral function
nn = 2:2:20;
err = 0*nn;
for k = 1:length(nn)
    N = nn(k);
    x = -.98:0.02:1;
    F = 0*x;
    FN = F;
    for j = 1:length(x) 
        f = chebfun('cos(t^2)', [-1 x(j)]);
        fN = chebfun('cos(t^2)', [-1 x(j)],N);
        F(j) = sum(f);
        FN(j) = sum(fN);
        F = [0 F]; FN = [0 FN]; x = [-1 x];
    end
    err(k) = norm(F-FN);
end
figure
semilogy(nn,err,'r*')
grid on
xlabel('$N$','interpreter','latex')
ylabel('$\|f-p_N\|$','interpreter','latex')
set(gca,'fontsize',labelfontsize)
txt='Latex/FIGURES\integral';
saveas(gcf,txt,figformat)
\end{verbatim}
\newpage
\section*{Problem 2}
In this problem we are going t approximate the function 
$$ f(x)=e^{\sin(5x)}$$
using Chebyshev's interpolation of degree ten (using eleven points) and using the orthonormal basis provided by the \textit{Legendre polynomials}. It will be shown that the Legendre polynomials, which we show in the figure 1, give the best approximation of that degree as we proved in Problem 1.

\begin{figure}[H]
\centering     %%% not \center
{\includegraphics[scale=0.75]{LegendrePols.eps}}
\caption{Legendre Polynomials.}
\end{figure}

In the following figure we show the function $f$ and the two approximations made. To the naked eye both of them seem very accurate, however we can see how the Legendre polynomials give us a better approximation by looking at the error in figure 3. 
\begin{figure}[H]
\centering     %%% not \center
{\includegraphics[scale=0.6]{Approximations_p2.eps}}
\caption{Approximations using Chebyshev and Legendre polynomials.}
\end{figure}
The sudden drops in the error are in fact the points used to do the approximation, so the error is indeed zero. Since we are using logarithmic scale we see thos drops. We can see how the error of Chebyshev's method is higher. We can compute the $L_2$ error of the two methods,
\begin{align*}
e_{Chebyshev}=0.07800109011,~~~~~~e_{Legendre}=0.05183155462,
\end{align*}
and see how the Legendre polynomials, as proved in problem 1, give us the best approximation.
\begin{figure}[H]
\centering     %%% not \center
{\includegraphics[scale=0.6]{Error_p2.eps}}
\caption{Error of the approximations.}
\end{figure}
\subsection*{Matlab code for this problem}
\begin{verbatim}
%% Problem 2
clear all
close all
clc
format long
legendfontsize=12;
labelfontsize=14;
f = @(x) exp(sin(5*x));
f= chebfun(f);
f_cheb = chebfun(f,11);
xx=linspace(-1,1,1000);
P=legpoly(0:10,'norm');

figure
plot(P)
grid on
xlabel('$x$','fontsize',labelfontsize,'interpreter','latex')
saveas(gcf,'Latex/FIGURES/LegendrePols','epsc')
saveas(gcf,'Latex/FIGURES/LegendrePols','fig')

f_N=0;
for k=1:11
    f_N=f_N+(f'*P(:,k))*P(:,k);
end
%%
figure
plot(f,'linewidth',2)
hold on
plot(f_N,'linewidth',2)
plot(f_cheb,'linewidth',2)
grid on
legend({'$f(x)=e^{\sin(5x)}$','Legendre','Chebishev'}...
    ,'fontsize',legendfontsize,'interpreter','latex','location','north')
xlabel('$x$','fontsize',labelfontsize,'interpreter','latex')
saveas(gcf,'Latex/FIGURES/Approximations_p2','epsc')
saveas(gcf,'Latex/FIGURES/Approximations_p2','fig')

Error_cheb=norm(f-f_cheb,2)
Error_app=norm(f-f_N,2)
%%
figure
semilogy(abs(f-f_cheb))
hold on
semilogy(abs(f-f_N))
grid on
axis([-1 1 1e-4 0.35])
legend({'$e_{Cheb}$','$e_{Leg}$'},'fontsize',legendfontsize,'interpreter','latex')
xlabel('$x$','fontsize',labelfontsize,'interpreter','latex')
saveas(gcf,'Latex/FIGURES/Error_p2','epsc')
saveas(gcf,'Latex/FIGURES/Error_p2','fig')

\end{verbatim}
\newpage

Consider the Schroedinger equation
\begin{align*}
i\hbar \frac{\partial\psi(x, t)}{\partial t}  = -\frac{\hbar^2}{2m}\frac{\partial^2}{\partial x^2}\psi(x,t) + V (x)\psi(x,t).
\end{align*}
Here $\hbar$ is Planck’s constant, $m$ is the mass of the particle,
and $V (x)$ is the external potential energy. $\psi$ is the wave
function. The particle density is given by $u(x, t) = |\psi(x, t)|^2$.
Assume units for space, time and mass such that $\hbar = m = 1$ holds.

\section*{Problem 3}
\textbf{Solve the nonlinear boundary value problem}
\begin{align*}
y''=y'+\cos(y),~~y(0)=0,~~y(\pi)=1.
\end{align*}
\textbf{Use finite differences to discretize the problem and Newton’s method to solve the resulting system of equations. Estimate (numerically) the number of discretized points needed to obtain a solution that is accurate to 4 digits at $x = \pi/6$.}

We can express the previous differential equation as
\begin{align*}
y''-y'-\cos(y)=0,
\end{align*}
so that its roots will give us the solution. To do so we need to discretize the derivatives using finite differences in order to be able to use Newton's method. Then, the differential equation takes the linear form
\begin{align*}
y'' - y_x - cos(y) &= D_2y - D_1y - cos(y)\\
&=My-\cos(y) =0~,
\end{align*}
where the matrix $M = D_2 - D_1$. The matrices $D_1$ and $D_2$ will give us the discrete values of the first and second derivative, respectively, and are obtained using second order centered finite differences. For $D_1$,
\begin{align*}
&y' = \frac{1}{2(\Delta x)}
\begin{bmatrix}
-1 & 0 & 1 & ~&  \dots & 0 \\
0 & -1 & 0 & 1 & \dots & 0 \\ 
\vdots & ~&  \ddots & \ddots & \ddots & \vdots \\
0 & \dots & ~ & -1 & 0 & 1 \\
\end{bmatrix}y +
\begin{bmatrix}
0 \\
\vdots \\
0 \\
\frac{1}{2 \Delta x}
\end{bmatrix} = D_1y + g_1,
\end{align*}
where the column vector $g_1$ contains is found by imposing the boundary conditions. Note that the matrices are calculated in such a way that they give us the derivatives for the interior nodes, since the boundary nodes we know their values thanks to the boundary conditions. For $D_2$,
\begin{align*}
& y'' = \frac{1}{(\Delta x)^2} \begin{bmatrix}
1 & -2 & 1 & ~&  \dots & 0 \\
0 & 1 & -2 & 1 & \dots & 0 \\ 
\vdots & ~&  \ddots & \ddots & \ddots & \vdots \\
0 & \dots & ~ & 1 & -2 & 1 \\
\end{bmatrix}y + \begin{bmatrix}
0 \\
\vdots \\
0 \\
\frac{1}{( \Delta x)^2}
\end{bmatrix} = D_2y + g_2~,
\end{align*}
where the column vector $g_2$ contains is found by imposing the boundary conditions. We now define
\begin{align*}
\vec{F}(\vec{y}) = My - cos(y) + g_2 - g_1~,
\end{align*}
where $\vec{y}$ contains the interior nodes. The solution to our differential equations will be the vector $y$ that makes $\vec{F}(\vec{y})=0$. To calculate those roots we will use Newton's method. In order to be able to use it, we need to calculate the Jacobian,
\begin{align*}
J_{jk}=\frac{\partial}{y_j}F(y_k).
\end{align*}
First note that
\begin{align*}
\frac{\partial }{\partial y_j} \sum_{l=1}^{N}M_{kl}y_l = M_{kl}
\end{align*}
and
\begin{align*}
\frac{\partial }{\partial y_j} \cos(y_k) = \begin{dcases}
	-\sin(y_k) & \text{for}~k = j\\
		0 & \text{for}~j \neq k 
\end{dcases}
\end{align*}
Hence, the Jacobian can be coded as
\begin{align*}
J(\vec{F}(\vec{y})) = M + \text{diag}(\sin(y)).
\end{align*}

We will first run the code in a while loop to find out how many nodes we need to have an $L_1$ norm of $y^{new}(\pi/6)-y^{old}(\pi/6)$ less than a $tol=10^{-4}$. Using 31 nodes I obtained an $L_1$ norm of $7.269\cdot 10^{-5}$. To be safe, since it is a low number of nodes and the code runs fast, we could run it for double number of elements $N$. Using then 61 nodes we obtain an $L_1$ norm of the error of $7.603\cdot 10^{-6}$. Using 61 nodes we have obtained the solution showed in the next figure

\begin{figure}[H]
\center{\includegraphics[scale=.75]{prob3sol.eps}}
\caption{Solution for 61 points}
\end{figure}

\subsection*{Matlab code for this problem}

\begin{verbatim}
%% Problem 3
close all
clc
format long
% Find an answer with sufficient level of convergence
tol=1e-4;
i=1;
N=6;
err=1;
yy=zeros(6+1,1);
while err>tol
    i=i+1;
    y_old=yy(N/6);
    N = i*6;
    xi=linspace(0,pi,N+1)';
    [F,J,y0]=FandJ(N);
    [yy,niter] = newton(F,J,y0,1e-14);
    y_new=yy(N/6);
    % Check solution
    err=max(abs(y_new-y_old));
end
% Results
nodes=N+1
err
% Run it for N=60
N=60;
xi=linspace(0,pi,N+1)';
[F,J,y0]=FandJ(N);
[yy,niter] = newton(F,J,y0,1e-14);
y_new=yy(N/6);
% Plot the solution
figure
plot(xi,[0; yy; 1],'linewidth',1.5)
hold on
plot(xi(N/6+1),y_new,'r.','markerSize',12)
grid on
leg1= legend('Solution', 'Value at $\pi/6$');
set(leg1,'interpreter','latex','FontSize',17);
set(gca,'fontsize',axisfontsize)
xlabel('$x$','interpreter','latex','fontsize',labelfontsize)
ylabel('$y$','interpreter','latex','fontsize',labelfontsize)
saveas(gcf,'Latex/FIGURES/prob3sol','epsc')
\end{verbatim}
\subsubsection*{Function to calculate F and J}
\begin{verbatim}
function [F,J,y0]=FandJ(N)
x = linspace(0,pi,N+1)';
h=pi/N;
xi = x(2:N);
y0 = xi/pi;

D1 = (gallery('tridiag',N+1,-1,0,1))/(2*h);
D1(1,:) = [];
D1(end,:) = [];
D1(:,1) = [];
D1(:,end) = [];

D2 = (gallery('tridiag',N+1,1,-2,1))/(h^2);
D2(1,:) = [];
D2(end,:) = [];
D2(:,1) = [];
D2(:,end) = [];

% g1 = [zeros(N-2,1)];
% g1(N-2) = (1/(2*h));
% g2 = [zeros(N-2,1)];
% g2(N-2) = (1/(h^2));
% e1 = g2 - g1;

M = D2 - D1;
e1 = [zeros(N-1,1)];
e1(N-1) = ((2-h)/(2*h^2));

F = @(y) (M*y)-cos(y)+e1;
J = @(y) M + diag(sin(y));
end
\end{verbatim}
\newpage
\section*{Problem 4}
\begin{questions}

\question{
Use conjugate gradient on the steepest descent problem we did in
class: 
\begin{displaymath}
	A = \left[ 
	\begin{array}{rr} 
	4 & -2 \\
	-2 & 2
	\end{array}
	\right] ,~~
	b = \left[ 
	\begin{array}{r} 
	-2\\
	2
	\end{array}
	\right] ,~~
	x_0 = \left[ 
	\begin{array}{r} 
	0\\
	0
	\end{array}
	\right] ,~~
	x = \left[ 
	\begin{array}{r} 
	0\\
	1
	\end{array}
	\right] .
\end{displaymath}
}
\begin{solution}
Since $x_0=0$, $r_0=b$ and $\beta_1=0$. Then,
\begin{align*}
d_1=r_0+\beta_1d_0=r_0=b,
\end{align*}
and,
\begin{align*}
\alpha_1=\frac{r_0^Tr_0}{d_1^TAd_1}=1/5.
\end{align*}
Then new guess is
\begin{align*}
x_1=x_0+\alpha_1d_1=\left[
	\begin{array}{r r} 
	-2/5 \\
	2/5 \\
	\end{array} \right],
\end{align*}
and the new residual is
\begin{align*}
r_1=r_0+\alpha_1Ad_1=\left[
	\begin{array}{r r} 
	2/5 \\
	2/5 \\
	\end{array} \right],
\end{align*}
We repeat the process again,
\begin{align*}
&\beta_2=\frac{r_1^Tr_1}{r_0^Tr_0}=1/25,\\
&d_2=r_1+\beta_2d_1=\left[
	\begin{array}{r r} 
	8/25 \\
	12/25 \\
	\end{array} \right],\\
&\alpha_2=\frac{r_1^Tr_1}{d_2^TAd_2}=5/4,\\
&x_2=x_1+\alpha_2d_2=\left[
	\begin{array}{r r} 
	0 \\
	1 \\
	\end{array} \right].\\		
\end{align*}
\end{solution}
\end{questions}


\end{document}