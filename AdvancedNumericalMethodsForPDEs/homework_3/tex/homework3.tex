% --------------------------------------------------------------
% This is all preamble stuff that you don't have to worry about.
% Head down to where it says "Start here"
% --------------------------------------------------------------

\documentclass[12pt,answers]{exam}

\usepackage[margin=1in]{geometry}
\usepackage{amsmath,amsthm,amssymb}
\usepackage{graphicx}
\usepackage{subfigure}
\usepackage{float}
\usepackage{color}
\usepackage{amsmath}
\usepackage{amsthm}
\usepackage{amsfonts}
\usepackage{amssymb}
\usepackage{mathrsfs}
\usepackage{verbatim}
\usepackage{listings}
\usepackage[comma, authoryear]{natbib}
\bibliographystyle{plainnat}
\usepackage{mathtools}
\usepackage[shortlabels]{enumitem}
\usepackage{pythontex}
\usepackage[makeroom]{cancel}


\usepackage{listings}
\usepackage{xcolor}


\definecolor{codegreen}{rgb}{0,0.6,0}
\definecolor{codegray}{rgb}{0.5,0.5,0.5}
\definecolor{codepurple}{rgb}{0.58,0,0.82}
\definecolor{backcolour}{rgb}{0.95,0.95,0.92}
\definecolor{darkblue}{rgb}{0.1,0.1,1}

\lstdefinestyle{mystyle}{
    backgroundcolor=\color{backcolour},
    commentstyle=\color{codegreen},
    keywordstyle=\color{darkblue},
    numberstyle=\tiny\color{codegray},
    stringstyle=\color{codepurple},
    basicstyle=\ttfamily\footnotesize,
    breakatwhitespace=false,
    breaklines=true,
    captionpos=b,
    keepspaces=true,
    numbers=left,
    numbersep=1pt,
    showspaces=false,
    showstringspaces=false,
    showtabs=false,
    tabsize=2
}

\lstset{style=mystyle}

\graphicspath{{../figures/}}
\newcommand{\R}{\mathbb{R}}
\newcommand{\C}{\mathbb{C}}
\newcommand{\Z}{\mathbb{Z}}
\newcommand{\N}{\mathbb{N}}
\newcommand\tab[1][1cm]{\hspace*{#1}}
\newcommand{\K}{\mathbb{K}}
\newcommand{\zero}{\mathbb{O}}

\newenvironment{theorem}[2][Theorem]{\begin{trivlist}
\item[\hskip \labelsep {\bfseries #1}\hskip \labelsep {\bfseries #2.}]}{\end{trivlist}}
\newenvironment{lemma}[2][Lemma]{\begin{trivlist}
\item[\hskip \labelsep {\bfseries #1}\hskip \labelsep {\bfseries #2.}]}{\end{trivlist}}
\newenvironment{exercise}[2][Exercise]{\begin{trivlist}
\item[\hskip \labelsep {\bfseries #1}\hskip \labelsep {\bfseries #2.}]}{\end{trivlist}}
\newenvironment{reflection}[2][Reflection]{\begin{trivlist}
\item[\hskip \labelsep {\bfseries #1}\hskip \labelsep {\bfseries #2.}]}{\end{trivlist}}
\newenvironment{proposition}[2][Proposition]{\begin{trivlist}
\item[\hskip \labelsep {\bfseries #1}\hskip \labelsep {\bfseries #2.}]}{\end{trivlist}}
\newenvironment{corollary}[2][Corollary]{\begin{trivlist}
\item[\hskip \labelsep {\bfseries #1}\hskip \labelsep {\bfseries #2.}]}{\end{trivlist}}

\begin{document}

% --------------------------------------------------------------
%                         Start here
% --------------------------------------------------------------

%\renewcommand{\qedsymbol}{\filledbox}

\title{\textbf{Advanced Numerical Methods for PDEs}\\ \Large{Homework 3}}%replace X with the appropriate number
\author{Francisco Castillo}


\maketitle
Consider the 2D problem
\begin{align*}
\partial_tu(x,y,t) + \partial_xf^x &+ \partial_yf^y = 0,~~ x,y\in[0,1]\\
f^x = u - \partial_xu,&~~ f^y = - \partial_yu
\end{align*}
with boundary conditions
\begin{align*}
f^x(x=0,y,t) = 0,~~f^x(x=1,y,t) = 0,\\
f^y(x,y=0,t) = 0,~~f^y(x,y=1,t) = 0,
\end{align*}
and initial conditions
\begin{align*}
u(x,y,t=0) = u^I(x,y) = \delta(x-0.5)\delta(y-0.5)
\end{align*}
\section*{Problem 1}
\begin{questions}

\question{Let $\mu$ be a measure on a ring $\mathcal{B}$. Show that $\mu$ is finitely subadditive and $\sigma$-subadditive. (See Lemma 7.11).}

\begin{solution}
  \begin{proof}
Let $\mu$ be a measure on a ring $\mathcal{B}$. By \textit{Definition 7.10}, $\mu$ is additive. Then, by \textit{Lemma 7.8}, $\mu$ is subbadditive, i.e. $\mu(A_1\cup A_2)\leq\mu(A_1)+\mu(A_2)$. We prove that $\mu$ is finitely subadditive,
\begin{equation}\label{eq:subadditive}
\mu\left(\bigcup_{j=1}^nA_j\right)=\sum_{j=1}^n\mu\left(A_j\right)~,
\end{equation}
by induction. It is trivial for $n=1$ is given by the definition of subadditivity for $n=2$. Assume that (\ref{eq:subadditive}) is true for $n$. Now
\begin{align*}
\mu\left(\bigcup_{j=1}^{n+1}A_j\right)&=\mu\left(\bigcup_{j=1}^nA_j\cup A_{n+1}\right)\\
&\leq \mu\left(\bigcup_{j=1}^nA_j\right)+\mu\left(A_{n+1}\right),~~\text{by subadditivity for two sets,}\\
&\leq \sum_{j=1}^n\mu\left(A_j\right)+\mu\left(A_{n+1}\right),~~ \text{by induction hypothesis (\ref{eq:subadditive}),}\\
&=\sum_{j=1}^{n+1}\mu\left(A_j\right).
\end{align*}
Thus, $\mu$ is finitely subadditive.

Now we prove that $\mu$ is $\sigma$-subadditive. By \textit{Definition 7.10}, $\mu$ is continuous from below and we just proved that it is finitely subadditive. Let $(A_n)$ be a sequence of sets in $\mathcal{B}$ such that $\cup_{n\in\N}A_n\in\mathcal{B}$. Set $B_n=\bigcup_{j=1}^nA_j$. Then $(B_n)$ is an increasing sequence in $\mathcal{B}$ with $\bigcup_{n=1}^{\infty}B_n=\bigcup_{n=1}^{\infty}A_n\in\mathcal{B}$. Since $\mu$ is continuous from below,
\begin{align*}
\mu\left(\bigcup_{j=1}^{\infty}A_j\right)=\mu\left(\bigcup_{n=1}^{\infty}B_n\right)=\lim_{n\rightarrow\infty}\mu(B_n).
\end{align*}
Since $\mu$ is finitely subadditive,
\begin{align*}
\mu\left(\bigcup_{j=1}^{\infty}A_j\right)=\lim_{n\rightarrow\infty}\mu(B_n)&=\lim_{n\rightarrow\infty}\mu\left(\bigcup_{j=1}^nA_j\right)\\
&\leq \lim_{n\rightarrow\infty}\sum_{j=1}^n\mu(A_j)\\
&=\sum_{j=1}^{\infty}\mu(A_j).
\end{align*}
Thus,
\begin{align*}
\mu\left(\bigcup_{j=1}^{\infty}A_j\right)\leq \sum_{j=1}^{\infty}\mu(A_j),
\end{align*}
and $\mu$ is $\sigma$-subadditive.
\end{proof}
\end{solution}
\end{questions}
\newpage
\section*{Problem 2}
\begin{questions}

\question{Show: For each $n\in\N$, $\R^n$ is separable. Hint: Show $\Q^n$ is dense in $\R^n$.}

\begin{solution}
  \begin{proof}
Let $x=(x^1,...,x^n)\in\R^n$ where the upper indices are not powers. Since $\R\subseteq\overline{\Q}$ (by \textit{example 4.9}), we can define $(x_k)\in\Q^n$ such that each $x_k=(x^1_k,...,x^n_k)\in\Q^n$ and $\lim x_k=x$. Thus, $(x_k)$ is a sequence of rational vectors converging to the vector $x$ which is in $\R^n$. In addition, each $x\in\R^n$ is a limit point of $\Q^n$ and therefore $x\in\overline{Q^n}$.
Thus, $\R^n\subseteq\overline{Q^n}$ and $\Q^n$ is a countable dense subset of $\R^n$. By \textit{definition 4.31}, $\R^n$ is separable.
  \end{proof}
\end{solution}


\end{questions}
\newpage

Consider the Schroedinger equation
\begin{align*}
i\hbar \frac{\partial\psi(x, t)}{\partial t}  = -\frac{\hbar^2}{2m}\frac{\partial^2}{\partial x^2}\psi(x,t) + V (x)\psi(x,t).
\end{align*}
Here $\hbar$ is Planck’s constant, $m$ is the mass of the particle,
and $V (x)$ is the external potential energy. $\psi$ is the wave
function. The particle density is given by $u(x, t) = |\psi(x, t)|^2$.
Assume units for space, time and mass such that $\hbar = m = 1$ holds.

\section*{Problem 3}
\begin{questions}

\question{How does the spacing depend on $e$?
}
\begin{solution}
We obtain the spacing by focusing in the last digit $a_{23}$ that is multiplied by $2^e$. Hence,
\begin{align*}
1~ulp=2^{-23}2^e=2^{e-23}.
\end{align*}

\end{solution}
\end{questions}

\newpage
\section*{Problem 4}
\begin{questions}
  \question{Solve the SE for $0 < t < 1$ for $N = 3, 10, 50$ with a time step
$\Delta t$ such that 
\begin{align*}
\Delta t||A||^2 = \Delta t\sqrt{\rho(A^HA)} = 1
\end{align*} holds. Here $\rho$ is the spectral radius $\rho(A^HA) = max_n |\lambda_n|$ with $\lambda_n$ the eigenvalues of $A^HA$ and $A^H$ the hermitian of the complex matrix $A$: $A^H = (A^T)^*$. Use the \textsc{Matlab} function \texttt{eig} to compute the eigenvalues. Use
\begin{align*}
\psi^I(x) = \left\{\begin{array}{ll}
      1 & -\pi < x < -\frac{\pi}{2} \\
      0 &  -\frac{\pi}{2} < x < \pi \\
\end{array}\right.
\end{align*}
as an initial condition, and the Cranck - Nicholson scheme for the time discretization.}
\begin{solution}
From the previous problem, recover the equation
\begin{align*}
\overrightarrow{c}' (t) = A\overrightarrow{c}(t),
\end{align*}
and apply the Crank-Nicholson scheme,
\begin{align*}
\frac{\overrightarrow{c}^{k+1}-\overrightarrow{c}^{k}}{\Delta t} = A\frac{1}{2}\left(\overrightarrow{c}^{k+1}+\overrightarrow{c}^{k}\right),
\end{align*}
which yields
\begin{align*}
\overrightarrow{c}^{k+1} - A\frac{\Delta t}{2} \overrightarrow{c}^{k+1} = \overrightarrow{c}^{k} + A\frac{\Delta t}{2} \overrightarrow{c}^{k}.
\end{align*}
Solving for $\overrightarrow{c}^{k+1}$,
\begin{align*}
\overrightarrow{c}^{k+1}= \left(\mathbb{I} - A\frac{\Delta t}{2}\right)^{-1}\left(\mathbb{I} + A\frac{\Delta t}{2}\right) \overrightarrow{c}^{k}.
\end{align*}
Note that we use the Crank-Nicholson scheme on the \textit{coefficients} of the expansion. Hence, we need to calculate the coefficients of the initial condition,
\begin{align*}
c_n(0) = \left\langle\phi_n,\psi^I(x)\right\rangle &= \frac{1}{2\pi}\int_{-\pi}^{\pi}e^{-inx}\psi^I(x)dx\\
&= \frac{1}{2\pi}\int_{-\pi}^{-\pi/2}e^{-inx}dx\\
&= \frac{i}{2\pi n}e^{in\pi/2}\left(1-e^{in\pi/2}\right), ~~\text{for}~~ n\neq 0,
\end{align*}
and, 
\begin{align*}
c_0(0) = \left\langle\phi_0,\psi^I(x)\right\rangle &= \frac{1}{2\pi}\int_{-\pi}^{\pi}\psi^I(x)dx\\
&= \frac{1}{2\pi}\int_{-\pi}^{-\pi/2}dx\\
&= \frac{1}{4}.
\end{align*}
\end{solution}
\end{questions}

\end{document}