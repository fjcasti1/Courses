\begin{questions}
  \question{Solve the SE for $0 < t < 1$ for $N = 3, 10, 50$ with a time step
$\Delta t$ such that 
\begin{align*}
\Delta t||A||^2 = \Delta t\sqrt{\rho(A^HA)} = 1
\end{align*} holds. Here $\rho$ is the spectral radius $\rho(A^HA) = max_n |\lambda_n|$ with $\lambda_n$ the eigenvalues of $A^HA$ and $A^H$ the hermitian of the complex matrix $A$: $A^H = (A^T)^*$. Use the \textsc{Matlab} function \texttt{eig} to compute the eigenvalues. Use
\begin{align*}
\psi^I(x) = \left\{\begin{array}{ll}
      1 & -\pi < x < -\frac{\pi}{2} \\
      0 &  -\frac{\pi}{2} < x < \pi \\
\end{array}\right.
\end{align*}
as an initial condition, and the Cranck - Nicholson scheme for the time discretization.}
\begin{solution}
From the previous problem, recover the equation
\begin{align*}
\overrightarrow{c}' (t) = A\overrightarrow{c}(t),
\end{align*}
and apply the Crank-Nicholson scheme,
\begin{align*}
\frac{\overrightarrow{c}^{k+1}-\overrightarrow{c}^{k}}{\Delta t} = A\frac{1}{2}\left(\overrightarrow{c}^{k+1}+\overrightarrow{c}^{k}\right),
\end{align*}
which yields
\begin{align*}
\overrightarrow{c}^{k+1} - A\frac{\Delta t}{2} \overrightarrow{c}^{k+1} = \overrightarrow{c}^{k} + A\frac{\Delta t}{2} \overrightarrow{c}^{k}.
\end{align*}
Solving for $\overrightarrow{c}^{k+1}$,
\begin{align*}
\overrightarrow{c}^{k+1}= \left(\mathbb{I} - A\frac{\Delta t}{2}\right)^{-1}\left(\mathbb{I} + A\frac{\Delta t}{2}\right) \overrightarrow{c}^{k}.
\end{align*}
Note that we use the Crank-Nicholson scheme on the \textit{coefficients} of the expansion. Hence, we need to calculate the coefficients of the initial condition,
\begin{align*}
c_n(0) = \left\langle\phi_n,\psi^I(x)\right\rangle &= \frac{1}{2\pi}\int_{-\pi}^{\pi}e^{-inx}\psi^I(x)dx\\
&= \frac{1}{2\pi}\int_{-\pi}^{-\pi/2}e^{-inx}dx\\
&= \frac{i}{2\pi n}e^{in\pi/2}\left(1-e^{in\pi/2}\right), ~~\text{for}~~ n\neq 0,
\end{align*}
and, 
\begin{align*}
c_0(0) = \left\langle\phi_0,\psi^I(x)\right\rangle &= \frac{1}{2\pi}\int_{-\pi}^{\pi}\psi^I(x)dx\\
&= \frac{1}{2\pi}\int_{-\pi}^{-\pi/2}dx\\
&= \frac{1}{4}.
\end{align*}
\end{solution}
\end{questions}