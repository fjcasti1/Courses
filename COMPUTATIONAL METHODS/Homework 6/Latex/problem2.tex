\textbf{Suppose the discretization of a boundary value problem, such as the Poisson equation, leads to the matrix equation}
\begin{align*}
uA + Bu = f,
\end{align*}
\textbf{that needs to be solved for $u$. Here $u$ is $n \times m$, $A$ is $m \times m$, $B$ is $n \times n$, and $f$ is $n \times m$.}

\textbf{Assuming both $A$ and $B$ are non-defective matrices, with $A = V_A \Lambda_AV_A^{-1}$ and $B = V_B \Lambda_BV_B^{-1}$, show that the solution $u$ is given by $u = V_B UV_A^{-1}$, where}
\begin{align*}
U_{i,j} = \frac{F_{i,j}}{\alpha_j + \beta_i},
\end{align*}
\textbf{$F = V_B^{-1}fV_A$, and $\alpha_k$ and $\beta_k$ are the diagonal entries of $\Lambda_A$ and $\Lambda_B$ respectively.}

We start with the eigenvalue decomposition of the matrices $A$ and $B$,
\begin{align*}
uV_A \Lambda_AV_A^{-1} + V_B \Lambda_BV_B^{-1} u = f,
\end{align*}
and premultiplying by $V_B^{-1}$ multiplying from the right by $V_A$,
\begin{align*}
V_B^{-1}uV_A\Lambda_A + \Lambda_B V_B^{-1} u V_A = V_B^{-1} f V_A.
\end{align*}
Then let $U = V_B^{-1}uV_A$ and $F = V_B^{-1} f V_A$ so that. Therefore, we have
\begin{align*}
U\Lambda_A + \Lambda_B U = F,
\end{align*}
which in matrix form is
\begin{align*}
&\begin{bmatrix}
\tilde{u}_{11}& \cdots & \tilde{u}_{1m} \\
\tilde{u}_{21}& \cdots & \tilde{u}_{2m} \\
\vdots& ~ & \vdots \\
\tilde{u}_{n1}& \cdots & \tilde{u}_{nm}
\end{bmatrix} \begin{bmatrix}
\alpha_1 & 0 & \cdots & 0 \\
0 & \alpha_2 & ~ & \vdots \\
\vdots & ~& \ddots & ~ \\
0 & \cdots & 0 & \alpha_m
\end{bmatrix} + \begin{bmatrix}
\beta_1 & 0 & \cdots & 0 \\
0 & \beta_2 & ~ & \vdots \\
\vdots & ~& \ddots & ~ \\
0 & \cdots & 0 & \beta_m
\end{bmatrix}\begin{bmatrix}
\tilde{u}_{11}& \cdots & \tilde{u}_{1m} \\
\tilde{u}_{21} & \cdots & \tilde{u}_{2m} \\
\vdots& ~ & \vdots \\
\tilde{u}_{n1} & \cdots & \tilde{u}_{nm}
\end{bmatrix} = \begin{bmatrix}
\tilde{f}_{11} & \cdots & \tilde{f}_{1m} \\
\tilde{f}_{21} & \cdots & \tilde{f}_{2m} \\
\vdots & ~ & \vdots \\
\tilde{f}_{n1} & \cdots & \tilde{f}_{nm}
\end{bmatrix}.
\end{align*}
We can infer from the matrix equation above that
\begin{align*}
U_{i,j} = \frac{F_{i,j}}{\alpha_j + \beta_i}.
\end{align*}
Then, to get back our solution, we use $u = V_B U V_A^{-1}$.

\textbf{Using part (a), solve the Poisson equation $\Delta u = \sin(x)\cos(100y)$ on $[0,1] \times [0,1]$ (with zero Dirichlet boundary conditions) using 100 points in the $x$ direction and 300 in the $y$ direction. Estimate the accuracy of your solution.}

Coding into \textsl{Matlab} the previous method, we obtain a solution for $u$ shown in the figure below,

\begin{figure}[H]
\center{\includegraphics[scale=.75]{problem2_u.png}}
\caption{Solution0 $u$ of the poisson equation.}
\end{figure}

The error, found by comparing $uD2A + D2Bu$ and $f$, where $D2A$ and $D2B$ are the centered finite difference matrices for $x$ and $y$ respectively, is $2.10806\cdot 10^{-12}$. 

\subsection*{Matlab code for this part}
\begin{verbatim}
%% Problem 2
Nx=100; Ny=300;
hx=1/(Nx-1); hy=1/(Ny-1);

D2x=gallery('tridiag',Nx,1,-2,1)/(hx^2);
D2y=gallery('tridiag',Ny,1,-2,1)/(hy^2);

[Vx,LambdaX]=eig(full(D2x));
[Vy,LambdaY]=eig(full(D2y));
[x,y]=meshgrid(linspace(0,1,100),linspace(0,1,300));
f=sin(x).*cos(100*y);
F=Vy'*f*Vx;
U=zeros(size(f));
lambdax=diag(LambdaX);
lambday=diag(LambdaY);
% whos U F lambdax lambday
for j=1:Nx
    U(:,j)=F(:,j)./(lambday+lambdax(j));
end

u=Vy*U*Vx';
% Calculate error
err=norm(u*D2x+D2y*u-f,inf)
figure
surf(x,y,u)
shading interp
xlabel('$x$','Interpreter','latex')
ylabel('$y$','Interpreter','latex')
set(get(gca,'ylabel'),'rotation',0)
txt=[path,'problem2_u'];
saveas(gcf,txt,'png')

\end{verbatim}