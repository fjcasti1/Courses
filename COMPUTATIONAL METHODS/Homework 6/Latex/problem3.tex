\textbf{ Using finite differences on the given grid and the data in $u$ and $v$ from \texttt{velocitydata.mat}, solve the equation of temperature in a fluid flow}
\begin{align*}
T_t + uT_x + vT_y = \alpha(T_{xx}+T_{yy}),~~(x,y)\in (0,1) \times (0,1),~~t>0,
\end{align*} 
\textbf{where $\alpha = 0.0005$. Assume that $T=0$ at $t=0$ and that $T(t,x,y) = xy\tanh(10t)^2$ on the boundary. use \texttt{contourf} to plot the level curves to the temperature at times $t = 10,50,150,300$.}

We can appreciate that the equation to solve has the same form than the vorticity equation using the frozen values for the velocity given as data. Just adjusting the given code we obtain the following results.

\begin{figure}[H]
\centering     %%% not \center
\hspace*{\fill}
\subfigure[$t=1$ s.]{\includegraphics[scale=0.55]{T_10.eps}}
\hfill
\subfigure[$t=2$ s.]{\includegraphics[scale=0.55]{T_50.eps}}
\hspace*{\fill}

\hspace*{\fill}
\subfigure[$t=3$ s.]{\includegraphics[scale=0.55]{T_150.eps}}
\hfill
\subfigure[$t=5$ s.]{\includegraphics[scale=0.55]{T_300.eps}}
\hspace*{\fill}
\caption{Temperature Contours at different values of time.}
\end{figure}

\subsection*{Matlab code for this part}
\begin{verbatim}
clear variables; close all; clc
DATA=load('velocitydataPlatte.mat');
u=DATA.u;
v=DATA.v;
xx=DATA.xx;
yy=DATA.yy;
fluidflowFDTemperature(xx,yy,u,v)

% Fluid flow in a cavity (Navier-Stokes equations)
% Stream-function and vorticity formulation
%
% Rodrigo Platte, Arizona State University, April 2013.

function fluidflowFDTemperature(xx,yy,u,v)
% Parameters
path='Latex/FIGURES/';
N = 150;
alpha = 0.0005;
dt = min(0.2/(alpha*N^2),1e-3);

h = 1/N; % N+1 points in each direction

% Initial condition & pre-allocate memory (start with zeros at t = 0)
T = 0*xx;

% boundary condition
ii = 2:N; jj = 2:N;  % index for interior nodes
Ti = zeros(N-1,N-1); % interior values of vorticity

count = 0;
time = 0;
outputTime=[10 50 150 300];
endtime=outputTime(end);
n=1;
% main loop
while time<endtime
	if ( time < outputTime(n) && time+dt >= outputTime(n) ) %#ok<ALIGN>
        dt=outputTime(n)-time;
        n=n+1;
    else
        dt = min(0.2/(alpha*N^2),1e-3);
    end
   % advance Temperature with forward Euler
   Ti = Ti + dt*Trhs(u,v,T,alpha);
   T(ii,jj)=Ti;
   
   % update Temperature at boundary
   T = BCT(xx,yy,time,T);
   
   time = time+dt;
   count = count + 1;
   
   % Plot results every now and then
   if ismember(time,outputTime);
       plotresults(time,T);    
   end

end
    % Right-hand side of vorticity equation
    function rhs = Trhs(u,v,T,alpha)
        Tx = (T(ii,jj+1)-T(ii,jj-1))/(2*h);
        Ty = (T(ii+1,jj)-T(ii-1,jj))/(2*h);
        LT = (T(ii,jj+1)+T(ii,jj-1)+T(ii+1,jj)+T(ii-1,jj)-4*T(ii,jj))/h^2;
        rhs = -u(ii,jj).*Tx - v(ii,jj).*Ty + alpha*LT;          
    end

    % Enforce boundary condition (temperature)
    function T = BCT(xx,yy,t,T)
        T(1,:)=xx(1,:).*yy(1,:).*tanh(10*t).^2;
        T(N+1,:)=xx(N+1,:).*yy(N+1,:).*tanh(10*t).^2;
        T(:,1)=xx(:,1).*yy(:,1).*tanh(10*t).^2;
        T(:,N+1)=xx(:,N+1).*yy(:,N+1).*tanh(10*t).^2;
    end

    % Plot results
    function plotresults(time,T)
        figure();set(gcf,'Visible', 'off');
        pcolor(xx,yy,T)
        shading interp
        H = colorbar; set(H,'fontsize',16)
        caxis([0 1])
        colorbar
        axis square
        drawnow      
        xlabel('$x$','Interpreter','latex')
        ylabel('$y$','Interpreter','latex')
        set(get(gca,'ylabel'),'rotation',0)
        txt=[path,'T_' num2str(time)];
        saveas(gcf,txt,'epsc')
    end

end

\end{verbatim}