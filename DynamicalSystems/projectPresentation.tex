\documentclass[compress]{beamer}
\input{BEAMERoptions.tex}					                                         \usepackage[english]{babel}
\usepackage{graphics}
\usepackage{multimedia} % for movies and sound
\usepackage{times}

\title{The Knife Edge Viscometer}
\author[F. Castillo-Carrasco]{\emph{Francisco Castillo-Carrasco} }
\date{\today}
% Show ASU logo in title page
%%%%%%%%%%%%%%%%%%%%%%
\institute[Mathematics and Statistics]{
\includegraphics[height=.9cm]{ASUlogo.pdf} \\
{\color{ASUred} SCHOOL OF \textbf{MATHEMATICAL AND STATISTICAL SCIENCES}}}
%%%%%%%%%%%%%%%%%%%%%




	%%%%%%%%%%%%%%%%%%%%%%%%%%%%%%%%%%%%%%%%%%
%%%%%%%%%%%% Presentation Starts Here %%%%%%%%%%%%%%%%
	%%%%%%%%%%%%%%%%%%%%%%%%%%%%%%%%%%%%%%%%%%

\begin{document}
%%% Title frame %%%%%
\begin{frame}[plain]
	\titlepage
\end{frame}

%%%%%%%%%%%%%%%%%%%%%%%%%%%%%%%%%%
\begin{frame}
  \frametitle{Outline}
  \tableofcontents[pausesections]
\end{frame}
%%%%%%%%%%%%%%%%%%%%%%%%%%%%%%%%%%

\section[Introduction]{Introduction}

\subsection{The Knife Edge Viscometer}
\begin{frame} \frametitle{The Knife Edge Viscometer}
\begin{figure}
\centering
\includegraphics[scale=0.8]{KEviscometer.png}
\end{figure}
\end{frame}

\subsection{Quick Motivation}
\begin{frame} \frametitle{Quick Motivation}
Why do we study this problem?
\begin{itemize}
\item How to measure surface shear viscosity remains a controversial issue.

\item Monomolecular layers are key in broad areas
\begin{itemize}
\item Pharmaceuticals: interfacial processing.
\item Food processing: surfactants.
\item Natural: Gas absorption into a fluid (lungs, oceans).
\end{itemize}
\item Applications where a high degree of mixing is desired (microbioreactors), although at a low level of shear stress.
\end{itemize}
\end{frame}

\subsection{Governing Equations}
\begin{frame} \frametitle{Governing Equations}
\begin{columns}
\column{0.5\textwidth}
\includegraphics[scale=.95]{scheme.pdf}
\column{0.5\textwidth}
\pause
\vspace{-8mm}
{\footnotesize
\begin{block}{Cylindrical Coordinates}
$\psi\equiv$ Stream function\\
$\gamma\equiv$ Angular momentum\\
$\eta\equiv$ Azimuthal vorticity
\begin{align*}
\textbf{u} = (u,v,w) &= \left(-\frac{1}{r}\frac{\partial\psi}{\partial z},\frac{\gamma}{r},\frac{1}{r}\frac{\partial\psi}{\partial r}\right)\\
\nabla\times\textbf{u} &= \left(-\frac{1}{r}\frac{\partial\gamma}{\partial z},\eta,\frac{1}{r}\frac{\partial\gamma}{\partial r}\right)
\end{align*}
\vspace{-4mm}
\end{block}}
\end{columns}
\pause
{\small
\vspace{-5mm}
\begin{align*}
\frac{\partial\gamma}{\partial
t}-\frac{1}{r}\frac{\partial\psi}{\partial
z}\frac{\partial\gamma}{\partial
r}+\frac{1}{r}\frac{\partial\psi}{\partial
r}\frac{\partial\gamma}{\partial
z}&=\frac{1}{Re}\left(\frac{\partial^2\gamma}{\partial
z^2}+\frac{\partial^2\gamma}{\partial
r^2}-\frac{1}{r}\frac{\partial\gamma}{\partial
r}\right)\\
\frac{\partial \eta}{\partial t}-\frac{1}{r}\frac{\partial
\psi}{\partial z}\frac{\partial \eta}{\partial
r}+\frac{1}{r}\frac{\partial \psi}{\partial r}\frac{\partial
\eta}{\partial z}+\frac{\eta}{r^2}\frac{\partial \psi}{\partial
z}-\frac{2\gamma}{r^3}\frac{\partial \gamma}{\partial
z}&=\frac{1}{Re}\left(\frac{\partial^2\eta}{\partial
z^2}+\frac{\partial^2\eta}{\partial
r^2}+\frac{1}{r}\frac{\partial\eta}{\partial
r}-\frac{\eta}{r^2}\right)\\
\frac{\partial^2\psi}{\partial z^2}+\frac{\partial^2\psi}{\partial
r^2}-\frac{1}{r}\frac{\partial\psi}{\partial r}&=-r\eta
\end{align*}}
\end{frame}


\subsection{Boundary Conditions}

\begin{frame} \frametitle{Boundary Conditions}
\begin{columns}
\column{0.4\textwidth}
\includegraphics[scale=0.8]{scheme.pdf}
\column{0.6\textwidth}
\pause
\vspace{-6mm}
{\small
\begin{block}{Bottom (no slip)}
\vspace{-5mm}
\begin{align*}
\psi=\gamma=0~\text{ at } z=0\\
\eta=\left.-\frac{1}{r}\frac{\partial^2\psi}{\partial z^2}\right|_{z=0}
\end{align*}
\end{block}}
\pause
\vspace{-1mm}
{\small
\begin{block}{End Wall (no slip)}
\vspace{-5mm}
\begin{align*}
\psi=\gamma=0~\text{ at } r=A_R\\
\eta=\left.-\frac{1}{A_R}\frac{\partial^2\psi}{\partial r^2}\right|_{r=A_R}
\end{align*}
\end{block}}
\pause
\vspace{-1mm}
{\small
\begin{block}{Axis (symmetry)}
\vspace{-5mm}
\begin{align*}
\psi=\gamma=\eta=0~\text{ at } r=0\\
\text{Are we sure?  }
\left(\left.-\frac{1}{r}\frac{\partial\psi}{\partial r}\right|_{r=0}= 0\right)
\end{align*}
\end{block}}
\end{columns}
\end{frame}


\begin{frame} \frametitle{Boundary Conditions (II)}
\begin{columns}
\column{0.4\textwidth}
\includegraphics[scale=0.8]{scheme.pdf}
$$v = \frac{\gamma}{r}$$
\column{0.6\textwidth}
{\small
\begin{block}{Top (stress balance)}
\vspace{-5mm}
\begin{align*}
\psi=0~\text{ at } z=A_H\\
\eta=\left.-\frac{1}{r}\frac{\partial^2\psi}{\partial z^2}\right|_{z=A_H}\\
\frac{\partial^2 v}{\partial r^2}+\frac{1}{r}\frac{\partial v}{\partial r}-\frac{v}{r^2} = \frac{1}{Bo}\frac{\partial v}{\partial z}~~~~ \forall r\neq A_R/2, z=A_H\\
v=1 ~~~~ r= A_R/2, z=A_H
\end{align*}
\end{block}}
\pause
{\small
\begin{block}{The limit $Bo\rightarrow\infty$}
\vspace{-5mm}
\begin{align*}
\frac{\partial^2 v}{\partial r^2}+\frac{1}{r}\frac{\partial v}{\partial r}-\frac{v}{r^2} = 0
\end{align*}
\pause
The bulk flow and the monolayer are decoupled!
Analytic solution possible
\end{block}}
\end{columns}
\end{frame}

\section[Results]{Results}

\subsection{Observables}
\begin{frame}\frametitle{Observables}
\begin{columns}
\column{0.4\textwidth}
\begin{block}{Global}
\begin{itemize}
\item Kinetic Energy
\begin{align*}
E_k = \frac{1}{2}\int \|\textbf{u}\|^2dV
\end{align*}
\item Enstrophy
\begin{align*}
E_w = \int \|\nabla\times\textbf{u}\|^2dV
\end{align*}
\item Angular Momentum
\begin{align*}
E_{\gamma} = \int \gamma^2dV
\end{align*}
\end{itemize}
\end{block}
\column{0.4\textwidth}
\begin{block}{Local}
We probe the value of the three velocity components:
\begin{align*}
u &= -\frac{1}{r}\frac{\partial\psi}{\partial z}\\
v &= \frac{\gamma}{r}\\
w &= \frac{1}{r}\frac{\partial\psi}{\partial r}
\end{align*}
 at the point $\left(\frac{3}{4}A_H,\frac{3}{4}A_R\right)$.
\end{block}
\end{columns}
\end{frame}

\subsection{Parameter Sweep}

\begin{frame}\frametitle{Parameter Sweep}
\begin{figure}
\centering
\includegraphics[scale=1.2]{parGraphalpha0e0.pdf}
\end{figure}
\end{frame}

\begin{frame}\frametitle{Scaling}
\begin{figure}
\centering
\includegraphics[scale=1]{bifCurveScaled_E.pdf}
\end{figure}
\end{frame}

\subsection{Time-series at the bifurcation}
\begin{frame}\frametitle{At the Hopf Bifurcation}
We now focus on the results for $Bo = 20$ and $Re = 1270, 1300$ (right after the bifurcation).

\centering
\href{https://mathpost.asu.edu/~kiko/research/alpha0.html}{\beamergotobutton{Zero Forcing Frequency Webpage}}

Thanks to Jason Yalim.
\end{frame}
\subsection{Movies}
\begin{frame}\frametitle{Flow Field Animations}
\movie[externalviewer]{MOVIES IMPOSSIBLE TO EMBED IN THE BEAMER PRESENTATION}{movie.mp4}
\end{frame}

\begin{frame}
\centering
\Huge
Thank you.
\end{frame}

\end{document}
