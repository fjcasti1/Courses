%\documentclass[a4paper,landscape]{article}
\documentclass[a4paper]{article}
\usepackage{amsmath, amssymb}
\usepackage{multirow}
\usepackage{mathrsfs}
\usepackage{graphicx}
\usepackage{amsthm}
\usepackage{multicol}

%\usepackage[font={small}, margin=1cm]{caption}
\usepackage[margin=.1in]{geometry}
\setlength{\parindent}{0pt}
\renewcommand{\arraystretch}{0.8}
\newcommand{\C}{\mathbb{C}}
\newcommand{\F}{\mathcal{F}}
\newcommand{\K}{\mathbb{K}}
\newcommand{\N}{\mathbb{N}}
\newcommand{\Q}{\mathbb{Q}}
\newcommand{\R}{\mathbb{R}}
\newcommand{\Z}{\mathbb{Z}}
\newcommand{\ra}{\rightarrow}

\begin{document}

\fontsize{10.5}{6}
\selectfont
\begin{multicols}{2}
%\section*{CHAPTER 0:}
%{\bf L2 Inner Product:} The $L^2$ inner product on $L^2([a,b])$ is defined as\newline $\left\langle f,g\right\rangle_{L^2}=\int_a^bf(t)\overline{g(t)}dt$.
%{\bf l2 Inner Product:} The space $l^2$ is the set of all sequences $x_i\in\C$ with $\sum_{-\infty}^{\infty}|x_n|^2<\infty$. The inner product on $l^2$ i defined as $\left\langle X,Y\right\rangle_{l^2}=\sum_{n=-\infty}^{\infty}x_n\overline{y_n}$.
%{\bf Schwartz Inequality:}$\left|\left\langle X,Y\right\rangle\right|\leq||X||||Y||$
{\bf Triangle Inequality:}$\left|\left| X+Y\right|\right|\leq||X||+||Y||$
%{\bf Orthogonal Projection:} Suppose $V$ is an inner product space and $V_0$ is an $N$-dimensional subspace with orthonormal basis $\{e_1,e_2,\dots , e_N\}$. The orthogonal projection of a vector $v\in V$ onto $V_0$ is given by $v_0=\sum_{j=1}^{N}\left\langle v,e_j\right\rangle e_j$. In addition, $\left|\left|v-v_0\right|\right|=\min_{w\in V_0}\left|\left|v-w\right|\right|$
%{\bf Adjoints:} If $T:V\rightarrow W$ is a linear operator between two inner product spaces, the adjoint of $T$ is the linear operator $T^*:W\rightarrow V$, such that $\left\langle T(v),w\right\rangle_W=\left\langle v,T*(w)\right\rangle_V$.

\section*{CHAPTER 1: FOURIER SERIES}
%\subsubsection*{Real Fourier Series}

\underline{\textsl{\bf REAL FOURIER SERIES}}
{\bf Orthonormal Basis:} The set of functions $\{\frac{\sin(k\pi x/a)}{\sqrt{\pi}},\frac{1}{\sqrt{2\pi}},\frac{\cos(k\pi x/a)}{\sqrt{\pi}}\}$ with $k=1,2,\dots$, is an orthonormal set of functions in $L^2([-a,a])$.
{\bf Fourier Coefficients:} If $f(t)=a_0+\sum_{k=1}^{\infty}a_k\cos(k\pi t/a)+\sum_{k=1}^{\infty}b_k\sin(k\pi t/a)$ on the interval $-a\leq t\leq a$, then $a_0=\frac{1}{2a}\int_{-a}^{a}f(t)dt$, $a_k=\frac{1}{a}\int_{-a}^{a}f(t)\cos(k\pi t/a)dt$ and $b_k=\frac{1}{a}\int_{-a}^{a}f(t)\sin(k\pi t/a)dt$.
%\subsubsection*{Complex Fourier Series}

\underline{\textsl{\bf COMPLEX FOURIER SERIES}}
{\bf Orthonormal Basis:} The set of functions $\{\frac{1}{\sqrt{2a}}e^{i\frac{n\pi}{a} t}, n=0,\pm 1,\pm2,\dots\}$ is an orthonormal basis for $L^2([-a,a])$.
{\bf Fourier Coefficients:} If \newline $f(t)=\sum_{n=-\infty}^{\infty}\alpha_ne^{i\frac{n\pi}{a} t}$, then $\alpha_n=\frac{1}{2a}\int_{-a}^{a}f(t)e^{-i\frac{n\pi}{a} t}dt$
%\subsubsection*{Convergence Theorems}

\underline{\textsl{\bf CONVERGENCE THEOREMS}}
{\bf Riemann-Lebesgue Lemma:} Suppose $f$ is a piecewise continuous function on the interval $[a,b]$. Then $\lim_{k\rightarrow\infty}\int_a^bf(x)\cos(kx)dx=\lim_{k\rightarrow\infty}\int_a^bf(x)\sin(kx)dx=0$.
{\bf Convergence at a Point of Continuity:} Suppose $f$ is a continuous and $2\pi$-periodic function. Then for each point $x$, where the derivative of $f$ is defined, the Fourier series of $f$ converges to $f(x)$.
{\bf Convergence at a Point of Discontinuity:} Suppose $f$ is periodic function and piecewise continuous. Suppose $x$ is a point where $f$ is left and right differentiable (but not necessarily continuous). Then the Fourier series of $f$ at $x$ converges to $\frac{f(x-0)+f(x+0)}{2}$, i.e., converges to the average of the left and right limits of $f$.
{\bf Uniform Convergence:} The Fourier series of a continuous, piecewise smooth $2\pi$-periodic function $f(x)$ converges uniformly to $f(x)$ on $[-\pi,\pi]$.
{\bf Lemma 1.33:} Suppose $f(x)=a_0+\sum_{k=1}^{\infty}a_k\cos(kx)+\sum_{k=1}^{\infty}b_k\sin(kx)$ with $\sum_{k=1}^{\infty}|a_k|+|b:k|<\infty$. Then the Fourier series converges uniformly and absolutely to the function $f(x)$.
{\bf Convergence in the Mean:} Suppose $f$ is an element of $L^2([-\pi,\pi])$. Let $f_N(x)=a_0+\sum_{k=1}^Na_k\cos(kx)+\sum_{k=1}^Nb_k\sin(kx)$, where $a_k$ and $b_k$ are the Fourier coefficients of $f$. Then $f_N$ converges to $f$ in $L^2([-\pi,\pi])$, that is, $\left|\left|f_N-f\right|\right|_{L^2}\rightarrow 0$ as $N\rightarrow\infty$.
{\bf Parseval's Equation - Real Version: } Suppose $f(x)=a_0+\sum_{k=1}^{\infty} a_k\cos(kx)+\sum_{k=1}^{\infty}b_k\sin(kx)\in L^2[-\pi,\pi]$. Then $\frac{1}{\pi}\int_{-\pi}^{\pi}|f(x)|^2dx=2|a_0|^2+\sum_{k=1}^{\infty}\left(|a_k|^2+|b_k|^2\right)$.
{\bf Parseval's Equation - Complex Version: } Suppose $f(x)=\sum_{k=-\infty}^{\infty} \alpha_ke^{ikx}\in L^2[-\pi,\pi]$. Then \newline $\frac{1}{2\pi}||f||^2=\frac{1}{2\pi}\int_{-\pi}^{\pi}|f(x)|^2dx=\sum_{k=-\infty}^{\infty}|\alpha_k|^2$.

\section*{CHAPTER 2:FOURIER TRANSFORM}
{\bf Definition:} If $f$ is a continuously differentiable function with $\int_{-\infty}^{\infty}|f(t)|dt<\infty$, then $f(t)=\frac{1}{\sqrt{2\pi}}\int_{-\infty}^{\infty}\hat{f}(\lambda)e^{i\lambda t}d\lambda$, where $\hat{f}(\lambda)$ is the Fourier transform of $f(t)$ given by $\hat{f}(\lambda)=\frac{1}{\sqrt{2\pi}}\int_{-\infty}^{\infty}f(t)e^{-i\lambda t}dt$
%\newline
{\bf Properties:} 
\begin{itemize}
\item $\mathcal{F}[\alpha f+\beta g]=\alpha \mathcal{F}[f]+\beta \mathcal{F}[g]$ // $\mathcal{F}^{-1}[\alpha f+\beta g]=\alpha \mathcal{F}^{-1}[f]+\beta \mathcal{F}^{-1}[g]$
\end{itemize}
{\bf Convolution:} Suppose $f$ and $g$ are two square integrable functions. The convolution of $f$ and $g$ is defined by $(f*g)(t)=\int_{-\infty}^{\infty}f(t-x)g(x)dx=\int_{-\infty}^{\infty}f(x)g(t-x)dx$. 
{\bf Fourier Transform of the Convolution:} \newline$\mathcal{F}[f*g]=\sqrt{2\pi}\mathcal{F}[f]\cdot\mathcal{F}[g]$,   $~~\mathcal{F}^{-1}[\hat{f}\cdot\hat{g}]=\frac{1}{\sqrt{2\pi}}(f*g)$.
{\bf Pancherel Theorem:} The Fourier transform, and its inverse, preserves the $L^2$ inner product. $\left\langle\mathcal{F}[f],\mathcal{F}[g]\right\rangle_{L^2}=\left\langle f,g\right\rangle_{L^2}$ and $\left\langle\mathcal{F}^{-1}[f],\mathcal{F}^{-1}[g]\right\rangle_{L^2}=\left\langle f,g\right\rangle_{L^2}$.
%\subsubsection*{Linear Filters}

\underline{\textsl{\bf LINEAR FILTERS}}
{\bf Time Invariance:} A transformation $L$ (mapping signals to signals) is said to be time-invariant if for any signal $f$ and any real number $a$, $L[f_a](t)=(Lf)(t-a)$ for all $t$. In other words, $L$ is time-invariant if the time shifted input signal $f(t-a)$ is transformed by $L$ into the time shifted output signal $(Lf)(t-a)$.
{\bf Lemma 2.16:} Let $L$ be a linear, time-invariant transformation and let $\lambda$ be any fixed real number. Then, there is a function $h$ with $L(e^{i\lambda t})=\sqrt{2\pi}\hat{h}(\lambda)e^{i\lambda t}$. In other words, the output signal from a time-invariant filter of a sinusoidal input is also sinusoidal with the same frequency.
{\bf Convolution in Filters:} Let $L$ be a linear, time-invariant transformation on the space of signals that are piecewise continuous functions. Then there exists an integrable function, $h$, such that $L(f)=f*h$ for all signals $f$.
{\bf Causal Filters:} A causal filter is one for which the output signal begins after the input signal has started to arrive. Let $L$ be a time-invariant filter with response function $h$ (i.e., $Lf=f*h$). $L$ is a causal filter if and only if $h(t)=0$ for all $t<0$.
{\bf Theorem 2.20:} Suppose $L$ is a causal filter with response function $h$. Then the system function associated with $L$ is $\hat{h}(\lambda)=\frac{\mathcal{L}[h](i\lambda)}{\sqrt{2\pi}}$.
%\subsubsection*{The Sampling Theorem}

\underline{\textsl{\bf THE SAMPLING THEOREM}}
{\bf Definition 2.22:} A function $f$ is said to be frequency band limited if there exists a constant $\Omega>0$ such that $f(\lambda)=0$ for $|\lambda|>\Omega$. Note: $\Omega$ is the smallesq frequency for which the preceding equation is true.
{\bf Shannon-Whittaker Sampling Theorem:} Suppose that $\hat{f}(\lambda)$ is piecewise smooth and continuous and that $\hat{f}(\lambda)=0$ for $|\lambda|>\Omega$, where $\Omega$ is some fixed, positive frequency. Then $f=\mathcal{F}^{-1}[f]$ is completely determined by its values at the poiints $t_j=\frac{j\pi}{\Omega}, j=0,\pm 1,\pm 2,\dots$. More precisely, $f$ has the following series expansion: $f(t)=\sum_{j=-\infty}^{\infty}f\left(\frac{j\pi}{\Omega}\right)\frac{\sin(\Omega t-j\pi)}{\Omega t-j\pi}$, where the series converges uniformly.

\section*{CHAPTER 3: DISCRETE FOURIER TRANSFORM}
{\bf Set of n-periodic sequences:} Let $\mathcal{S}_n$ be the set of $n$-periodic sequences of complex numbers. Each element $y={y_j}_{j=-\infty}^{\infty}$ in $\mathcal{S}_n$, can be thought of as a periodic discrete signal where $y_j$ is the value of the signal at a time node $t=t_j$. The sequence $y_j$ is $n$-periodic if $y_{k+n}=y_k$ for any integer $k$.
{\bf Definition:} Suppose $y={y_k}$ is an element of $\mathcal{S}_n$. Let $\mathcal{F}_n(y)=\hat{y}$. That is, $\hat{y}_k=\sum_{j=0}^{n-1}y_j\overline{w}^{jk}$, where $w=e^{\frac{2\pi}{n}i}$. Then $y=\mathcal{F}^{-1}(\hat{y})$ is given by $y_j=\frac{1}{n}\sum_{k=0}^{n-1}\hat{y}_kw^{jk}$.
%\newline
{\bf Properties:} 
\begin{itemize}
\item Shifts or translations. If $y\in\mathcal{S}_n$ and $z_k=y_{k+1}$, then $\mathcal{F}[z]_j=w^j\mathcal{F}[y]_j$
\item Convolutions. If $y\in\mathcal{S}_n$ and $z\in\mathcal{S}_n$, then the sequence \newline $[y*z]_k:=\sum_{j=0}^{n-1}y_jz_{k-j}$ is also in $\mathcal{S}_n$. The sequence $y*z$ is called the convolution of the sequences $y$ and $z$.
\item The Convolution Theorem. $\mathcal{F}[y*z]_k=\mathcal{F}[y]_k\mathcal{F}[z]_k$
\item If $y\in\mathcal{S}_n$ is a sequence of real numbers, then $\mathcal{F}[y]_{n-k}=\overline{\mathcal{F}[y]}_k$, for $k\in[0,n-1]$, or $\hat{y}_{n-k}=\overline{\hat{y}}_k$
\end{itemize}

\section*{CHAPTER 4: HAAR WAVELET ANALYSIS}
{\bf Haar Scaling function:} The Haar scaling function is defined as $\phi(x) = 1$ if $x\in[0,1]$.
{\bf Definition:} Suppose $j$ is any nonnegative integer. The space of step functions at level $j$, denoted by $V_j$, is defined to be the space spanned by the set $\{\dots,\phi(2^j+1),\phi(2^j),\phi(2^j-1),\phi(2^j-2),\dots\}$.
{\bf Theorem 4.5:} A function $f(x)$ belongs to $V_0$//$V_j$ if and only if $f(2^jx)$//$f(2^{-j}x)$) belongs to $V_j$//$V_0$.
{\bf Theorem 4.6:} The set of functions $\{2^{j/2}\phi(2^jx-k);k\in\Z\}$ is an orthonormal basis of $V_j$.
{\bf Haar Wavelet:} The Haar wavelet is function $\psi(x) = \phi(2x)-\phi(2x-1)$.
{\bf Theorem 4.8:} Let $W_j$ be the space of functions of the form $\sum_{k\in\Z}a_k\psi(2^jx-k)$, $a_k\in\R$ (only a finite number of $a_k$ are nonzero). $W_j$ is the orthogonal complement of $V_j$ in $V_{j+1}$ and $V_{j+1}=V_j\bigoplus W_j$.
{\bf Theorem 4.9:} The space $L^2(\R)$ can be decomposed as an infinite orthogonal direct sum $L^2(\R)=V_0\bigoplus W_0\bigoplus W_1\dots$. In particular, each $f\in L^2(\R)$ can be written as $f=f_0+\sum_{j=0}^{\infty}w_j$, where $f_0\in V_0$ and $w_j\in W_j$.

\underline{\textsl{\bf SAMPLE}}
If the signal is continuous, $y=f(t)$, where $t$ represents time, choose the top level $j=J$ so that $2^j$ is larger than the Nyquist rate for the signal. Let $a_k^J=f(k/2^J)$. The top level $a_k^J$ is set equal to the $k$th term in the sampled signal, and $2^J$ is taken to be the sampling rate. In any case, we have the highest-level approximation to $f$ given by $f_J=\sum_{k\in\Z}a_k^j\phi(2^Jx-k)$.

\underline{\textsl{\bf DECOMPOSITION}}
{\bf Lemma 4.10:} The following relations hold for all $x\in\R$. $\phi(2^jx)=\left(\phi(2^{j-1}x)+\psi(2^{j-1}x)\right)/2$. $\phi(2^jx-1)=\left(\phi(2^{j-1}x)-\psi(2^{j-1}x)\right)/2$.
{\bf Theorem 4.12:} Suppose $f_j(x)=\sum_{k\in\Z}a_k^j\phi(2^jx-k)\in V_j$. Then $f_j$ can be decomposed as $f_j=w_{j-1}+f_{j-1}$, where $w_{j-1}=\sum_{k\in\Z}b_k^{j-1}\psi(2^{j-1}x-k)\in W_{j-1}$ and $f_{j-1}=\sum_{k\in\Z}a_k^{j-1}\phi(2^{j-1}x-k)\in V_{j-1}$, with $b_k^{j-1}=\frac{a_{2k}^j-a_{2k+1}^j}{2}$ and $a_k^{j-1}=\frac{a_{2k}^j+a_{2k+1}^j}{2}$.

\underline{\textsl{\bf RECONSTRUCTION}}
{\bf Theorem 4.12:} If $f=f_0+w_0+w_1+\dots +w_{j-1}$ with $f_{0}(x)=\sum_{k\in\Z}a_k^0\phi(x-k)\in V_0$ and $w_{j}=\sum_{k\in\Z}b_k^{j}\psi(2^jx-k)\in W_{j}$ for $0\leq j\leq J$, then $f(x)=f_J(x)=\sum_{k\in\Z}a_k^J\phi(2^Jx-k)\in V_J$. The $a_k^J$ are determined recursively by $a_k^j = a_l^{j-1}+b_l^{j-1}$ if $k=2l$ is even and $a_k^j = a_l^{j-1}-b_l^{j-1}$ if $k=2l+1$ is odd.

\section*{CHAPTER 5: MULTIRESOLUTION ANALYSIS}
{\bf Definition:} Let $V_j, j=\dots -1,0,1\dots,$ be a sequence of subspaces of cuntions in $L^2(\R)$. The collection of spaces $\{V_j,j\in\Z\}$ is called a \textit{multiresolution analysis with scaling funciton $\phi$} if the following conditions hold. {\bf 1.} (Nested) $V_j\subset V_{j+1}$. {\bf 2.} (Density) $\overline{\cup V_j}=L^2(\R)$. {\bf 3.} (Separation) $\cap V_j=\{0\}$ {\bf 4.} (Scaling) See Theorem 4.5(b) {\bf 5.} (Orthonormal basis) The function $\phi$ belongs to $V_0$ and the set $\{\phi(x-k);k\in\Z\}$ is an orthonormal basis for $V_0$.
{\bf Theorem 5.5:} Suppose $\{V_j; j\in\Z\}$ is a multiresolution analysis with scaling function $\phi$. Then for any $\in\Z$, the set of functions $\{\phi_{jk}(x)=2^{j/2}\phi(2^jx-k);k\in\Z\}$ is an orthonormal basis for $V_j$. 

\underline{\textsl{\bf THE SCALING RELATION}}
{\bf Theorem 5.6:} Suppose $\{V_j; j\in\Z\}$ is a multiresolution analysis with scaling function $\phi$. Then the following scaling relation holds: $\phi(x) = \sum_{k\in\Z} p_k\phi(2x - k)$, where $p_k = 2\int_{-\infty}^{\infty}\phi(x)\overline{\phi(2x - k)}dx$. Moreover, we also have $\phi_{j-1,l}=2^{-1/2}\sum_{k\in\Z}p_{k-2l}\phi_{jk}$. \newline {\bf Remark:} When the support of $\phi$ is compact, only a finite number of $p_k$ are nonzero, because when $|k|$ is large enough, the support of $\phi(2x - k)$ will be outside of the support of $\phi(x)$.
{\bf Theorem 5.9:} Suppose $\{V_j; j\in\Z\}$ is a multiresolution analysis with scaling function $\phi$. Then, provided the scaling relation can be integrated termwise, the following equalities hold: {\bf 1.} $\sum_{k\in\Z}p_{k-2l}\overline{p_k}=2\delta_{l0}$. {\bf 2.} $\sum_{k\in\Z}|p_{k}|^2=2$. {\bf 3.} $\sum_{k\in\Z}p_{k}=2$. {\bf 4.} $\sum_{k\in\Z}p_{2k}=1$ and $\sum_{k\in\Z}p_{2k+1}=1$.
{\bf Theorem 5.10:} Suppose $\{V_j; j\in\Z\}$ is a multiresolution analysis with scaling function $\phi = \sum_{k}p_{k}\phi(2x-k)$.  Let $W_j$ be the span of $\{\psi(2^jx-k); k\in\Z\}$ where $\psi(x) = \sum_{k\in\Z}(-1)^{k}\overline{p_{1-k}}\phi(2x-k)$. Then, $W_j\subset V_{j+1}$ is the orthogonal complement of $V_j$ in $V_{j+1}$. Furthermore, $\{\psi_{jk}(x) := 2^{j/2}\psi(2^jx - k); k\in\Z\}$ is an orthonormal basis for the $W_j$.
{\bf Theorem 5.11:} Let $\{V_j, j\in\Z\}$ be a multiresolution analysis with scaling function $\phi$. Let $W_j$ be the orthogonal complement of $V_j$ in $V_{j+1}$. Then $L^2(\R)=\dots\bigoplus W_{-1}\bigoplus W_{0}\bigoplus W_{1}\bigoplus\dots$. In particular, each $f\in L^2(\R)$ can be uniquely expressed as a sum $\sum_{k\in\Z}w_k$ with $w_k\in W_k$ and where the $w_k$'s are mutually orthogonal. Equivalently, the set of all wavelets, $\{\psi_{jk}\}_{j,k\in\Z}$, is an orthonormal basis for $L^2(\R)$. 

\underline{\textsl{\bf DECOMPOSITION}}
{\bf Orthogonal Form:} $a^{j-1}_l = 2^{-1}\sum_{k\in\Z}\overline{p_{k-2l}}a_k^j$ and $b^{j-1}_l = 2^{-1}\sum_{k\in\Z}(-1)^kp_{1-k+2l}a_k^j$
{\bf ON Form:} $\left\langle f,\phi_{j-1,l}\right\rangle = 2^{-1/2}\sum_{k\in\Z}\overline{p_{k-2l}}\left\langle f,\phi_{jk}\right\rangle$ and $\left\langle f,\psi_{j-1,l}\right\rangle = 2^{-1/2}\sum_{k\in\Z}(-1)^kp_{1-k+2l}\left\langle f,\phi_{jk}\right\rangle$

\underline{\textsl{\bf RECONSTRUCTION}}
{\bf Orthogonal Form:} $a^{j}_k = \sum_{l\in\Z}p_{k-2l}a_l^{j-1}+\sum_{l\in\Z}(-1)^k\overline{p_{1-k+2l}}b_l^{j-1}$
{\bf ON Form:} $\left\langle f,\phi_{jk}\right\rangle = 2^{-1/2}\sum_{l\in\Z}p_{k-2l}\left\langle f,\phi_{j-1,l}\right\rangle+2^{-1/2}\sum_{l\in\Z}(-1)^k\overline{p_{1-k+2l}}\left\langle f,\psi_{j-1,l}\right\rangle$
\section*{APPENDIX}
{\bf Identities:} $\sin^2x = (1-\cos 2x)/2, \cos^2x = (1+\cos 2x)/2$, $e^{ix}=\cos x + i \sin x, e^{-ix}=\cos x - i \sin x$,  \newline$e^{ix}+e^{-ix}=2\cos x$,\newline $e^{ix}-e^{-ix}=2i\sin x$. \newline%$2 \cosh x=(e^x+e^{-x}), 2\sinh x=(e^x-e^{-x}). \cosh^2 x-\sinh^2x=1$. \newline
{\bf Sum and Difference Formula:} $\sin(A\pm B)=\sin A \cos B\pm \cos A \sin B$. $\cos(A\mp B)=\cos A \cos B\pm \sin A \sin B$. $\tan(A \pm B)=(\tan A\pm \tan B)/(1\mp \tan A \tan B)$. \newline
{\bf Double Angle Formula:}  $\sin(2A)=2 \sin A \cos A$. $\cos(2A)=\cos^2 A-\sin^2 A=2 \cos^2 A-1=1-2\sin^2 A$. $\tan(2A)=(2\tan A)/(1-\tan^2 A)$.  \newline
%{\bf Half Angle Formula:} $\sin(A/2)=\pm \sqrt{(1-\cos A)/2}$. $\cos(A/2)=\pm \sqrt{(1+\cos A)/2}$. \newline
%{\bf Product to Sum:} $\cos A \cos B=(1/2)(\cos(A+B)+\cos(A-B))$. $\sin A \sin B=(1/2)(\cos(A-B)-\cos(A+B)$. $\sin A \cos B=(1/2)(\sin(A+B)+\sin(A-B)$.  \newline
{\bf Sum to Product:} $\sin A\pm \sin B=2\sin((A\pm B)/2)\cos((A\mp B)/2)$. $\cos A - \cos B=-2\sin((A+B)/2)\sin((A-B)/2)$. $\cos A + \cos B=2\cos((A+B)/2)\cos((A-B)/2)$. \newline
{\bf Geometric Sum:} $\sum_{k=0}^N z^k=\frac{1-z^{N+1}}{1-z}$. $\sum_{k=0}^{\infty}z^k=\frac{1}{1-z}$.  \newline
%{\bf General ODE Solutions:}  $y''=y(t)\implies y=c_1e^{-t}+c_2e^t \qed \; dy/dt+p(t)y=g(t) \implies y=(\int u(t)g(t))/u(t) + c$ where $u(t)=$exp$(\int p(t)dt) \qed \; y'=x; x'=y \implies x=c_1 \cosh t + c_2 \sinh t, y=c_1\sinh t+c_2 \cosh t$ or $x=c_1e^t+c_2e^{-t}, y=c_1e^t-c_2e^{-t} \qed \; y'=-x; x'=y \implies y=c_1 \cos t + c_2 \sin t, x=c_1\sin t-c_2 \cos t \qed \; x'=x+y; y'=-x+y \implies x=e^t(c_1 \cos t+c_2 \sin t); y=e^t(-c_1\sin t +c_2 \cos t) \qed \; v'=\gamma v, v(z,0)=u_0 \implies v=u_0e^{\gamma t} \qed$
{\bf Integrals: }
\begin{align*}
&\int (a+bx)\cos(kx)dx=\frac{(a+bx)\sin(kx)}{k}+\frac{bcos(kx)}{k^2}+C\\
&\int (a+bx)\sin(kx)dx=\frac{b\sin(kx)}{k^2}-\frac{(a+bx)cos(kx)}{k}+C\\
&\int (a+bx)e^{ikx}dx=\frac{e^{ikx}(b-ik(a+bx))}{k^2}+C\\
\end{align*}

\end{multicols}
\end{document}

