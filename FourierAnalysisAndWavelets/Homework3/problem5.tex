\begin{questions}

\question{
Let 
\begin{align*}
\phi(x)= \left\{
\begin{array}{ll}
      1 & $if $ 0\leq x<1, \\
      0 & $otherwise.$ \\
\end{array} 
\right.
\end{align*}
Show that
\begin{align*}
\left(\phi*\phi\right)(x)= \left\{
\begin{array}{ll}
      1-|x-1| & $if $ 0\leq x<2, \\
      0 & $otherwise.$ \\
\end{array} 
\right.
\end{align*}
}
\begin{solution}
Given the definition of the convolution,
\begin{align*}
\left(\phi*\phi\right)(x)=\int_{-\infty}^{\infty}\phi(x-t)\phi(t)dt,
\end{align*}
note that the convolution will be zero for those values of $x$ that make the functions $\phi(t)$ and $\phi(x-t)$ not overlap. A simple graph of the functions demonstrates that they overlap as long as $x\in[0,2)$. We have now two different situations:
\begin{itemize}
\item $x\in[0,1)$. Then,
\begin{align*}
\left(\phi*\phi\right)(x)&=\int_{-\infty}^{\infty}\phi(x-t)\phi(t)dt\\
&=\int_{0}^{x}\phi(x-t)\phi(t)dt\\
&=\int_{0}^{x}dt\\
&=x,
\end{align*}
which we can rewrite as
\begin{align*}
\left(\phi*\phi\right)(x)=1-(1-x)
\end{align*}
\item $x\in[1,2)$. Then,
\begin{align*}
\left(\phi*\phi\right)(x)&=\int_{-\infty}^{\infty}\phi(x-t)\phi(t)dt\\
&=\int_{x-1}^{1}\phi(x-t)\phi(t)dt\\
&=\int_{x-1}^{1}dt\\
&=1-(x-1)
\end{align*}
We can now summarize both results as
\begin{align*}
\left(\phi*\phi\right)(x)=1-|x-1|~~\text{for }x\in[0,2)
\end{align*}
\end{itemize}
\end{solution}
\end{questions}
