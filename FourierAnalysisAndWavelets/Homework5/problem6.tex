\begin{questions}

\question{Reconstruct $g\in V_3$, given these coefficients in its Haar wavelet decomposition:
\begin{align*}
a^2&=[1/2,2,5/2,-3/2],\\
b^2&=[-3/2,-1,1/2,-1/2].
\end{align*}
The first entry in each list corresponds to $k=0$. Sketch $g$.
}
\begin{solution}
It is easy to reconstruct $g\in V_3$ using the relations from \textsl{Theorem 4.14},
\begin{align*}
g(x)=\sum_{k=0}^{2^3-1}a_k^3\phi(2^3x-k)\in V_3,
\end{align*}
where
\begin{align*}
a_k^3=\left\{
\begin{array}{ll}
	 a_l^2+b_l^2 & $if $ k=2l $ is even$, \\
	 a_l^2-b_l^2 & $if $ k=2l+1 $ is odd$.
\end{array} 
\right.\\
\end{align*}
Therefore,
\begin{align*}
a^3&=[a_0^2+b_0^2,a_0^2-b_0^2,a_1^2+b_1^2,a_1^2-b_1^2,a_2^2+b_2^2,a_2^2-b_2^2,a_3^2+b_3^2,a_3^2-b_3^2]\\
a^3&=[-1,2,1,3,3,2,-2].
\end{align*}
Hence,
\begin{align*}
g(x)=\sum_{k=0}^{7}a_k^3\phi(8x-k),
\end{align*}
which we can expand as
\begin{align*}
g(x)=-\phi\left(8x\right)+2\phi\left(8x-1\right)+\phi\left(8x-2\right)+3\phi\left(8x-3\right)+3\phi\left(8x-4\right)+2\phi\left(8x-5\right)-2\phi\left(8x-6\right)-\phi\left(8x-7\right),
\end{align*}
and it is graphed in the next figure.
\begin{figure}[H]
\centering     %%% not \center
{\includegraphics[scale=0.6]{gp6.png}}
\caption{Function $g=\sum_{k=0}^{7}a_k^3\phi(8x-k)$.}
\end{figure}
\end{solution}
\end{questions}
