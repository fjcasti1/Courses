\begin{questions}

\question{
(a) What is the normalized IEEE single-precision representation of the number $5.5$?
}
\begin{solution}
We can find the binary representation of the number as follows. First, the integer part
\begin{align*}
5/2 &= 2+1;\\
2/2 &= 1+0;\\
1/2 &= 0+1;\\
\end{align*}
which implies
\begin{align*}
5_{10}=101_2.
\end{align*}
Second, the decimal part
\begin{align*}
0.5\times 2=1+0;
\end{align*}
which gives
\begin{align*}
0.5_{10}=0.1_2.
\end{align*}
Finally, we have that $5.5_{10}=101.1_2$. We express that in the normalized format and obtain
\begin{align*}
5.5_{10}=1.011000\dots 000\times 2^2.
\end{align*}
This means that $a_2=a_3=1$ and the rest of $a_j$ are zero.
\end{solution}
\question{
(b) If we change the significand of $5.5$ by one ulp, by how much does the value of the floating point representation change? Express your answer as a power of $2$.
}
\begin{solution}
We change the \textit{unit of last place}, $a_{23}$, from $0$ to $1$. This produces a change in the value of $\Delta x=2^{-23+2}=2^{-21}$
\end{solution}
\end{questions}