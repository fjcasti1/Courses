\begin{questions}

\question{Consider IEEE single-precision representations. (a) Is $1,000,000.0$ exactly representable in IEEE single precision?
}
\begin{solution}
Yes, we find the representation doing the same as in Problem 1, although this is a much longer case. We obtained 
\begin{align*}
1,000,000.0=1.11101000010010000000000\times 2^{19}
\end{align*}
\end{solution}
\question{(b) What is the smallest positive integer $M$ that does not have an exact IEEE single-precision representation?
}
\begin{solution}
Since we have 24 bits of mantisa, the largest value that those bits can represent is $2^{24}=16777216$. Therfore the next integer will not be representable. Therefore the smallest positive integer $M$ that does not have an exact IEEE single-precision representation is
\begin{align*}
M=2^{24}+1=16,777,217.
\end{align*} 
\end{solution}
\end{questions}

