\begin{questions}

\question{(a) What real number is represented by $(+, 1.5, 0)$ $(-, 1.5, 1)$ $(+, 1.5, 2)$ $(-, 1.5, −1)$?
}
\begin{solution}
\begin{align*}
(+, 1.5, 0)&=1.5\\
(-, 1.5, 1)&=-2^{1.5}=-2\sqrt{2}\\
(+, 1.5, 2)&=2^{2^{1.5}}=2^{2\sqrt{2}}=4^{\sqrt{2}}\approx 7.103\\
(-, 1.5, −1)&=-2^{-1.5}=-\frac{1}{2^{1.5}}=-\frac{1}{2\sqrt{2}}
\end{align*}
\end{solution}
\question{
(b) What is the set of representable values for $l = 0, \pm1, \pm2, \pm3, \pm4$?
}
\begin{solution}
\begin{itemize}
\item $l=0$,
\begin{align*}
S=\{x;x=\pm s, s\in [1,2)\}.
\end{align*}
\item $l=\pm 1$,
\begin{align*}
S=\{x;|x|= 2^{\pm s}, s\in [1,2)\}.
\end{align*}
\item $l=\pm 2$,
\begin{align*}
S=\{x;|x|= 2^{\pm 2^s}, s\in [1,2)\}.
\end{align*}
\item $l=\pm 3$,
\begin{align*}
S=\{x;|x|= 2^{\pm 2^{2^s}}, s\in [1,2)\}.
\end{align*}
\item $l=\pm 3$,
\begin{align*}
S=\{x;|x|= 2^{\pm 2^{2^{2^s}}}, s\in [1,2)\}.
\end{align*}
\end{itemize}
\end{solution}
\question{
(c) The number $G = 10^100$ is called a $googol$; $10^G$ is a $googolplex$. Express the largest representable value of (+, s, 4) as an approximate power of $G$.
}
\begin{solution}
First we have that the largest value of $(+,s,4)$ is
\begin{align*}
x\approx 2^{2^{2^{2^2}}}=2^{2^{2^{4}}}=2^{2^{16}}.
\end{align*}
To express it as an approximate power of $G$ we have
\begin{align*}
x&=G^y=\left(10^{100}\right)^y=10^{100y}.
\end{align*}
Using logarithms we get
\begin{align*}
\log_{10}x=100y\Rightarrow
y=\frac{1}{100}\log_{10}\left(2^{2^{16}}\right)=\frac{2^{16}}{100}\log_{10}\left(2\right)\approx
197.
\end{align*}
\end{solution}
\question{
(d) Is the largest representable value of (+, s, 5) greater or less than a googolplex? Explain.
}
\begin{solution}
Let $z=10^G$ and $w=(+,s,5)$. Using logarithms as before
\begin{align*}
\log_{10}z=10^{100}=G,
\end{align*}
and
\begin{align*}
\log_{10}w=2^{2^{16}}\log_{10}2=x\log_{10}2=G^y\log_{10}2.
\end{align*}
It is obvious that $\log_{10}2<<G$ and therefore
\begin{align*}
\log_{10}w=G^y\log_{10}2<\frac{G^y}{G}=G^{y-1}>G=\log_{10}z.
\end{align*}
Thus,
\begin{align*}
\log_{10}w=\log_{10}z\Rightarrow (+,s,5)>10^G.
\end{align*}
\end{solution}
\newpage
\question{
(e) Invent your own terminology as necessary to describe the largest representable values
of $(+, s, 6)$ and $(+, s, 7)$.
}
\begin{solution}
Let $b^{\#_n^e}$ denote the number obtained by calculating the consecutive power of $b$ $n$ times and ending with the power $e$. For example,
\begin{align*}
5^{\#_3^2}=5^{5^{5^2}},
\end{align*}
and
\begin{align*}
8^{\#_4^6}=8^{8^{8^{8^6}}}.
\end{align*}
With that notation we have,
\begin{align*}
(+, s, 6)= +2^{\#_6^s},
\end{align*}
and
\begin{align*}
(+, s, 7)= +2^{\#_7^s}.
\end{align*}
\end{solution}
\end{questions}
