\begin{questions}

\question{Suppose a double precision floating point number is stored on a
computer using 64 bits in the following way: sign 1 bit, exponent 8
bits, and mantissa 55 bits.  A given real number $r$ is written as
\begin{align*}
r = \pm m 2^n
\end{align*}
where the mantissa $m$ satisfies $1/2 \le m < 1$ and $-128\leq n \le 127$.  Give the following numbers in both base 2 and base 10
scientific notation:
\begin{enumerate}
\item [(a)] What is the largest positive number $realmax$ that can be stored?
\item [(b)] What is the smallest positive number $realmin$ that can be stored?
\item [(c)] What is the machine epsilon $\epsilon_M$ (take the leading 1 in $m$ to be a phantom)?
\end{enumerate}
}
\begin{solution}
We calculate the largest positive number as
\begin{align*}
realmax\approx 2^{127}\approx 1.7014\cdot 10^{38},
\end{align*}
and the smallest positive number as
\begin{align*}
realmin = 2^{-128}\approx 1.4694\cdot 10^{-39}.
\end{align*}
Lastly, the machine epsilon is
\begin{align*}
\varepsilon_M=\frac{1}{2}2^{-55}=2^{-56}\approx 1.3878\cdot 10^{-17}
\end{align*}
\end{solution}
\end{questions}
