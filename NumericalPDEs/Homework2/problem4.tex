\begin{questions}

\question{ (a) Using the trapezoidal rule integration method to approximate
$Q^n \approx \overline{Q}(t) = \int u(x, t) dx$, show that
discretizations of $u_t + f_x = 0$ of the form
\begin{displaymath} 
	u^{n+1}_i = u^n_i - \frac{\Delta t}{\Delta x}
	\left(F_{i+\half} - F_{i-\half}\right)
\end{displaymath}
are conservative. }

\begin{solution}

Using the trapezoidal rule we get the following
\begin{align*}
Q^n = \int u^n(x,t)dx = \sum_{j=0}^{N-1} \frac{1}{2}\left(u^n_{j+1}+u^n_j\right)\Delta x~,
\end{align*}
and
\begin{align*}
Q^{n+1} = \int u^{n+1}(x,t)dx &= \sum_{j=0}^{N-1} \frac{1}{2}\left(u^{n+1}_{j+1}+u^{n+1}_j\right)\Delta x\\
&=\sum_{j=0}^{N-1} \frac{1}{2}\left(u^n_{j+1}+u^n_j\right)\Delta x+\frac{1}{2}\Delta t \sum_{j=0}^{N-1}\left(F_{j-1/2}-F_{j+3/2}\right)\\
&= Q^n +\frac{1}{2}\Delta t \sum_{j=0}^{N-1}\left(F_{j-1/2}-F_{j+3/2}\right).
\end{align*}
From here it is easy to find,
\begin{align*}
\frac{dQ}{dt} \approx \frac{Q^{n+1}-Q^n}{\Delta t} &= \frac{1}{2} \sum_{j=0}^{N-1}\left(F_{j-1/2}-F_{j+3/2}\right) \\
& = \frac{1}{2}\left(F_{-1/2}+F_{1/2}-F_{N-1/2}-F_{N+1/2}\right),
\end{align*}
where we have used that the series is a telescoping series.
We can define the flux at the two boundaries as
\begin{align*}
F_0 = \frac{1}{2}\left( F_{-1/2}+F_{1/2}\right),
\end{align*}
and
\begin{align*}
F_N = \frac{1}{2}\left( F_{N-1/2}+F_{N+1/2}\right).
\end{align*}
In the previous, the nodes outside of the grid are ghost nodes. Finally we have obtained
\begin{align*}
\frac{dQ}{dt} \approx F_0-F_N,
\end{align*}
which proves that the discretization is conservative since the change of the conserved quantity $Q$ is the different between the inflow and the outflow.

\end{solution}

\question{ (b) For nonlinear diffusion, $f(u) = - D(u) u_x$
and $F_{i+\half} = f_{i+\half}$.  Show that if homogeneous Neumann
boundary conditions $u_x(x_L, t) = 0 = u_x(x_R, t)$ are imposed via
ghost points, $Q$ is constant in time.}

\begin{solution}

If homogeneous boundary conditions are imposed, the quantity must be conserved because there is no flux through the boundaries. Formally, we have
\begin{align*}
F_0 = -D(u_0)u_x|_{x=x_0} = 0,
\end{align*}
and
\begin{align*}
F_N = -D(u_N)u_x|_{x=x_N}=0,
\end{align*}
which makes
\begin{align*}
\frac{dQ}{dt} \approx 0~.
\end{align*}
Hence, $Q$ is constant in time.
\end{solution}
\end{questions}
