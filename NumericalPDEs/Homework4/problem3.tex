\begin{questions}

\question{
{\em $2 \times 2$ SOR example.}\/ Calculate the first two iterates
$x_1$ and $x_2$ for Jacobi, Gauss-Seidel, and SOR with $x_0 = (0, 0)$
for $A x = b$ with
\begin{displaymath}
        A = \left[
	\begin{array}{r r} 
	2 & -1 \\
	-1 & 2 \\
	\end{array} \right] ,~~
        b = \left[
	\begin{array}{r} 
	3\\
	-3\\
	\end{array} \right]
\end{displaymath}
\begin{displaymath}
        M_{SOR} = \left[
	\begin{array}{r r} 
	\frac{2}{\omega} & 0 \\
	-1 & \frac{2}{\omega} \\
	\end{array} \right] ,~
        B_{SOR} = \left[
	\begin{array}{c c} 
	1 - \omega & \frac{\omega}{2} \\
	\frac{\omega}{2} (1-\omega) & \left(1-\frac{\omega}{2}\right)^2 \\
	\end{array} \right]
\end{displaymath}
\begin{displaymath}
        \omega_{opt} = 4 (2 - \sqrt{3}) \approx 1.0718,~
        \lambda_1 = \lambda_2 = \omega_{opt} - 1  = \rho_{SOR} \approx 0.0718 .
\end{displaymath}
The exact solution is $x = (1, -1)$, $x_2^J = (3/4, -3/4)$, and
$x_2^{GS} = (9/8, -15/16)$.  Calculate $||e_2^J||_1$,
$||e_2^{GS}||_1$, and $||e_2^{SOR}||_1$.  Note that the SOR $x_2$ is
much closer to the exact solution.

}
\begin{solution}
\begin{itemize}
\item \textbf{Jacobi:}
We start with the Jacobi iteration, $x^{(k+1)}=x^{(k)}-D^{-1}r^{(k)}$, where
\begin{align*}
D=\left[
	\begin{array}{r r} 
	2 & 0 \\
	0 & 2 \\
	\end{array} \right]\Rightarrow D^{-1}=\left[
	\begin{array}{r r} 
	1/2 & 0 \\
	0 & 1/2 \\
	\end{array} \right].
\end{align*}
Since $x_0=0$, $r_0=b$. Then,
\begin{align*}
x_1=D^{-1}b=\left[
	\begin{array}{r r} 
	3/2 \\
	-3/2 \\
	\end{array} \right],~~r_1=Ax_1-b=\left[
	\begin{array}{r r} 
	3/2 \\
	-3/2 \\
	\end{array} \right].
\end{align*}
The second guess is
\begin{align*}
x_2=x_1-D^{-1}r_1=\left[
	\begin{array}{r r} 
	3/2 \\
	-3/2 \\
	\end{array} \right]-\left[
	\begin{array}{r r} 
	1/2 & 0 \\
	0 & 1/2 \\
	\end{array} \right]\left[
	\begin{array}{r r} 
	3/2 \\
	-3/2 \\
	\end{array} \right]=\left[
	\begin{array}{r r} 
	3/4 \\
	-3/4 \\
	\end{array} \right].
\end{align*}
Lastly, we compute the 1 error norm, $||e_J^{(2)}||_1=1/2$.
\item \textbf{Gauss Seidel:}
We start with the Gauss-Seidel iteration, $x^{(k+1)}=M^{-1}\left(M-A\right)x^{(k)}-M^{-1}b$, where
\begin{align*}
M=D+L=\left[
	\begin{array}{r r} 
	2 & 0 \\
	-1 & 2 \\
	\end{array} \right]\Rightarrow M^{-1}=\frac{1}{4}\left[
	\begin{array}{r r} 
	2 & 0 \\
	1 & 2 \\
	\end{array} \right],
\end{align*}
and 
\begin{align*}
M-A=-U=\left[
	\begin{array}{r r} 
	0 & 1 \\
	0 & 0 \\
	\end{array} \right]\Rightarrow M^{-1}\left(M-A\right)=\frac{1}{4}\left[
	\begin{array}{r r} 
	0 & 2 \\
	0 & 1 \\
	\end{array} \right]
\end{align*}
Since $x_0=0$,
\begin{align*}
x_1=M^{-1}b=\left[
	\begin{array}{r r} 
	-3/4 \\
	-3/4 \\
	\end{array} \right].
\end{align*}
The second guess is
\begin{align*}
x_2=M^{-1}\left(M-A\right)x^{(1)}-M^{-1}b=\frac{1}{4}\left[
	\begin{array}{r r} 
	0 & 2 \\
	0 & 1 \\
	\end{array} \right]\left[
	\begin{array}{r r} 
	-3/4 \\
	-3/4 \\
	\end{array} \right]-\frac{1}{4}\left[
	\begin{array}{r r} 
	2 & 0 \\
	1 & 2 \\
	\end{array} \right]\left[
	\begin{array}{r r} 
	3 \\
	-3 \\
	\end{array} \right]=\left[
	\begin{array}{r r} 
	9/8 \\
	-15/16 \\
	\end{array} \right].
\end{align*}
Finally, we compute the 1 error norm, $||e_{GS}^{(2)}||_1=3/16$.

\end{itemize}
\end{solution}
\end{questions}
