\begin{questions}

\question{
Compute the first two conjugate gradient iterates $x_1$ and $x_2$
with $x_0 = (0,0)$ with and without preconditioning to the solution $x
= (0,1)$ of $A x = b$:
\begin{displaymath}
	A = \left[ 
	\begin{array}{rr} 
	9 & 1 \\
	1 & 1
	\end{array}
	\right] ,~~
	b = \left[ 
	\begin{array}{r} 
	1\\
	1
	\end{array}
	\right] ,~~
	M = \left[ 
	\begin{array}{rr} 
	9 & 0 \\
	0 & 1
	\end{array}
	\right] .
\end{displaymath}
Calculate $||e_1^{CG}||_1$ and $||e_1^{PCG}||_1$.  Note that while
both CG and PCG give the exact $x$ in two steps $x_2 = (0, 1)$, PCG
gives a much better $x_1$.
}
\begin{solution}
\begin{itemize}
\item \textbf{CG:}
We start without using preconditioning. Since $x_0=0$, $r_0=b$ and $\beta_1=0$. Then,
\begin{align*}
d_1=r_0+\beta_1d_0=r_0=b,
\end{align*}
and,
\begin{align*}
\alpha_1=\frac{r_0^Tr_0}{d_1^TAd_1}=1/6.
\end{align*}
Then new guess is
\begin{align*}
x_1=x_0+\alpha_1d_1=\left[
	\begin{array}{r r} 
	1/6 \\
	1/6 \\
	\end{array} \right],
\end{align*}
and the new residual is
\begin{align*}
r_1=r_0+\alpha_1Ad_1=\left[
	\begin{array}{r r} 
	-2/3 \\
	2/3 \\
	\end{array} \right],
\end{align*}
We repeat the process again,
\begin{align*}
&\beta_2=\frac{r_1^Tr_1}{r_0^Tr_0}=4/9,\\
&d_2=r_1+\beta_2d_1=\left[
	\begin{array}{r r} 
	-2/9 \\
	10/9 \\
	\end{array} \right],\\
&\alpha_2=\frac{r_1^Tr_1}{d_2^TAd_2}=3/4,\\
&x_2=x_1+\alpha_2d_2=\left[
	\begin{array}{r r} 
	0 \\
	1 \\
	\end{array} \right].\\		
\end{align*}
\item \textbf{PCG:}
Now we use preconditioning. Since $x_0=0$, $r_0=b$ and $\beta_1=0$. We solve
\begin{align*}
Mz_0=r_0\Rightarrow z_0=\left[
	\begin{array}{r r} 
	1/9 \\
	1 \\
	\end{array} \right].
\end{align*}
Then,
\begin{align*}
d_1=z_0+\beta_1d_0=r_0=z_0,
\end{align*}
and,
\begin{align*}
\alpha_1=\frac{z_0^Tr_0}{d_1^TAd_1}=5/6.
\end{align*}
Then new guess is
\begin{align*}
x_1=x_0+\alpha_1d_1=\left[
	\begin{array}{r r} 
	5/54 \\
	5/6 \\
	\end{array} \right],
\end{align*}
and the new residual is
\begin{align*}
r_1=r_0+\alpha_1Ad_1=\left[
	\begin{array}{r r} 
	-2/3 \\
	2/27 \\
	\end{array} \right],
\end{align*}
We repeat the process again,
\begin{align*}
&Mz_1=r_1\Rightarrow z_1=\left[
	\begin{array}{r r} 
	-2/27 \\
	2/27 \\
	\end{array} \right],\\
&\beta_2=\frac{z_1^Tr_1}{z_0^Tr_0}=4/81,\\
&d_2=z_1+\beta_2d_1=\left[
	\begin{array}{r r} 
	-50/729 \\
	10/81 \\
	\end{array} \right],\\
&\alpha_2=\frac{z_1^Tr_1}{d_2^TAd_2}=27/20,\\
&x_2=x_1+\alpha_2d_2=\left[
	\begin{array}{r r} 
	0 \\
	1 \\
	\end{array} \right].\\		
\end{align*}
\end{itemize}
To finish we compute
\begin{align*}
||e^{(1)}_{CG}||_1&=1\\
||e^{(1)}_{PCG}||_1&=0.2593.
\end{align*}
The $PCG$ method gives indeed a much better $x_1$.
\end{solution}
\end{questions}
