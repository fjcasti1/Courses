\begin{questions}

\question{
Show that Lax-Friedrichs is conservative by verifying that
the numerical flux function
\begin{displaymath}
        F_{i+\half} = \half \left(f(w_i) + f(w_{i+1})\right) -
        \frac{\Delta x}{2 \Delta t} \left(w_{i+1} - w_i\right)
\end{displaymath}
correctly produces the Lax-Friedrichs method for $w_t + f(w)_x = 0$.
}
\begin{solution}
We start with the conservative form of for the given PDE,
\begin{align*}
w_i^{n+1}=w_i^n+\frac{\Delta t}{\Delta x}\left(F_{i-1/2}-F_{i+1/2}\right),
\end{align*}
introducing the given numerical flux function,
\begin{align*}
w_i^{n+1}=w_i^n+\frac{\Delta t}{\Delta x}\left[\half \left(f(w_{i-1}) + f(w_{i})-f(w_{i}) - f(w_{i+1})\right)-\frac{\Delta x}{2 \Delta t} \left(w_{i} - w_{i-1}-w_{i+1}+w_i\right)\right].
\end{align*}
Manipulating this expression we obtain the Lax-Friedrichs method,
\begin{align*}
w_i^{n+1}&=w_i^n+\frac{\Delta t}{2\Delta x} \left(f(w_{i-1})-f(w_{i+1})\right)-\frac{1}{2} \left(-w_{i+1} +2w_i-w_{i-1}\right),\\
&=\frac{\Delta t}{2\Delta x} \left(f(w_{i-1})-f(w_{i+1})\right)-\frac{1}{2} \left(-w_{i+1} -w_{i-1}\right),\\
&=\frac{1}{2} \left(w_{i+1} +w_{i-1}\right)-\frac{\Delta t}{2\Delta x} \left(f(w_{i+1})-f(w_{i-1})\right).
\end{align*}
Hence, the Lax-Friedrichs method is conservative.
\end{solution}
\end{questions}
