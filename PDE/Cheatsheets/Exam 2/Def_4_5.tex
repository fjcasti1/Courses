{\bf D 4.5}. Two vectors $u$ and $v$ in an IP space are called orthogonal if $\langle u,v \rangle=0$. A set $M$ of vectors is called orthogonal if $\langle u,v \rangle=0$ for all $u,v \in M, u \neq v. \; M$ is called orthonormal if, in addition, $||u||=1$ for all $u \in M$. A set $B$ in the inner product space $X$ is called a Schauder basis of $X$ if for any $u \in X$ and $\epsilon >0$ there exist $n \in \N$ and $v_1, \dots, v_n \in B, \alpha_1, \dots, \alpha_n \in \K$ such that $||u-\sum_{j=1}^n\alpha_jv_j||\leq \epsilon$. In other words, $B$ is called a Schauder basis if, for any $u \in X$ and $\epsilon >0$, there exists a finite subset $F$ of $B$ and elements $\alpha_v\in \K, v \in F$, such that $||u-\sum_{v \in F}\alpha_vv||\leq \epsilon$.  In yet other words, $B$ is a Schauder basis if the set of (finite) linear combinations of elements in $B$ is dense in $X$. An orthonormal Schauder basis is briefly called an orthonormal basis. 