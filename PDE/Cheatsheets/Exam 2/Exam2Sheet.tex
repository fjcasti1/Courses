\documentclass[a4paper,landscape]{article}
\usepackage{amsmath, amssymb}
\usepackage{multirow}
\usepackage{mathrsfs}
\usepackage{graphicx}
\usepackage{amsthm}
\usepackage{multicol}

%\usepackage[font={small}, margin=1cm]{caption}
\usepackage[margin=.1in]{geometry}
\setlength{\parindent}{0pt}
\renewcommand{\arraystretch}{0.8}
\newcommand{\C}{\mathbb{C}}
\newcommand{\F}{\mathcal{F}}
\newcommand{\K}{\mathbb{K}}
\newcommand{\N}{\mathbb{N}}
\newcommand{\Q}{\mathbb{Q}}
\newcommand{\R}{\mathbb{R}}
\newcommand{\Z}{\mathbb{Z}}
\newcommand{\ra}{\rightarrow}

\begin{document}

\fontsize{7.5}{6}
\selectfont
\begin{multicols}{3}
% CHAPTER 4: Elements of Fourier Series
%{\bf D 4.1}. Inner prod props  $\forall u,v,w \in X, \alpha \in \K$: (i) $\langle u,v\rangle=\overline{\langle v,u\rangle}. \implies \langle u,u\rangle$ is real. (ii) $\langle \alpha u,v\rangle = \alpha \langle u,v\rangle$ (assos). (iii) $\langle u+v,w\rangle=\langle u,w\rangle+\langle v,w\rangle$ (dist). (iv) $\langle u,u\rangle > 0$ if $u \neq 0$. If $\K=\R$, (i) becomes $\langle u,v\rangle=\langle v,u\rangle$. By (i) and (ii), (v) $\langle u, \alpha v\rangle=\bar{\alpha}\langle u,v\rangle$ and $\langle \alpha u,\alpha v\rangle=|\alpha|^2\langle u,v\rangle$. (vi) $\langle u,v+w\rangle=\langle u,v\rangle+\langle u,w\rangle$.  (vii) $\langle 0,u\rangle=\langle 0u,u\rangle=0\langle u,u\rangle=0$, (viii) $\langle u,0\rangle= \overline{\langle 0,u\rangle}=0$. 
%As an illustration, we show (ii) for the Euclidean product on $\R^n.\; x\cdot(y+z)=x\cdot(y_1+z_1,\dots,y_n+z_n)=\sum_{j=1}^nx_j(y_j+z_j)=\sum_{j=1}^n(x_jy_j+x_jz_j)=\sum_{j=1}^nx_jy_j+\sum_{j=1}^nx_jz_j=x\cdot y + x \cdot z$. Examples of other inner products on $\R^n$ are $\langle x,y\rangle=\sum_{j=1}^n\alpha_jx_jy_j$ where $\alpha_j>0, j=1,\dots,n$, are given real numbers. 
% Section 4.1: Inner Product Spaces
% 4.1.1 The Cauchy-Schwarz inequality
{\bf T 4.2}. Let $X$ be an IP space. If $u,v \in X$, then $|\langle u,v \rangle|^2 \leq \langle u,u \rangle\langle v,v \rangle$, equality iff $u, v$ are linearly dependent. 
% {\it Proof}. We can assume that $u,v \neq 0$, otherwise both sides of the inequality are 0 by (vii) and (viii).  Let us first assume that $u$ and $v$ are linearly dependent.  Since $v \neq 0, u=tv$ for some $t \in \K$ and $\langle u,v \rangle=t\langle v,v \rangle$, while, by (v), $\langle u,u \rangle=\langle tv,tv \rangle=|t|^2\langle v,v \rangle$. This implies equality in the Cauchy-Schwarz inequality.  We now assume that $u,v$ are linearly independent.  Let $t \in \K$ be arbitrary.  Then $u \neq tv$ and $0<\langle u-tv,u-tv \rangle=\langle u,u \rangle-\langle u,tv \rangle-\langle tv,u \rangle+|t|^2\langle v,v \rangle=\langle u,u \rangle-2\mathcal{R}(t\langle v,u \rangle+|t|^2\langle v,v \rangle$. We set $t=\langle u,v \rangle/\langle v,v \rangle. \; 0<\langle u,u \rangle-2\mathcal{R}(|\langle u,v \rangle|^2/\langle v,v \rangle)+(|\langle u,v \rangle|^2/\langle v,v \rangle^2)\langle v,v \rangle=\langle u,u \rangle-|\langle u,v \rangle|^2/\langle v,v \rangle$. We reorganize the terms and obtain the strict Cauchy-Schwarz inequality. 
%{\bf C 4.3}. $||u||=\sqrt{\langle u,u \rangle}$ is a norm on $X$.  
%{\it Proof}. Let $u,v \in X$. Then $||u+v||^2=\langle u+v,u+v\rangle=\langle u,u\rangle + 2 \mathcal{R}\langle u,v\rangle+\langle v,v\rangle$. By the Cauchy-Schwarz inequality, $||u+v||^2\leq ||u||^2+2||u||\;||v||+||v||^2=(||u||+||v||)^2$. Let $\alpha \in \K$. Then $||\alpha u||^2=\langle \alpha u,\alpha u\rangle=\alpha \bar{\alpha}\langle u,u\rangle=|\alpha|^2||u||^2$. We take square roots and obtain $||\alpha u||=|\alpha| \; ||u||$. Let $u \neq 0$. Then $\langle u,u\rangle \neq 0$ and $||u|| \neq 0$. 
%% Example 4.4
{\bf Expl 4.4}. Let $I=[a,b]$ and $X$ be the set of cont funct $f:I\ra \C, X = C([a,b],\C)=C^{(c)}[a,b]. \; X$ is a vector subspace of $\C^I$. Define an IP on $X$ by $\langle f,g \rangle = \int_a^bf(t)\overline{g(t)}dt$. (4.1). Recall that the integral of a continuous complex-valued function $h$ can be reduced to that of continuous real-valued functions by using real and imaginary parts and setting $\int_a^b h(t)dt=\int_a^b \mathcal{R}h(t)dt+i\int_a^b \mathcal{I}h(t)dt$. We check just one of the inner product properties, $\overline{\langle g,f \rangle}=\overline{\int_a^b g(t)\overline{f(t)}dt}=\int_a^b \overline{g(t)\overline{f(t)}}dt=\int_a^b \overline{g(t)}\cdot \overline{\overline{f(t)}}dt=\langle f,g\rangle$. The norm is given by $||f||=(\int_a^b|f(t)|^2dt)^{1/2}$. (4.2) As an application of the Cauchy-Schwartz inequality we have $\int_a^b |f(t)|dt=|\langle|f|,1\rangle|\leq ||f||||1||=(\int_a^b|f(t)|^2dt)^{1/2}(b-a)^{1/2}$. Alternatively we can choose $\langle f,g\rangle = 1/(b-a)\int_a^b f(t)\overline{g(t)}dt$ or $X$ as the space of Riemann integrable functions or the space of square (Lebesgue) integrable functions $L^2([a,b],\C)$. 
% Section 4.2: General Fourier Expansion
%{\bf D 4.5}. Two vectors $u$ and $v$ in an IP space are called orthogonal if $\langle u,v \rangle=0$. A set $M$ of vectors is called orthogonal if $\langle u,v \rangle=0$ for all $u,v \in M, u \neq v. \; M$ is called orthonormal if, in addition, $||u||=1$ for all $u \in M$. A set $B$ in the inner product space $X$ is called a Schauder basis of $X$ if for any $u \in X$ and $\epsilon >0$ there exist $n \in \N$ and $v_1, \dots, v_n \in B, \alpha_1, \dots, \alpha_n \in \K$ such that $||u-\sum_{j=1}^n\alpha_jv_j||\leq \epsilon$. In other words, $B$ is called a Schauder basis if, for any $u \in X$ and $\epsilon >0$, there exists a finite subset $F$ of $B$ and elements $\alpha_v\in \K, v \in F$, such that $||u-\sum_{v \in F}\alpha_vv||\leq \epsilon$.  In yet other words, $B$ is a Schauder basis if the set of (finite) linear combinations of elements in $B$ is dense in $X$. An orthonormal Schauder basis is briefly called an orthonormal basis. 
%% Example 4.6
{\bf Exmp 4.6}. (a) $\C^n$ is an inner product space (actually a Hilbert space, i.e. a complete inner product space) with the inner product $\langle x,y \rangle = \sum_{j=1}^nx_j\bar{y}_j$.  An orthonormal basis for $\C^n$ is given by $\{e^j\}_{j=1,\dots,n}$ where $e^j$ is the vector with all coordinates 0 except the $j^{th}$ which is 1. (b) Let $\ell^2$ be the space of all real sequences $(x_j)$ with $\sum_{j=1}^n|x_j|^2 < \infty. \; \ell^2$ is a Hilbert space with inner product $\langle x,y \rangle = \sum_{j=1}^{\infty} x_j y_j, x=(x_j), y=(y_j)\in \ell^2$. An orthonormal basis is given by $\{e^k\}_{k\in \N}$ where $e^k$ is the sequence with $e_j^k=0, j \neq k, e_j^k=1, j=k$.  
{\bf P 4.7}. Let $M$ be an orthonormal set in an inner product space $X$ over $\K$ and $F$ be a finite subset of $M$. Then the following hold for all $u \in X$:  (a) $||u-\sum_{v \in F}\langle u,v \rangle v||^2 = ||u||^2-\sum_{v \in F}|\langle u,v \rangle |^2 $. (b) If $G$ is also a finite subset of $M$ and $F \subseteq G, ||u-\sum_{v \in F}\langle u,v \rangle v|| \geq ||u-\sum_{v \in G}\langle u,v \rangle v||$. (c) (Bessel's inequality) $\sum_{v \in F}|\langle u,v \rangle|^2\leq ||u||^2$.  (d) (Best approximation) For any choice of $\alpha_v \in \K, v \in F, ||u-\sum_{v \in F}\langle u,v \rangle v||\leq ||u-\sum_{v \in F}\alpha_v v||$. 
%{\it Proof}. Let $F$ be a finite subset $M$ and $\{\alpha_v; v\in F\}$ a subset of $\K$. By the properties of the inner product, $||u-\sum_{v \in F}\alpha_v v||^2=\langle u-\sum_{v \in F}\alpha_v v, u-\sum_{v \in F}\alpha_v v\rangle = \langle u,u \rangle -\sum_{v \in F}\langle u, \alpha_v v\rangle -\sum_{v \in F}\langle \alpha_v v, u \rangle + \sum_{v,w \in F}\langle  \alpha_v v,\alpha_w w\rangle =||u||^2-\sum_{v \in F}\bar{\alpha}_v \langle u,v \rangle - \sum_{v \in F}\alpha_v \overline{\langle u,v \rangle}+\sum_{v,w \in F}\alpha_v \bar{\alpha}_w\langle u,w \rangle$. By orthonormality, $||u-\sum_{v \in F}\alpha_v v||^2=||u||^2+\sum_{v\in F}\alpha_v \bar{\alpha}_v-\sum_{v \in F}\bar{\alpha}_v \langle u,v \rangle - \sum_{v \in F}\alpha_v \overline{\langle u,v \rangle}=||u||^2+\sum_{v\in F}((\alpha_v - \langle u,v \rangle )(\overline{\alpha_v - \langle u,v \rangle)}- \langle u,v \rangle \overline{\langle u,v \rangle}).$ Since $|z|^2 = z \bar{z}$ for any complex number $z, ||u-\sum_{v \in F}\alpha_v v||^2=||u||^2 + \sum_{v \in F}|\alpha_v -\langle u,v \rangle|^2-\sum_{v \in F}|\langle u,v\rangle |^2$. (4.3) (a) follows by setting $\alpha_v = \langle u,v \rangle$ for $v \in F$. (b) and (c) follow from (a). By (4.3) and part (a) $||u-\sum_{v \in F}\alpha_v v||^2\geq||u||^2 - \sum_{v \in F}|\langle u,v \rangle|^2 = ||u-\sum_{v \in F}\langle u,v\rangle v ||^2 \qed$ 
{\bf T 4.8}. Let $M$ be an orthonormal basis of the IP space $X$ and $u \in X$. Then, for any $\epsilon >0, \exists$ a finite set $F \subseteq M$ s.t. for all finite sets $G$ with $F \subseteq G \subseteq M, ||u-\sum_{v \in G}\langle u,v\rangle v ||\leq \epsilon$ and $||u||^2 \leq \sum_{v \in G}|\langle u,v\rangle|^2 + \epsilon^2$. 
%{\it Proof}. Let $\epsilon>0$. As $M$ is an orthonormal basis, there exists some finite subset $F$ of $M$ and coefficients $\alpha_v \in \K, v\in F$, such that $||u - \sum_{v \in F} \alpha_v v|| \leq \epsilon$. Let $G$ be a finite set, $F \subseteq G \subseteq M$. Set $\alpha_v = 0$ for $v \in G \backslash F$. By the best approximation property (Proposition 4.7 (d)), applied to $G,  ||u-\sum_{v \in G}\langle u,v\rangle v ||\leq ||u - \sum_{v \in G} \alpha_v v|| \leq ||u - \sum_{v \in F} \alpha_v v|| \leq \epsilon$. By Proposition 4.7 (a), applied to $G, 0 \leq ||u||^2 -  \sum_{v \in G}|\langle u,v\rangle|^2=||u-\sum_{v \in G}\langle u,v \rangle v ||^2 \leq \epsilon^2 \qed$ If $M$ is an orthonormal basis of $X$, then every vector $u \in X$ is uniquely determined by its Fourier coefficients $\langle u,v \rangle, v \in M$. 
%{\bf C 4.9}. Let $M$ be an orthonormal basis of the IP space $X$ and $u_1, u_2 \in X$. If $\langle u_1, v\rangle=\langle u_2, v \rangle$ for all $v \in M$, then $u_1=u_2$. 
%{\it Proof}. Let $u_1, u_2 \in X$ and $\langle u_1, v\rangle=\langle u_2, v \rangle$ for all $v \in M$. Let $\epsilon >0$. By Theorem 4.8, there exist finite subsets $F_1$ and $F_2$ of $M$ such that, if $j=1,2$ and $G$ is a finite set with $F_j \subseteq G \subseteq M$, then $||u_j-\sum_{v\in G}\langle u_j,v\rangle v|| <\epsilon/2$.  Set $G = F_1 \cup F_2$. Then $G$ is a finite subset of $M$ and contains $F_1$ and $F_2$. By the triangle inequality, $||u_1-u_2||=||u_1-\sum_{v\in G}\langle u_1,v\rangle v+\sum_{v\in G}\langle u_2,v\rangle v-u_2|| \leq ||u_1-\sum_{v\in G}\langle u_1,v\rangle v||+||\sum_{v\in G}\langle u_2,v\rangle v-u_2|| < \epsilon/2+\epsilon/2=\epsilon$. Since $\epsilon >0$ has been arbitrary, $||u_1-u_2||=0$ and $u_1=u_2 \qed$  Recall that a nonempty set $M$ is denumerable iff there exists a bijective function $f: \N \ra M$. 
{\bf T 4.10}. Let $M$ be a denumerable orthonormal basis of the IP space $X$ and $f: \N\ra M$ be bijective.  Then $u = \sum_{n=1}^{\infty}\langle u,f(n)\rangle f(n)$ (Fourier expansion) and $||u||^2 =  \sum_{n=1}^{\infty}|\langle u,f(n)\rangle|^2$. (Parseval's relation)
% {\it Proof}. Let $\epsilon >0$. Choose a finite set $F \subseteq M$ according to Theorem 4.8. Since $f$ is surjective, for each $v \in F$ there exists some $n_v \in \N$ such that $f(n_v)=v$ Let $N = \max_{v\in F}n_v$. Since $F$ is finite, $N \in \N$. Let $m \in \N$ and $m \geq N$. Then $G=f(\{1,\dots, m\})\supseteq F$ and, by Theorem 4.8 and the injectivity of $f, \; \epsilon >||u - \sum_{v \in G}\langle u,v\rangle v||=||u - \sum_{n=1}^m \langle u,f(n)\rangle f(n)||$ and $\sum_{n=1}^m |\langle u,f(n)\rangle|^2 \leq ||u||^2\leq \sum_{n=1}^m \langle u,f(n)\rangle|^2+\epsilon^2 \qed$ The first inequality follows from Bessel's inequality. 
{\bf T 4.11}. Let $B$ be a denumerable orthonormal basis of the IP space $X, I \subseteq \R$, and $u: I \ra X$ such that, for all $v \in B, \langle u(t),v\rangle$ is a (uniformly) cont func of $t \in I$. Let $\{\alpha_v; v \in B\}$ be a family in $\R_+$ such that $|\langle u(t),v\rangle | \leq \alpha_v$ for all $t \in I$ and $v \in B$. Assume that there is some bijective $f:  \N\ra B$ s.t. the series $\sum_{n=1}^{\infty} \alpha_{f(n)}^2$ converges in $\R$. Then $u$ is (unif) cont.  
%{\it Proof}. Let $\epsilon > 0$. Since $\sum_{n=1}^{\infty}\alpha_{f(n)}^2$ converges in $\R$, there exists some $m \in \N$ such that $\sum_{n=m+1}^{\infty}\alpha_{f(n)}^2 <(\epsilon/4)^2$. Set $u_m(t) = \sum_{n=1}^m \langle u(t), f(n) \rangle f(n)$. For all $t \in I$, by Fourier expansion,  $u(t) - u_m(t) = \sum_{n=m+1}^{\infty} \langle u(t), f(n) \rangle f(n)$, and by Parseval's relation,  $||u(t) - u_m(t)||^2 = \sum_{n=m+1}^{\infty} |\langle u(t), f(n) \rangle|^2\leq \sum_{n=m+1}^{\infty} \alpha_{f(n)}^2 <(\epsilon/4)^2$. So $u_m(t) \ra u(t)$ as $m \ra \infty$ uniformly for $t \in I$. Since, for each $v \in F, \langle u(t), v \rangle$ is a (uniformly) continuous function of $t \in I, u_m$ is (uniformly) continuous as finite sum of (uniformly) continuous functions.  Let $r \in I$. Then there exists some $\delta >0$ such that $||u_m(t)-u_m(r)|| < \epsilon/2$ for all $t \in I$ with $|t-r|<\delta$. So let $t \in I$ and $|t-r|<\delta$. By the triangle inequality, $||u(t)-u(r)|| \leq ||u(t)-u_m(t)||+||u_m(t)-u_m(r)|| +||u_m(r)-u(r)||\leq \epsilon/4+\epsilon/2+\epsilon/4=\epsilon$.  The proof of uniform continuity is similar. $\qed$ 
% Exercises
%{\bf Exercises} Recall that a Hilbert space is an inner product space that is a Banach space under the norm induced by the inner product. 
{\bf Exercise 4.2.1} (Riesz-Fisher Theorem). Let $\{v_m;m\in \N\}$ be an orthonormal set in a Hilbert space $H$ over $\K$ and $(\alpha_m)$ a sequence in $\K$. Show: The series $\sum_{m=1}^{\infty} \alpha_m v_m$ exists in $H$ iff $\sum_{m=1}^{\infty} |\alpha_m|^2 < \infty$. Further, if one and then both of these statements hold, $||\sum_{m=1}^{\infty} \alpha_m v_m||^2=\sum_{m=1}^{\infty}|\alpha_m|^2$. {\it Proof}. We define partial sums in $H$, $x_n=\sum_{m=1}^n \alpha_m v_m$, and in $\K$, $\beta_n = \sum_{m=1}^n |\alpha_m|^2$.  By the properties of the inner product and orthonormality, $||x_n - x_k||^2 = ||\sum_{m=k+1}^n \alpha_m v_m||^2=\langle\sum_{m=k+1}^n \alpha_m v_m, \sum_{j=k+1}^n \alpha_j v_j \rangle = \sum_{m=k+1}^n \sum_{j=k+1}^n \alpha_m \bar{\alpha}_j \langle  v_m,  v_j \rangle =\sum_{m=k+1}^n|\alpha_m|^2 = |\beta_n-\beta_k|$. This shows that $(x_n)$ is a Cauchy sequence in $H$ iff $(\beta_n)$ is a Cauchy sequence in $\R$. Assume that $\sum_{m=1}^{\infty}|\alpha_m|^2$ converges.  Then $(\beta_n)$ is a Cauchy sequence in $\R$ and $(x_n)$ is a Cauchy sequence in $H$.  Since $H$ is complete, $(x_n)$ converges, i.e., $\sum_{n=1}^{\infty} \alpha_n x_n$ converges. The other direction follows similarly.  Finally, by continuity of the norm and orthonormality, $||\sum_{m=1}^{\infty} \alpha_m v_m||^2=\lim_{n\ra \infty}||x_n||^2=\lim_{n\ra \infty}\sum_{m=1}^{\infty}|\alpha_m|^2=\sum_{m=1}^{\infty}|\alpha_m|^2 \qed$
{\bf Exercise 4.2.2}. Let $X$ be a Hilbert space. Let $M=\{v_m;m\in \N\}$ be an orthonormal subset of $X$. Show: $\sum_{m=1}^{\infty} \langle u,v_m\rangle v_m$ converges for every $u \in X$. Warning: This means that the Fourier series of $u$ converges, but it may happen that it does not equal $u$ (unless $M$ is an orthonormal basis). {\it Proof}. Combine Bessel's inequality with Exercise 4.2.1 choosing $\alpha_v = \langle u,v \rangle \qed$ 
{\bf Exercise 4.2.3}. Let $X$ be an inner product space and $M$ a denumerable orthonormal subset of $X$. Show (a) If $M$ is an orthonormal basis and $x \in X$, then $\langle x,v \rangle = 0$ for all $v \in M$ implies that $x = 0$. (b) If $X$ is an Hilbert space and if, for all $x \in X, \langle x,v \rangle = 0$ for all $v \in M$ implies that $x = 0$, then $M$ is an orthonormal basis. {\it Proof}. (a) Let $M=\{v_m; m \in \N\}$ be an orthonormal basis and $x \in X$. Then $x$ is represented by its Fourier series, $x = \sum_{m=1}^{\infty}\langle x, v_m\rangle v_m$. Assume that $\langle x,v \rangle = 0$ for all $v \in M$. Then $x =0$. (b) Let $x \in X$. By Exercise 4.2.2, the series $\sum_{m=1}^{\infty}\langle x, v_m \rangle v_m =: y$ converges. By orthonormality and continuity of the inner product, $\langle y, v_k\rangle = \langle x, v_k \rangle$ for all $k\in \N$. So $\langle y-x, v \rangle = 0$ for all $v \in M$. By assumption, $y-x = 0$ i.e., $x = y = \sum_{m=1}^{\infty}\langle x, v_m \rangle v_m$. This means that $M$ is an orthonormal basis $\qed$  
% Section 4.3: Classic Fourier series
{\bf T 4.12}. The set $B=\{v_j;j\in \Z\}$ with $v_j(x)=e^{ijx}$ is an orthonormal basis for the following IP spaces with the IP $\langle f,g \rangle = 1/(2 \pi) \int_{-\pi}^{\pi} f(x)\overline{g(x)}dx: X=C([-\pi, \pi],\C), X = L^2([-\pi, \pi],\C)$ and the space of Riemann integrable functions on $[-\pi, \pi]$ with values in $\C$. Further the Fourier series $\sum_{j=-\infty}^{\infty}\hat{f}_je^{ijx}, \hat{f}_j=1/(2\pi)\int_{-\pi}^{\pi}f(x)e^{-ijx}dx$, converges to $f$ in the sense that for every $\epsilon >0 \exists$ some $N \in \N$ with $\int_{-\pi}^{\pi}|f(x)-\sum_{j=-k}^m\hat{f}_je^{ijx}|^2dx<\epsilon^2$ (4.4) for all $k,m \in \N$ with $k,m>N$. 
%{\it Proof}. Let $j \neq k$. Then $\int_{-\pi}^{\pi}e^{ijx}e^{-ikx}dx=\int_{-\pi}^{\pi}e^{i(j-k)x}dx=1/((j-k)i)(e^{i(j-k)\pi}-e^{-i(j-k)\pi})=1/((j-k)i)(\cos(j-k)\pi)-\cos(-(j-k)\pi)+2i\sin((j-k)\pi)=0$, and $\int_{-\pi}^{\pi}e^{ijx}e^{-ijx}dx=2 \pi$. This shows that $B$ is an orthonormal set. Let $g: [-\pi, \pi] \ra \C$ be continuous and $g(-\pi)=g(\pi)$. Let $\mathbb{T}=\{z \in \C; |z|=1\}$. Since every $z \in \mathbb{T}$ can be uniquely written as $z = e^{ix}$ with some $x \in (-\pi,\pi]$, the definition $f(z)=g(e^{-ix})$ is sound and provides a function $f: \mathbb{T} \ra \C$. Since $g$ is continuous and $g(-\pi)=g(\pi)$, $f$ is continuous. By the Stone-Weierstra$\ss$ approximation theorem, there exists a sequence $(p_m)$ of polynomials $p_m(z)=\sum_{j=-m}^m c_jz^j, z \in \mathbb{T}$, such that $p_m(z) \ra f(z)$ as $m \ra \infty$ uniformly for $z \in \mathbb{T}$. Hence $q_m(x) := \sum_{j=-m}^m c_je^{ijx} \ra g(x), m \ra \infty$, uniformly for $x \in [-\pi,\pi]$. (The latter is the Weierstra$\ss$' Trigonometric Approximation Theorem). Since $C[-\pi,\pi]$ is dense in $L^2[-\pi,\pi]$, for every $h \in L^2[-\pi,\pi]$ there exists a sequence $(g_n)$ in $C[-\pi,\pi]$ such that $||h-g_n||\ra 0$. The $g_n$ can be modified to satisfy $g_n(-\pi)=g_n(\pi)$ and still $||h-g_n||\ra 0$. This implies that $h$ can be approximated by trigonometric polynomials $q_m$ in $L^2[-\pi,\pi]$. This shows that $B$ is a Schauder basis. Now let $\epsilon >0$. By Theorem 4.8, there exists some finite subset $F$ of $\Z$ such that $||f-\sum_{j\in G}\hat{f}_jv_j||<\epsilon/\sqrt{2 \pi}$ for all finite sets $G$ with $F \subseteq G \subseteq \Z$. Since $F$ is finite, we can choose $N\in \N$ such that $F \subseteq \{n \in \Z;|n|\leq N\}$. Now let $k,m>N$. Let $G=\{n\in \Z;-k\leq n \leq m\}$. Then $F \subseteq G \subseteq \Z$ and so $\epsilon/\sqrt{2 \pi}>||f-\sum_{j\in G}\hat{f}_j v_j|| = ||f-\sum_{j=-k}^m\hat{f}_jv_j||$. We take squares and obtain (4.4) $\qed$ 
% 4.3.1 Complex Fourier series
%{\bf D 4.13}. Let $I$ be an interval. A function $f$ from $I$ into a normed vector space $Z$ is called Lipschitz cont if $\exists \Lambda >0$ s.t. $||f(y)-f(x)||\leq \Lambda|y-x|$ for all $x,y \in I$. 

{\bf P 4.14}. Let $f: \R \ra \C$ be Lipschitz cont and $2\pi$-periodic. Then $\sum_{j\in\Z}|\hat{f}_j|$ converges in $\R$. More precisely, if $\Lambda$ is a Lipschitz const for $f$, then $\sum_{j\in M}|\hat{f}_j| \leq 2 \Lambda + |\hat{f}_0|$ for all finite subsets $M$ of $\Z$. 
%{\it Proof}. To approximate $f$ by continuously differentiable functions, we define $f_n(x)=n\int_x^{x+(1/n)}f(y)dy=\int_0^1f(x+(y/n))dy$. Then $f_n$ inherits the $2\pi$-periodicity from $f$ and $|f_n(x)-f(x)|=|\int_0^1[f(x+(y/n))-f(x)]dy|\leq \int_0^1|f(x+(y/n))-f(x)|dy\leq |\int_0^1\Lambda (y/n)dy = \Lambda/(2n)$. By the fundamental theorem of calculus, $f_n$ is continuously differentiable, $f_n'(x)=n[f(x+(1/n))-f(x)]$ and so $|f_n'(x)| \leq \Lambda, x \in \R$. Let $g$ be continuously differentiable and $2\pi$-periodic and $j \neq 0$. Then $\hat{g}_j=(1/2\pi)\int_{-\pi}^{\pi}g(y)e^{-ijy}dy=(1/2\pi)\int_{-\pi}^{\pi}g(y)(-1/ij)(d/dy)e^{-ijy}dy$. We integrate by parts. Since $g$ and $e^{-ijy}$ are $2\pi$-periodic, the boundary terms cancel and $\hat{g}_j=(1/2\pi)(1/ij)\int_{-\pi}^{\pi}g'(y)e^{-ijy}dy=(1/ij)\hat{g'}_j$. For $m\in \N$, set $M_m=\{-m, \dots, -1,1,\dots, m\}$. Then $\sum_{j\in M_m}|\hat{g}_j|=\sum_{j\in M_m}(1/j)|\hat{g'}_j|$. By the Cauchy-Schwarz inequality in $\R^{2m}, (\sum_{j\in M_m}|\hat{g}_j|)^2\leq (\sum_{j\in M_m}1/j^2)(\sum_{j\in M_m}|\hat{g'}_j|^2)$. By Bessel's inequality, $(\sum_{j\in M_m}|\hat{g}_j|)^2\leq 2(\sum_{j=1}^{\infty}1/j^2)||g'||^2\leq 4||g'||^2$ and so $\sum_{j\in M_m}|\hat{g}_j|\leq 2||g'||$. Let $\epsilon>0$. By the considerations at the beginning of this proof, there exists some continuously differentiable $2\pi$-periodic function $g$ with $|f(x)-g(x)|\leq \epsilon/2m$ and $|g'(x)|\leq \Lambda$ for all $x \in [-\pi,\pi]$. So $\sum_{j\in M_m}|\hat{f}_j|\leq \sum_{j\in M_m}|\hat{g}_j|+\sum_{j\in M_m}|\widehat{f-g}_j|\leq 2 ||g'||+\sum_{j\in M_m}(1/2\pi)|\int_{-\pi}^{\pi}(f(y)-g(y))e^{-ijy}dy|\leq 2\Lambda+\epsilon$. Since this holds for every $\epsilon>0, \sum_{j\in M_m}|\hat{f}_j|\leq 2\Lambda$. Let $F$ be a finite subset of $\Z$. Then there exists some $m \in \N$ such that $F\subseteq M_m \cup \{0\}$. So $\sum_{j\in F}|\hat{f}_j|\leq \sum_{j\in M_m}|\hat{f}_j|+|\hat{f}_0|\leq 2\Lambda + |\hat{f}_0|<\infty \qed$ 
{\bf R 4.15}. This proof also shows: If $g: [-\pi,\pi] \ra \C$ is absolutely cont and $2\pi$-periodic and $g' \in L^2[-\pi,\pi]$, then $\sum_{j\in \Z}|\hat{g}_j|\leq 2||g'||+ |\hat{g}_0|$. Every Lipschitz cont funct $f$ is absolutely cont with $|f'(x)| \leq \Lambda$ for a.a. $x$. 
{\bf L 4.16}. Let $f: [-L,L]\ra \C$ be Lipschitz cont, $f(L) = f(-L)$. Extend f in an $2L$-periodic way, $f(y+2kL)=f(y): k \in \Z, -L<y\leq L$. Extension of $f$ is Lipschitz cont with same Lipschitz const. 
%{\it Proof}. Assume that $f$ is Lipschitz continuous on $[-L,L]$. So there exists some $\Lambda > 0$ such that $|f(y)-f(x)|\leq \Lambda|y-x|$ for all $x,y \in [-L,L]$. Now let $x,y \in \R$. Then $x = 2kL+\tilde{x}$ and $y = 2\ell L+\tilde{y}$ with $k, \ell \in \Z$ and $\tilde{x}, \tilde{y} \in (-L, L]$. Case 1: $k = \ell$. Then $|f(y)-f(x)|=|f(\tilde{y})-f(\tilde{x})|\leq \Lambda|\tilde{y}-\tilde{x}|=\Lambda|y-x|$. Case 2: $|k-\ell|\geq 2$. Without loss of generality, $k<\ell$ and so $\ell\geq k+2$. Then $y-x=2(\ell - k)L+\tilde{y}-\tilde{x} \geq 4L-2L=2L$. So $|f(y)-f(x)|=|f(\tilde{y})-f(\tilde{x})|\leq \Lambda|\tilde{y}-\tilde{x}|\leq \Lambda(2L)\leq\Lambda|y-x|$. Case 3: $|k-\ell|=1$. Without loss of generality, $\ell=k+1$. Then $y-x=2L+\tilde{y}-\tilde{x}>0$. Since $f(-L)=f(L), |f(y)-f(x)|=|f(\tilde{y})-f(\tilde{x})|=|f(\tilde{y})-f(-L)+f(L)-f(\tilde{x})|\leq \Lambda|\tilde{y}+L|+\Lambda|L-\tilde{x}|$. Since $L-\tilde{x} \geq 0$ and $L+\tilde{y}\geq 0, |f(y)-f(x)|\leq \Lambda((L+\tilde{y})+(L-\tilde{x}))\leq \Lambda(2L+\tilde{y}-\tilde{x})=\Lambda(y-x). \qed$
{\bf T 4.17}. Let $f: [-L,L] \ra \C$ be Lipschitz cont, $f(-L)=f(L)$. Then $f$ is the uniform limit of its Fourier series, $f(x)=\sum_{j\in \Z}\hat{f}_je^{i\lambda_jx}, \hat{f}_j=(1/2L)\int_{-L}^L f(y)e^{-i\lambda_jy}dy, \lambda_j=j\pi/L$. 
%{\it Proof}. We extend $f$ an an $2L$-periodic way.  The extended $f$ is Lipschitz continuous with same Lipschitz constant. We define $g(x)=f(xL/\pi), x \in \R$.  Then $g$ is Lipschitz continuous and $2\pi$-periodic. Let $I$ be the interval $[-\pi,\pi]$ and $X$ the vector space $B(I)$ of bounded complex-valued functions on $I$. We define a norm on $B(I)$ by $||g||_{\infty}=\sup_{x\in I}|g(x)|$. With this norm, $B(I)$ is a Banach space. For $j \in \Z$, define $g_j(x)=\hat{g}_je^{ijx}, x \in I$. Then $g_j\in B(I)$ and $||g_j||_{\infty} \leq |\hat{g}_j|$. By Propostion 4.14 and Lemma 4.16, $\sum_{j\in \Z}||g_j||_{\infty}$ converges in $\R$. By the Weierstra$\ss$ majorant test, the Fourier series of $g, \sum_{j\in \Z} g_j$ converges in $B(I)$, i.e. $\sum_{j\in \Z}\hat{g}_je^{ijx}$ converges in $\C$, unformly for $x \in I$. The Fourier series of $g$ converges to $g$ in $L^2[-\pi,\pi]$ by Theorem 4.10. This implies that $g$ and its Fourier series are equal almost everywhere.  Since both are continuous, they are equal.  In combination, $g$ is the uniform limit of its Fourier series, and so is $f$. The formula for the Fourier coefficients of $f$ follows by substitution. $\qed$
% Exercises
{\bf Exercise 4.3.1}. Let $B= \{\cos(jx);j\in \N\} \cup\{\sin(jx); j \in \N\} \cup \{1/\sqrt{2}\}$. Show that $B$ is an orthonormal basis of $L^2([-\pi,\pi],\R)$ with inner product $\langle f,g \rangle = (1/\pi) \int_{-\pi}^{\pi}fg$. Hint:  Use that $\{e^{ijx};j \in \Z\}$ is an orthonormal basis of $L^2([-\pi,\pi],\C)$ and express $\sin x$ and $\cos x$ in terms of $e^{ix}$ and $e^{-ix}$. {\it Proof}. Recall that $\cos(jx)=(1/2)(e^{ijx}+e^{-ijx}), \sin(jx)=(1/2i)(e^{ijx}-e^{-ijx})$. Since $e^{ijx}$ and $e^{ikx}$ are orthogonal to each other for $j \neq k$, so are $\cos jx$ and $\cos kx$, and $\cos jx$ and $\sin kx$, and $\sin jx$ and $\sin kx$ for $j \neq k$. $\sin jx$ and $\cos jx$ are orthogonal to each other because their product is odd about 0 and the integral yields 0. Further $(1/\pi)\int_{-\pi}^{\pi}\cos jx \cos jx dx = (1/\pi)\int_{-\pi}^{\pi}(1/4)(e^{ijx}+e^{-ijx})(e^{ijx}+e^{-ijx}) dx $. Notice that $\int_{-\pi}^{\pi}e^{ijx}e^{ijx}=2\pi\langle e^{ij\cdot},e^{-ij\cdot}\rangle_{\C}=0$ and $\int_{-\pi}^{\pi}e^{-ijx}e^{-ijx}=2\pi\langle e^{-ij\cdot},e^{ij\cdot}\rangle_{\C}=0$. Here $\langle \cdot, \cdot \rangle_{\C}$ denotes the inner product for $C([-\pi,\pi], \C)$. So $(1/\pi)\int_{-\pi}^{\pi}\cos(jx) \cos (jx) dx =1$. Similarly for the sines. In order to show that this is an orthonormal basis, we use Exercise 4.2.3. Let $f \in L^2([-\pi,\pi], \R)$ and $(1/\pi)\int_{-\pi}^{\pi}f(x)\sin(jx) dx =0, (1/\pi)\int_{-\pi}^{\pi}f(x)\cos(jx) dx =0, j\in \N, (1/\pi)\int_{-\pi}^{\pi}f(x)(1/\sqrt{2})=0$. By Euler's formula, for all $j \in \Z, (1/2\pi)\int_{-\pi}^{\pi}f(x)e^{ijx} dx =(1/2\pi)\int_{-\pi}^{\pi}f(x)\cos(jx) dx +i(1/2\pi)\int_{-\pi}^{\pi}f(x)\sin(jx) dx =0$. Since $\{e^{ijx}; j \in \Z\}$ is an orthonormal basis, $f=0$ by Exercise 4.2.3(a). Exercise 4.2.3 (b) implies that $B$ is an orthonormal basis for $L^2([-\pi,\pi],\R) \qed$
{\bf Exercise 4.3.2}. Let $f: [a,b] \ra \R$ be continuous and assume that there exists a partition $a = t_0<\cdots <t_m=b$ such that $f$ is differentiable with bounded derivative on each interval $(t_{j-1}, t_j)$. Show $f$ is Lipschitz continuous. {\it Proof}. Let $x,y \in [z,b], x < y$. Modifying the partition of $[a,b]$, we can find a partition $x=r_0<\cdots <r_k=y$ such that $f$ is continuously differentiable on each $(r_{j-1},r_j)$ and $L_j:=\sup_{r_{j-1}<s<r_j}|f'(s)|<\infty$. More precisely $r_1, \dots, r_{k-1} \in \{t_1,\dots,t_{m-1}\}$. Let $j \in \{0,\dots, k\}$. and $r_{j-1}\leq s<t\leq r_j$. By the mean value theorem of calculus, $f(t)-f(s)=f'(r)(t-s)$ for some $r \in (s,t)$. So $|f(t)-f(s)| \leq L_j|t-s|$. Since $f$ is continuous, we can take the limit $s \ra r_{j-1}$ and $t \ra r_j$ and $|f(r_j)-f(r_{j-1})|\leq L_j(r_j-r_{j-1})$. We telescope, $|f(y)-f(x)|=|\sum_{j=1}^k[f(r_j)-f(r_{j-1})]| \leq \sum_{j=1}^k L_j(r_j-r_{j-1})$. Set $\Lambda = \max_{j=1}^k L_j$. Then $|f(y)-f(x)| \leq \Lambda \sum_{j=1}^k (r_j-r_{j-1})=\Lambda (y-x)$. Here we have telescoped again. 
{\bf Exercise 4.3.3}. Let $f: [-L,L] \ra \R$ be Lipschitz continuous, $f(L)=f(-L)$, and $A_j$ and $B_j$ be the Fourier cosine and sine coeffiecints respectively. Show: $\sum_{j=0}^{\infty}|A_j|<\infty, \sum_{j=1}^{\infty}|B_j|<\infty$. Hint: Use the analogous result for complex Fourier coefficients. {\it Proof}. After a change of variables, we can assume that $L=\pi$. Notice that, for $j\in \N, \hat{f}_j=1/(2\pi)\int_{-\pi}^{\pi}f(x)e^{-ijx}dx= 1/(2\pi)\int_{-\pi}^{\pi}f(x)[\cos(jx)-i\sin(jx)]dx=(1/2)(A_j-iB_j)$. Since $f$ has real values, $A_j$ and $B_j$ are real numbers and $|\hat{f}_j|=(1/2)\sqrt{A_j^2+B_j^2}\geq (1/2)\max\{|A_j|,|B_j|\}$. Since $f$ is Lipschitz continuous and $f(-\pi)=f(\pi)$, by Theorem 4.14, $\sum_{j=1}^{\infty}|A_j| \leq 2 \sum_{j=1}^{\infty}|\hat{f}_j|\leq 2\sum_{j=-\infty}^{\infty}|\hat{f}_j|<\infty$. Similarly for $|B_j|. \qed$
% 4.3.2 Real Fourier series
%{\bf 4.3.2 Real FS} Let $f: [-L,L]\ra \R$ be Lipschitz cont, $f(L)=f(-L)$. Extend $f$ in an $2L$-periodic way, $f(y+2kL)=f(y); k\in \Z, -L<y \leq L$. By L 4.16, the extended $f$ is Lipschitz cont (with same const). Define $g: \R \ra \R$ by $g(x)=f(xL/\pi), x \in \R$. Then $g$ is Lipschitz cont and $2\pi$-periodic.  By T 4.17, $g$ is the uniform limit of Fourier sums $\sum_{j=-m}^m \hat{g}_je^{ijx}=\hat{g}_0+\sum_{j=1}^m (\hat{g}_je^{ijx}+\hat{g}_{-j}e^{-ijx})$. For $j \geq 1, \hat{g}_je^{ijx}+\hat{g}_{-j}e^{-ijx}=1/(2\pi)\int_{-\pi}^{\pi}g(y)e^{-ijy}dy(\cos jx+i \sin jx)+ 1/(2\pi) \int_{-\pi}^{\pi} g(y)e^{ijy}dy(\cos jx-i \sin jx)$. Set $A_j=(1/\pi)\int_{-\pi}^{\pi}g(y)\cos jy dy, B_j = (1/\pi) \int_{-\pi}^{\pi} g(y)\sin jy dy$. Then $ \hat{g}_je^{ijx}+\hat{g}_{-j}e^{-ijx}=(1/2)[(A_j-iB_j)(\cos jx + i \sin jx)+(A_j+iB_j)(\cos jx - i\sin jx)]=A_j\cos jx + B_j \sin jx$. For $j=0, \hat{g}_0=(1/2\pi)\int_{-\pi}^{\pi}g(y)dy=:A_0$. So $g(x)= \lim_{m\ra \infty}(A_0+\sum_{j=1}^m(A_j\cos jx + B_j \sin jx))$ with the convergence being uniform in $x \in [-\pi,\pi]$. Since $f(x)=g(x\pi/L)$, ...
{\bf T 4.18}. Let $f: [-L,L]\ra \R$ be Lipschitz cont and $f(L)=f(-L)$. Then $f(x) = A_0 + \sum_{j=1}^{\infty}(A_j\cos(\lambda_jx) + B_j \sin(\lambda_jx)), \lambda_j=(j\pi/L)$, with the convergence being uniform in $x \in [-L,L]$ and $A_j=(1/L)\int_{-L}^L f(y)\cos(\lambda_jy)dy, j \in \N, B_j=(1/L)\int_{-L}^L f(y)\sin(\lambda_jy)dy, j \in \N, A_0=1/(2L)\int_{-L}^L f(y)dy$. The formula of $A_j$, the Fourier cosine coef of $f$, follows by a change of variable from $A_j=(1/\pi)\int_{-\pi}^{\pi} f(yL/\pi)\cos(jy)dy$. Similarly for $B_j$, the Fourier sine coef of $f$. 
{\bf L 4.19}. Let $f: [0,L]\ra \R$ be Lipschitz continuous, $f(0)=0=f(L)$. Then $f(x)=\sum_{j=1}^{\infty}B_j\sin(\lambda_jx), B_j=(2/L)\int_0^Lf(y)\sin(\lambda_jy)dy, \lambda_j = (j \pi/L)$, with the convergence being uniform in $[0,L]$. {\it Proof}. Extend $f$ to $[-L,L]$ in an odd fashion by defining $f(-x)=-f(x)$ for $x \in (0,L]$. We first check whether this is also Lipschitz cont. Critical case is $-L \leq x <0 \leq y \leq L$. Since $f(0)=0, |f(y)-f(x)| \leq |f(y)| + |f(x)|=|f(y)-f(0)|+|f(0)-f(-x)|\leq \Lambda(y+(-x)) \leq \Lambda (y-x)$. Since $f(L)=0, f(-L)=-f(L)=0$ and $f(-L)=f(L)$. By T 4.18, $f(x)=A_0+\sum_{j=1}^{\infty}(A_j\cos(\lambda_jx) + B_j \sin(\lambda_jx)), \lambda_j=(j\pi/L)$, with the convergence being uniform in $x \in [-L,L]$ and $A_j=(1/L)\int_{-L}^L f(y)\cos(\lambda_jy)dy, B_j=(1/L)\int_{-L}^L f(y)\sin(\lambda_jy)dy$. Since cosine is even and sine is odd, for $j \in \N, A_j=(1/L)\int_0^L( f(y)+f(-y))\cos(\lambda_jy)dy=0$ and $B_j=(1/L)\int_0^L (f(y)-f(-y))\sin(\lambda_jy)dy=(2/L)\int_0^L f(y)\sin(\lambda_jy)dy, A_0=(1/2L)\int_{-L}^L f(y)dy=0 \qed$
{\bf R 4.20}. Actually, $\sum_{j=1}^{\infty}|B_j| < \infty$ under the conds of L 4.19.  See Ex 4.3.3. 
% Exercises
%{\bf Exercise 4.3.3}. Let $f: [-L,L] \ra \R$ be Lipschitz continuous, $f(L)=f(-L)$, and $A_j$ and $B_j$ be the Fourier cosine and sine coeffiecints respectively. Show: $\sum_{j=0}^{\infty}|A_j|<\infty, \sum_{j=1}^{\infty}|B_j|<\infty$. Hint: Use the analogous result for complex Fourier coefficients. {\it Proof}. After a change of variables, we can assume that $L=\pi$. Notice that, for $j\in \N, \hat{f}_j=1/(2\pi)\int_{-\pi}^{\pi}f(x)e^{-ijx}dx= 1/(2\pi)\int_{-\pi}^{\pi}f(x)[\cos(jx)-i\sin(jx)]dx=(1/2)(A_j-iB_j)$. Since $f$ has real values, $A_j$ and $B_j$ are real numbers and $|\hat{f}_j|=(1/2)\sqrt{A_j^2+B_j^2}\geq (1/2)\max\{|A_j|,|B_j|\}$. Since $f$ is Lipschitz continuous and $f(-\pi)=f(\pi)$, by Theorem 4.14, $\sum_{j=1}^{\infty}|A_j| \leq 2 \sum_{j=1}^{\infty}|\hat{f}_j|\leq 2\sum_{j=-\infty}^{\infty}|\hat{f}_j|<\infty$. Similarly for $|B_j|. \qed$
{\bf Exercise 4.3.4}. Let $f: [0,L] \ra \R$ be Lipschitz continuous. Show that $f(x)=\sum_{j=0}^{\infty}A_j\cos(jx\pi/L)$ with the convergence being uniform in $[0,L], \sum_{j=0}^{\infty}|A_j|<\infty$, and $A_j=(2/L)\int_0^L f(y)\cos(jy\pi/L)dy, j \in \N$, and $A_0=(1/L)\int_0^L f(y)dy$. {\it Proof}. We extend $f$ to $[-L,L]$ by defining $f(-x)=f(x)$ for $x \in (0,L]$. We first need to check whether this extension is also Lipschitz continuous. The critical case is $-L \leq x <0 \leq y \leq L$. By the triangle inequality, $|f(y)-f(x)| \leq |f(y)-f(0)|+|f(0)-f(-x)|\leq \Lambda(y+(-x)) \leq \Lambda (y-x)$. By construction, $f(-L)=f(L)$. By Theorem 4.18, $f(x)=A_0+\sum_{j=1}^{\infty}(A_j\cos(\lambda_jx) + B_j \sin(\lambda_jx)), \lambda_j=(j\pi/L)$, with the convergence being uniform in $x \in [-L,L]$ and $A_j=(1/L)\int_{-L}^L f(y)\cos(\lambda_jy)dy, B_j=(1/L)\int_{-L}^L f(y)\sin(\lambda_jy)dy$. By the previous exercise, $\sum_{j=0}^{\infty}|A_j|<\infty$. Since cosine is even and sine is odd, for $j \in \N, A_j=(1/L)\int_0^L( f(y)+f(-y))\cos(\lambda_jy)dy=(2/L)\int_0^L f(y)\cos(\lambda_jy)dy$ and  $B_j=(1/L)\int_0^L (f(y)-f(-y))\sin(\lambda_jy)dy= 0, A_0=1/(2L)\int_{-L}^L f(y)dy=1/(2L)\int_0^L (f(y)+f(-y))dy=(1/L)\int_0^L f(y)dy \qed$
%{\bf Exercies 4.3.5}. Show that $\{v_j;j\in\N\}$ with $v_j(x)=\sin(jx)$ is an orthonormal basis of $L^2([0,\pi],\R)$ with the inner product $\langle f,g \rangle =(2/\pi)\int_0^{\pi}f(x)g(x)dx$. Conclude that, for $f \in C([0,\pi],\R), \int_0^{\pi}|f(x)-\sum_{j=1}^mB_j\sin(jx)dx|^2dx \ra 0, m\ra \infty, B_j=(2/\pi)\int_0^{\pi}f(x)\sin(jx)dx$. {\it Proof}. By Exercise 4.3.1, $B=\{\cos(jx);j\in\N\}\cup \{\sin(jx);j\in\N\}\cup \{1/\sqrt{2}\}$ is an orthonormal basis of $L^2([-\pi,\pi],\R)$ with inner product $\langle f,g \rangle = (1/\pi)\int_{-\pi}^{\pi}fg$. In particular $\tilde{B} = \{\sin(jx);j\in\N\}$ is an orthonormal subset of $L^2([-\pi,\pi],\R)$. So, for $j \neq k, 0 = \int_{-\pi}^{\pi}\sin(jx)\sin(kx)dx=2\int_0^{\pi}\sin(jx)\sin(kx)dx$, because $\sin(jx)\sin(kx)$ is an even function of $x$. By the same token, $1 = (1/\pi)\int_{-\pi}^{\pi}\sin^2(jx)dx= (2/\pi)\int_0^{\pi}\sin^2(jx)dx$. So $\tilde{B}$ is an orthonormal subset of $L^2([0,\pi], \R)$. To show that it is an orthonormal basis, we use Exercise 4.2.3:  We let $f \in L^2([0,\pi], \R)$ with $\int_0^{\pi}f(x)\sin(jx)dx=0$ for all $j \in \N$ and show that $f=0$. Extend $f$ to an odd function on $[-\pi,\pi]$ by setting $f(-x)=-f(x)$ for $x \in (0,\pi]$. Then, for all $j \in \Z, \int_{-\pi}^{\pi}f(x)\cos(jx)dx=0$, because $f(x)\cos(jx)$ is an odd function of $x$. For all $j \in \N, \int_{-\pi}^{\pi}f(x)\sin(jx)dx=2\int_0^{\pi}f(x)\sin(jx)dx=0$, because $f(x)\sin(jx)$ is an even function of $x$. Since $B$ is an orthonormal basis of $L^2([-\pi,\pi],\R), f=0$ by Exercise 4.2.3 (a). So $\tilde{B}$ is an orthonormal basis by Exercise 4.2.3 (b).  Alternatively, one can use the density of $C[0,\pi]$ in $L^2[0,\pi]$ with Lemma 4.19 to show that the sine functions form a Schauder basis in $L^2[0,\pi] \qed$
{\bf Exercise 4.3.6}. Show that $\{v_j;j\in\Z_+\}$ with $v_j(x) = \cos(jx)$ and $v_0=\sqrt{1/2}$ is an orthonormal basis of $L^2([0,\pi],\R)$ with the inner product $\langle f,g \rangle = \int_0^{\pi}f(x)g(x)dx$. Conclude that, for $f \in L^2([0,\pi],\R), \int_0^{\pi}|f(x)-\sum_{j=0}^m B_j\cos(jx)dx|^2dx \ra 0, m \ra \infty, B_j=(2/\pi)\int_0^{\pi}f(x)\cos(jx)dx, j \in \N, B_0=(\sqrt{2}/\pi)\int_0^{\pi}f(x)dx$. {\it Proof}. By Exercise 4.3.1, $B=\{\cos(jx);j\in\N\}\cup \{\sin(jx);j\in\N\}\cup \{1/\sqrt{2}\}$ is an orthonormal basis of $L^2([-\pi,\pi],\R)$, with inner product $\langle f,g \rangle = (1/\pi)\int_{-\pi}^{\pi}fg$. In particular $\tilde{B} = \{v_j;j\in\N\}$ is an orthonormal subset of $L^2([-\pi,\pi],\R)$. So, for $j \neq k, 0 = \int_{-\pi}^{\pi}v_j(x) v_k(x) dx=2\int_0^{\pi}v_j(x) v_k(x)dx$, because $v_j v_k$ is an even function. By the same token, for $j\in \N, 1 = (1/\pi)\int_{-\pi}^{\pi}v_j(x)^2 dx= (2/\pi)\int_0^{\pi}v_j(x)^2 dx$. For $j = 0$ this property easily is directly verified. So $\tilde{B}$ is an orthonormal subset of $L^2([0,\pi], \R)$. To show that it is an orthonormal basis, we use Exercise 4.2.3:  We let $f=L^2([0,\pi], \R)$ with $\int_0^{\pi}f(x)v_j(x)dx=0$ for all $j \in \Z_+$ and show that $f=0$. Extend $f$ to an even function on $[-\pi,\pi]$ by setting $f(-x)=f(x)$ for $x \in (0,\pi]$. Then, for all $j \in \Z, \int_{-\pi}^{\pi}f(x)\sin(jx)dx=0$, because $f(x)\sin(jx)$ is an odd function of $x$. For all $j \in \Z_+, \int_{-\pi}^{\pi}f(x)v_j(x)dx=2\int_0^{\pi}f(x)v_j(x)dx=0$, because $f(x)v_j(x)$ is an even function of $x$. Since $B$ is an orthonormal basis of $L^2([-\pi,\pi],\R), f=0$ by Exercise 4.2.3 (a). So $\tilde{B}$ is an orthonormal basis by Exercise 4.2.3 (b).  $\qed$
% Generalized (weak) solutions to the vibrating string equation
%{\bf vibrating string} (VSE) (PDE)$(\partial_t^2-c^2\partial_x^2)u=0, 0 \leq x \leq L, t \in \R$, (IC) $u(x,0)=f(x), \partial_tu(x,0)=g(x), 0 \leq x \leq L$, (BC) $u(0,t)=0=u(L,t), t \in \R$. (4.5) Assume that $f$ and $g$ are cont and $f(0)=0=f(L)$ and $g(0)=0=g(L)$. Extend $f$ and $g$ to $[-L,L]$ in an odd way and then to $\R$ in an $2L$-periodic way. Then $f$ and $g$ are cont on $\R$ and odd about 0 and $L$ (L 3.13). The d'Alembert form $u(x,t)=(1/2)(f(x+ct)+f(x-ct))+ 1/(2c) \int_{x-ct}^{x+ct}g(y)dy$. (4.6) The same proofs as for T 3.14 show  $u(0,t) = 0 = u(L,t)$ for all $t\in \R$; and $u(x,0)=f(x)$ for all $x \in [0,L]$. 
%{\bf D 4.21}. A funct $u: [0,L] \times \R \ra \R$ is a classical sol of (4.5) if $u$ is twice cont diff on $[0,L]\times \R$ and satisfies (4.5). Let $\phi: [0,L] \times \R$ be twice cont diff, $\phi(0)=0=\phi(L)$. Let $u$ be a classical sol of (3.38). Since $u$ is twice cont diff, the following derivative exists and satisfies $(d^2/dt^2)\int_0^L \phi(x) u(x,t) dx = \int_0^L \phi(x) \partial_t^2u(x,t)dx=\int_0^L\phi(x)c^2\partial_x^2u(x,t)dx$. Int by parts. Since both $\phi$ and $u$ are 0 for $x = 0,L, (d^2/dt^2)\int_0^L \phi(x)u(x,t)dx=-c^2 \int_0^L \phi'(x) \partial_x u(x,t)dx =c^2\int_0^L\phi''(x)u(x,t)dx$. Similarly  $(d/dt_{[t=0]}) \int_0^L \phi(x) u(x,t)dx = \int_0^L\phi(x)g(x)dx$.  
{\bf D 4.22}. A twice cont diff func $\phi: [0,L] \ra \R$ with $\phi(0)=0=\phi(L)$ is a test func for the VSE. A func $u: [0,L]\times \R \ra \R$ is called a gen sol of (4.5) if $u(t, \cdot) \in L^2[0,L]$ for all $t \in \R, u(x,0)=f(x)$ for a.a. $x \in [0,L]$ and, for every test function $\phi, \int_0^L \phi(x)u(x,t)dx$ is a twice cont diff func of $t \in \R$ and $(d^2/dt^2)\int_0^L \phi(x) u(x,t) dx= c^2\int_0^L \phi''(x)u(x,t)dx$ and $(d/dt_{[t=0]})\int_0^L\phi(x)u(x,t)dx=\int_0^L\phi(x)g(x)dx$. 
{\bf T 4.23}. Any classical sol $u$ of (4.5) is a gen sol of (4.5). A gen sol is uniquely determined by finding its Fourier sine series, $u(x,t)=\sum_{j=1}^{\infty}B_j(t)\sin(\lambda_jx), \lambda_j =j\pi/L, B_j(t)= (2/L)\int_0^L u(x,t) \sin(\lambda_jx)dx, j \in \N$. (4.7) For each $t \in \R$ the convergence of $\sum_{j=1}^{\infty}B_j(t)\sin(\lambda_j \cdot)$ holds in $L^2[0,L]$. Let $j \in \N$ and choose $\phi(x)=\sin(\lambda_jx). \; \phi$ is a test func and so the following derivatives exist and satisfy $(d^2/dt^2)\int_0^L u(x,t)\sin(\lambda_jx)dx=c^2\int_0^L u(x,t)(d^2/dx^2) \sin(\lambda_jx)dx=-c^2\lambda_j^2\int_0^L u(x,t) \sin(\lambda_jx)dx$ and $(d/dt_{[t=0]})\int_0^Lu(x,t) \sin(\lambda_jx)dx=\int_0^Lg(x)\sin(\lambda_jx)dx, \int_0^Lu(x,0) \sin(\lambda_jx)dx=\int_0^Lf(x)\sin(\lambda_jx)dx$. $\implies$ Fourier sine coef of a gen sol, $B_j(t)$, satisfy the ODEs $B_j''+c^2 \lambda_j^2B_j=0$ and the ICs $B_j(0)=(2/L)\int_0^L f(y)\sin(\lambda_jy)dy, B_j'(0)=(2/L)\int_0^L g(y)\sin(\lambda_jy)dy$. Gen sol, $a_j\cos(c \lambda_jt)+b_j\sin(c\lambda_jt)=B_j(t)$. From IC obtain  $a_j = B_j(0) = (2/L) \int_0^L  f(y)\sin(\lambda_jy)dy$ (4.8) and $c\lambda_jb_j=B_j'(0)=(2/L)\int_0^L g(y) \sin(\lambda_jy)dy$. (4.9) Subst and get Fourier sine series for any gen sol, $u(x,t)=\sum_{j=1}^{\infty}[a_j\cos(c\lambda_jt)+b_j \sin(c \lambda_jt)] \sin(\lambda_jx)$. (4.10) The d'Alembert form provides a gen sol if $f$ and $g$ are cont on $[0,L]$ and $f(0)=0=f(L)$ and $g(0)=0=g(L)$ and $f$ and $g$ are extended in an odd and $2L$-periodic way.  Set $v(x,t)=(1/2)(f(x+ct)+f(x-ct))$. (4.11) After a change of variables, $\int_0^L\phi(x)v(x,t)dx=(1/2)\int_{ct}^{L+ct}\phi(y-ct)f(y)dy+(1/2)\int_{-ct}^{L-ct}\phi(y+ct)f(y)dy$. Since $\phi$ is twice cont diff and $f$ is cont, we can use the Leibnitz rule, $\implies$ expression is diff and $(d/dt)\int_0^L\phi(x)v(x,t)dx=(c/2)[\phi(L)f(L+ct)-\phi(0)f(ct)]-(c/2)\int_{ct}^{L+ct}\phi'(y-ct)f(y)dy-(c/2)[\phi(L)f(L-ct)-\phi(0)f(-ct)]+(c/2)\int_{-ct}^{L-ct}\phi'(y+ct)f(y)dy$.  Since $\phi(0)=0=\phi(L), (d/dt)\int_0^L\phi(x)v(x,t)dx=-(c/2)\int_{ct}^{L+ct}\phi'(y-ct)f(y)dy+(c/2)\int_{-ct}^{L-ct}\phi'(y+ct)f(y)dy$ (4.12). This expression is 0 at $t = 0$. Use Leibnitz rule again, $(d^2/dt^2)\int_0^L\phi(x)v(x,t)dx=-(c^2/2)[\phi'(L) f(L+ct)-\phi'(0)f(ct)] + (c^2/2) \int_{ct}^{L+ct}\phi''(y-ct)f(y)dy-(c^2/2)[\phi'(L)f(L-ct)-\phi'(0)f(-ct)]+(c^2/2)\int_{-ct}^{L-ct} \phi''(y+ct)f(y)dy$.  Since $f$ is odd about 0 and $L$, the boundary terms cancel each other and, after reversing the subst $(d^2/dt^2)\int_0^L\phi(x)v(x,t)dx=\int_0^L c^2\phi''(x)v(x,t)dx$. (4.13)  Set $w(x,t) = 1/(2c) \int_{x-ct}^{x+ct} g(y)dy$. (4.14) Since $g$ is cont, $w$ is diff wrt $t$ and $x$ and $\partial_t w(x,t)=(1/2)[g(x+ct)+g(x-ct)], \partial_x w(x,t)=(1/2c)[g(x+ct)-g(x-ct)]$. (4.15)  At $t =0$, the first expression is $g(x)$. Since $\partial_t w$ is cont, we can diff under the int and obtain $(d/dt)\int_0^L\phi(x)w(x,t)dx=\int_0^L\phi(x)(1/2)[g(x+ct)+g(x-ct)]dx$. The same consideration as before with $g$ replacing $f$ (see (4.12)) yields $(d^2/dt^2)\int_0^L\phi(x)w(x,t)dx=(-c/2) \int_{ct}^{L+ct}\phi'(y-ct)g(y)dy+(c/2)\int_{-ct}^{L-ct}\phi'(y+ct)g(y)dy$. After refersing the subst, $(d^2/dt^2) \int_0^L \phi(x) w(x,t)dx = - \int_0^L \phi'(x) (c/2)[g(x+ct)-g(x-ct)]dx$.  Observe from (4.15) that $(d^2/dt^2) \int_0^L \phi(x) w(x,t)dx = - \int_0^L \phi'(x) c^2 \partial_x w(x,t)dx$. We int by parts, recall $w(0,t)=0=w(L,t)$, and obtain $(d^2/dt^2) \int_0^L\phi(x)w(x,t)dx=\int_0^L\phi''(x)c^2w(x,t)dx$. Since $u(x,t)=v(x,t)+w(x,t)$, so $\int_0^L\phi(x)u(x,t)dx$ is twice diff and $(d^2/dt^2) \int_0^L \phi(x) u(x,t)dx = \int_0^Lc^2\phi''(x)u(x,t)dx$ and $(d/dt)\int_0^L\phi(x)u(x,t)dx=\int_0^L\phi(x)g(x)dx, t=0$. 
%{\bf Exercise 4.4.1}. Consider the wave equation (PDE) $(\partial_t^2-\partial_x^2)u=0, 0 \leq x \leq \pi, t \in \R$, (IC) $u(x,0)=f(x), \partial_t u(x,0)=g(x), 0\leq x\leq \pi$, (BC) $\partial_xu(0,t)=0=\partial_xu(\pi,t), t \in \R$. (4.17) with $f$ and $g$ in $C[0,\pi]$. (a) Develop the appropriate notions of classical and generalized solutions.  Show that every classical solution is a generalized solution. (b) Show that a generalized solution is uniquely determined. (c) Check whether the d'Alembert formula provides a generalized solution. {\it Proof}. (a) We assume that the initial data $f$ and $g$ of (4.17) are merely elements in $C[0,\pi]$. {\bf 4.24 Definition}. A function $u: [0,L] \times \R \ra \R$ is called a classical solution of (4.17) if $u$ is twice continuously differentiable on $[0,L] \times \R$ and satisfies (4.17). Let $\phi: [0,L] \ra \R$ be twice continuously differentiable, $\phi'(0)=0=\phi'(L)$. Further let $u$ be a classical solution of (4.17). Since $u$ is twice continuously differentiable, the following derivative exists and satisfies the equation $(d^2/dt^2)\int_0^L \phi(x)u(x,t)dx=\int_0^L\phi(x)\partial_t^2u(x,t)dx=\int_0^L\phi(x)c^2\partial_x^2u(x,t)dx$. We integrate by parts. Since both $\phi'$ and $\partial_xu$ are 0 for $x=0,L, (d^2/dt^2)\int_0^L \phi(x)u(x,t)dx=-c^2\int_0^L\phi'(x)\partial_xu(x,t)dx=c^2\int_0^L\phi''(x)u(x,t)dx$. Similarly we find $(d/dt_{[t=0]})\int_0^L\phi(x)u(x,t)dx=\int_0^L\phi(x)g(x)dx$. These findings motivate the following definition and prove that following result.
%{\bf D 4.24}. A function $u:[0,L]\times\R\rightarrow\R$ is called a classical solution of $(4.17)$ if $u$ is continuously differentiable on $[0,L]\times\R$ and satisfies $(4.17)$.
%{\bf D 4.25}. A twice continuously differentiable function $\phi: [0,L] \ra \R$ with $\phi'(0)=0=\phi'(L)$ is called a test function for (4.17). A function $u:[0,L] \times \R \ra \R$ is called a generalized solution of (4.17) if $u(t,\cdot) \in L^2[0,L]$ for all $t \geq 0, u(x,0)=f(x)$ for a.a. $x \in [0,L]$ and, for every test function $\phi, \int_0^L \phi(x)u(x,t)dx$ is a twice continuously differentiable function of $t \geq 0$ and $(d^2/dt^2) \int_0^L \phi(x) u(x,t)dx = c^2 \int_0^L \phi''(x) u(x,t)dx$ and $ (d/dt_{[t=0]})\int_0^L\phi(x)u(x,t)dx=\int_0^L\phi(x)g(x)dx$. 
%{\bf 4.26 Theorem}. Any classical solution $u$ of (4.17) is a generalized solution of (4.17).  The point of a generalized solution is that a function can solve a differential equation without being differentiable. (b) The uniqueness of a generalized solution is shown by calculating its Fourier cosine series. Notice that $v_j(x)=\cos(jx, j \in \Z_+$, are test functions. This works analogously to the zero boundary problem. Because of the different boundary conditions, we use the Fourier cosine representation $u(x,t)= \sum_{j=0}^{\infty}A_j(t)\cos(\lambda_jx), \lambda_j=j\pi/L, A_j(t)=(2/L)\int_0^Lu(x,t)\cos(\lambda_jx)dx, j \in \N, A_0(t)=(1/L)\int_0^Lu(x,t)dx$. (4.18) Let $j \in \Z_+$ and choose $\phi(x)=\cos(\lambda_jx). \; \phi$ is a test function and so the following derivatives exist and satisfy $(d^2/dx^2)\int_0^L u(x,t)\cos(\lambda_jx)dx=c^2\int_0^Lu(x,t)(d^2/dx^2)\cos(\lambda_jx)dx=-c^2\lambda_j^2\int_0^Lu(x,t)\cos(\lambda_jx)dx$ and $(d/dt_{[t=0]})\int_0^L u(x,t)\cos(\lambda_jx)dx=\int_0^Lg(x)\cos(\lambda_jx)dx, \int_0^L u(x,0)\cos(\lambda_jx)dx=\int_0^L f(x)\cos(\lambda_jx)dx$.  This implies that the Fourier cosine coefficients of a generalized solution, $A_j(t)$, for $j \in \N$, satisfy the ODEs $A_j''+c^2\lambda_j^2A_j=0$ and the initial contitions $A_j(0)=(2/L)\int_0^Lf(y)\cos(\lambda_jy)dy, A_j'(0)=(2/L)\int_0^L g(y)\cos(\lambda_jy)dy$. The ODEs have the general solutions, $a_j\cos(c\lambda_jt)+b_j\sin(c\lambda_jt)=A_j(t)$. From the initial conditions we obtain that $a_j=A_j(0)=(2/L)\int_0^L f(y) \cos(\lambda_jy)dy$ (4.19) and $c\lambda_jb_j=A_j'(0)=(2/L)\int_0^L g(y) \cos(\lambda_jy)dy$. (4.20) For $j=0, A_j''=0$, and the initial conditions $A_j(0)=(1/L)\int_0^L f(y) \cos(\lambda_jy)dy=a_0, A_j'(0)=(1/L)\int_0^L g(y) \cos(\lambda_jy)dy = b_0$. We integrate twice, $A_j(t)=tb_0+a_0$. We substitute and obtain the Fourier cosine series for any generalized solution, $u(x,t)=\sum_{j=1}^{\infty}[a_j\cos(c\lambda_jt)+b_j\sin(c\lambda_jt)]\cos(\lambda_jx)+b_0t+a_0$. (4.21) (c) Finally we show that the d'Alembert formula provides a generalized solution if $f$ and $g$ are continuous on $[0,L]$ and $f$ and $g$ are extended in an even and $2L$-periodic way.  The extended $f$ and $g$ are continuous. Set $v(x,t)=(1/2)(f(x+ct)+f(x-ct))$. (4.22) After a change of variables, $\int_0^L \phi(x)v(x,t)dx=(1/2)\int_{ct}^{L+ct}\phi(y-ct)f(y)dy+(1/2)\int_{-ct}^{L-ct}\phi(y+ct)f(y)dy$. Since $\phi$ is twice continuously differentiable and $f$ is continuous, we can use the Leibnitz rule that implies that this expression is differentiable and $(d/dt)\int_0^L \phi(x)v(x,t)dx=(c/2)[\phi(L)f(L+ct)-\phi(0)f(ct)]-(c/2)\int_{ct}^{L+ct}\phi'(y-ct)f(y)dy-(c/2)[\phi(L)f(L-ct)-\phi(0)f(-ct)]+(c/2)\int_{-ct}^{L-ct}\phi'(y+ct)f(y)dy$. Since $f$ is even about 0 and $L$, the boundary terms cancel each other and $(d/dt)\int_0^L \phi(x)v(x,t)dx=-(c/2)\int_{ct}^{L+ct}\phi'(y-ct)f(y)dy+(c/2)\int_{-ct}^{L-ct}\phi'(y+ct)f(y)dy$. (4.23) Notice that this expression is 0 at $t=0$. We can use Leibnitz rule another time, $(d^2/dt^2)\int_0^L \phi(x)v(x,t)dx=-(c^2/2)[\phi'(L)f(L+ct)-\phi'(0)f(ct)]+(c^2/2)\int_{ct}^{L+ct}\phi''(y-ct)f(y)dy-(c^2/2)[\phi'(L)f(L-ct)-\phi'(0)f(-ct)]+(c^2/2)\int_{-ct}^{L-ct}\phi''(y+ct)f(y)dy$. Since $\phi'(0)=0=\phi'(L)$, after reversing the substitution, $(d^2/dt^2)\int_0^L \phi(x)v(x,t)dx=\int_0^L c^2\phi''(x)v(x,t)dx$. (4.24) Set $w(x,t)=1/(2c)\int_{x-ct}^{x+ct}g(y)dy$ (4.25) After a substitution $w(x,t)=1/(2c)\int_{-ct}^{ct}g(x+y)dy$. (4.26) Since $g$ is continuous, $w$ is differentiable with respect to $t$ and $\partial_t w(x,t)=(1/2)[g(x+ct)+g(x-ct)]$. (4.27) At $t=0$, this expression is $g(x)$. Since $\partial_tw$ is continuous, we can differentiate under the integral and obtain $(d/dt)\int_0^L\phi(x)w(x,t)dt=\int_0^L\phi(x)(1/2)[g(x+ct)+g(x-ct)]$. The same consideration as before with $g$ replacing $f$ (see 4.12)) yields $(d^2/dt^2)\int_0^L \phi(x)w(x,t)dx=-(c/2)\int_{ct}^{L+ct}\phi'(y-ct)g(y)dy+(c/2)\int_{-ct}^{L-ct}\phi'(y+ct)g(y)dy$. After reversing the substitution, $(d^2/dt^2)\int_0^L \phi(x)w(x,t)dx=-\int_0^L \phi'(x)(c/2)[g(x+ct)-g(x-ct)]dx$. We observe from (4.14) that $(d^2/dt^2)\int_0^L \phi(x)w(x,t)dx=-\int_0^L\phi'(x)c^2\partial_xw(x,t)dx$.  We integrate by parts, recall $\phi'(0)=0=\phi'(L)$, and obtain $(d^2/dt^2)\int_0^L \phi(x)w(x,t)dx=\int_0^L\phi''(x)c^2w(x,t)dx$.  Since $u(x,t)=v(x,t)+w(x,t)$, we have shown that $\int_0^L \phi(x)u(x,t)dx$ is twice differentiable and $(d^2/dt^2)\int_0^L \phi(x)u(x,t)dx=\int_0^Lc^2\phi''(x)u(x,t)dx$ and $(d/dt)\int_0^L \phi(x)u(x,t)dx=\int_0^L \phi(x)g(x)dx, t=0 \qed$
% CHAPTER 5: The heat equation
{\bf Heat } Let $L,a >0$. (PDE) $( \partial_t-a\partial_x^2)u=0, 0 \leq x \leq L, t > 0$, (IC) $u(x,0)=f(x), 0 \leq x \leq L$, (BC) $u(0,t)= 0 = u(L,t), t>0$ (5.1). 
% Section 5.1: Existence and uniqueness of solutions via Fourier series
The sol, if $\exists$, can be written as a Fourier sine series (Lemma 4.19, Ex 4.3.5) $u(x,t)=\sum_{j=1}^{\infty}B_j(t)\sin(\lambda_jx), \lambda_j=j\pi/L=j\lambda_1$, with $B_j(t) = (2/L)\int_0^L u(y,t) \sin(\lambda_j y)dy$ (5.2), where, for fixed $t$, the series converges in the $L^2$-sense in $x$. If $u$ is a sol, it is suff smooth that we can diff under the int, $B_j'(t) = (2/L) \int_0^L \partial_tu(y,t)\sin(\lambda_jy)dy=(2/L)\int_0^L a\partial_y^2 u(y,t)\sin(\lambda_jy)dy$. Since the sines and $u$ satisfy zero boundary cond, we can int by parts twice and obtain the diff eq $B_j'(t)=-a \lambda_j^2B_j(t)$. The IC yields $B_j(0)=(2/L)\int_0^L f(y) \sin(\lambda_jy)dy$ (5.3). Solutions $B_j(t)=B_j(0)e^{-a\lambda_j^2t}$ (5.4). $\implies$ sol of (5.1) is uniquely determined. 
{\bf Existence} 
{\bf T 5.1}. Let $(c_j)$ be a seq of non-neg numbers s.t. $\sum_{n=1}^{\infty} c_n < \infty$. Let $D \subseteq \R^N$ and $(f_n)$ be a seq of func $f_n: D\ra \K$ s.t. $|f_n(x)| \leq c_n \forall x \in D$ and $n \in \N$. Then $\sum_{n=1}^{\infty}f_n(x)$ converges unif in $x \in D$. If each $f_n$ is (unif) cont on $D$, so is $\sum_{n=1}^{\infty}f_n$. 
%{\it Proof}. Choose $X=B(D,\K)$ or $X=BC(D, \K)$ or $X=BUC(D, \K)$, the spaces of bounded, bounded continuous, or bounded uniformly continuous functions from $D$ to $\K$, respectively. Endowed with the supremum norm, $||f||_{\infty}=\sup_{x \in D}|f(x)|$, they are Banach spaces.  Since $||f_n||_{\infty}\leq c_n$, the statements now follow from the Weierstra$\ss$ majorant test (convergence of a series follows from its absolute convergence) and the fact that the supremum norm is the norm of uniform convergence $\qed$
%{\bf T 5.2}. Let $(c_j)$ be a seq of non-neg numbers s.t. $\sum_{n=1}^{\infty} c_n < \infty$. Let $I$ be a bounded nondegenerate interval and $(f_n)$ be a seq of (cont) diff funcs $f_n:  I \ra \K$ s.t. $|f_n'(x)|\leq c_n$ for all $x \in I$ and $n \in \N$. Assume there is some $a \in I$ s.t. $\sum_{n=1}^{\infty}f_n(a)$ converges. Then $\sum_{n=1}^{\infty}f_n(x)$ and $\sum_{n=1}^{\infty}f_n'(x)$ converge unif in $x \in I, \sum_{n=1}^{\infty}f_n$ is (cont) diff and $(\sum_{n=1}^{\infty}f_n)'=\sum_{n=1}^{\infty}f_n'$.
{\bf T 5.3}. Let $(c_j)$ be a seq of non-neg numbers s.t. $\sum_{n=1}^{\infty}c_n < \infty$. Let $I_1$ and $I_2$ be two bounded nondegenerate intervals and $(f_n)$ be a seq of cont funcs $f_n: I_1 \times I_2 \ra \K$. Assume that each $f_n$ has partial derivatives wrt the first var and $|\partial_1 f_n(x,t)|\leq c_n$ for all $n \in \N, x \in I_1, t \in I_2$. Assume  $\sum_{n=1}^{\infty} f_n$ converges pointwise on $I_1 \times I_2$. Then $\sum_{n=1}^{\infty}f_n$ is partially diff wrt the first var and $\partial_1(\sum_{n=1}^{\infty} f_n) = (\sum_{n=1}^{\infty} \partial_1f_n)$, with the second series converging uniformly; this partial derivative is bounded. If each $\partial_1f_n$ is jointly cont on $I_1 \times I_2$, so is $\partial_1 (\sum_{n=1}^{\infty} f_n)$. If each $f_n$ is diff with both partial derivatives being cont and $|\partial_jf_n(x,t)|\leq c_n$ for $j=1,2$, then $\sum_{n=1}^{\infty}f_n$ is cont diff and we can interchange diff and sum (diff term by term). 
%{\it Proof}. Fix $t \in I_2$. Consider the functions $\phi_n(x) = f_n(x,t), x \in I_1$. By Theorem 5.2, $\sum_{n=1}^{\infty}\phi_n$ is differentiable on $I_1$ and differentiation and series commute. Since $t\in I_2$ has been arbitrary, this means that $\sum_{n=1}^{\infty}f_n$ is partially differentiable with repect to the first variable and the partial derivative of the series is the series of the partial derivatives. If the partial derivative of each $f_n$ is continuous on $I_1 \times I_2$, so is the partial derivative of the series by Theorem 5.1. If each $f_n$ has continuous partial derivatives with respect to both variables, so has the series. Thus the series is (totally) differentiable with continuous derivative$\qed$ Armed with these result, we can prove that our formal solutions are solutions indeed. 
%{\bf T 5.4}. Let $f$ be integrable and $\int_0^L|f(x)|dx < \infty$. Then the series $u$ in (5.2) with (5.4) and (5.3) converges unif on $[0,L] \times [\epsilon,\infty)$ for all $\epsilon >0$. Morover $u$ is infinitely often diff on $[0,L] \times (0, \infty)$ and satisfies the (PDE) and (BC) in (5.1).
%{\bf R 5.5}. The derivatives of $u$ can be found by diff term by term $\partial_x^k\partial_t^{\ell} u(x,t)=\sum_{j=1}^{\infty}B_j(0)(d^k/dx^k)\sin(\lambda_jx)(d^{\ell}/dt^{\ell})e^{-a\lambda_j^2t}$ with the series converging unif on every set $[0,L]\times [\epsilon, \infty), \epsilon > 0$. {\it Proof}. For $m \in \N$, set $u_m(x,t)=B_m(0)\sin(\lambda_mx)e^{-a \lambda_m^2t}$ (5.5). Then $u_m$ is infinitely often diff and $\partial_x^k\partial_t^{\ell} u_m(x,t)=B_m(0)(d^k/dx^k)\sin(\lambda_mx)(d^{\ell}/dt^{\ell})e^{-a\lambda_m^2t}$. Notice that  $|B_m(0)| \leq 2/L \int_0^L |f(x)|dx =: B_0 < \infty$. Let $t \geq \epsilon >0$. Then, by the form of $\lambda_m, |\partial_x^k\partial_t^{\ell} u_m(x,t)| \leq |B_m(0)| \lambda_m^{k+2 \ell} a^{\ell} e^{-a\lambda_m^2\epsilon} \leq B_0 c m^{k+2\ell}\eta^{(m^2)}$ for $\eta = e^{-a\lambda_1^2\epsilon} \in (0,1)$ and some $c >0$ that only depends on $k$ and $\ell$. It follows from the ratio test that the series $\sum_{m=1}^{\infty}B_0cm^{k+2\ell}\eta^{(m^2)}$ converges.  By T 5.1, each series $\sum_{m=1}^{\infty}\partial_x^k\partial_t^{\ell} u_m(x,t)$ (in particular the series for u) converges unif on $[0,L] \times [\epsilon, \infty)$. Let $x \in [0,L] = I_1$ and $t >0$. Choose $I_2=(t_1,t_2)$ with $0<t_1<t<t_2<\infty$. Applying T 5.3 repeatedly (or using induction) $\implies u = \sum_{m=1}^{\infty}u_m$ has partial derivatives of all order and can be diff term by term on $I_1 \times I_2$. So $u$ is infinitely often diff on $[0,L]\times (t_1,t_2)$ and can be differentiated term by term.  In particular, all partial derivatives of $u$ exist at $(x,t)$ and can be found by term by term differentiation.  Since each $u_m$ satisfies (PDE) and (BC), so does $u$ on $[0,L]\times (0,\infty) \qed$  We explore in what sense the initial condition (IC) is satisfied.
{\bf T 5.6}. Let $f: [0,L] \ra \R$ be Lipschitz cont, $f(L)=0=f(0)$. Then $u$ is cont on $[0,L] \times [0,\infty)$ and $u(x,0)=f(x)$ for all $x \in [0,L]$. In particular, $u(x,t) \ra f(x)$ as $t \ra 0$ uniformly in $x \in [0,L]$.  The Fourier sine series of $u$ converges to $u$ uniformly on $[0,L] \times [0,\infty)$. 
%{\it Proof}. The functions $u_m$ in (5.5) satisfy the estimate $|u_m(x,t)|\leq |B_m(0)|, x \in [0,L], t \geq 0, m \in \N$, with $B_m(0)$ being given by (5.3). Since $f$ is Lipschitz continuous and $f(0)=0=f(L), \sum_{m=1}^{\infty}|B_m(0)| < \infty$ by exercise 4.3.3 and odd extension.  By Theorem 5.1, $u =  \sum_{m=1}^{\infty} u_m$ converges uniformly and is continuous on $[0,L] \times [0,\infty)$. In particular, $u(x,0)=f(x)$ by Lemma 4.19, and $u$ is uniformly continuous on the closed bounded set $[0,L] \times [0,1]$. Let $\epsilon >0$. Then there exists some $\delta >0$ such that $|u(x,t)-u(y,0)| < \epsilon$ whenever $|x-y|+|t-0| < \delta$. In particular, $|u(x,t)-u(x,0)| < \epsilon$ whenever $0\leq t  < \delta \qed$  If $f(0) \neq 0$, then $u(0,t)$ does not converge to $f(0)$ as $t \ra 0$ because, $u(0,t)=0$ for all $t >0$. The analogous statement holds for $L$. We will see later that, for uniform convergence to the initial data, it is sufficient that $f$ is continuous rather than Lipschitz continuous. Without continuity assumptions, we still have convergence in the $L^2$-sense.  
{\bf T 5.7}. Assume that $f: [0,L] \ra \R$ is integrable and $\int_0^L|f(x)|^2dx<\infty$. Then the series $u$ in (5.2) with (5.4) and (5.3) satisfies $\int _0^L |u(x,t) - f(x)|^2dx \ra 0, t \ra 0$. Actually, $u(\cdot, t)$ is a uniformly cont func of $t \in \R_+$ with values in $L^2[0,L]$. Notice that $\int_0^L|f(x)|^2dx<\infty$ is a stronger assumption than $\int_0^L|f(x)|dx<\infty$ in T 5.4. 
%{\it Proof}. Let $\langle \phi, \psi \rangle = (2/L)\int_0^L \phi(x) \psi(x)dx$ be the inner product of choice on $L^2([0,L],\R)$, the space of square integrable functions.  Then $\{ v_j; j \in \N\}$ with $v_j(x)=\sin(\lambda_j x)$ is an orthonormal basis.  By the considerations at the beginning of this section, $\langle u(\cdot, t), v_m \rangle = \langle f, v_m \rangle e^{-a \lambda_m^2 t}$ (5.6), which are uniformly continuous functions on $\R_+$. Further $|\langle u(\cdot, t), v_m \rangle | \leq | \langle f, v_m \rangle |$ and, by Parseval's relation $\sum_{m \in \N}|\langle f, v_m \rangle|^2 = ||f||^2$. The assertion now follows from Theorem 4.11. $\qed$
% Behavior of solutions as $t \ra \infty$
{\bf T 5.8}. Let $u$ be the sol of the heat eq with zero boundary conditions and initial data $f \in L^2([0,L],\R)$. Then $||u(\cdot, t)|| \leq ||f||e^{-a\lambda_1^2 t}, t \geq 0$. 
%{\it Proof}. With $\{v_m\}$ being the orthonormal basis of sine functions, $u(\cdot, t)=\sum_{m=1}^{\infty}\langle f, v_m \rangle v_m e^{-a \lambda_m^2 t}$. By orthonormality and Parseval's relation, $||u(\cdot, t)||^2 = \sum_{m=1}^{\infty}|\langle f, v_m \rangle |^2 e^{-2a \lambda_m^2 t}\leq e^{-2a \lambda_1^2 t}\sum_{m=1}^{\infty}|\langle f, v_m \rangle|^2=e^{-2a \lambda_1^2 t} ||f||^2$. This implies the assertion$\qed$
{\bf T 5.9}. Let $f: [0,L] \ra \R$ be integrable and $\int_0^L|f(x)|dx<\infty$ and $u$ the sol of the heat eq from T 5.4.  Then $u(x,t) \ra 0$ as $t \ra \infty$ uniformly in $x \in [0,L]$. {\it Proof}. Recall  $|u(x, t)|=B_0 \sum_{m=1}^{\infty} e^{-a \lambda_m^2 t}, B_0=(2/L) \int_0^L |f(x)|dx$. For $t > 0$, with $\kappa = (\pi/L)^2$, since $\lambda_m^2 \leq \kappa^2m^2$, $|u(x, t)|=B_0 \sum_{m=1}^{\infty} e^{-a \lambda_m^2 t} \leq B_0 \sum_{m=1}^{\infty} (e^{-a \kappa t})^m$. Using the geometric series formula, for $t >0$, $|u(x, t)|=B_0 (e^{-a \kappa t})/(1-e^{-a \kappa t}) \ra 0, t \ra \infty \qed$
%{\bf Large-time behavior of a diffusing population. Critical patch size} 
%Let $L, a >0$.  (PDE) $(\partial_t -a \partial_x^2)w = r(t) w, 0 \leq x \leq L, t > 0$, (IC) $w(x,0)=f(x), 0 \leq x \leq L$, (BC) $w(0,t) = 0 = w(L,t), t > 0$ (5.7).  Assume that $r: \R_+ \ra \R$ is continuous.  Set $u(x,t)=w(x,t)g(t), g(t)=exp(-\int_0^tr(s)ds)$. Notice that $g'(t)=-r(t)g(t)$ and $(\partial_t - a \partial_x^2)u=g(t)(\partial_t - a \partial_x^2)w - r(t)u = g(t) ((\partial_t - a \partial_x^2)w-r(t)w)$. Hence, $w$ is a solution of (5.15) iff $u$ is a solution of the heat equation (5.1). From $u, w$ inherits the Fourier representation $w(x,t)=\sum_{m=1}^{\infty}\langle f, v_m\rangle v_m(x)e^{(\bar{r}(t)-a \lambda_m^2)t}, t>0$ (5.8), where $\bar{r}(t)$ are the time average of the per capita growth rates, $\bar{r}(t)=1/t \int_0^tr(s)ds$. By Parsval's relation, $||w(\cdot, t)||^2 = \sum_{m=1}^{\infty} |\langle f, v_m\rangle|^2e^{2(\bar{r}(t)-a \lambda_m^2)t}, t>0$ (5.9), Assume that $f \in L^2[0,L]$ and that the asymptotic time average $\bar{r}(\infty):=\lim_{t \ra \infty} \bar{r}(t)$ (5.10) exists.  Case 1: Let $\bar{r}(\infty)<a \lambda_1^2$. Then $||w(\cdot, t)||^2 \leq \sum_{m=1}^{\infty}|\langle f, v_m\rangle|^2e^{2(\bar{r}(t)-a \lambda_1^2)t} = ||f||^2e^{2(\bar{r}(t)-a \lambda_1^2)t} \ra 0$ (5.11). Indeed, let $0 < \epsilon < a \lambda_1^2 - \bar{r}(\infty)$. Then there exists some $T>0$ such that, for all $t \geq T, a \lambda_1^2 - \bar{r}(t) > \epsilon$ and  $e^{2(\bar{r}(t)-a \lambda_1^2)t} <e^{-\epsilon t}$. Case 2: Let $\bar{r} (\infty)>a \lambda_1^2$ and $\langle f, v_1 \rangle \neq 0$. Then $||w(\cdot, t)||^2 \geq |\langle f, v_1\rangle|^2e^{2(\bar{r}(t)-a \lambda_1^2)t} \ra \infty, t \ra \infty$ (5.12). Notice that $\langle f, v_1 \rangle >0$ if $f \geq 0$ and not $f = 0$ a.e., because $v_1(x) >0$ for all $x \in (0,L)$. So it depends on the sign of $\bar{r}(\infty)-(a \pi^2/L^2)$ as to whether the population dies out or survives. 
%In our model, where we have neglected any overcrowding, survival means that the population size tends to infinity.  So a large per capita growth rate and a large habitat are beneficial for survival and so is a small diffusion rate.
% Exercises
%{\bf Exercise 5.1.1} Let $L, a >0$. Consider the problem (PDE) $(\partial_t - a \partial_x^2)u = 0, 0 \leq x \leq L, t > 0$, (IC) $u(x,0)=f(x), 0 \leq x \leq L$, (BC) $\partial_xu(0,t)=0=\partial_xu(L,t), t > 0$ (5.13). This equation is a model for heat diffusion in a (finite) rod of length $L$. The no flux boundary condition means that both ends of the rod are insulated.  (a) Use Fourier cosine series to solve (5.13), at least as far as (PDE) and (BC) are concerned, under an appropriate condition for $f$. (b) Explore two assumptions for $f$ under which (IC) is satisfied in meaningful though not necessarily literal ways. (c) Show that $\int_0^L u(x,t)dx = \int_0^L u(x,0)dx$ for all $t \geq 0$.  Hint: These integrals are related to the Fourier cosine coefficient of index zero.  (d) Show that $u(x,t) \ra (1/L)\int_0^L f(x)dx$ as $t \ra \infty$, uniformly in $x \in [0,L]$. {\it Solution}. (a) Assume that $f: [0,L] \ra \R$ is integrable and $\int_0^L|f(y)|dy < \infty$. Define $u_m(x,t) = A_m \cos(\lambda_mx)e^{-a \lambda_m^2 t}, m \in \N \cup \{0\} = \Z_+$, with $\lambda_m = m\pi/L, A_m=2/L\int_0^L f(y) \cos(\lambda_my)dy, m \in \N, A_0=1/L\int_0^Lf(y)dy$. Then $u_m$ is infinitely often differentiable, $\partial_t u_m(x,t) = -a \lambda_m^2 u_m(x,t) = a \partial_x^2 u_m(x,t), x \in [0,L], t \geq 0$ and $\partial_x u_m(x,t) = -A_m \sin(\lambda_mx) e^{-\lambda_m^2 t} = 0$ for $x = 0, L$ and $ t \geq 0$. For all $k, \ell \in \Z_+$, $\partial_x^k \partial_t^{\ell} u_m(x,t) = A_m(d^k/dx^k)\cos(\lambda_mx)(d^{\ell}/dt^{\ell})e^{-a \lambda_m t}$. Notice that $|A_m| \leq 2/L \int_0^L |f(x)|dx =: A$. Let $t \geq \epsilon >0$. Then, by the form of $\lambda_m, |\partial_x^k \partial_t^{\ell} u_m(x,t)| \leq |A_m|\lambda_m^{k+2\ell}a^{\ell}e^{-\lambda_m^2 \epsilon} \leq A c m^{k+2 \ell} \eta^m$ for $\eta = e^{-a(\pi/L)^2 \epsilon} \in (0,1)$ and some $c >0$ that only depends on $k$ and $\ell$. It follows from the ratio test that the series $\sum_{m=1}^{\infty} A c m^{k+2 \ell} \eta^{2m}$ converges. By Theorem 5.1, each series $\sum_{m=0}^{\infty}\partial_x^k \partial_t^{\ell} u_m(x,t)$ (in particular the series for $u$) converges uniformly on $[0,L] \times [ \epsilon, \infty)$. Applying Theorem 5.3 repeatedly (or use induction) implies that $u = \sum_{m=1}^{\infty} u_m$ has partial derivatives of all order and can be differentiated term by term.  This holds on $[0,L] \times [\epsilon, \infty)$ for every $\epsilon >0$. So $u$ is infinitely often differentiable on $[0,L] \times (0, \infty)$ and can be differentiated term by term. Since each $u_m$ satisfies (PDE) and (BC), so does $u$ on $[0,L] \times (0, \infty)$. (b) First condition:  Let $f: [0,L] \ra \R$ be Lipschitz continuous. Then $u$ is continuous on $[0,L] \times [0, \infty)$ and $u(x,0) = f(x)$ for all $x \in [0,L]$. In particular, $u(x,t) \ra f(x)$ as $t \ra 0$, uniformly in $x \in [0,L]$. Notice that $|u_m(x,t)| \leq |A_m|, x \in [0,L], t \geq 0$ with $A_m$ being the appropriate Fourier cosine coefficient. Since $f$ is Lipschitz continuous, $\sum_{m=0}^{\infty} |A_m| < \infty$ by Exercise 4.3.3 and even extension of $f$. By Thorem 5.1, $u = \sum_{m=0}^{\infty}u_m$ converges uniformly and is continuous on $[0,L] \times [0, \infty)$. Further $u(x,0)= \sum_{m=0}^{\infty} A_m \cos(\lambda_mx)=f(x)$ by Exercise 4.3.4. The last statement follows as in Theorem 5.6.  Second condition:  Assume that $f: [0,L] \ra \R$ is integrable and $\int_0^L|f(x)|^2 dx < \infty$. Then the series $u$ in (5.2) with (5.4) and (5.3) satisfies $|| u(\cdot, t) - f||^2 = 2/L \int_0^L |u(x,t)-f(x)|^2 dx \ra 0, t \ra 0$. Let $\langle \phi, \psi \rangle = 2/L \int_0^L \phi(x) \psi(x) dx$ be the inner product of choice on $L^2([0,L], \R)$, the space of square integrable functions.  Then $\{v_m;m\in \Z_+\}$ with $v_m=\cos(\lambda_mx)$ for $m \in \N$ and $v_0 = 2^{-1/2}$ is an orthonormal basis, $U(t) := u(\cdot, t) =  \sum_{m=0}^{\infty} \langle f, v_m \rangle e^{-a \lambda_m^2 t} v_m$, with convergence holding in mean-square norm for $t \geq 0$ and uniformly on $[0,L]$ for $t>0$. Notice that $\langle U(t), v_m \rangle = \langle f, v_m \rangle e^{-a \lambda_m^2 t}$ is uniformly continuous on $\R_+$, $|\langle U(t), v_m \rangle| \leq \langle f, v_m \rangle$ and, by Parseval's relation, $\sum_{n=0}^{\infty} |\langle f, v_m \rangle|^2 = ||f||^2 < \infty$. By Theorem 4.11, the function $U: \R_+ \ra X, X = L^2([0,L]), U(t)=u(\cdot)$, is continuous on $\R_+ = [0,\infty)$ and $U(0)=f$. (c) By orthonormality and $\lambda_0 = 0$, for all $t \geq 0, \langle u(\cdot, t), v_0 \rangle = \langle f, v_0 \rangle = (2 \cdot 2^{-1/2})/L\int_0^L f(x)dx$. This implies the assertion. (d) Notice that $u(x,t)=-1/L\int_0^Lf(x)dx=u(x,t)-A_0 = \sum_{m=1}^{\infty}A_m \cos(\lambda_mx)e^{-a \lambda_m^2 t}$. So $|u(x,t)-1/L\int_0^Lf(x)dx\leq \sum_{m=1}^{\infty} |A_m |e^{-a \lambda_m^2 t} \leq A \sum_{m=1}^{\infty} e^{-a \lambda_m^2 t}$.  The proof now continues as the proof of Theorem 5.9 $\qed$
%{\bf Exercise 5.1.2}. Let $f: [0,L] \ra \R$ be integrable and $\int_0^L |f(x)|dx < \infty$, i.e. $f \in L^1([0,L],\R)$. Consider the heat equation with zero boundary condition and initial data $f$. Show:  there exists a function $u: [0,L] \times (0,\infty) \ra \R$ that solves (PDE) and (BC) and satisfies the initial condition in the following weak sense:  If $\phi: [0,L] \ra \R$ is Lipshitz continuous and $\phi(0) = 0 = \phi(L)$, then $\int_0^L \phi(x)u(x,t)dx \ra \int_0^L\phi(x)f(x)dx, t \ra 0$. Hint: Notice (and prove) that  $\int_0^L \phi(x)u(x,t)dx =\int_0^L f(x)v(x,t)dx$ where $v$ is the solution of the heat equation with initial data $\phi$. {\it Proof}. Let $\langle \phi, \psi \rangle = 2/L \int_0^L \phi(x) \psi(x) dx$ be the inner product of choice on $L^2([0,L], \R)$, the space of square integrable functions.  Then $\{v_j; j \in \N\}$ with $v_j(x)=\sin(\lambda_jx)$ is an orthonormal basis. For $t > 0, u(\cdot, t)=\sum_{m=1}^{\infty} \langle f, v_m \rangle v_m e^{-a \lambda_m^2 t}$ (5.14) converges uniformly on $[0,L]$ (equivalently in the supremum norm) (Theorem 5.4).  Here, abusing the notation ($f$ not necessarily in $L^2$), we have written $\langle f, v_m \rangle$ for $1/(2L) \int_0^L f(x) v_m(x)dx$ which is defined because $v_m$ is continuous. Then, for $t >0, \langle \phi, u(\cdot, t)\rangle = \sum_{m=1}^{\infty}\langle f, v_m \rangle \langle \phi, v_m \rangle e^{-a \lambda_m^2 t}$. By Theorem 5.6, $w(\cdot, t) =  \sum_{m=1}^{\infty}\langle \phi, v_m \rangle e^{-a \lambda_m^2 t}$ converges uniformly on $[0,L]$, uniformly in $t \in \R_+$, and $w$ is continuous on $[0,L] \times \R_+$. So $1/(2L) \int_0^L f(x)w(x,t)dx=\sum_{m=1}^{\infty}\langle \phi, v_m \rangle \langle f, v_m \rangle e^{-a \lambda_m^2 t} = \langle \phi, u(\cdot, t)\rangle $. Since $w(\cdot, t) \ra \phi$ as $t \ra 0$, uniformly on $[0,L], 1/(2L)\int_0^L \phi(x) u(x,t) dx = \langle \phi, u(\cdot, t)\rangle \ra 1/(2L)\int_0^L \phi(x) f(x) dx, t \ra 0$. This implies the assertion. $\qed$
{\bf Exercise 5.1.3}. Let $L, a > 0$. Consider the problem (PDE) $(\partial_t - a \partial_x^2)u = 0, -\pi \leq x \leq \pi, t > 0$, (IC) $u(x,0)=f(x), -\pi \leq x \leq \pi$, (BC) $u(-\pi,t)=u(\pi,t), \partial_x u(-\pi, t) =\partial_x u(\pi, t), t > 0$ (5.15).  (a) Use complex Fourier series to solve (5.15), at least as far as (PDE) and (BC) are concerned, under an appropriate condition for $f$. (b) Explore two assumptions for $f$ under which (IC) is satisfied in meaningful though not necessarily literal ways. Cf. Theorem 5.6 and 5.7. (c) Show that $u$ is real-valued if $f$ is real-valued. (d) Show that $\int_{-\pi}^{\pi} u(x,t)dx = \int_{-\pi}^{\pi} u(x,0)dx$ for all $t \geq 0$.  Hint: These integrals are related to the Fourier cosine coefficient of index zero.  (e) Show that $u(x,t) \ra (1/2 \pi)\int_{-\pi}^{\pi} f(x)dx$ as $t \ra \infty$, uniformly in $x \in [-\pi,\pi]$ provided that $\int_{-\pi}^{\pi}|f(x)|dx < \infty$. {\it Proof}. (a) We try to find $u$ as a complex Fourier series $u(x,t)=\sum_{m\in \Z} C_m(t)e^{imx}, C_m(t) = 1/(2 \pi)\int_{-\pi}^{\pi} u(x,t)e^{-imx}dx$ (5.16). If $\partial_t u(x,t)$ exists and is continuous on $[-\pi, \pi] \times (0, \infty)$, we can interchange time differentiation and integration, $C_m'(t) = 1/(2\pi) \int_{-\pi}^{\pi} \partial_t u(x,t) e^{-imx} dx = 1/(2\pi) \int_{-\pi}^{\pi} a\partial_x^2 u(x,t) e^{-imx} dx$. We integrate by parts twice; the boundary terms cancel because of the periodic boundary conditions, $C_m'(t) = -a m^2 C_m$. Further $C_m(0) = 1/(2 \pi)\int_{-\pi}^{\pi} f(x) e^{-imx}dx = \hat{f}_m$. We solve the ordinary differential equation, $C_m(t)=\hat{f}_m e^{imx} e^{-am^2t}$. That the series in (5.16) converges uniformly on $[-\pi, \pi] \times [ \epsilon, \infty)$ for any $\epsilon >0$ and solves (PDE) in (5.15) is shown analogously to the proof of Theorem 5.4. Define $u_m(x,t)=\hat{f}_m e^{imx} e^{-am^2t}$. Then $u_m$ is infinitely often differentiable and satisfies the heat equation, $\partial_t u_m(x,t)=-am^2u_m(x,t) =  a \partial_x^2u_m(x,t)$. For all $k, \ell \in \Z_+$ and $m \in \Z, \partial_x^k \partial_t^{\ell} u_m(x,t)=\hat{f}_m(im)^k(-am^2)^{\ell} e^{imx} e^{-am^2 t}$ and $|\partial_x^k \partial_t^{\ell} u_m(x,t)|=|\hat{f}_m| |i|^k |m|^k a^{\ell}m^{2 \ell} |e^{imx}| e^{-am^2 t} \leq |\hat{f}_m|  |m|^{k+2\ell} a^{\ell}  e^{-am^2 t}$. For $m \in \Z, |\hat{f}_m|\leq 1/(2 \pi) \int_{-\pi}^{\pi}|f(x)| |e^{-imx}|dx \leq 1/(2 \pi) \int_{-\pi}^{\pi}|f(x)| dx =: A$. Let $\epsilon > 0$. For $m \in \Z$ and $k, \ell \in \Z_+$ and $t \in [\epsilon, \infty), |\partial_x^k \partial_t^{\ell} u_m(x,t)| \leq A |m|^{k+2 \ell}a^{\ell} e^{-am^2 \epsilon}$. By the ration test, $\sum_{m=1}^{\infty}A m^{k+2\ell}a^{\ell}e^{-am^2 \epsilon}$ converges in $\R$. By the Weierstra$\ss$ test, for each $\ell, k \in \Z_+$, the following series converge uniformly for $x \in [-\pi, \pi], t \in [\epsilon, \infty), \sum_{m=1}^{\infty}\partial_x^k \partial_t^{\ell}u_m(x,t), \sum_{m=1}^{\infty}\partial_x^k \partial_t^{\ell}u_{-m}(x,t), \sum_{m\in \Z} \partial_x^k \partial_t^{\ell}u_m(x,t)$ with the third being the sum of the first and second.  By Theorem 5.3, $u$ is infinitely often partially differentiable on $[-\pi, \pi] \times (0, \infty)$ and $\partial_x^k \partial_t^{\ell}u(x,t) =  \sum_{m\in \Z}\partial_x^k \partial_t^{\ell}u_m(x,t)$. In particular, $u$ satisfies the heat equation on $[-\pi, \pi] \times (0, \infty)$.  Since $e^{imx}$ is $2\pi$-periodic for all $m \in \Z, u(\pi,t)=u(-\pi,t)$ for all $t >0$ follows from the uniform convergence of the series in (5.16) converges uniformly on $[-\pi, \pi] \times [\epsilon, \infty)$ for any $\epsilon > 0$. Further, $\partial_x u(x,t) = \sum_{m\in \Z}C_m(t)mie^{imx}$ with convergence being uniform for $x \in [-\pi, \pi], t \in  [\epsilon, \infty)$ for any $\epsilon > 0$. By the same token as before $\partial_x u(\pi,t)=\partial_x u(-\pi,t), t>0$. (b) This is analogous to Theorem 5.6 and 5.7. We first assume that $f$ is Lipschitz continuous on $[-\pi, \pi]$ and $f(\pi) = f( -\pi)$. By Lemma 4.16, $f$ can be extended to a Lipschitz continuous $2 \pi$-periodic function on $\R$. by Proposition 4.14, the following series converges, $\sum_{m \in \Z}|\hat{f}_m| = |\hat{f}_0|+\sum_{m=1}^{\infty}(|\hat{f}_m|+|\hat{f}_{-m}|)$. For all $m \in \Z, |C_m(t)|\leq |\hat{f}_m| |e^{-am^2t}|\leq |\hat{f}_m|$. Thus $|C_m(t)e^{imx}|= |C_m(t)| |e^{imx}|=|C_m(t)|\leq |\hat{f}_m|, m \in \Z$. For all $m\in \N, |C_m(t)e^{imx} + C_{-m}(t)e^{-imx}|\leq |C_m(t)e^{imx}| + |C_{-m}(t)e^{-imx}| \leq |\hat{f}_m| + |\hat{f}_{-m}|$. By the Weierstra$\ss$ test. $\sum_{m\in \Z}C_m(t) e^{imx}=C_0(t) + \sum_{m=1}^{\infty}(C_m(t)e^{imx} + C_{-m}(t)e^{-imx})$ converges uniformly on $[-\pi, \pi] \times [0, \infty)$. This implies that $u$ is continuous on $[-\pi, \pi] \times [0, \infty)$. We now assume that $f$ is integrable and $\int_{-\pi}^{\pi}|f(x)|^2 dx < \infty$. Let $\langle \phi, \psi \rangle = 1/(2\pi)\int_{-\pi}^{\pi} \phi(x) \overline{\psi(x)} dx$ be the inner product of choice on $L^2([-\pi, \pi], \C)$, the space of square integrable functions. Then $\{v_j; j\in \Z\}$ with $v_j(x)=e^{ijx}$ is an orthonormal basis.  We intend to apply Theorem 4.11. Notice that $g: \N \ra \Z$ with $g(2n) = n$ and $g(2n-1)=-n, n\in \N$, is a bijection.  By the considerations at the beginning of this section, $\langle u(\cdot, t), v_m \rangle = \langle f, v_m \rangle e^{-a\lambda_m^2 t}$ (5.17), which are uniformly continous functions of $t \in \R_+$. Further $|\langle u(\cdot, t), v_m \rangle| \leq  |\langle f, v_m \rangle|$  and, by Parseval's relation (Theorem 4.10), $\sum_{m \in \Z}|\langle f, v_m \rangle|^2 = ||f||^2$. Theorem 4.11 implies that $U: \R_+ \ra L^2([-\pi, \pi], \C)$ with $U(t) = u(\cdot, t)$ is continuous. (c) This is analogous to Theorem 4.18. For fixed $t > 0, u(x,t)$ is the limit (uniformly in $x \in [-\pi, \pi]$) of Fourier sums $\sum_{m=-n}^n \hat{f}_m e^{-am^2t} e^{imx} = \hat{f}_0 + \sum_{m=1}^n (\hat{f}_me^{imx} +\hat{f}_{-m}e^{-imx})e^{-am^2t}$. The same proof as for Theorem 4.18 shows that $\hat{f}_me^{imx} +\hat{f}_{-m}e^{-imx}$ is real if $f$ is real-valued. (d) Let $\langle \phi, \psi \rangle = 1/(2\pi)\int_{-\pi}^{\pi} \phi(x) \overline{\psi(x)} dx$ be the inner product on the Hilbert space $L^2([-\pi, \pi], \C)$. Let $\{v_m; m\in \Z\}$ be the orthonormal basis with $v_m(x)=e^{imx}$. Notice that $v_0(x) = 1$. By orthonormality and part (a), $\langle u(\cdot, t), v_0 \rangle = C_0(t) = \langle f, v_0\rangle = \langle u(\cdot, 0), v_0 \rangle$. This implies the assertion. (e) For all $t \geq 0, |u(x,t)-1/(2\pi)\int_{-\pi}^{\pi}f(x)dx| = | \sum_{0 \neq m\in \Z} \hat{f}_m e^{-am^2 t}e^{imx}| \leq  \sum_{0 \neq m\in \Z} |\hat{f}_m| e^{-am^2 t} \leq A \sum_{m=1}^{\infty}  e^{-am^2 t}$ with $A = 1/\pi \int _{-\pi}^{\pi} | f(x)| dx$. The estimate can be continued by $\leq A \sum_{m=1}^{\infty}  e^{-am t}= \sum_{m=1}^{\infty}  A (e^{-a t})^m = A (e^{-at})/(1-e^{-at}) \ra^{t \ra \infty} 0 \qed$
%{\bf Exercise 5.1.4} Let $L, a > 0$. We change the heat diffustion problem into a problem for a diffusing population with density-dependent effect, (PDE) $(\partial_t - a \partial_x^2)u = r(t)u, 0 \leq x \leq L, t>0$, (IC) $u(x,0) = f(x), 0 \leq x \leq L$, (BC) $u(0,t) = 0 = u(L, t), t > 0$ (NL) $ r(t)=b-2c/L \int_0^L u(t,y) \sin(\lambda_1 y) dy$  with $b,c >0$. NL stands for nonlinear term. Assume that $f$ is nonnegative, continuous, and strictly positive somewhere on $(0,L)$. Derive the ODEs for the Fourier sine coefficients and consider their solutions. Notice that the first Fourier sine coefficient of a solution satisfies the logistic (aka Verhulst) differential equation and has a positive value at $t = 0$. Show that the Fourier sine series is a solution. (The convergence of the appropriate series will be uniform on every interval $[\epsilon, 1/\epsilon]$ with $0 < \epsilon < 1)$. Define the carrying capacity $K = (b-a(\pi/L)^2)/c$. Show: If $K \leq 0$, then $\int_0^L|u(x,t)|^2 dx \ra 0$ as $t \ra \infty$. $K > 0$, then $\int_0^L|u(x,t)|^2 dx \ra K^2$ as $t \ra \infty$.  Now assume in addition that $f$ is Lipschitz continuous and $f(0)=0=f(L)$. Show: If $K \leq 0$, then $u(x,t) \ra 0$ as $t \ra \infty$ unformly for $x \in [0,L]$. If $K>0$, then $u(x,t) \ra K \sin(\pi x/L)$ as $t \ra \infty$ unformly for  $x \in [0,L]$.
% Section 5.2: Maximum principle for the heat equation
%{\bf Max principle for the heat equation} Let $\Omega$ be an open bounded subset of $\R^n. \bar{\Omega}$ denotes the closure of $\Omega$ and $\partial \Omega$ the boundary of $\Omega$. In the previous section, we have considered $\Omega = (0,L) \subseteq \R$. Then $\bar{\Omega} = [0,L]$ and $\partial \Omega = \{0,L\}$. Let $I$ be an interval.  For a function $u: \Omega \times I \ra \R$ that is once partially diff wrt $t \in I$ and twice partially diff wrt $x_k$ at each $x \in \Omega, k = 1, \dots , n$, let $(Lu)(x,t) = \sum_{k=1}^n [a_k \partial_{x_k}^2 u(x,t) + b_k\partial_{x_k}u(x,t)], x \in \Omega, t \in I$ (5.19), with $a_k \geq 0$ and $b_k \in \R$ for $k = 1, \dots, n$. 
{\bf T 5.10}. Let $T \in (0,\infty)$ and $u: \Omega \times (0,T] \ra \R$. Assume that $u$ is once partially diff wrt $t \in (0,T]$ and twice partially diff wrt $x_k$ at each $x \in \Omega, t \in (0,T], k = 1, \dots, n$, Let $c: \Omega \times (0, T] \ra \R$ be strictly neg. Assume the differential inequality $(\partial_t - L)u \leq c(x,t)u, x \in \Omega, t \in (0,T]$. Then $u$ has no positive max in $\Omega \times (0,T]: \exists$ no $t\in (0,T], x \in \Omega$ such that $u(x,t) \geq u(y,s)$ for all $s \in (0,T], y \in \Omega$, and $u(x,t) > 0$.
% {\it Proof} (by cont). Suppose that such $t \in (0, T], x \in \Omega$ exist. Then, for all $s \in (0, t], u(x,t) \geq u(x,s)$ and $(u(x,s)-u(x,t))/(s-t) \geq 0$ because the numerator is nonpositive and the denominator is neg. So $\partial_t u(x,t) = \lim_{s \ra t} (u(x,s)-u(x,t))/(s-t)  \geq 0$. (It is possible that $t = T$. Then the last limit is only a limit from the left and $\partial_t u(x,t)>0$ could occur.) Define $\phi(r) = u(x_1 + r, x_2, \dots, x_n, t)$ for $r \in (-\delta, \delta)$ and $\delta >0$ small enough s.t. $(x_1 + r, x_2, \dots, x_n) \in \Omega$ whenever $r \in (-\delta, \delta)$. Then $\phi(0) \geq \phi(r)$ for all $r \in (-\delta, \delta)$ and $\phi$ is twice diff on $(-\delta, \delta)$. Thus $0 = \phi'(0)=\partial_{x_1}u(x,t), 0 \geq \phi''(0) = \partial_{x_1}^2 u(x,t)$. Similarly $\partial_{x_k} u(x,t) = 0$ and $0 \geq \partial_{x_k}^2 u(x,t)$ for all $k = 1, \dots, n$. Since $u(x,t)>0$ and $c(x,t)<0, \partial_t u(x,t)-(Lu)(x,t)\geq 0 > c(x,t)u(x,t)$, a contradition to the differential inequality in the assumptions of this theorem$\qed$.
{\bf T 5.11}. Let $T \in (0, \infty)$ and $u: \bar{\Omega} \times [0,T] \ra \R$ be cont. Let $c: \Omega \times (0,T) \ra \R$ be bounded above. Assume that $u$ is once partially diff wrt $t \in (0,T)$ and twice partially diff wrt $x_k$ at each $x \in \Omega, t \in (0, T), k = 1, \dots, n$. Assume $(\partial_t -L)u \leq cu, x \in \Omega, t \in (0,T), u(x,0) \leq 0, x \in \bar{\Omega}, u(x,t)\leq 0, x \in \partial \Omega, t \in (0,T)$. Then $u(x,t) \leq 0$ for all $x \in \bar{\Omega}, t \in [0,T]$. %{\it Proof}. First assume $c$ is strictly neg. Suppose that $u (x,t) >0$ for some $x \in \bar{\Omega}, t \in [0,T)$. Since $u$ is cont, it takes a max on $\bar{\Omega} \times [0,t]$; this max is pos. So there exists $z \in \bar{\Omega}, r \in [0, t]$ such that $u(z,r)>0$ and $u(z,r) \geq u(y,s)$ for all $y \in \bar{\Omega}, s \in [0,t]$. By our assumptions $z \in \Omega, r \in (0, t]$. But this contradicts T 5.10 with $t$ instead of $T$. So $u(x,t) \leq 0$ for all $x \in \bar{\Omega}, t \in [0, T)$. Since $u$ is cont on $\bar{\Omega} \times [0,T], u(x,t)\leq 0$ for all $x \in \bar{\Omega}, t \in [0, T]$. Now assume that $c$ is bounded above. Then $\exists$ some $\kappa > 0$ s.t. $\kappa > c(x,t)$ for all $x \in \Omega, t \in (0,T)$. Define $v(x,t)=u(x,t)e^{-\kappa t}, t \in [0,T], x \in \bar{\Omega}$. Then $\partial_t v(x,t)-(Lv)(x,t)= e^{-\kappa t}(\partial_t u(x,t) - (Lu)(x,t) - \kappa v(x,t) \leq (c(x,t)- \kappa) v(x,t)$.  Now, $c(x,t) - \kappa < 0$ for all $x \in \Omega$ and $t \in (0,T)$ and $v \leq 0$ on $\partial \Omega \times [0,T]$ and on $\Omega \times \{0\}$. By our previous consideration, applied to $v$ rather than $u, v \leq 0$ on $\bar{\Omega} \times [0,T]$. This implies that $u \leq 0$ on $\bar{\Omega} \times [0,T] \qed$ T 5.11 implies uniqueness of sols. 
%{\bf C 5.12}. Let $T \in (0, \infty)$ and $c, F: \Omega \times (0,T)\ra \R, c$ bounded above, and $g: \partial \Omega \times (0,T) \ra \R, f: \Omega \ra \R$. Then there exists at most one cont func $u: \bar{\Omega} \times [0,T] \ra \R$ s.t. $u$ is once partially diff wrt $t \in (0,T)$ and twice partially diff wrt $x_k$ at each $x \in \Omega, t \in (0,T), k=1,\dots, n$, and $(\partial_t - L - c)u = F(x,t), x \in \Omega, t \in (0,T), u(x,0)=f(x), x \in \Omega, u(x,t) = g(x,t), x \in \partial \Omega, t \in (0,T)$. {\it Proof}. Assume there are two, $u_1$ and $u_2$. Set $u = u_1 - u_2$. Since $u$ is cont, $u = 0$ on $\partial \Omega \times [0,T]$ and on $\bar{\Omega} \times \{0\}, (\partial_t - L - c)u = 0$ on $\Omega \times (0,T)$. By T 5.11, $\pm u \leq 0$. This implies $u=0 \qed$

% 5.2.1 Estimates and qualitative behavior of solutions
%{\bf Estimates and qualitative behavior of solutions} Theorem 5.11 also provides estimates for solutions. 
{\bf T 5.13}. Let $T \in (0, \infty)$ and $u: \bar{\Omega} \times [0,T] \ra \R$ be cont. Let $c, F: \Omega \times (0, T) \ra \R, F$ bounded and $c$ non-pos. Assume that $u$ is once partially diff wrt $t \in (0, T)$ and twice partially diff wrt $x_k$ at each $x \in \Omega, t \in (0, T), k = 1, \dots, n$. Assume $(\partial_t - L - c)u = F(x,t), x \in \Omega, t \in (0,T)$. Let $M, N \geq 0$ s.t. $|u(x,0)| \leq M, x \in \bar{\Omega}, |u(x,t)|\leq M+tN, x \in \partial \Omega, t \in [0,T)$, and $|F(x,t)| \leq N$ for all $x \in \Omega, t \in (0, T)$. Then $|u(x,t)|\leq M+tN$ for all $x \in \bar{\Omega}, t \in [0, T]$. 
%{\it Proof}. Define $v_{\pm}(x,t)= \pm u(x,t)-(M+tN)$ for $x \in \bar{\Omega}$ and $ t \in [0, T]$. Since $c$ is non-positive, $(\partial_t - L - c)v_{\pm} = \pm F(x,t) + c(M+tN)-N \leq 0$ on $ \Omega \times (0,T)$. Further $v_{\pm} \leq 0$ on $\partial \Omega \times [0, T]$ and on $\Omega \times \{0\}$. By T 5.11, $v_{\pm} \leq 0$ on $\bar{\Omega} \times [0,T]$. So $\pm u(x,t)\leq M+tN$, i.e. $|u(x,t)|\leq M+Nt \qed$ The estimates for sols imply that sols depend cont on initial and boundary data and on forcing functions. 
{\bf T 5.14}. Let $T \in (0,\infty)$ and $u_1, u_2: \bar{\Omega}\times [0,T] \ra \R$ be cont. Let $c, F_1, F_2: \Omega \times (0,T) \ra \R, F_j$ bounded and $c$ non-pos and $g_1, g_2: \partial \Omega \times [0, T] \ra \R, f_1, f_2: \bar{\Omega} \ra \R$. Assume $u_1$ and $u_2$ are once partially diff wrt $t \in (0, T)$ and twice partially diff wrt $x_k$ at each $x \in \Omega, t \in (0, T), k = 1, \dots, n$. Assume, for $j = 1,2, (\partial_t - L - c)u_j = F_j(x,t), x \in \Omega, t \in (0,T), u_j(x,0)=f_j(x), x \in \Omega, u_j(x,t)=g_j(x,t), x \in \partial \Omega, t \in (0, T)$. Let $\delta, \epsilon > 0$ and $|F_1(x,t)-F_2(x,t)|\leq \delta, x \in \Omega, t \in (0,T), |f_1(x) - f_2(x)| \leq \epsilon, x \in \Omega, |g_1(x,t) - g_2(x,t)|\leq \epsilon + \delta t, x \in \partial \Omega, t \in (0,T)$. Then $|u_1(x,t) - u_2 (x,t)| \leq \epsilon + \delta t$ for all $x \in \bar{\Omega}, t \in [0,T]$. 
%{\it Proof}. Set $u = u_1-u_2$ and $F = F_1 - F_2$. Then $(\partial_t - L - c)u = F$ on $\Omega \times (0, T)$ with $|F(x,t)|\leq \delta$ for all $x \in \Omega$ and $t \in (0, T)$. Further $|u(x,t)| \leq \epsilon + \delta t$ whenever $x \in \partial \Omega, t \in (0, T)$ or $x \in \Omega$ and $t = 0$. Since $u$ is cont, this also holds whenever $x \in \partial \Omega, t \in [0,T]$ or $x \in \bar{\Omega}, t=0$. By T 5.13, $|u(x,t)|\leq \epsilon + \delta t \qed$ 
% Example 5.15
%{\bf Exmp 5.15}. Consider the problem $(\partial_t - \partial_x^2)u=0$ on $(0,1) \times (0, \infty), u(0,t)=0=u(1,t), t \geq 0, u(x,0)=x(1-x)g(x), 0 \leq x \leq 1$ with some cont nonneg func $g: [0,1] \ra \R$. Show: $ \exists \alpha, \beta >0$ such that the cont sol $u: [0,L] \times [0,\infty) \ra \R$ ($\exists$ by T 5.16) satisfies $0 \leq u(x,t) \leq \beta x(1-x)e^{-\alpha t}, 0 \leq x \leq 1, t \geq 0$. We can choose $\alpha = 8$ and $\beta = \sup g$. {\it Proof}. Set $v(x,t)=\beta x (1-x)e^{-\alpha t}$ with $\alpha$ and $\beta$ tbd later.  Then $(\partial_t - \partial_x^2)v(x,t) = - \alpha \beta x(1-x)e^{-\alpha t} + \beta 2 e^{-\alpha t} = \beta e^{-\alpha t}(2-\alpha x(1-x))$. The local and global max of the func $x(1-x)$ is taken at $x = 1/2$ and equals $1/4$. So $(\partial_t - \partial_x^2)v(x,t) \geq \beta e^{-\alpha t}(2-\alpha/4)$. We choose $\alpha = 8$ and have $(\partial_t - \partial_x^2)v \geq 0$. In order to achieve $u(x,0)\leq v(x,0)$ we choose $\beta \geq g(x)$ for all $x \in [0,1]$. This is possible because $g$ is cont and thus bounded.  Define $ w = u - v$. Then $w$ is cont on $[0,1] \times [0, \infty), (\partial_t - \partial_x^2)w \leq 0$ on $[0,1] \times (0, \infty), w(x,t) = 0$ for $x = 0, x = 1, t \geq 0, w(x,0) \leq 0$ for all $x \in [0,1]$. So $w \leq 0$ by Theorem 5.11. The nonneg of $u$ follows by applying T 5.11 to $-u \qed$
% Exercises
%{\bf Ex 5.2.1.} Let $T>0$ and $u: \overline{\Omega}\times [0,T]$ be cont.  Let $c, F : \Omega \times (0,T) \ra \R$, $F$ non-neg and $c$ bounded above and $f:\Omega\ra \R$ be non-neg.  Let the partial differential operator $L$ be as in the text before.  Assume that $u$ is once partially diff wrt $t \in (0,T)$ and twice partially diff wrt $x_k$ at each $x \in \Omega, k=1,\dots, n$. Assume $(\partial_t-L-c)u = F(x,t), \quad x \in \Omega, \quad t \in (0,T), u(x,t)=0, \quad x \in \partial \Omega, \quad t \in (0,T), u(x,0)=f(x), \quad x \in \Omega$. Show $u(x,t)\geq 0$ for all $x \in \overline{\Omega}, t \in [0,T]$. {\it Proof}. Let $v=-u$. Since $F$ and $f$ are non-neg, $(\partial_t-L-c)v = -F(x,t) \leq 0, x \in \Omega, t \in (0,T), v(x,t) = 0,  x \in \partial \Omega, t \in (0,T), v(x,0)=-f(x) \leq 0, x \in \Omega.$ By T 5.11, $-u = v \leq 0$. This implies $u \geq 0\qed$.
{\bf Ex 5.2.3}.  Let $T>0$ and $u: \overline{\Omega}\times [0,T]$ be cont.  Let $c, F : \Omega \times (0,T) \ra \R$, $F$ bounded and $c$ bounded above.   Assume $u$ is once partially diff wrt $t \in (0,T)$ and twice partially diff wrt $x_k$ at each $x \in \Omega, k=1,\dots, n$. Assume $(\partial_t - L-c)u = F(x,t), x \in \Omega, t \in (0,T),$ Let $M, N \geq 0$ such that $|u(x,t)| \leq M$ whenever $x \in \partial \Omega, t \in [0,T] \text{ or } x \in \overline{\Omega}, t = 0$ and $|F(x,t)| \leq N \text{ for all } x \in \Omega, t \in (0,T).$ Show:  $|u(x,t)| \leq (M + tN)e^{\kappa t}$ for all $x \in \overline{\Omega}, t \in [0,T]$, where $\kappa \geq 0$ is chosen s.t. $c(x,t) \leq \kappa$ for all $x \in \Omega, t \in (0,T)$. Hint:  Consider $v(x,t)=u(x,t)e^{-\kappa t}$.
{\it Proof}. Define $v(x,t)=u(x,t)e^{-\kappa t}$.  Then $\partial_tv(x,t) - (Lv)(x,t) = e^{-\kappa t}(\partial_tu(x,t) - (Lu)(x,t))-\kappa v(x,t)= (c(x,t) - \kappa)v(x,t) + F(x,t)e^{-\kappa t}$. So $\partial_tv(x,t) - (Lv)(x,t) - \tilde{c}(x,t) v(x,t)= \tilde{F}(x,t)$, where $\tilde{c}(x,t) = c(c,t) - \kappa \leq 0, \tilde{F}(x,t)= F(x,t)e^{-\kappa t}$ and so $|\tilde{F}(x,t)| \leq |F(x,t)|\leq N, x \in \Omega, t \in (0,T)$. Further $|v(x,t)| \leq |u(x,t)| \leq M$ whenever $x \in \partial \Omega, t \in [0,T]$ or $x \in \bar{\Omega}, t = 0$. By T 5.13, $|v(x,t)| \leq M + N t$. So 
 $|u(x,t)|=|v(x,t)|e^{\kappa t} \leq (M+Nt)e^{\kappa t}, x \in \bar{\Omega}, t \in [0,T] \qed$. 
%{\bf Exercise 5.2.6}. (a) Find the sol $w$ of $-\partial_x^2 w(x)=2, \quad 0<x<1, w(0) = 0 = w(1).$ (b) Suppose that $u$ is the sol to $(\partial_t - \partial_x^2)u=2 \quad \text{ on } (0,1) \times (0, \infty), u(0,t)=0=u(1,t), \quad t \geq 0, u(x,0)=0, \quad 0 \leq x \leq 1.$ Show that 
$x(1-x)(1-e^{-8t})\leq u(x,t) \leq x(1-x), \quad 0 \leq x \leq 1.$ Remark:  $w$ in part (a) is the steady state for the equation in part (b). {\it Proof}. (a) We integrate twice to get
$\partial_x w(x)=-2x+c_1 \implies w(x)=-x^2+c_1+c_2$. Now we check boundary conditions. $w(0)=0=c_2$ and $w(1)=0=-1+c_1$ and so we have that $c_1 = 1$ and so $w(x)=-x^2+x=x(1-x)$. (b) We let $v(x,t)=x(1-x)(1-e^{-8t})$.  Then $\partial_t v(x,t)=x(1-x)8e^{-8t}$ and $\partial_x v(x,t)=-2x(1-e^{-8t})$ and $\partial_x^2=-2(1-e^{-8t})$.  We put this together to get $(\partial_t - \partial_x^2)v=x(1-x)8e^{-8t}+2(1-e^{-8t})=2[4x(1-x)e^{-8t}+(1-e^{-8t})]=2[e^{-8t}(4x(1-x)-1)+1]$.  Given that $f(x)=x(1-x), f'(x)=1-2x=0$ at $x=1/2$ and so $f(x)$ has a max at $1/4$.  So  $2[e^{-8t}(4x(1-x)-1)+1] \leq 2[e^{-8t}(4(1/4)-1)+1]=2$
Therefore $(\partial_t-\partial_x^2)v \leq 2$.  Now we set $\tilde{v}=v-u$. Then $(\partial_t - \partial_x^2) \tilde{v}=(\partial_t-\partial_x^2)v-(\partial_t-\partial_x^2)u \leq 2-2=0$.  Further, for $\partial \Omega = \{0,1\}, \tilde{v}(0,t)=(v-u)(0,t)=v(0,t)-u(0,t)=0(1-0)(1-e^{-8t})-0=0, t \geq 0$ and $\tilde{v}(1,t)=(v-u)(1,t)=v(1,t)-u(1,t)=1(1-1)(1-e^{-8t})-0=0, t \geq 0$ and $\tilde{v} (x,0) = (v-u)(x,0) = v(x,0)-u(x,0)=x(1-x)(1-1)-0=0, \; 0 \leq x \leq 1.$ So we have $(\partial_t -\partial_x^2)\tilde{v} \leq 0, \text{ on } (0,1)\times (0,\infty), \tilde{v}(0,t) = 0 = \tilde{v} (1,t) \quad t \geq 0, \tilde{v}(x,0)=0 \quad 0 \leq x \leq 1.$ By T 5.11, $\tilde{v} \leq 0$ and therefore $v \leq u$.  Now, recall $w$ from part (a).  Set $\tilde{w}=u-w$. Then 
$(\partial_t-\partial_x^2)\tilde{w}=(\partial_t-\partial_x^2)u-(\partial_t-\partial_x^2)w = 2-2=0$.  Further, for $\partial \Omega = \{0,1\}, \tilde{w}(0,t)=u(0,t)-w(0,t)=0-0=0, t \geq 0$ and $\tilde{w} (1,t) =u(1,t)-w(1,t)=0-0=0, t \geq 0$ and $\tilde{w}(x,0)=u(x,0)-w(x,0)=0-x(1-x) \leq 0, \; 0 \leq x \leq 1.$ And so we have $(\partial_t - \partial_x^2)\tilde{w}=0$ on $(0,1)\times (0,\infty), \tilde{w}(0,t)=0=\tilde{w}(1,t), t \geq 0, \tilde{w}(x,0) \leq 0, 0 \leq x \leq 1$. So by T 5.11, we have that $\tilde{w} \leq 0$ and so $u \leq w$ and so $v \leq u \leq w, 0 \leq x \leq 1\qed$.
{\bf Exercise 5.2.11}.  Let $L,T >0$ and $u: [0,L] \times [0,T] \ra \R$ be continuous, and sufficiently often differentiable and satisfy $0 \leq \partial_t u(x,t)-x^3(L-x)^5\partial_x^2 u(x,t)+a\partial_x u(x,t) + (L-x)u(x,t), 0 <x<L, \quad 0<t<T, 0 \leq u(0,t), \quad u(L,t) \geq 0, \quad t \in [0,T], 0 \leq u(x,0), \quad 0 \leq x \leq L.$ Here $a \in \R$. Show:  $u(x,t) \geq 0$ for all $x \in [0,L], t \in [0,T]$. Do not use the maximum principle, but do the proof from scratch. 
{\it Proof}.  By contradiction assume there exists a $y \in [0,L]$ and an $r \in [0,T]$ such that $u(y,r) <0$. Consider $u: [0,L] \times [0,r]$. Since $u$ is compact $u$ has a minimum in $[0,L] \times [0,r]$.  Denote such a minimum as $u_m = u(x_m, t_m)$ with $x_m\in[0,L]$ and $t_m \in [0,r]$.  Therefore $u(x_m, t_m) \leq u(z,s)$ for any $z \in [0,L]$ and $s \in [0,r]$. Particularly, $u(x_m,t_m) \leq u(y,r) < 0$.  Since $u_m$ is a minimum, $ \partial_xu(x,t)|_{(x_m,t_m)} = 0, \quad \text{ and } \quad \partial_x^2u(x,t)|_{(x_m,t_m)} \geq 0.$ Now note that $\partial_t u(x,t)|_{(x_m,t_m)} = \lim_{s \ra t_m^-} (u(x_m, s)-u(x_m, t_m))/(s-t_m) \leq 0$ because the numerator is positive and the denominator is negative. Having these results, we can evaluate the PDE at the minimum point $(x_m, t_m)$. $\partial_t u(x_m,t_m)-x^3(L-x_m)^5\partial_x^2 u(x_m,t_m)+a\partial_x u(x_m,t_m) + (L-x_m)u(x_m,t_m).$ We have shown that the third term is zero so we are left with $\partial_t u(x_m,t_m)-x^3(L-x_m)^5\partial_x^2 u(x_m,t_m)+ (L-x_m)u(x_m,t_m),$ We have that $(L = x_m)>0$ and we have shown that the 1st and 3rd terms are negative and so $\partial_t u(x_m,t_m)-x^3(L-x_m)^5\partial_x^2 u(x_m,t_m)+a\partial_x u(x_m,t_m) + (L-x_m)u(x_m,t_m) <0$ which contradicts our PDE $\qed$. 
% Identities
{\bf Identities:} $\sin^2x = (1-\cos 2x)/2, \cos^2x = (1+\cos 2x)/2$, $e^{ix}=\cos x + i \sin x, e^{-ix}=\cos x - i \sin x$,  \newline$e^{ix}+e^{-ix}=2\cos x$,\newline $e^{ix}-e^{-ix}=2i\sin x$. \newline%$2 \cosh x=(e^x+e^{-x}), 2\sinh x=(e^x-e^{-x}). \cosh^2 x-\sinh^2x=1$. \newline
{\bf Sum and Difference Formula:} $\sin(A\pm B)=\sin A \cos B\pm \cos A \sin B$. $\cos(A\mp B)=\cos A \cos B\pm \sin A \sin B$. $\tan(A \pm B)=(\tan A\pm \tan B)/(1\mp \tan A \tan B)$. \newline
{\bf Double Angle Formula:}  $\sin(2A)=2 \sin A \cos A$. $\cos(2A)=\cos^2 A-\sin^2 A=2 \cos^2 A-1=1-2\sin^2 A$. $\tan(2A)=(2\tan A)/(1-\tan^2 A)$.  \newline
%{\bf Half Angle Formula:} $\sin(A/2)=\pm \sqrt{(1-\cos A)/2}$. $\cos(A/2)=\pm \sqrt{(1+\cos A)/2}$. \newline
%{\bf Product to Sum:} $\cos A \cos B=(1/2)(\cos(A+B)+\cos(A-B))$. $\sin A \sin B=(1/2)(\cos(A-B)-\cos(A+B)$. $\sin A \cos B=(1/2)(\sin(A+B)+\sin(A-B)$.  \newline
{\bf Sum to Product:} $\sin A\pm \sin B=2\sin((A\pm B)/2)\cos((A\mp B)/2)$. $\cos A - \cos B=-2\sin((A+B)/2)\sin((A-B)/2)$. $\cos A + \cos B=2\cos((A+B)/2)\cos((A-B)/2)$. \newline
{\bf Geometric Sum:} $\sum_{k=0}^N z^k=\frac{1-z^{N+1}}{1-z}$. $\sum_{k=0}^{\infty}z^k=\frac{1}{1-z}$.  \newline
%{\bf General ODE Solutions:}  $y''=y(t)\implies y=c_1e^{-t}+c_2e^t \qed \; dy/dt+p(t)y=g(t) \implies y=(\int u(t)g(t))/u(t) + c$ where $u(t)=$exp$(\int p(t)dt) \qed \; y'=x; x'=y \implies x=c_1 \cosh t + c_2 \sinh t, y=c_1\sinh t+c_2 \cosh t$ or $x=c_1e^t+c_2e^{-t}, y=c_1e^t-c_2e^{-t} \qed \; y'=-x; x'=y \implies y=c_1 \cos t + c_2 \sin t, x=c_1\sin t-c_2 \cos t \qed \; x'=x+y; y'=-x+y \implies x=e^t(c_1 \cos t+c_2 \sin t); y=e^t(-c_1\sin t +c_2 \cos t) \qed \; v'=\gamma v, v(z,0)=u_0 \implies v=u_0e^{\gamma t} \qed$
{\bf Integrals: }
\begin{align*}
&\int (a+bx)\cos(kx)dx=\frac{(a+bx)\sin(kx)}{k}+\frac{bcos(kx)}{k^2}+C\\
&\int (a+bx)\sin(kx)dx=\frac{b\sin(kx)}{k^2}-\frac{(a+bx)cos(kx)}{k}+C\\
&\int (a+bx)e^{ikx}dx=\frac{e^{ikx}(b-ik(a+bx))}{k^2}+C\\
\end{align*}

\end{multicols}
\end{document}