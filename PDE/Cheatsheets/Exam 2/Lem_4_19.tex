{\bf L 4.19}. Let $f: [0,L]\ra \R$ be Lipschitz continuous, $f(0)=0=f(L)$. Then $f(x)=\sum_{j=1}^{\infty}B_j\sin(\lambda_jx), B_j=(2/L)\int_0^Lf(y)\sin(\lambda_jy)dy, \lambda_j = (j \pi/L)$, with the convergence being uniform in $[0,L]$. {\it Proof}. Extend $f$ to $[-L,L]$ in an odd fashion by defining $f(-x)=-f(x)$ for $x \in (0,L]$. We first check whether this is also Lipschitz cont. Critical case is $-L \leq x <0 \leq y \leq L$. Since $f(0)=0, |f(y)-f(x)| \leq |f(y)| + |f(x)|=|f(y)-f(0)|+|f(0)-f(-x)|\leq \Lambda(y+(-x)) \leq \Lambda (y-x)$. Since $f(L)=0, f(-L)=-f(L)=0$ and $f(-L)=f(L)$. By T 4.18, $f(x)=A_0+\sum_{j=1}^{\infty}(A_j\cos(\lambda_jx) + B_j \sin(\lambda_jx)), \lambda_j=(j\pi/L)$, with the convergence being uniform in $x \in [-L,L]$ and $A_j=(1/L)\int_{-L}^L f(y)\cos(\lambda_jy)dy, B_j=(1/L)\int_{-L}^L f(y)\sin(\lambda_jy)dy$. Since cosine is even and sine is odd, for $j \in \N, A_j=(1/L)\int_0^L( f(y)+f(-y))\cos(\lambda_jy)dy=0$ and $B_j=(1/L)\int_0^L (f(y)-f(-y))\sin(\lambda_jy)dy=(2/L)\int_0^L f(y)\sin(\lambda_jy)dy, A_0=(1/2L)\int_{-L}^L f(y)dy=0 \qed$