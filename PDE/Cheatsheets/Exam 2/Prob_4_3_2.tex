{\bf Exercise 4.3.2}. Let $f: [a,b] \ra \R$ be continuous and assume that there exists a partition $a = t_0<\cdots <t_m=b$ such that $f$ is differentiable with bounded derivative on each interval $(t_{j-1}, t_j)$. Show $f$ is Lipschitz continuous. {\it Proof}. Let $x,y \in [z,b], x < y$. Modifying the partition of $[a,b]$, we can find a partition $x=r_0<\cdots <r_k=y$ such that $f$ is continuously differentiable on each $(r_{j-1},r_j)$ and $L_j:=\sup_{r_{j-1}<s<r_j}|f'(s)|<\infty$. More precisely $r_1, \dots, r_{k-1} \in \{t_1,\dots,t_{m-1}\}$. Let $j \in \{0,\dots, k\}$. and $r_{j-1}\leq s<t\leq r_j$. By the mean value theorem of calculus, $f(t)-f(s)=f'(r)(t-s)$ for some $r \in (s,t)$. So $|f(t)-f(s)| \leq L_j|t-s|$. Since $f$ is continuous, we can take the limit $s \ra r_{j-1}$ and $t \ra r_j$ and $|f(r_j)-f(r_{j-1})|\leq L_j(r_j-r_{j-1})$. We telescope, $|f(y)-f(x)|=|\sum_{j=1}^k[f(r_j)-f(r_{j-1})]| \leq \sum_{j=1}^k L_j(r_j-r_{j-1})$. Set $\Lambda = \max_{j=1}^k L_j$. Then $|f(y)-f(x)| \leq \Lambda \sum_{j=1}^k (r_j-r_{j-1})=\Lambda (y-x)$. Here we have telescoped again. 