{\bf Exercies 4.3.5}. Show that $\{v_j;j\in\N\}$ with $v_j(x)=\sin(jx)$ is an orthonormal basis of $L^2([0,\pi],\R)$ with the inner product $\langle f,g \rangle =(2/\pi)\int_0^{\pi}f(x)g(x)dx$. Conclude that, for $f \in C([0,\pi],\R), \int_0^{\pi}|f(x)-\sum_{j=1}^mB_j\sin(jx)dx|^2dx \ra 0, m\ra \infty, B_j=(2/\pi)\int_0^{\pi}f(x)\sin(jx)dx$. {\it Proof}. By Exercise 4.3.1, $B=\{\cos(jx);j\in\N\}\cup \{\sin(jx);j\in\N\}\cup \{1/\sqrt{2}\}$ is an orthonormal basis of $L^2([-\pi,\pi],\R)$ with inner product $\langle f,g \rangle = (1/\pi)\int_{-\pi}^{\pi}fg$. In particular $\tilde{B} = \{\sin(jx);j\in\N\}$ is an orthonormal subset of $L^2([-\pi,\pi],\R)$. So, for $j \neq k, 0 = \int_{-\pi}^{\pi}\sin(jx)\sin(kx)dx=2\int_0^{\pi}\sin(jx)\sin(kx)dx$, because $\sin(jx)\sin(kx)$ is an even function of $x$. By the same token, $1 = (1/\pi)\int_{-\pi}^{\pi}\sin^2(jx)dx= (2/\pi)\int_0^{\pi}\sin^2(jx)dx$. So $\tilde{B}$ is an orthonormal subset of $L^2([0,\pi], \R)$. To show that it is an orthonormal basis, we use Exercise 4.2.3:  We let $f \in L^2([0,\pi], \R)$ with $\int_0^{\pi}f(x)\sin(jx)dx=0$ for all $j \in \N$ and show that $f=0$. Extend $f$ to an odd function on $[-\pi,\pi]$ by setting $f(-x)=-f(x)$ for $x \in (0,\pi]$. Then, for all $j \in \Z, \int_{-\pi}^{\pi}f(x)\cos(jx)dx=0$, because $f(x)\cos(jx)$ is an odd function of $x$. For all $j \in \N, \int_{-\pi}^{\pi}f(x)\sin(jx)dx=2\int_0^{\pi}f(x)\sin(jx)dx=0$, because $f(x)\sin(jx)$ is an even function of $x$. Since $B$ is an orthonormal basis of $L^2([-\pi,\pi],\R), f=0$ by Exercise 4.2.3 (a). So $\tilde{B}$ is an orthonormal basis by Exercise 4.2.3 (b).  Alternatively, one can use the density of $C[0,\pi]$ in $L^2[0,\pi]$ with Lemma 4.19 to show that the sine functions form a Schauder basis in $L^2[0,\pi] \qed$