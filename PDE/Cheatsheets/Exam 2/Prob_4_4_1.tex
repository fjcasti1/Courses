{\bf Exercise 4.4.1}. Consider the wave equation (PDE) $(\partial_t^2-\partial_x^2)u=0, 0 \leq x \leq \pi, t \in \R$, (IC) $u(x,0)=f(x), \partial_t u(x,0)=g(x), 0\leq x\leq \pi$, (BC) $\partial_xu(0,t)=0=\partial_xu(\pi,t), t \in \R$. (4.17) with $f$ and $g$ in $C[0,\pi]$. (a) Develop the appropriate notions of classical and generalized solutions.  Show that every classical solution is a generalized solution. (b) Show that a generalized solution is uniquely determined. (c) Check whether the d'Alembert formula provides a generalized solution. {\it Proof}. (a) We assume that the initial data $f$ and $g$ of (4.17) are merely elements in $C[0,\pi]$. {\bf 4.24 Definition}. A function $u: [0,L] \times \R \ra \R$ is called a classical solution of (4.17) if $u$ is twice continuously differentiable on $[0,L] \times \R$ and satisfies (4.17). Let $\phi: [0,L] \ra \R$ be twice continuously differentiable, $\phi'(0)=0=\phi'(L)$. Further let $u$ be a classical solution of (4.17). Since $u$ is twice continuously differentiable, the following derivative exists and satisfies the equation $(d^2/dt^2)\int_0^L \phi(x)u(x,t)dx=\int_0^L\phi(x)\partial_t^2u(x,t)dx=\int_0^L\phi(x)c^2\partial_x^2u(x,t)dx$. We integrate by parts. Since both $\phi'$ and $\partial_xu$ are 0 for $x=0,L, (d^2/dt^2)\int_0^L \phi(x)u(x,t)dx=-c^2\int_0^L\phi'(x)\partial_xu(x,t)dx=c^2\int_0^L\phi''(x)u(x,t)dx$. Similarly we find $(d/dt_{[t=0]})\int_0^L\phi(x)u(x,t)dx=\int_0^L\phi(x)g(x)dx$. These findings motivate the following definition and prove that following result.