{\bf Exercise 5.2.11}.  Let $L,T >0$ and $u: [0,L] \times [0,T] \ra \R$ be continuous, and sufficiently often differentiable and satisfy $0 \leq \partial_t u(x,t)-x^3(L-x)^5\partial_x^2 u(x,t)+a\partial_x u(x,t) + (L-x)u(x,t), 0 <x<L, \quad 0<t<T, 0 \leq u(0,t), \quad u(L,t) \geq 0, \quad t \in [0,T], 0 \leq u(x,0), \quad 0 \leq x \leq L.$ Here $a \in \R$. Show:  $u(x,t) \geq 0$ for all $x \in [0,L], t \in [0,T]$. Do not use the maximum principle, but do the proof from scratch. 
{\it Proof}.  By contradiction assume there exists a $y \in [0,L]$ and an $r \in [0,T]$ such that $u(y,r) <0$. Consider $u: [0,L] \times [0,r]$. Since $u$ is compact $u$ has a minimum in $[0,L] \times [0,r]$.  Denote such a minimum as $u_m = u(x_m, t_m)$ with $x_m\in[0,L]$ and $t_m \in [0,r]$.  Therefore $u(x_m, t_m) \leq u(z,s)$ for any $z \in [0,L]$ and $s \in [0,r]$. Particularly, $u(x_m,t_m) \leq u(y,r) < 0$.  Since $u_m$ is a minimum, $ \partial_xu(x,t)|_{(x_m,t_m)} = 0, \quad \text{ and } \quad \partial_x^2u(x,t)|_{(x_m,t_m)} \geq 0.$ Now note that $\partial_t u(x,t)|_{(x_m,t_m)} = \lim_{s \ra t_m^-} (u(x_m, s)-u(x_m, t_m))/(s-t_m) \leq 0$ because the numerator is positive and the denominator is negative. Having these results, we can evaluate the PDE at the minimum point $(x_m, t_m)$. $\partial_t u(x_m,t_m)-x^3(L-x_m)^5\partial_x^2 u(x_m,t_m)+a\partial_x u(x_m,t_m) + (L-x_m)u(x_m,t_m).$ We have shown that the third term is zero so we are left with $\partial_t u(x_m,t_m)-x^3(L-x_m)^5\partial_x^2 u(x_m,t_m)+ (L-x_m)u(x_m,t_m),$ We have that $(L = x_m)>0$ and we have shown that the 1st and 3rd terms are negative and so $\partial_t u(x_m,t_m)-x^3(L-x_m)^5\partial_x^2 u(x_m,t_m)+a\partial_x u(x_m,t_m) + (L-x_m)u(x_m,t_m) <0$ which contradicts our PDE $\qed$. 