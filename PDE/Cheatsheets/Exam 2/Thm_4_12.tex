{\bf T 4.12}. The set $B=\{v_j;j\in \Z\}$ with $v_j(x)=e^{ijx}$ is an orthonormal basis for the following IP spaces with the IP $\langle f,g \rangle = 1/(2 \pi) \int_{-\pi}^{\pi} f(x)\overline{g(x)}dx: X=C([-\pi, \pi],\C), X = L^2([-\pi, \pi],\C)$ and the space of Riemann integrable functions on $[-\pi, \pi]$ with values in $\C$. Further the Fourier series $\sum_{j=-\infty}^{\infty}\hat{f}_je^{ijx}, \hat{f}_j=1/(2\pi)\int_{-\pi}^{\pi}f(x)e^{-ijx}dx$, converges to $f$ in the sense that for every $\epsilon >0 \exists$ some $N \in \N$ with $\int_{-\pi}^{\pi}|f(x)-\sum_{j=-k}^m\hat{f}_je^{ijx}|^2dx<\epsilon^2$ (4.4) for all $k,m \in \N$ with $k,m>N$. 
%{\it Proof}. Let $j \neq k$. Then $\int_{-\pi}^{\pi}e^{ijx}e^{-ikx}dx=\int_{-\pi}^{\pi}e^{i(j-k)x}dx=1/((j-k)i)(e^{i(j-k)\pi}-e^{-i(j-k)\pi})=1/((j-k)i)(\cos(j-k)\pi)-\cos(-(j-k)\pi)+2i\sin((j-k)\pi)=0$, and $\int_{-\pi}^{\pi}e^{ijx}e^{-ijx}dx=2 \pi$. This shows that $B$ is an orthonormal set. Let $g: [-\pi, \pi] \ra \C$ be continuous and $g(-\pi)=g(\pi)$. Let $\mathbb{T}=\{z \in \C; |z|=1\}$. Since every $z \in \mathbb{T}$ can be uniquely written as $z = e^{ix}$ with some $x \in (-\pi,\pi]$, the definition $f(z)=g(e^{-ix})$ is sound and provides a function $f: \mathbb{T} \ra \C$. Since $g$ is continuous and $g(-\pi)=g(\pi)$, $f$ is continuous. By the Stone-Weierstra$\ss$ approximation theorem, there exists a sequence $(p_m)$ of polynomials $p_m(z)=\sum_{j=-m}^m c_jz^j, z \in \mathbb{T}$, such that $p_m(z) \ra f(z)$ as $m \ra \infty$ uniformly for $z \in \mathbb{T}$. Hence $q_m(x) := \sum_{j=-m}^m c_je^{ijx} \ra g(x), m \ra \infty$, uniformly for $x \in [-\pi,\pi]$. (The latter is the Weierstra$\ss$' Trigonometric Approximation Theorem). Since $C[-\pi,\pi]$ is dense in $L^2[-\pi,\pi]$, for every $h \in L^2[-\pi,\pi]$ there exists a sequence $(g_n)$ in $C[-\pi,\pi]$ such that $||h-g_n||\ra 0$. The $g_n$ can be modified to satisfy $g_n(-\pi)=g_n(\pi)$ and still $||h-g_n||\ra 0$. This implies that $h$ can be approximated by trigonometric polynomials $q_m$ in $L^2[-\pi,\pi]$. This shows that $B$ is a Schauder basis. Now let $\epsilon >0$. By Theorem 4.8, there exists some finite subset $F$ of $\Z$ such that $||f-\sum_{j\in G}\hat{f}_jv_j||<\epsilon/\sqrt{2 \pi}$ for all finite sets $G$ with $F \subseteq G \subseteq \Z$. Since $F$ is finite, we can choose $N\in \N$ such that $F \subseteq \{n \in \Z;|n|\leq N\}$. Now let $k,m>N$. Let $G=\{n\in \Z;-k\leq n \leq m\}$. Then $F \subseteq G \subseteq \Z$ and so $\epsilon/\sqrt{2 \pi}>||f-\sum_{j\in G}\hat{f}_j v_j|| = ||f-\sum_{j=-k}^m\hat{f}_jv_j||$. We take squares and obtain (4.4) $\qed$ 