{\bf T 5.6}. Let $f: [0,L] \ra \R$ be Lipschitz cont, $f(L)=0=f(0)$. Then $u$ is cont on $[0,L] \times [0,\infty)$ and $u(x,0)=f(x)$ for all $x \in [0,L]$. In particular, $u(x,t) \ra f(x)$ as $t \ra 0$ uniformly in $x \in [0,L]$.  The Fourier sine series of $u$ converges to $u$ uniformly on $[0,L] \times [0,\infty)$. 
%{\it Proof}. The functions $u_m$ in (5.5) satisfy the estimate $|u_m(x,t)|\leq |B_m(0)|, x \in [0,L], t \geq 0, m \in \N$, with $B_m(0)$ being given by (5.3). Since $f$ is Lipschitz continuous and $f(0)=0=f(L), \sum_{m=1}^{\infty}|B_m(0)| < \infty$ by exercise 4.3.3 and odd extension.  By Theorem 5.1, $u =  \sum_{m=1}^{\infty} u_m$ converges uniformly and is continuous on $[0,L] \times [0,\infty)$. In particular, $u(x,0)=f(x)$ by Lemma 4.19, and $u$ is uniformly continuous on the closed bounded set $[0,L] \times [0,1]$. Let $\epsilon >0$. Then there exists some $\delta >0$ such that $|u(x,t)-u(y,0)| < \epsilon$ whenever $|x-y|+|t-0| < \delta$. In particular, $|u(x,t)-u(x,0)| < \epsilon$ whenever $0\leq t  < \delta \qed$  If $f(0) \neq 0$, then $u(0,t)$ does not converge to $f(0)$ as $t \ra 0$ because, $u(0,t)=0$ for all $t >0$. The analogous statement holds for $L$. We will see later that, for uniform convergence to the initial data, it is sufficient that $f$ is continuous rather than Lipschitz continuous. Without continuity assumptions, we still have convergence in the $L^2$-sense.  