{\bf T 5.7}. Assume that $f: [0,L] \ra \R$ is integrable and $\int_0^L|f(x)|^2dx<\infty$. Then the series $u$ in (5.2) with (5.4) and (5.3) satisfies $\int _0^L |u(x,t) - f(x)|^2dx \ra 0, t \ra 0$. Actually, $u(\cdot, t)$ is a uniformly cont func of $t \in \R_+$ with values in $L^2[0,L]$. Notice that $\int_0^L|f(x)|^2dx<\infty$ is a stronger assumption than $\int_0^L|f(x)|dx<\infty$ in T 5.4. 
%{\it Proof}. Let $\langle \phi, \psi \rangle = (2/L)\int_0^L \phi(x) \psi(x)dx$ be the inner product of choice on $L^2([0,L],\R)$, the space of square integrable functions.  Then $\{ v_j; j \in \N\}$ with $v_j(x)=\sin(\lambda_j x)$ is an orthonormal basis.  By the considerations at the beginning of this section, $\langle u(\cdot, t), v_m \rangle = \langle f, v_m \rangle e^{-a \lambda_m^2 t}$ (5.6), which are uniformly continuous functions on $\R_+$. Further $|\langle u(\cdot, t), v_m \rangle | \leq | \langle f, v_m \rangle |$ and, by Parseval's relation $\sum_{m \in \N}|\langle f, v_m \rangle|^2 = ||f||^2$. The assertion now follows from Theorem 4.11. $\qed$