\documentclass[a4paper,landscape]{article}
\usepackage{amsmath, amssymb}
\usepackage{multirow}
\usepackage{mathrsfs}
\usepackage{graphicx}
\usepackage{amsthm}
\usepackage{multicol}

%\usepackage[font={small}, margin=1cm]{caption}
\usepackage[margin=.1in]{geometry}
\setlength{\parindent}{0pt}
\renewcommand{\arraystretch}{0.8}
\newcommand{\C}{\mathbb{C}}
\newcommand{\F}{\mathcal{F}}
\newcommand{\K}{\mathbb{K}}
\newcommand{\N}{\mathbb{N}}
\newcommand{\Q}{\mathbb{Q}}
\newcommand{\R}{\mathbb{R}}
\newcommand{\Z}{\mathbb{Z}}
\newcommand{\ra}{\rightarrow}

\begin{document}
\fontsize{7.5}{6}
\selectfont
\begin{multicols}{3}

%\fontsize{12}{14}
\selectfont

% The Green's function
{\bf Green's function (GF)} HE w/ 0 BCs, $(\partial_t - a \partial_x^2)u = 0$, on $[0,L] \times (0, \infty), u(x,0)= f(x), 0 \leq x \leq L, u(0,t) = 0 = u(L, t), t > 0$ (5.25), is given by the Fourier sine series $u(x,t) = \sum_{m=1}^{\infty} B_m e^{-a \lambda_m^2 t} \sin(\lambda_m x), \lambda_m = m \pi / L = m \lambda_1, B_m = 2/L \int_0^L f(y) \sin(\lambda_m y) dy$ (5.26). Sub 3rd into  1st eq, $u(x,t) = \sum_{m=1}^{\infty} (2/L \int_0^L f(y) \sin(\lambda_m y) dy) e^{-a \lambda_m^2 t} \sin(\lambda_m x)$. Interchange series and int, $u(x,t) = \int_0^L G(x,y,t)f(y) dy, G(x,y,t)=\sum_{m=1}^{\infty} G_m(x,y,t), t \in (0, \infty), x,y, \in [0, L]$, $G_m(x,y,t) = (2/L) \sin(\lambda_m x) \sin(\lambda_m y)e^{-a \lambda_m^2 t}$ (5.27). Notice $|\partial_x^j \partial_y^k \partial_t^{\ell} G_m(x,y,t)| \leq (2/L) a^{\ell} \lambda_m^{j+k+2\ell} e^{-a \lambda_m^2 t}$ (5.28). From ratio test, for $t > 0, \sum_{m=1}^{\infty}(2/L) a^{\ell} \lambda_m^{j+k+2\ell} e^{-a \lambda_m^2 t} < \infty$.
By T 5.2, $G$ is inf often diff on $[0, L]^2 \times (0, \infty)$ and $\partial_x^j \partial_y^k \partial_t^{\ell} G(x,y,t) = \sum_{m=1}^{\infty} \partial_x^j \partial_y^k \partial_t^{\ell} G_m(x,y,t)$ (5.29). This justifies (5.27). Also $|\partial_x^j \partial_y^k \partial_t^{\ell} G(x,y,t)| \leq \sum_{m=1}^{\infty} (2/L) a^{\ell}$
$\lambda_m^{j+k+2\ell} e^{-a \lambda_m^2 t} < \infty$ (5.30). $G$ satisfies HE because each $G_m$ does, $\partial_t G(x,y,t) = a \partial_x^2 G(x,y,t) = a \partial_y^2 G(x,y,t), 0 = G(0,y,t) = G(L,y,t)=G(x,0,t) = G(x,L,t), 0 \leq x, y \leq L, t > 0$ (5.31). From (5.27), $G(x,y,t)=G(y,x,t), 0 \leq x, y \leq L, t > 0$ (5.32). 
% 5.17 Proposition
{\bf P5.17}. $G(x,y,t) \geq 0 \; \forall \; x, y \in [0,L], t>0$. {\it Prf}. Suppose $G(x,\tilde{y}, t)<0$ for some $x, \tilde{y} \in [0,L], t>0$. Then  $x, \tilde{y} \in (0,L)$. Since $G$ is cont, $\exists \; \delta > 0$ s.t. $(\tilde{y} - \delta, \tilde{y} + \delta) \subseteq (0,L), G(x,y,t) \leq - \delta, |y - \tilde{y}| < \delta$. Set $f(y)=[1-|\tilde{y} - y|/\delta]_+, y \in [0,L]$, where $[r]_+ = \max\{r,0\}$ is the pos part of $r$. Then $f$ is Lip cont and non-neg, $f(y)=0$ whenever $y \notin (\tilde{y} - \delta, \tilde{y} + \delta)$. Let $u$ be sol of HE with ID $f$. By T 5.16, $u$ is cont on $[0,L] \times [0, \infty)$. By T 5.11, applied to $-u, 0 \leq u(x,t)=\int_0^L G(x,y,t)f(y) dy= \int_{\tilde{y} - \delta}^{\tilde{y} + \delta} G(x,y,t)f(y) dy \leq  - \delta \int_{\tilde{y} - \delta}^{\tilde{y} + \delta} f(y) dy \leq - \delta \int_{\tilde{y} - \delta}^{\tilde{y} + \delta} [1-| y - \tilde{y}|/\delta] dy = -\delta^2\int_{-1}^1 (1-|y|)dy = - \delta^2$, a contradiction $\qed$

% 5.18 Proposition
{\bf P5.18}. (a) If $f: [0,L] \ra \R$ is cont and $f(0) = 0 = f(L)$, then $\int_0^L G(x,y,t) f(y)dy \ra f(x), t \ra 0$, unif in $x \in [0,L]$. (b) $\forall  \; x \in [0,L], t>0, \int_0^L G(x,y,t) dy \leq 1$, and, if $0<a<b<L,  \int_0^L G(x,y,t)dy \ra 1, t \ra 0$, unif in $x \in [a,b]$. {\it Prf}. (a) This follows from T 5.16 and (5.27). (b) For $n \geq 3L$, define $f_n(x)=nx, 0 \leq x \leq 1/n, f_n(x)=1, 1/n \leq x \leq L-1/n, f_n(x)=n(L-x), L-1/n \leq x \leq L$. Then $f_n$ is Lip cont by E 4.3.2, $0 \leq f_n(x) \leq 1 \; \forall x \in [0,L]$, and $f_n(0)=0=f_n(L)$. Let $u_n$ be sol of HE with ID $f_n$ and 0 BCs. By T 5.11, applied to $-u_n$ and $u_n-1$, we have $0 \leq u_n \leq 1$. By (5.27), $1 \geq u_n(x,t) = \int_0^L G(x,y,t)f_n(y)dy \geq \int_{1/n}^{L-1/n} G(x,y,t) dy$. For fixed $t>0, G(x,y,t) \leq c_t \; \forall \; x,y \in [0,L]$ with some const $c_t>0$, we have $1 \geq \int_0^L G(x,y,t) dy - c_t (2/n)$. Take the limit as $n \ra \infty$ and obtain the 1st inequality.  For the 2nd statement, choose $n \in \N$ so large that $(1/n) < a$ and $L - (1/n) > b$. Then $f_n(x)=1 \; \forall \; x \in [a,b]$. Since $u_n$ is unif cont on $[0,L] \times [0,1], 1 \geq \int_0^L G(x,y,t)dy \geq \int_0^L G(x,y,t) f_n(y) dy \ra_{t \ra 0} f_n(x) = 1$, unif in $x \in [a,b] \qed$
% 5.19 thm
{\bf T5.19}. Let $f: (0,L) \ra \R$ be cont and bdd, $0 < a < b < L$. Then $\int_0^L G(x,y,t)f(y)dy \ra f(x), t \ra 0$, unif in $x \in [a,b]$. {\it Prf}. Consider the same fctns $f_n$ as previous proof and choose $n$ so large that $(1/n) < a$ and $L-(1/n) > b$. Then $f_n(x)=1 \; \forall \; x \in [a,b]$. Then $f f_n$ has a cont extension $g_n$ on $[0,L], g_n(0) = 0 = g_n(L)$. Let $x\in (0,L)$. By the TI, $|\int_0^L G(x,y,t)f(y)dy - f(x)|=|\int_0^L G(x,y,t)f(y)(1-f_n(y))dy +\int_0^L G(x,y,t)f(y)f_n(y)dy - f(x)f_n(x)-f(x)(1-f_n(x))|\leq |\int_0^L G(x,y,t)f(y)(1-f_n(y)dy| +|\int_0^L G(x,y,t)g_n(y)dy - g_n(x)|+|f(x)(1-f_n(x))|$. We show that each of the last 3 exps tends to 0 as $ t\ra 0$ unif in $x \in [a,b]$. For the 2nd exp this follows from P 5.18.  The last equals 0 for $x \in [a,b]$ by our choice of $n$. As for the 1st, since $f$ is bdd on $(0,L)$, choose $M > 0$ s.t. $|f(x)| \leq M \; \forall \; x \in (0,L)$. Then, for $x \in [a,b]$, since $G \geq 0, |\int_0^L G(x,y,t)f(y)(1-f_n(y)dy| \leq \int_0^L G(x,y,t)|f(y)(1-f_n(y))|dy \leq M \int_0^L G(x,y,t)(1-f_n(y)dy = M(\int_0^L G(x,y,t)dy- \int_0^L G(x,y,t)f_n(y)dy) \ra_{t \ra 0} M(1-f_n(x))=0$, unif in $x \in [a,b]$, by P 5.18$\qed$
% 5.20 Corollary
{\bf C5.20}. If $f: (0,L) \ra \R$ is cont and bdd, then $u(x,t)= \int_0^L G(x,y,t)f(y) dy, t>0, 0 \leq x \leq L, u(x,t)=f(x), t=0, 0<x<L$, defines a sol of the HE in the following sense:  $u$ is defined and cont on $[0,L] \times [0, \infty)$ except at $(0,0)$ and $(L,0), (\partial_t - a\partial_x^2)u = 0$ on $[0,L] \times (0,\infty), u(x,0)=f(x), x \in (0,L), u(0,t)=0=u(L,t), t >0$. {\it Prf}. Everything has already been proved except that $u$ is cont at points $(x,0)$ with $0 < x<L$. Let $\epsilon > 0$ and $0<x<L$. Choose $\delta_1>0$ s.t. $|x - \delta_1, x + \delta_1| \subseteq (0,L)$ and $|f(x)-f(y)|<\epsilon/2$ whenever $|y-x|< \delta_1$. By T 5.19, $u(y,t) \ra f(y)$ as $t\ra 0$, unif in $y \in |x - \delta_1, x + \delta_1|$. So $\exists \;\delta_2 >0$ s.t. $|u(y,t)-f(y)| < \epsilon/2, t \in [0, \delta_2], |y-x| < \delta_1$. Set $\delta = \min \{\delta_1, \delta_2\}$. Then, if $|y-x| + |t| < \delta, |u(y,t)-u(x,0)| \leq |u(y,t)- f(y)|+|f(y)-f(x)|< \epsilon/2+\epsilon/2=\epsilon \qed$
%Exercises
{\bf E 5.3.1}. Show that there is at most one GF. More precisely: $\exists$ at most one cont fctn $G: [0,L]^2 \times [0, \infty) \ra \R$ s.t. $u(x,t)=\int_0^L G(x,y,t) f(y) dy$ solves the HE with initial values $f$ and zero BCs $ \forall $ Lipschitz cont $f: [0,L] \ra \R$ with $f(0) = 0 = f(L)$. Hint:  Assume that there are 2 and show that they are equal.  This proof is similar to the one that shows that the GF is non-neg. {\it Proof}. Assume that there are 2 GFs $G_1$ and $G_2$. Then $u_j(x,t)=\int_0^L G_j(x,y,t)f(y)dy$ solve the HE with initial values $f$ and zero BCs.  By C 5.12, $u_1=u_2$. So $\int_0^L G_1(x,y,t)f(y)dy=\int_0^L G_2(x,y,t)f(y)dy\; \forall$ Lipschitz cont fctns $f: [0, L] \ra \R$ with $f(0)=0=f(L), t \geq 0, x \in [0,L]$. Suppose that $G_1(x,y,t) \neq G_2(x,y,t)$ for some $x,y \in [0,L], t>0$. Without LOG we assume that $G_1(x,y,t) > G_2(x,y,t)$. Since $G_1$ and $G_2$ are cont,  $\exists \; \delta >0$ s.t. $G_1(x,z,t) \geq G_2(x,z,t)+ \delta, z \in (y-2\delta, y+2\delta)\cap[0,L]$.  We can choose $\delta >0$ so small that $y+2\delta <L$ or $y-2\delta >0$. We assume that $y+ 2 \delta <L$. The other case is similar. We define $f(z)=0; 0 \leq z \leq y, f(z) = z-y; y\leq z \leq y+\delta, f(z) = 2\delta +y-z; y+\delta \leq z \leq y+2 \delta, f(z)=0; y+2\delta \leq z \leq L$. Then $f$ is Lipschitz cont, $f(0)=0=f(L)$, and $0=\int_0^L (G_1(x,z,t)-G_2(x,z,t))f(z)dz = \int_y^{y+2\delta}(G_1(x,z,t)-G_2(x,z,t))f(z)dz \geq \delta \int_y^{y+2\delta}f(z)dz= \delta ( \int_y^{y+\delta}(z-y)dz+\int_{y+\delta}^{y+2\delta}(2 \delta + y - z)dz=2\delta\int_0^{\delta}zdz=\delta^3>0$, a contradiction$\qed$. 
{\bf E 5.3.2}. Let $G$ be the GF. Show: $G(x,y,t+r)=\int_0^L G(x,z,t)G(z,y,r) dz, 0 \leq x,y\leq L, t,r \geq 0$. There are several ways to prove this.  One way is to use the Fourier sine representation formula and orthonormality.  Another is to fix $r$ and, for fixed by arbitrary cont $f: [0,L] \ra \R$ with $f(0)=0=f(L)$ consider $u(x,t)=\int_0^L G(x,y,t+r)f(y)dy$ and $v(x,t)=\int_0^L G(x,z,t)g(z)dz$ with $g(x)=\int_0^LG(x,y,r)f(y)dy$. {\it Proof}. Recall that $G(x,y,t) = \sum_{m=1}^{\infty} (2/L) \sin(\lambda_mx) \sin(\lambda_m y) e^{-a\lambda_m^2 t}$. Then $\int_0^L G(x,z,t) G(z,y,r) dz = (2/L) \sum_{m=1}^{\infty}\sin(\lambda_mx)  e^{-a \lambda_m^2 t}\sum_{m=1}^{\infty} [(2/L) \int_0^L \sin(\lambda_mz) \sin(\lambda_k z) dz] \sin(\lambda_ky)  e^{-a\lambda_k^2 r})$. Because of orthonormality, the integrals in the brackets are 0 if $k \neq m$ and 1 if $k=m$. Then $\int_0^L G(x,z,t)G(z,y,r) dz$ = $(2/L) \sum_{m=1}^{\infty} \sin(\lambda_mx)  e^{-a\lambda_m^2 t} \sin(\lambda_my)  e^{-a\lambda_m^2 r}= (2/L) \sum_{m=1}^{\infty} \sin(\lambda_m x) \sin(\lambda_m y)  e^{-a\lambda_m^2 (t+r)}=G(x,y,t+r) \qed$
{\bf E 5.3.3} Let $f$ be int and $\int_0^L |f(y)| dy < \infty$. Let $u$ be the soln of HE with ID $f$ and 0 BCs. Show:  $\int_0^L |u(t,x) - f(x)| dx \ra 0, t \ra 0$. Hint:  Use the GF and (without prf) that for any $\epsilon > 0 \; \exists$ a cont fctn $g: [0, L] \ra\R$ s.t. $\int_0^L| f(y) - g(y) | dy < \epsilon$. {\it Prf}. Let $\epsilon > 0$.  Then $\exists$ a cont fctn $g: [0, L] \ra \R$ s.t. $\int_0^L |f(y) - g(y)| dy < \epsilon/8$. For $n \geq 3L$, define $f_n(x) = nx, 0 \leq x \leq 1/n; f_n(x) = 1, 1/n \leq x \leq L - 1/n; f_n(x) = n(L-x), L-1/n \leq x \leq L$.  Then $f_n$ is Lip cont by E4.3.2, $0 \leq f_n(x) \leq 1 \; \forall \; x \in [0, L]$, and $f_n(0) = 0 = f_n(L)$. Set $g_n = g f_n$. Then $\int_0^L |g - g_n| = \int_0^L |g| (1-f_n) = \int_0^{1/n}|g| + \int_{L-(1/n)}^L |g| \leq (2/n) \sup_{[0, L]} |g|$. By choosing n large enough, we can achieve that $\int _0^L |g - g_n| < \epsilon/8$. By the TI, $\int_0^L |f-g_n| < \epsilon/4$. The fctn $g_n$ is cont and $g_n(0) = 0 = g_n(L)$. Recall $u(x,t) = \int_0^L G(x,y,t) f(y) dy$. Set $v_n(x,t) = \int_0^L G(x,y,t) g_n(y)dy$. By T5.19, $\exists \; \delta >0$ s.t. $|v_n(x,t) - g_n(x) | \epsilon/(8L), 0 \leq t < \delta, 0 \leq x \leq L$. By TI $ \int_0^L |u(x,t) - f(x)|dx \leq  \int_0^L |u(x,t) - v_n(x,t)|dx + \int_0^L |v_n(x,t) - g_n(x)|dx  + \int_0^L |g_n(x) - f(x)|dx$.  We change the order of int and use that $G$ is symmetric in $x$ and $y$ and so $\int_0^L G(x,y,t) dx \leq 1 \; \forall \; t > 0, y \in [0, L]$ (P5.18(b)), $\int_0^L |u(x,t) - v_n(x,t)|dx = \int_0^L | \int_0^L G(x,y,t)(f(y) - g_n(y))dy|dx \leq  \int_0^L ( \int_0^L G(x,y,t)|f(y) - g_n(y)|dy)dx =  \int_0^L ( \int_0^L G(x,y,t) dx)|f(y) - g_n(y)|dy \leq  \int_0^L |f(y) - g_n(y)|dy$. We subst these inequalities into each other:  $\forall \; t \in (0, \delta), \int_0^L |u(x,t) - f(x)|dx \leq 2 \int_0^L|g_n - f| + \int_0^L \epsilon/(8L) < \epsilon/2 + \epsilon/8 < \epsilon \qed$
{\bf E 5.3.4}. Let $f: [0, L] \ra \R$ be twice cont diff, $f(0) = 0 = f(L)$. Let $G$ be the GF for the HE. Set $u(x,t) = \int_0^L G(x,y,t) f(y) dy, x \in [0,L], t \in (0, \infty), u(x,0) = f(x), x \in [0, L]$. (a) Show that $u$ has cont PDs $\partial_t u(x,t)$ on $[0, L] \times (0, \infty)$ and on $(0, L) \times [0, \infty)$ and $\partial_t u(x,t) = \int_0^L G(x,y,t) a f''(y) dy, x \in [0,L], t \in (0, \infty), \partial_t u(x,0) = af''(x), x \in (0, L)$ (5.33). Hint: First  prove the 1st statement in (5.33) interchanging diff and int. (b) Show that $\partial_t u$ satisfies the HE with ID $af'', \; (\partial_t - a \partial_x^2 )\partial_t u = 0$ on $[0, L] \times (0, \infty), \partial_t u(x,0) = a f''(x), x \in (0, L), \partial_t u(0,t) = 0 = \partial_t u(L,t), t \in (0, \infty)$. {\it Prf}. (a) Let $t \in (0, \infty)$ and $x \in [0, L]$. By (5.28), one can interchange diff and int, $\partial_t u(x,t) = \int_0^L \partial_t G(x,y,t) f(y) dy$. By (5.31) $\partial_t u(x,t) = \int_0^L a\partial_x^2 G(x,y,t) f(y) dy$. We IBP twice and use that $f(0) = 0 = f(L)$ and $G(x,0,t) = 0 = G(x,L,t), \partial_t u(x,t) = \int_0^L G(x,y,t) a f''(y) dy, t \in (0, \infty), x \in [0,L]$. Define $v: [0, L] \times [0, \infty)$ by $v(x,t) =\int_0^L G(x,y,t) a f''(y) dy, x \in [0,L], t \in (0, \infty), v(x,0) = a f''(x), x \in [0,L]$ (5.34). By C5.20, $v$ is cont on $[0, L] \times [0, \infty)$ except at $(0,0)$ and $(L,0)$. Let $x \in (0,L)$ and $t > 0$. Notice that $f$ is Lip cont and $u$ is cont on $[0,L] \times [0, \infty)$ by T5.6. By the mean-value thm, $(u(x,t)-u(x,0))/(t-0) = \partial_2 u(x,s) = v(x,s)$ for some $x \in (0,t)$. Let $\epsilon >0$. Since $v(x,\cdot)$ is cont on $[0, \infty)$ and $v(x,0) = a f''(x)\; \exists \;\delta >0$ s.t. $|v(x,r) - a f''(x)| < \epsilon \; \forall r \in [0, \delta)$. Let $t \in [0, \delta)$. Then $s \in [0, \delta]$ and $|(u(x,t)-u(x,0))/(t-0) - a f''(x)| = |v(x,s)-a f''(x)| < \epsilon$. Hence $a f''(x) = \lim_{t \ra 0} (u(x,t)-u(x,0))/(t-0) = \partial_t u(x,0)$. (b) This follows from C5.20 $\qed$
{\bf E 5.3.5}. Find a fctn like the GF for the equation $(\partial_t - a \partial_x^2)u = 0, 0 \leq x \leq L, t >0, \partial_x u(0,t)=0=\partial_x u(L,t), t>0, u(x,0) = f(x), 0 \leq x \leq L$. {\it Prf}. By E 5.1.1, the solution is given by $u(x,t) = \sum_{m=0}^{\infty}A_m \cos(\lambda_mx)e^{-a \lambda_m^2t}$, with $\lambda_m=m\pi/L, A_m=(2/L)\int_0^Lf(y)\cos(\lambda_my)dy, m \in \N, A_0=(1/L)\int_0^L f(y)dy$. Then $u(x,t)=\int_0^L N(x,y,t)f(y)dy$ with $N(x,y,t)=(1/L)+\sum_{m=1}^{\infty}(2/L)\cos(\lambda_mx)\cos(\lambda_my)e^{-a\lambda_m^2t}\qed$
% The inhomogeneous heat equation
{\bf Inhomog HE} with 0 BCs (Dirichlet BCs) has the form (PDE) $(\partial_t- a \partial_x^2)u = F(x,t), 0 \leq x \leq L, t \in (0,T)$, (IC) $u(x,0) = f(x), 0 <x<L$, (BC) $u(0,t)=0=u(L,t), 0 < t \leq T$ (5.35). Again $L, a > 0$. Here $F: [0,L] \times (0,T) \ra \R$ is cont and bdd and $f: (0,L)\ra \R$ is cont and bdd. Assume $u$ and $\partial_t u$ are cont on $[0,L] \times (0,T)$. Let $G$ be the GF from the previous Section and $0<s<t<T$. Because of (5.30), we can interchange int and diff in the following, $\partial_s \int_0^L G(x,y,s)u(y,t-s)dy = \int_0^L \partial_s [G(x,y,s)u(y,t-s)]dy= \int_0^L [\partial_s G(x,y,s) u(y,t-s)+  G(x,y,s) \partial_s u(y,t-s)]dy =^{(5.31)}  \int_0^L a\partial_y^2 G(x,y,s) u(y,t-s)dy - \int_0^L  G(x,y,s) [a\partial_y^2 u(y,t-s)+F(y,t-s)]dy= -\int_0^L  G(x,y,s) F(y,t-s)dy$, where the last equation follows from IBP.  We int this equation over $s$ from $\epsilon$ to $t, 0<\epsilon < t, \int_0^L G(x,y,t)u(y,0)dy - \int_0^L G(x,y,\epsilon) u(y, t-\epsilon) dy = -\int_{\epsilon}^t(\int_0^L G(x,y,s) F(y,t-s)dy)ds$. Take the lim of the rhs as $\epsilon \ra 0$. As for the 2nd expn on the lhs, $|\int_0^L G(x,y,\epsilon)u(y,t-\epsilon)dy - u (x,t)| \leq |\int_0^L G(x,y,\epsilon)(u(y,t-\epsilon) - u (y,t))dy| + |\int_0^L G(x,y,\epsilon)u(y,t)dy - u (x,t)|$. The last expn tends to 0 by P 5.18 (a). By part (b) of this very prop, the last but one expn can be est by $\sup_{0 \leq y \leq L} |u(y,t-\epsilon) - u(y,t)|$ which tends to 0 as $\epsilon \ra 0$, because $u$ is unif cont on $[0,L] \times [t/2, t]$. 
{\bf T5.21}. If $u$ is a soln of (5.35) and $u$ and $F$ are cont and bdd on $[0,L] \times (0, T)$, then $u(x, t) = \int_0^L G(x,y,t)f(y)dy+\int_0^t \int_0^L G(x,y,s)F(y,t-s)dsdy, 0<t<T, 0\leq x \leq L$. Under which assumptions for $F$ is this a soln? Since we already studied the homog IVP in detail, we can set $f=0$ and, after a subst, set $\tilde{u}(x,t)=\int_0^t \int_0^L G(x,y,t-s)F(y,s)dsdy$ (5.36). Interchanging int and the series rep of $G$, (5.27), we obtain the Fourier sin rep, $\tilde{u}(x,t)=\sum_{m=1}^{\infty} \tilde{u}_m(x,t), \tilde{u}_m(x,t)$ = $\sin(\lambda_m x)(2/L)\int_0^t \int_0^L \sin(\lambda_m y)e^{-a \lambda_m^2(t-s)} F(y,s)dsdy$ = $\sin(\lambda_m x) e^{-a \lambda_m^2 t} (2/L)\int_0^t e^{a \lambda_m^2 s}(\int_0^L \sin(\lambda_m y) F(y,s)dy)ds$ (5.37).  
%We could have also obtained this expression directly from expanding $\tilde{u}$ and $F$ into Fourier sine series and deriving ODEs for the Fourier coeffs of $\tilde{u}$ involving the Fourier coeffs of $F$ (E 5.4.1).
We have the est $|\tilde{u}_m(x,t)|\leq 2 \sup |F| \int_0^t e^{-a \lambda_m^2(t-s)}ds$
$\leq (2\sup|F|)/(a \lambda_m^2)$. Since the series over the rhs converges (recall that $\lambda_m$ is proportional to $m$), the series for $\tilde{u}$ converges unif on $[0,L] \times [0,T]$ and $\tilde{u}$ is cont on $[0,L] \times [0,T], \tilde{u}(x,0) = 0\; \forall \; x \in [0,L]$. By the product rule and the fund thm of calc, $\partial_t \tilde{u}_m(x,t)$ = $-a \lambda_m^2 \tilde{u}_m(x,t) + \sin(\lambda_m x) (2/L) \int_0^L \sin (\lambda_m y) F(y, t) dy$ (5.38)  $=a \partial_x^2 \tilde{u}_m(x,t) + \sin ( \lambda_m x) (2/L) \int_0^L \sin(\lambda_m y) F(y, t) dy$ (5.39). We make a change of variables in the int for $\tilde{u}_m$ in (5.37),  $\partial_t \tilde{u}_m(x,t) = -a \lambda_m^2  \sin(\lambda_m x) (2/L) \int_0^t \int_0^L \sin(\lambda_m y) e^{-a \lambda_m^2 s} F(y,t-s)dy ds +\sin(\lambda_m x) (2/L) \int_0^L \sin(\lambda_m y) F(y,t)dy$ (5.40). We change the order of int and combine the spatial ints, $\partial_t \tilde{u}_m(x,t)$ = $\sin(\lambda_m x) (2/L) \int_0^L \sin(\lambda_m y) H(y,t)dy, H(y,t) = -a \lambda_m^2  \int_0^t  e^{-a \lambda_m^2 s}$
$F(y,t-s)ds + F(y,t)$. Since $a \lambda_m^2  \int_0^t  e^{-a \lambda_m^2 s} ds = 1 - e^{-a \lambda_m^2 t}, H(y,t)=a \lambda_m^2  \int_0^t  e^{-a \lambda_m^2 s} [F(y,t) - F(y,t-s)]ds -a \lambda_m^2  \int_0^t  e^{-a \lambda_m^2 s} ds F(y,t)+ F(y,t) =a \lambda_m^2  \int_0^t  e^{-a \lambda_m^2 s} [F(y,t) - F(y,t-s)]ds + e^{-a \lambda_m^2 t} F(y,t)$.  Assume $F$ is Lip cont in the time var:  $\exists \; \Lambda >0$ s.t. $|F(y,t)- F(y,s)| \leq \Lambda|t-s|, 0 \leq y \leq L, 0\leq s,t \leq T$. Then $|H(y,t)| \leq a \lambda_m^2  \int_0^t  e^{-a \lambda_m^2 s} \Lambda s ds + e^{-a \lambda_m^2 t} |F(y,t)|$. Subst $r=a \lambda_m^2 s, |H(y,t)| \leq (L/(a \lambda_m^2) \int_0^{ta\lambda_m^2}e^{-r} r dr + e^{-a\lambda_m^2t} |F(y,t)| \leq  (L/(a \lambda_m^2) + e^{-a\lambda_m^2t} |F(y,t)|$. Now $|\partial_t \tilde{u}_m(x,t)| \leq | \sin(\lambda_m x)| (2/L) \int_0^L $
$| \sin(\lambda_my)| |H(y,t)| dy \leq (2/L) \int_0^L  |H(y,t)| dy $. Sub est $|H(y,t)|, |\partial_t\tilde{u}_m(x,t)|\leq (2 \Lambda)/(a \lambda_m^2) + e^{-a\lambda_m^2t} (2/L) \int_0^L  |F(y,t)| dy$ (5.41). By T5.3, $u$ is PD wrt $t$ on $[0,L] \times (0, T]$ and $\partial_t u = \sum_{m=1}^{\infty}\partial_t \tilde{u}_m$ with the series converging unif on $[0,L] \times [\epsilon, T]$ for every $\epsilon \in (0,T)$. Similarly one shows that $u$ is twice PD wrt $x$ and $\partial_x^2 u = \sum_{m=1}^{\infty}\partial_x^2 \tilde{u}_m$ with the series converging unif on $[0,L] \times [\epsilon, T]$ for every $\epsilon \in (0,T)$.By (5.39), $\partial_t \tilde{u} - a \partial_x^2\tilde{u} = \sum_{m=1}^{\infty}\sin(\lambda_mx) (2/L) \int_0^L \sin(\lambda_my) F(y,t) dy$. where the convergence on the rhs is unif on $[0,L] \times [\epsilon, T]$ for every $\epsilon \in (0,T)$. Since the sines form an orthonormal basis, the rhs equals the Fourier sine series of $F(\cdot, t)$ and so equals $F(x,t)$ for a.a. $x$. Further, for fixed $t$, the convergence holds in $L^2[0,L]$ wrt the space var.  Since both $\partial_t \tilde{u} - a \partial_x^2\tilde{u} $ and $F$ are cont, $\partial_t \tilde{u} - a \partial_x^2\tilde{u} = F$. Further $\tilde{u}(x,0) = 0$ and $\tilde{u}(0,t) = 0 = \tilde{u}(L,t)$. 
{\bf T 5.22}. Let $f: (0,L) \ra \R$ be cont and bdd, $F: [0, L] \times [0,T] \ra \R$ cont and Lip cont in the time var.  Then $\exists$ a fctn $u: [0,L] \times [0,T] \ra \R$ that is cont except possibly at $(0,0)$ and $(L,0)$ and solves (5.35). The soln $u$ is given as in T5.21. 
% 5.4.1 Inhomogeneous boundary contitions
{\bf 5.4.1 Inhomog BCs} HE: (PDE) $(\partial_t - a \partial_x^2)u=0, 0 \leq x \leq L, t \in (0,T)$, (IC) $u(x,0)= 0, 0 \leq x \leq L$, (BC) $u(0,t)=g(t), u(L,t)=h(t), 0 < t \leq T$ (5.42). Ansatz $u(x,t)= v(x,t) + U(x,t), U(x,t)=(1-(x/L))g(t) + (x/L) h(t)$ (5.43). Then $u$ solves (5.42) iff $v$ satisfies (PDE) $(\partial_t - a \partial_x^2)v=- (1-(x/L))g'(t) - (x/L) h'(t), 0 \leq x \leq L, 0 < t < T$, (IC) $v(x,0)= -U(x,0), 0 \leq x \leq L$, (BC) $v(0,t)=0, v(L,t)=0, 0 < t \leq T$ (5.44). If $g'$ and $h'$ are Lip cont, a soln can be found.  It is given by $v(x,t)=- \int_0^L U(y,0) G(x,y,t) dy - \int_0^t \int_0^L G(x,y,t-s)[(1-(y/L))g'(s)+(y/L) h'(s)]dy ds$.
%Exercises
{\bf E5.4.1}. Let $F: [0,L] \times [0,T] \ra \R$ be cont and Lipschitz cont in the space variable, $F(0,t) = 0 = F(L, t)\; \forall \; t \in [0,T]$. Derive (5.37) for a solution $\tilde{u}$ of $(\partial_t - a \partial_x^2)\tilde{u} = F$, on $[0,L] \times [0,T], u(x,0)=0, 0 \leq x \leq L, \tilde{u}(0,t) = 0 = \tilde{u}(L,t), 0 \leq t \leq T$, directly expressing $\tilde{u}$ and $F$ by their Fourier sine series and deriving and solving ODEs for the Fourier coefficients of $\tilde{u}$.

{\bf E5.4.2}. Let $u:[0,L] \times (0, T) \ra \R$ be cont. Assume that the partial derivatives $\partial_tu(x,t)$ exist $ \forall \; x \in [0,L]$ and $t \in (0,T)$ and that $\partial_t u$ is cont on $[0,L] \times (0,T)$. Show: $\forall $ cont $\phi: [0,L] \ra \R, \int_0^L \phi(x) u(x,t)dx$ is differentiable in $t \in (0, T)$ and $d/dt\int_0^L \phi(x)u(x,t)dx = \int_0^L \phi(x) \partial_t u(x,t)dx$. Hint: Show (why?) $\int_0^L \phi(x)((u(x,s)-x(x,t))/(s-t)-\partial_t u(x,t))dx \ra 0, s \ra t$. You may like to use the mean value thm. {\it Proof}. Set $v(t) = \int_0^L \phi(x)u(x,t)dx,  w(t) = \int_0^L \phi(x)\partial_t u(x,t)dx$. The task is to show that $v$ is differentiable on $(0, T)$ and $v' = w$. To this end, we work with the definition of the derivative. Let $s., t \in (0, T)$. Since integration is a linear operation, $(v(s) - v(t))/(s-t)-w(t)= \int_0^L \phi(x)((u(x,s)-x(x,t))/(s-t)-\partial_t u(x,t))dx$. Then $|(v(s) - v(t))/(s-t)-w(t)| \leq \int_0^L |\phi(x)| |((u(x,s)-x(x,t))/(s-t)-\partial_t u(x,t))|dx$. Since $\phi: [0, L] \ra \R$ is cont, $\exists M > 0$ s.t. $|\phi(x)|\leq M \; \forall x \in [0,L]$. So $|(v(s) - v(t))/(s-t)-w(t)| \leq M \int_0^L  |((u(x,s)-x(x,t))/(s-t)-\partial_t u(x,t))|dx \leq M L \sup_{0 \leq x \leq L}  |((u(x,s)-x(x,t))/(s-t)-\partial_t u(x,t))|$ (5.29). Let $x \in [0, L]$. By the mean value thm, $\exists r_x$ between $s$ and $t$ s.t. $|((u(x,s)-x(x,t))/(s-t)-\partial_t u(x,t))|= |\partial_2u(x,r_x)-\partial_2u(x,t)|$. Here $\partial_2 u$ denotes the partial derivative of $u$ wrt the 2nd variable, time. Choose some $\delta_0 >0$ s.t. $[t-\delta_0, t+\delta_0] \subseteq (0,T)$. Since $\partial_2 u$ is cont on $[0,L] \times (0, T)$, it is uniformly cont on $[0,L] \times[t-\delta_0, t+\delta_0]$. Let $\epsilon > 0$. Then $\exists \delta \in (0, \delta_0)$ s.t. $|\partial_2u(x,r)-\partial_2u(x,t)|< \epsilon/(2LM)$ if $|r-t| < \delta$. Let $|s-t| < \delta$. Since $r_x$ is between $s$ and $t, |r_x - t| \leq |s-t| < \delta$ and so $|((u(x,s)-x(x,t))/(s-t)-\partial_t u(x,t))|= |\partial_2u(x,r_x)-\partial_2u(x,t)|<\epsilon/(2LM)$. Since this holds $ \forall \; x \in [0,L], \sup_{0\leq x \leq L}|((u(x,s)-x(x,t))/(s-t)-\partial_t u(x,t))| \leq \epsilon/(2LM)$. By (5.29), $|((v(s)-v(t))/(s-t)-w(t))| \leq \epsilon/2<\epsilon \qed$

{\bf E5.4.3}. Let $u$ be as in E5.4.2. Assume $u$ is twice partially diff wrt $x$ on $[0,L] \times (0,T)$ and $\partial_xu$ and $\partial_x^2u$ are cont on $[0,L] \times (0,T)$ and $(\partial_t - a \partial_x^2)u = F(x,t), x \in [0,L], t \in (0, T), u(0,t)=0=u(L,t), t \in (0, T)$, where $F: [0, L] \times (0, T) \ra \R$ is cont. Show: For every twice contly diff $\phi: [0,L] \ra \R$ with $\phi(0) = 0 = \phi(L), \int_0^L \phi(x)u(x,t) dx$ is diff in $t \in (0,T)$ and $d/dt \int_0^L \phi(x)u(x,t) dx= \int_0^L a\phi''(x)u(x,t) dx + \int_0^L \phi(x)F(x,t) dx, 0 < t < T$. {\it Prf}. By the previous exercise, $\int_0^L \phi(x)u(x,t) dx$ is diff in $t \in (0, T)$ and $d/dt \int_0^L \phi(x)u(x,t) dx=\int_0^L \phi(x) \partial_t u(x,t) dx=\int_0^L \phi(x) (a \partial_x^2 u(x,t) + F(x,t)) dx= \int_0^L \phi(x) a \partial_x^2 u(x,t) dx + \int_0^L \phi(x) F(x,t) dx$.  Since $\phi$ is twice contly diff, we can IBP twice.  Since $\phi(0)=0=\phi(L)$ and $u(0,t)=0=u(L,t)$, we do not obtain any terms at the int limits and $d/dt \int_0^L \phi(x)u(x,t) dx=  \int_0^L a \phi''(x)  u(x,t) dx + \int_0^L \phi(x) F(x,t) dx \qed$

{\bf E5.4.4}. Let $F: [0,L]\times [0,T) \ra \R$ be cont. Define $u(x,t)= \int_0^t \int_0^L G(x,y,t-s)F(y,s)dy ds,  0 < t < T,  0 \leq x \leq L.$ Show:  For every twice contly diff $\phi: [0,L] \ra \R$ with $\phi(0) = 0 = \phi(L), \int_0^L \phi(x)u(x,t)dx$ is diff in $t \in (0,T)$ and $d/dt \int_0^L  \phi(x) u(x,t) dx = \int_0^L  a\phi''(x)u(x,t)dx + \int_0^L  \phi(x)F(x,t)dx, \quad 0 < t < T.$ Hint:  Notice and prove that $ \int_0^L  \phi(x)u(x,t)dx = \int_0^L ( \int_0^t v(y,t-s) F(y,s)ds) dy$ with $v(y,t) =  \int_0^L  \phi(x)G(x,y,t)dy = \int_0^L  G(y,z,t)\phi(z)dz,  t>0, y \in [0,L]$ (5.45), and use E 5.3.4.  You may interchange int and diff and use Leibnitz rule without prf. 
{\it Prf}. Changing the order of int, we obtain $\int_0^L \phi(x) u(x,t) dx = \int_0^L \phi(x)( \int_0^t \int_0^L G(x,y,t-s) F(y,s)ds dy )dx$ =  $\int_0^L \int_0^t (\int_0^L G(x,y,t-s)\phi(x) dx) F(y,s)ds dy$ =  $\int_0^L (\int_0^t v(y,t-s) F(y,s)ds) dy$ (5.46) with $v$ in (5.45). Recall that $G(x,y,t) = G(y,x,t)$. By E 5.3.4, $\partial_t v(y,t) = \int_0^L G(y,z,t) a \phi''(z) dz \; \exists$ and is cont for $t > 0, y \in [0,L]$ and is bdd on $[0,L] \times [0, \infty)$ because $\phi''$ is bdd on $(0, L)$ and $\int_0^L G(y,z,t) dz \leq 1$. Further $v$ can be cont extended to $[0, L] \times [0, \infty)$ with $v(y,0) = \phi(y)$. So we can apply Leibnitz rule and obtain $\partial_t \int_0^t v(y,t-s) F(y,s)ds  =  v(y,0) F(y,t) + \int_0^t \partial_t v(y,t-s) F(y,s))ds$ = $\phi(y) F(y,t) + \int_0^t ( \int_0^L G(y,z,t) a \phi''(z))F(y,s)ds$.  We interchange diff and int in (5.46), $\partial_t \int_0^L \phi(x) u(x,t)dx = \int_0^L \partial_t ( \int_0^t v(y,t-s)F(y,s)ds)dy$ = $\int_0^L (\phi(y) F(y,t) + \int_0^t ( \int_0^L G(y,z,t-s) a \phi''(z)dz ) F(y,x) ds) dy$.  We change the order of int back and use $G(y,z,t-s) = G(z,y,t-s), \partial_t \int_0^L \phi(x) u(x,t)dx = \int_0^L \phi(x) F(y,t)dy + \int_0^L a \phi''(z)u(z,t)dz \qed$

{\bf E5.4.14}. Let $u: [0,L] \times [0,T] \ra \R$ be cont and twice contly diff. Let $F: [0,L] \times [0,T] \ra \R$ be cont.  Assume $\partial_t u(x,t) - a \partial_x^2 u(x,t) = F(x,t), 0 \leq t \leq T, 0 \leq x \leq L, \partial_x u(0,t) = 0 = \partial_x u(L,t), t \in [0, T], u(x,0)=0, 0 \leq x \leq L$. Derive a Fourier series rep of $u$.  You may interchange int, diff and series without prf.  You do not need to discuss convergence of the series. {\it Prf}. Because of the BCs, we choose a Fourier cosine rep, $u(x,t)=\sum_{m=0}^{\infty}A_m(t) \cos(\lambda_m x), \lambda_m$ = $(m \pi)/L, m \in \Z, A_m(t) = 2/L \int_0^L u(x,t) \cos(\lambda_m x) dx, m \in \N, A_0(t)= 1/L \int_0^L u(x,t) dx$ (5.47). Let $m \in \N$. Interchange diff and int, $A_m'(t) = 2/L \int_0^L \partial_t u(x,t) \cos(\lambda_m x) dx$. Use the PDE for $u, A_m'(t) = 2/L \int_0^L (a \partial_x^2 u(x,t) + F(x,t))\cos(\lambda_m x) dx$.  IBP twice and use that the boundary terms are 0 because of the BCs for $u$ and $\sin(\lambda_m x) = 0$ for $x = 0$ and $x = L,\; A_m'(t) = 2/L \int_0^L u(x,t) a \partial_x^2 \cos(\lambda_m x) dx + \hat{F}_m (t)$ with $\hat{F}_m (t)= 2/L \int_0^L F(x,t)\cos(\lambda_m x) dx$. By the diff properties of cosine, $A_m'(t) = -a \lambda_m^2 A_m(t) + \hat{F}(t)$. Further $A_m(0)=0$ by the initial conditions for $u$. We use the variation of consts formula, $A_m(t) = \int_0^t e^{-a \lambda_m^2 (t-s)}\hat{F}_m(s) ds$. Further, since $u$ satisfies the PDE and the no-flux BCs, $A_0'(t) = 1/L \int_0^L \partial_t u(x,t)dx$ = $1/L \int_0^L (a \partial_x^2 u(x,t)+ F(x,t))dx = (1/L)a [ \partial_x u(L,t)-\partial_x u(0,t)]+1/L\int_0^L F(x,t)dx = 1/L \int_0^L F(x,t)dx$, and $A_0 = 0$ because of the zero initial values for $u$. This implies $A_0(t)=\int_0^t 1/L \int_0^L F(x,s)dxds$.  Together with $A_m(t) = \int_0^t e^{-a \lambda_m^2 (t-s)}2/L ( \int_0^L F(x,s) \cos (\lambda_m x)dx)ds$, $m \in \N$, and $u(x,t)$ = $\sum_{m=0}^{\infty} A_m (t) \cos(\lambda_m x), t \geq 0, 0 \leq x \leq L$, This provides the Fourier series representation of $u \qed$
% Chapter 6 The Laplace equation
{\bf The LE}
Let $\Omega \subseteq \R^n$ be open and $u: \Omega \ra \R$ be twice diff.  Then the Laplace operator is  $\Delta u(x) = \sum_{j=1}^n \partial_j^2 u(x), x \in \Omega$ (6.1). The Laplace equation is for a cont fctn $u: \bar{\Omega} \ra \R$ that is twice diff on $\Omega$, (PDE) $\Delta u = 0$ on $\Omega$, (BC) $u(x) = f(x), x \in \partial \Omega$ (6.2), where $\partial \Omega =\bar{\Omega} \setminus \Omega$ is the boundary of $\Omega$ and $f: \partial \Omega \ra \R$ is given.  A twice contly diff fctn $u$ on $\Omega$ that satisfies $\Delta u =0$ on $\Omega$ is called harmonic on $\Omega$.
{\bf The LE on a rectangle}
Let $n=2$ and $\Omega = (0,L) \times (0,H)$ with $L, H >0$. Then $\bar{\Omega} = [0,L] \times [0,H]$ and $\partial \Omega = \cup_{k=1}^4 B_k$ consists of four line sections.   LE takes the form (PDE) $(\partial_x^2 + \partial_y^2) u(x,y) = 0, 0 < x < L, 0 < y < H$, (BC) $u(0,y) = g_1(y), u(L,y)=g_2(y), 0 < y < H, u(x,0)=h_1(x), u(x,H)=h_2(x), 0 < x < L$ (6.3).  By symmetry, it is sufficient to study the problem 
(PDE) $(\partial_x^2 + \partial_y^2) u(x,y) = 0, 0 < x < L, 0 < y < H$, (BC) $u(0,y) = g_1(y), u(L,y)=0, 0 < y < H, u(x,0)=0, u(x,H)=0, 0 < x < L$ (6.4). 
The form of the problem suggests to look for the soln in the form of a Fourier sine series in $y, u(x,y) = \sum_{m=1}^{\infty}B_m(x) \sin(\lambda_m y), \lambda_m = (m \pi) / H, B_m(x)=(2/H)\int_0^H u(x,y) \sin(\lambda_m y) dy$ (6.5). To determine $B_m$, we derive a diff eqn, $B_m''(x) = (2/H)\int_0^H \partial_x^2 u(x,y) \sin(\lambda_m y) dy$ = $- (2/H)\int_0^H \partial_y^2 u(x,y) \sin(\lambda_m y) dy$. We IBP twice using the zero BCs for both $u$ and the sine fctns, $B_m''(x) = -(2/H) \int_0^H  u(x,y) \partial_y^2 \sin(\lambda_m y) dy$ = $(2/H) \int_0^H  u(x,y) \lambda_m^2 \sin(\lambda_m y) dy = \lambda_m^2 B_m$. Further $B_m(L)=0, B_m(0) = (2/H)\int_0^H  g_1(y)  \sin(\lambda_m y) dy$ (6.6). A poss fund set of solns for this ODE is $e^{\lambda_m x}, e^{- \lambda_m x}$, but in view of the condition $B_m(L)=0$ the fund set $\cosh(\lambda_m (L-x)), \sinh(\lambda_m (L-x))$ is more practical.  Then $B_m(x)= A_m \sinh(\lambda_m (L-x))$ and, by (6.6), $A_m = 2/(H \sinh(\lambda_m L)) \int_0^H g_1(z) \sinh(\lambda_m z) dz$ (6.7). We combine (6.7) and (6.5), $u(x,y)= \sum_{m=1}^{\infty} u_m(x,y), u_m(x,y)= A_m \sin(\lambda_m y) \sinh(\lambda_m (L-x))$ (6.8). Then, if $0 \leq x \leq L$ and $0 \leq y \leq H, \partial_y^k u_m(x,y) = \pm \lambda_m^k A_m \sinh(\lambda_m (L-x))\{\sin(\lambda_m y)$ or $ \cos(\lambda_m y)\}$ and $\partial_x^k u_m(x,y) = \pm \lambda_m^k A_m \sin(\lambda_m y) \{\sinh(\lambda_m(L-x)), k \in \N, k$ even,  or $ \cosh(\lambda_m (L-x)), k \in \N, k$ odd$\}$. So $|\partial_x^k \partial_y^{\ell} u_m(x,y)| \leq \lambda_m^{k+\ell} |A_m| \{\sinh(\lambda_m(L-x))$ or $ \cosh(\lambda_m (L-x))\}\; 0 \leq x \leq L$. Since sinh and cosh are increasing on $\R_+, |\partial_x^k \partial_y^{\ell} u_m(x,y)| \leq \lambda_m^{k+\ell} |A_m| \{\sinh(\lambda_m L)$ or $ \cosh(\lambda_m L)\}\; 0 \leq x \leq L$. Recall $\cosh z - \sinh z = (e^z + e^{-z})/2- (e^z - e^{-z})/2 = e^{-z} \leq 1$. Again, since sinh is increasing, $\cosh(\lambda_m L) \leq \sinh(\lambda_m L) + 1 \leq (1 + 1/\sinh(\lambda_1 L))\sinh(\lambda_m L), m \in \N$. We combine these considerations and find a const $c > 0$ s.t. $|\partial_x^k \partial_y^{\ell} u_m(x,y)| \leq c \lambda_m^{k+\ell} |A_m| \sinh(\lambda_m L), k, \ell \in \N, 0 \leq x \leq L$. By T 5.3 and (6.7), $u$ is twice PD (and satisfies LE by construction) if $ \infty > \sum_{m=1}^{\infty} \lambda_m^2 |\int_0^H g_1(z) \sin(\lambda_m z) dz| = \sum_{m=1}^{\infty} |\int_0^H g_1(z) (d^2/dz^2)\sin(\lambda_m z) dz|$ (6.9). If $g_1$ is twice contly diff and $g_1(0)=0=g_1(H)$, by partial int the last expression equals $\sum_{m=1}^{\infty}  |\int_0^H g_1''(z) \sin(\lambda_m z) dz|$, which is finite if $g_1''$ is Lip cont and $g_1(0)=0=g_1''(H)$ (T 4.14 and E 4.3.3). We summarize. 
{\bf T 6.1} Let $g_1: [0,H] \ra \R$ be twice diff, $g_1''$ Lip cont and $g_1(0)=0=g_1(H), g_1''(0)=0=g_1''(H)$. Then $\exists$ a twice contly diff fctn $u: [0,L] \times [0,H] \ra \R$ that satisfies (PDE) $(\partial_x^2 + \partial_y^2)u(x,y) = 0, $ (BC) $u(0,y)= g_1(y), u(L,y) = 0, u(x,0) = 0, u(x,H) = 0, 0 \leq x \leq L, 0 \leq y \leq H$.
% Exercises
{\bf E 6.1.3}. Solve the LE with mixed BC, $(\partial_x^2 + \partial_y^2) u(x,y) = 0, 0 \leq x \leq L, 0 \leq y \leq H, u(0,y) = g(y), u(L,y) = 0, 0 \leq y \leq H, \partial_y u(x,0) = 0 = \partial_y u(x, H), 0 \leq x \leq L$. Make an educated guess which conditions $g$ must satisfy for $u$ to be a soln. Explain why you choose these conditions. Is $u$ unique? {\it Sol}. Look for the soln in the form of a Fourier cosine series in y, $u(x,y) = \sum_{m=0}^{\infty} A_m(x) \cos(\lambda_m y),  \lambda_m=m \pi/H, m \in \Z$, $A_m(x)=2/H \int_0^H u(x,y) \cos(\lambda_m y) dy,  m \in \N,  A_0(x) = 1/H \int_0^H u(x,y) dy.$ To determine $A_m$ we derive a diff eq. Start with the special case of $A_0. \; A_0(x)''=0 \implies A_0(x)'=C_1 \implies A_0(x)=C_1 x + C_2$. We look at BCs. $A_0(0) = 1/H \int_0^H g(y) dy = C_2. \; A_0(L) = 0 = C_1 L + C_2 \implies C_1= -1/(H L) \int_0^H g(y) dy.$  Put these together $A_0(x) = -1/(H L) \int_0^H g(y) dy + 1/H \int_0^H g(y) dy =1/H \int_0^H g(y) dy (1-x/L)$. Now we derive a diff eq for the general case $A_m''(x)=2/H \int_0^H \partial_x^2 u(x,y) \cos(\lambda_m y) dy=-2/H \int_0^H \partial_y^2 u(x,y) \cos(\lambda_m y) dy$. IBP $A_m''(x)$ = $-2/H [\partial_y u(x,y) \cos(\lambda_m y) ]_0^H -  2 \lambda_m/H\int_0^H \partial_y u(y,t) \sin(\lambda_m y)dy$. Because $\partial_y u(x,0) = 0 = \partial_y u(x,H)$ we have $A_m''(x)=  -  2 \lambda_m/ H \int_0^H \partial_y u(y,t) \sin(\lambda_m y)dy$. IBP again $A_m''(x)=  -2\lambda_m / H [u(x,y) \sin(\lambda_m y) ]_0^H +  2 \lambda_m^2 / H \int_0^H  u(y,t) \cos(\lambda_m y)dy$. Because $\sin(\lambda_m y) = 0$ for $y = 0$ and $y = H$ we have $A_m''(x)= 2 \lambda_m^2 / H \int_0^H  u(y,t) \cos(\lambda_m y)dy = \lambda_m^2 A_m (x).$ Further $A_m(L) = 0, A_m(0) = 2/H \int_0^H  g(y) \cos(\lambda_m y)dy$ (1). A poss fund set of solns for this ODE is $e^{\lambda_m x}, e^{-\lambda_m x}$, but in view of the condition $A_m(L) = 0$ the fund set $\cosh(\lambda_m(L-x)),  \sinh(\lambda_m(L-x))$ is more practical.  Then $A_m(x) = C_m \cosh\lambda_m(L-x)) + B_m \sinh(\lambda_m(L-x))$. Initial conditions (1) yields $C_m = 0,  B_m = 2/(H \sinh(\lambda_m L)) \int_0^H  g(z) \cos(\lambda_m z)dz$ (2). We combine all this to get $u(x,y) = \sum_{m=0}^{\infty} u_m(x,y), u_m(x,y) = B_m \cos(\lambda_m y) \sinh(\lambda_m(L-x)).$ Then, if $0 \leq x \leq L$ and $0 \leq y \leq H, \partial_y^k u_m(x,y) = \pm \lambda_m^k B_m \sinh(\lambda_m(L-x)) =\cos(\lambda_m y)$ or $ \sin(\lambda_m y)$ and $\partial_x^k u_m(x,y) = \pm \lambda_m^k B_m \cos(\lambda_m y) \sinh(\lambda_m (L-x)), k \in \N, k \text{ even},$ or $\cosh(\lambda_m (L-x)),  k \in \N, k \text{ odd}$. So, $ |\partial_x^k \partial_y^{\ell} u_m(x,y)| \leq \lambda_m^{k+ \ell} |B_m| \{\sinh(\lambda_m (L-x))$ or $        \cosh(\lambda_m (L-x)) \}, 0 \leq x \leq L. $  Since sinh and cosh are increasing on $\R_+,    |\partial_x^k \partial_y^{\ell} u_m(x,y)| \leq \lambda_m^{k+ \ell} |B_m| \{ \sinh(\lambda_m L)$ or $ \cosh(\lambda_m L) \}, 0 \leq x \leq L. $ Recall $\cosh z - \sinh z = (e^z + e^{-z})/2- (e^z - e^{-z})/2 = e^{-z} \leq 1.$  Again, since sinh is increasing, $\cosh(\lambda_m L) \leq \sinh(\lambda_m L) + 1 \leq ( 1 + 1/(\sinh(\lambda_1 L))) \sinh(\lambda_m L),  m \in \N.$ We combine these considerations and find a const $c >0$ s.t. $|\partial_x^k \partial_y^{\ell} u_m(x,y)| \leq \lambda_m^{k+ \ell} |B_m| \sinh(\lambda_m L),  k,\ell \in \N, \; 0 \leq x \leq L.$ By T 5.3 and (2), $u$ is twice partially diff (and satisfies LE by construction) if $ \infty > \sum_{m=1}^{\infty} \lambda_m^2 |\int_0^H g(z) \cos(\lambda_m z)dz|=\sum_{m=1}^{\infty}  |\int_0^H g(z) d^2/dz^2 \cos(\lambda_m z)dz|$.  If $g$ is twice cont diff and $g'(0) = 0 = g'(H)$, by partial integration the last expression equals $\sum_{m=1}^{\infty} |\int_0^H g''(z) \cos(\lambda_m z)dz |$ which is finite if $g''$ is Lip cont and $g''(0) = 0 = g''(H)$ (P 4.14 and E4.3.3). Our solution $u$ is unique because there is only one series and the series is unique. $\qed$

{\bf The LE on a disk}
Now $\Omega \subseteq \R^2$ is the open disk with center 0 and radius $a > 0$ and $\partial \Omega$ the circle with center 0 and radius $a$. We represent the solution in polar coords, $u(x, y) = v(r,\theta), x = r \cos \theta, y = r \sin \theta, 0 \leq r \leq a, \theta \in \R$, where $v(r,\theta)$ is $2 \pi$-periodic in $\theta$. The BC is easily expressed as $v(a,\theta) = f(\theta), \theta \in \R$ (6.15). Here $f: \R \ra \R$ is $2 \pi$-periodic and cont. We translate the Laplace operator into polar coords.  By the chain rule, $\partial_r v(r,\theta) = \partial_x u(x,y) \cos \theta + \partial_y u(x,y) \sin \theta, \; \partial_r^2 v(r,\theta) = \partial_x^2 u (\cos \theta)^2 + 2\partial_x \partial_y u \cos \theta \sin \theta + \partial_y^2 u (\sin \theta)^2, \; \partial_{\theta} v(r, \theta) = \partial_x u(x,y) (-r \sin \theta) + \partial_y u(x,y) r \cos \theta, \;\partial_{\theta}^2 v(r, \theta) = \partial_x^2 u r^2 (\sin \theta)^2 - \partial_x \partial_y u [r^2 \sin \theta \cos \theta] - \partial_x u r \cos \theta + \partial_y^2 u r^2 (\cos \theta)^2 - \partial_x \partial_y u [r^2 \sin \theta \cos \theta ] - \partial_y u r \sin \theta$. Thus $\partial_r^2 v(r, \theta) + (1/r^2) \partial_{\theta}^2 v(r, \theta) = \partial_x^2 u + \partial_y^2 u - (1/r) \partial_x u \cos \theta - (1/r) \partial_y u \sin \theta = \Delta u - (1/r) \partial_r v(r, \theta)$. The Laplacian of $u$ takes the polar coord form $\Delta u = (\partial_r^2 + (1/r) \partial_r + (1/r^2) \partial_{\theta}^2)v(r, \theta)$. The LE takes the form $(r^2 \partial_r^2 + r \partial_r + \partial_{\theta}^2)v(r, \theta) = 0, 0 \leq r < a, v(a, \theta) = f(\theta), \theta \in \R$ (6.16), with the understanding that $v(r, \theta)$ and $f(\theta)$ are $2 \pi$-periodic in $\theta$ We write $v$ is a complex Fourier series in $\theta, \; v(r, \theta) = \sum_{j=-\infty}^{\infty} \hat{v}_j(r)e^{ij\theta}, \hat{v}_j(r) = (1/(2\pi)) \int_{-\pi}^{\pi} v(r, \theta)e^{-ij \theta} d \theta$. If $v$ is smooth enough, $(r^2 \partial_r^2 + r \partial_r)\hat{v}_j(r) =  (1/(2\pi)) \int_{-\pi}^{\pi} (r^2 \partial_r^2 + r \partial_r) v(r, \theta)e^{-ij \theta} d \theta=  (1/(2\pi)) \int_{-\pi}^{\pi} (-1) \partial_{\theta}^2 v(r, \theta)e^{-ij \theta} d \theta$. Since $v (r, \theta)$ is $2 \pi$-periodic in $\theta, \partial_{\theta}^k v(r, -\pi) = \partial_{\theta}^k v(r, \pi)$ for $r \geq 0, k = 0, 1, \dots$. Since the analogous properties hold for $e^{-ij \theta}$, we IBP twice and obtain $(r^2 \partial_r^2 + r \partial_r-j^2)\hat{v}_j(r)=0, j \in \Z$. If $j = 0, 0 = (r \partial_r^2 + \partial_r)\hat{v}_0(r) = (d/dr)(r \hat{v}_0'(r))$. So $r \hat{v}_0'(r) = \alpha_0$ and $\hat{v}_0(r)=\alpha_0 \ln r + \beta_0$. The continuity of $\hat{v}_0$ at 0 enforces $\alpha_0 = 0$ and $\hat{v}_0$ is const, $\hat{v}_0(r) = \hat{v}_0(a) = (1/(2 \pi)) \int_{-\pi}^{\pi} f(\eta) d \eta = \hat{f}_0$. For $j \neq 0, \hat{v}_j$ satisfies Euler's equation which is solved by the ansatz $\hat{v}_j(r) = r^n$. This yields $0 = (n-1)n + n - j^2 = n^2 - j^2$. So $n = \pm j$ and a general solution is given by $\hat{v}_j(r) = \alpha_j r^{-j}+ \beta_j r^j$. Since $\hat{v}_j$ exists at $r = 0, \hat{v}_j(r) = \gamma_j r^{|j|}, j \in \Z, j \neq 0$. From the BC, (6.15), $\hat{v}_j(a) = (1/(2 \pi)) \int_{-\pi}^{\pi} f(\eta)e^{-i j \eta} d\eta = \hat{f}_j$ (6.17) and $\gamma_j = a^{-|j|}\hat{f}_j$. We sub this into the formula for $ \hat{v}_j, \hat{v}_j(r)=\hat{f}_j(r/a)^{|j|}, j \neq 0$. We sub this result into the Fourier series of $v, v(r,\theta) = \sum_{j \in \Z} \hat{f}_j(r/a)^{|j|}e^{ij\theta}$ (6.18) $ = \hat{f}_0 + \sum_{j=1}^{\infty}\hat{f}_{-j}(r/a)^j e^{-ij \theta} + \sum_{j=1}^{\infty}\hat{f}_j(r/a)^j e^{ij \theta}$ (6.19). The procedure is now alalogous to the one for the HE, $v(r, \theta) = \sum_{j=0}^{\infty} v_j(r, \theta), 0 \leq r < a, \theta \in \R$, with $v_0(r, \theta) = \hat{f}_0$ and  $v_j( r, \theta) = (r/a)^j (\hat{f}_{-j}e^{-ij \theta}+ \hat{f}_j e^{ij \theta}), 0 \leq r \leq a, \theta \in \R, j \in \N$. Notice that $|\hat{f}_j| \leq (1/(2\pi))\int_{-\pi}^{\pi} |f(\theta)|d\theta = c_f$ and $|e^{ij \theta}| = 1$. So $|\partial_r^k \partial_{\theta}^{\ell} v_j (r, \theta)| \leq ((j! j^{\ell})/(j-k)!)  a^{-k} (r/a)^{j-k} 2 c_f, k \leq j, 0 \leq r \leq a, \theta \in \R$, and $\partial_r^k \partial_{\theta}^{\ell} v_j (r, \theta)=0$ if $k > j$. By the ratio test $\sum_{j=k}^{\infty}  ((j! j^{\ell})/(j-k)!)  a^{-k} (r/a)^{j-k} 2 c_f < \infty, 0 \leq r < a$. By T 5.3, $v(r, \theta)$ is infinitely often diff at $0 \leq r < a, \theta \in \R$ and can be diff term by term. We check that each $v_j$ satisfies the LE in polar coord. Since $v_0$ is const, this holds for $v_0$. For $j = 1, \partial_r^2 v_1 = 0, r \partial_r v_1 =  v_1$ and $\partial_{\theta}^2 v_1 = - v_1$.  So $v_1$ satisfies the LE. For $j \geq 2, (r^2 \partial_r^2 + r \partial_r + \partial_{\theta}^2)v_j = (j(j-1)+j - j^2)v_j=0$. Since $v$ can be diff term by term for $r <a, \; v$ also satisfies the LE in the interior of the disk. Notice that $|v_j(r, \theta)| \leq |\hat{f}_{-j}|+|\hat{f}_j|, 0 \leq r \leq a$. Assume that $f$ is Lip cont. By T4.14, $\sum_{j=1}^{\infty} (|\hat{f}_{-j}|+|\hat{f}_j|) < \infty$. By T5.1, $v(r, \theta)$ is cont at $0 \leq r \leq a, \theta \in \R$. We summarize. 
{\bf T 6.3} Let $\Omega$ be a disk in $\R^2$ with the origin as center and $f: \partial \Omega \ra \R$ be Lipschitz cont. Then $\exists$ a fctn $u: \bar{\Omega} \ra \R$ s.t. $u$ is cont on $\bar{\Omega} \setminus \{(0,0)\}, \; u$ is infinitely often diff in $\Omega \setminus \{(0,0)\}$ and $\Delta u = 0 $ on $\Omega \setminus \{(0,0)\}$ and $u = f$ on $ \partial \Omega$. We do not obtain diff at the origin right away because the transformation from polar to rectangular coords is not invertible at the origin. Similarly as for the HE, we have a representation of $v$ via a Green's type fctn.  It holds, if $f$ is just cont. We use the definition of $\hat{f}_j$ in (6.17) and interchange series and int in (6.19), $v(r, \theta) = \int_{-\pi}^{\pi} f(\eta) G(r, \eta - \theta) d \eta, 2 \pi G(r, \theta) = \sum_{j=-\infty}^{\infty}(r/a)^{|j|}e^{-ij \theta}, 0 \leq r < a, \theta \in \R$ (6.20). Similarly as before, it can be shown that $G$ is infinitely often diff for $0 \leq r < a$ and $(r^2 \partial_r^2 + r \partial_r + \partial_{\theta}^2)G(r, \theta) = 0, 0 \leq r < a, \theta \in \R$. Notice that, if $f \equiv 1, v \equiv 1$ is a soln of (6.16). Certainly $v$ is sufficiently smooth to allow all the operations we have done before.  We set $v \equiv f \equiv 1$ in (6.20), $\int_{-\pi}^{\pi} G(r, \eta - \theta) d \eta = 1, 0 \leq r < a, \theta \in \R$ (6.21). Differently from the HE, we can obtain an explicit expression for the GF. $G$ can be rewritten as $2 \pi G(r, \theta)=\sum_{j=0}^{\infty}[(r/a)e^{i \theta}]^j + \sum_{j=0}^{\infty}[(r/a)e^{-i \theta}]^j -1$. The geometric series converge if $r < a, 2 \pi G(r, \theta) = 1/(1-(r/a)e^{i \theta})+ 1/(1-(r/a)e^{-i \theta})-1$. We bring the expression in parentheses into a common denominator $2 \pi G(r, \theta) = (1-(r/a)^2)/(1-(r/a)(e^{i \theta}+e^{-i \theta}) + (r/a)^2)$. This simplifies to $2 \pi G(r, \theta) = (a^2-r^2)/(a^2-2ra \cos \theta +r^2) > 0, 0 \leq r < a, \theta \in \R$ (6.22). We obtain Poisson's formula for the solution of the LE in polar coords, $v (r, \theta) = (1/(2 \pi))\int_{-\pi}^{\pi} f(\eta) (a^2-r^2)/(a^2-2ra \cos ( \eta - \theta) +r^2) d \eta, 0 \leq r < a$ (6.23). Since $r \cos(\eta - \theta) = r(\cos \eta \cos \theta + \sin \eta \sin \theta) = x \cos \eta + y \sin \eta$, we can express the soln in rect coords, $u(x,y) =  (1/(2 \pi))\int_{-\pi}^{\pi} f(\eta) (a^2-x^2-y^2)/(a^2-2ax \cos \eta - 2 a y \sin \eta + x^2 + y^2) d \eta, x^2 + y^2 < a^2$ (6.24). This shows that $u$ is inf often diff at the origin as well. By continuity, it also satisfies the LE at the origin.  Using the properties of the GF, we can extend T6.3 to cont boundary data. 
{\bf T 6.4} Let $\Omega$ be a disk in $\R^2$ and $f: \partial \Omega \ra \R$ be cont. Then there exists a cont fctn $u: \bar{\Omega} \ra \R$ such that $u$ is infinitely often diff in $\Omega$ and $\Delta u = 0$ on $\Omega$ and $u = f$ on $\partial \Omega$. See E 6.2.1. Poisson's formula in rectangular coords can be rewritten in a form that can be generalized to the ball $\Omega$ in $\R^2$ with radius $a$ and arbitrary center, $u(x)=(1/(A(\partial \Omega))) \int_{\partial \Omega} f(y) (a^2 - |x|^2)/(a^2 - 2 \langle x, y \rangle + |x|^2) d \sigma(y) = (1/(A(\partial \Omega))) \int_{\partial \Omega} f(y) (|y|^2 - |x|^2)/(|x-y|^2) d \sigma(y), x \in \Omega$ (6.25). Notice that $x$ and $y$ are now vectors in $\R^2, \; |x|$ is the Euclidean norm of $x$, and $|y| = a$ for $y \in \partial \Omega$. The symbol $d \sigma$ signalized that we take the survace integral over the sphere $\partial \Omega$ in $\R^2$ with radius $a. \; A(\partial \Omega)$ is the surface area of this sphere. This formula generalizes to $\R^n, u(x)=(1/(A(\partial \Omega))) \int_{\partial \Omega} f(y)(|y|^2 - |x|^2)/(|x-y|^n) d \sigma(y), x \in \Omega$ (6.26). See E 6.2.2.

% Exercises
{\bf E 6.2.1}. Use the properties of the GF to derive T6.4 from T6.3.  Hint:  Approximate cont boundary data by Lip cont boundary data.  Then use ideas from T5.19 and C5.20.  Here is a possible seq of steps. Step 1: For a cont $2\pi$-periodic $f: \R \ra \R$ construct a seq $(f_n)$ of Lip cont $2\pi$-periodic functions $f_n: \R \ra \R$ such that $f_n \ra f$ as $n \ra \infty$ uniformly on $\R$. Step 2:  Define $v_n(r,\theta) = \int_{-\pi}^{\pi} f_n(\eta) G(r,\eta-\theta) d\eta, v(r,\theta) = \int_{-\pi}^{\pi} f(\eta) G(r,\eta-\theta) d\eta$. Apply the cosiderations leading to T6.3 to $v_n$ and $v$ and show that $v_n(r, \theta) \ra v(r, \theta)$ as $n \ra \infty$ unif for $0 \leq r < a$ and $\theta \in \R$. Step 3:  Show that $v(r, \theta) \ra f(\theta), r \nearrow a$ unif in $\theta \in \R.$  Step 4: Show that, if we extend $v$ by $v(a, \theta) = f(\theta), \; v$ becomes cont on $[0,a] \times \R$.  {\it Prf}.  Step 1:   We construct a seq $(f_n)$ of Lip cont $2\pi$-periodic functions $f_n: \R \ra \R$ such that $f_n \ra f$ as $n \ra \infty$ unif on $\R$.  Define $f_n(\theta) = n \int_{\theta}^{\theta + (1/n)} f(\eta) d\eta,  n \in \N$. A simple change of var gives us $f_n (\theta) =  n \int_0^1 f(\theta + \eta/n) d\eta,  n \in \N$. Since $f$ is unif cont, $f_n \ra f$ as $n \ra \infty$ unif on $\R$.  Now we diff to get $f_n'(\theta) = n ( f(\theta + 1/n)-f(\theta)).$ Since  $f_n'$ is a function of $f(\theta)$ and $f$ is bdd, then $f_n'$ is also bdd and thus $f_n$ is Lip cont and since $f$ is $2\pi$-periodic, so is $f_n$. Step 2:  We now define $v_n(r,\theta) = \int_{-\pi}^{\pi} f_n (\eta) G(r,\eta-\theta) d\eta, v(r,\theta) = \int_{-\pi}^{\pi} f(\eta) G(r,\eta-\theta) d\eta$.  It follows from the considerations leading to T6.4 (since $G$ and $f_n$ are cont) that $v_n(r, \theta)$ and $v(r, \theta)$  are cont at $0 \leq r < a, \theta \in \R$.  Since $v_n$ and $v$ are $2\pi$- periodic in $\theta, \; v_n$ and $v$ are unif cont on $[0,a) \times \R$.  Since $G$ is non-neg $|v(r, \theta)-v_n(r,\theta)|=|\int_{-\pi}^{\pi} (f(\eta)-f_n(\eta)) G(r,\eta-\theta) d\eta| \leq \int_{-\pi}^{\pi} |f(\eta)-f_n(\eta)| G(r,\eta-\theta) d\eta \leq \int_{-\pi}^{\pi}  G(r,\eta-\theta)d\eta (\sup_{\eta \in \R}|f(\eta)-f_n(\eta)| ) \leq (\sup_{\eta \in \R}|f(\eta)-f_n(\eta)| ) \ra 0$, as $n \ra \infty.$ So $v_n(r, \theta) \ra v(r, \theta)$ as $n \ra \infty$ unif for $0 \leq r < a$ and $\theta \in \R$. Step 3:  $\forall \; n \in \N, 0 \leq r < a$ we apply the TI, $|v(r, \theta) - f(\theta)| \leq |v(r, \theta) - v_n(r, \theta)| + |v_n(r, \theta)-   f_n(\theta)| + | f_n(\theta)- f(\theta)|.$ Let $\epsilon > 0. \; \exists \; n \in \N$ s.t. $| f_n(\theta)- f(\theta)| \leq \epsilon/4$ and $|v(r, \theta) - v_n(r, \theta)| \leq \epsilon / 4 \; \forall \; r \in [0,a)$ and $\theta \in \R$.  Since $v_n$ is unif cont on $[0,a] \times \R, \exists \; \delta \in (0,a)$ s.t. $|v_n(r, \theta)-   f_n(\theta)| \leq \epsilon /4$ if $a-\delta < r < a$. Now we let $a-\delta < r < a$ and $\theta \in \R$. Then we put this together to get $|v(r, \theta) - f(\theta)| < \epsilon$. So $v(r, \theta) \ra f(\theta), r \nearrow a$ unif in $\theta \in \R.$  Step 4:  We know that $v$ is cont on $[0, a) \times \R$ and we know that $f$ is cont.  Let $\epsilon > 0. \; \exists \; \delta >0$ such that $|f(\eta)-f(\theta)| < \epsilon/2$ if $|\eta - \theta| < \delta$. We can choose $\delta$ s.t. $\delta \in (0, a)$ and $|v(r, \xi) - f(\xi)| < \epsilon/2$ whenever $a-\delta < r< a, \xi \in \R$. We use the TI to get $|v(r, \eta) - f(\theta)| \leq |v(r, \eta) - f(\eta)| + | f(\eta) - f(\theta)| < \epsilon.$  The transition between polar and rect coords is cont in both directions on $\R^2 \setminus \{(0,0)\}$ and so $u$ is cont on $\bar{\Omega} \setminus \{(0,0)\}$.  Equation (6.19), Poisson's formula in rectangluar coords, now shows the cont of $u$ at $(0, 0) \qed$ 
{\bf E 6.2.2}. (a) Let $|x|$ be the Euclidean norm of $x \in \R^n$. Show: $\Delta |x| = (n-1)/|x|\; \forall \; x \in \R^n, \; x \neq 0.$ (b) Let $y \in \R^n$ be fixed and define $u: \R^n \ra \R$ by $u(x) = (|y|^2 - |x|^2)|x-y|^{-n}$.  Show that $\Delta u(x) = 0\; \forall \; x \in \R^n, x \neq y$. {\it Prf}. (a) We diff once wrt $x_j$ and get $\partial_j |x| = \partial_j (\sum_{j=1}^n x_j^2)^{1/2} = 2 x_j 1/2 (\sum_{j=1}^n x_j^2)^{-1/2} = x_j/|x|$. Now we diff again wrt $x_j, \partial_j^2 |x| =  \partial_j x_j/|x|= \partial_j x_j (\sum_{j=1}^n x_j^2)^{-1/2} =  -x_j/2 (\sum_{j=1}^n x_j^2 )^{-3/2}(2 x_j)+(\sum_{j=1}^n x_j^2)^{-1/2} = 1/|x|- x_j^2/|x|^3.$  Sum over $j$ to get $ \Delta |x| = n/|x| - |x|^2 / |x|^3 = (n-1) / |x|.$  (b) Take the first PD wrt $x_j,
\partial_j u(x) = (|y|^2 - |x|^2)(-n)|x-y|^{(-n-1)}(1/2)(\sum_{j=1}^n(x_j-y_j)^2)^{-3/2}(2x_j-2y_j) - 2 x_j |x-y|^{-n} = - 2 x_j |x-y|^{-n}-n(|y|^2 - |x|^2)|x-y|^{(-n-2)}(x_j-y_j).$ Now we diff wrt $x_j, \partial_j^2 u(x) = -2|x-y|^{-n} + 2n x_j|x-y|^{(-n-2)}(x_j-y_j)+ 2 x_j n |x-y|^{(-n-2)}(x_j-y_j) + n(n+2)(|y|^2 - |x|^2)|x-y|^{(-n-4)}(x_j-y_j)^2- n (|y|^2 - |x|^2)|x-y|^{(-n-2)}.$  Sum over $j=1, \dots, n$ and define the inner product $\langle x, y \rangle = \sum_{j=1}^nx_jy_j, \Delta u(x) = -2n|x-y|^{-n} + 4n|x-y|^{(-n-2)}\langle x, x-y \rangle  + n (n+2) (|y|^2 - |x|^2)|x-y|^{(-n-4)}|x-y|^2 - n^2(|y|^2 - |x|^2)|x-y|^{(-n-2)}.$ Simplify $\Delta u(x) = -2n|x-y|^{-n} + 4n|x-y|^{(-n-2)}\langle x, x-y \rangle +2n(|y|^2 - |x|^2)|x-y|^{(-n-2)}.$ Factor out $2n|x-y|^{(-n-2)}$ to get $\Delta u(x) = 2n|x-y|^{(-n-2)} (-|x-y|^2 + 2\langle x, x-y \rangle +|y|^2 - |x|^2) = 0 \qed$

%Maximum principles for the Laplacian - Skip to 6.3.2. Weak maximum principle in higher space dimentions
{\bf Weak max principle} Let $\Omega \subseteq \R^n$ be open and bdd.  For a fctn $u: \Omega \ra \R$ that is twice PD wrt $x_k$ at each $x_k \in \Omega, k = 1, \dots, n$, let $(Lu)(x) = \sum_{k=1}^n [a_k \partial_k^2 u(x) + b_k \partial_k u(x)], x \in \Omega$ (6.27), with $a_k \geq 0$ for $k = 1, \dots, n$ and $\sum_{k=1}^n (a_k + |b_k|) > 0$. 
{\bf T6.7}. Assume that $u: \bar{\Omega} \ra \R$ is cont and twice PD on $\Omega$ as above and satisfies $(Lu)(x) \geq 0, x \in \Omega$. Then $\max_{\bar{\Omega}} u = \max_{\partial \Omega} u$. {\it Prf}. Case 1: $(Lu)(x) > 0\; \forall \; x \in \Omega$. Since $u$ is cont on $\bar{\Omega}, \exists \; x \in \bar{\Omega}$ s.t. $u(x) = \max_{\bar{\Omega}} u$. If $x \in \Omega$, then $\partial_j u(x) = 0$ and $\partial_j^2 u(x) \leq 0, j = 1, \dots, n$. So $(Lu)(x) \leq 0$, a contradiction. So $x \in \partial \Omega$ and the assertion follows. Case 2: $(Lu)(x) \geq 0\; \forall \; x \in \Omega$. For $\epsilon > 0$, set $u_{\epsilon} (x) = u(x) + \epsilon c \sum_{j=1}^n \xi_j x_j + \epsilon |x|^2, x \in \bar{\Omega}$, where $\xi_j \in \{0, 1, -1\}$ have the sign of $b_j, |x|$ denotes the Euclidean norm and $c > 0$ will be determined.  Then $\partial_j u_{\epsilon}(x) = \partial_j u(x) + \epsilon c \xi_j + 2 c x_j$ and $\partial_j^2 u_{\epsilon}(x) = \partial_j^2 u(x) + 2 \epsilon$. By (6.27) $(Lu_{\epsilon})(x) = (Lu)(x) + \epsilon \sum_{j=1}^n c|b_j| + 2 \epsilon \sum_{j=1}^n b_j x_j + 2 \epsilon \sum_{j=1}^n a_j$. By the Cauchy-Schwarz inequality in $\R^n$ and $(Lu)(x) \geq 0$, with $b = (b_1, \dots, b_n), (Lu_{\epsilon})(x) \geq \epsilon(c \sum_{j=1}^n|b_j|-2|b||x| + 2 \sum_{j=1}^n a_j)$. Recall that $|b| \leq \sum_{j=1}^n|b_j|$. Since $\Omega$ is bdd, $\exists \; c > 0$ s.t. $|x| \leq c/3 \; \forall \; x \in \Omega$. So $(Lu_{\epsilon})(x) \geq \epsilon \sum_{j=1}^n[|b_j|(c-2|x|) + 2 a_j] \geq \epsilon \sum_{j=1}^n[(c/3)|b_j| + 2 a_j] >0$. By case 1, $\max_{\bar{\Omega}} u_{\epsilon} = \max_{\partial \Omega} u_{\epsilon}$. Now, for $x \in \bar{\Omega}$, by the Cauchy-Schwartz inequality in $\R^n, u(x) \leq u_{\epsilon}(x) + \epsilon c \sqrt{n} |x| \leq u_{\epsilon}(x) + \epsilon c^2 \sqrt{n}$ and $u_{\epsilon}(x) \leq u(x) + \epsilon c^2(\sqrt{n}+1)$. So $\max_{\bar{\Omega}} u \leq \max_{\bar{\Omega}} u_{\epsilon} + \epsilon c^2 \sqrt{n} \leq \max_{\partial \Omega} u_{\epsilon} + \epsilon c^2 \sqrt{n} \leq \max_{\partial \Omega} u + \epsilon c^2(\sqrt{n}+1)$. Since this holds for any $\epsilon > 0, \max_{\bar{\Omega}}u \leq \max_{\partial \Omega} u$. The opposite inequality is trivial $\qed$
{\bf C6.8} Assume that $u: \bar{\Omega} \ra \R$ is cont and twice PD on $\Omega$ as above and satisfies $(Lu)(x) \leq 0, x \in \Omega$. Then $\min_{\bar{\Omega}}u = \min_{\partial \Omega} u$. {\it Prf}. Apply T6.7 to $-u$ and use $\min u = - \max(-u) \qed$

{\bf C6.9} Assume that $u: \bar{\Omega} \ra \R$ is cont and twice PD on $\Omega$ as above and satisfies $(Lu)(x) = 0, x \in \Omega$. Then $\min_{\partial \Omega}u \leq u(x) \leq  \max_{\partial \Omega} u, x \in \Omega$, and $\max_{\bar{\Omega}}|u| = \max_{\partial \Omega} |u|$. {\it Prf}. The 1st inequality is immediate from T6.7 and C6.8. Then, $\forall \; x\in \Omega, - \max_{\partial \Omega} |u| \leq u(x) \leq  \max_{\partial \Omega} |u|$ and $|u(x)| \leq  \max_{\partial \Omega} |u|$. Since this holds $\forall \; x \in \bar{\Omega}, \max_{\bar{\Omega}} |u| \leq  \max_{\partial \Omega} |u|$.  The opposite inequality is trivial. 
{\bf C6.10}. For given $f: \Omega \ra \R$ and $g: \partial \Omega \ra \R \; \exists$ at most one cont fctn $u: \bar{\Omega} \ra \R$ that is twice diff on $\Omega$ and satisfies $Lu=f$ on $\Omega, u = g$ on $\partial \Omega$. {\it Prf}. Assume $\exists$ two such fctns $u_1$ and $u_2$ s.t. $Lu_j = f$ on $\Omega, u_j = g$ on $\partial \Omega, j = 1,2$. Set $v = u_1 - u_2$.  The $Lv = 0$ on $\Omega, v=0$ on $\partial \Omega$. This implies $\max_{\bar{\Omega}} |v| = \max_{\partial \Omega} |v| = 0$. So $u_1(x) = u_2(x)\; \forall \; x \in \bar{\Omega} \qed$
% Exercises
Assume in all these exercises that $\Omega$ is a bdd open subset of $\R^n, \Delta$ the Laplacian and $L$ the PD operator defined in (6.27).
{\bf E6.3.1}. Prove from scratch:  if $u: \bar{\Omega} \ra \R$ is cont and is twice PD on $\Omega$ and satisfies $\Delta u \leq 0, x \in \Omega$, then $\min_{\bar{\Omega}} u = \min_{\partial \Omega} u$. {\it Prf}. Case 1: $\Delta u < 0\; \forall \;  x \in \Omega$. Since $u$ is cont on the closed bdd set $\bar{\Omega} \subseteq \R^n, \exists \; x \in \bar{\Omega}$ s.t. $u(x) \leq u(y)\; \forall \; y \in \bar{\Omega}$. Suppose $x \in \Omega$. Then, for $j = 1, \dots, n, \partial_j u(x) = 0$ and $\partial_j^2 u(x) \geq 0$. We sum over $j$ and obtain $\Delta u(x) \geq 0$, a contradiction.  This proves that $x \in \partial \Omega$ and $\min_{\bar{\Omega}} u = u(x) \geq \min_{\partial \Omega} u$. The opposite inequality holds because $\partial \Omega \subseteq \bar{\Omega}$. Case 2: $\Delta u(x) \leq 0 \; \forall \; x \in \Omega$. For $\epsilon >0$ set $u_{\epsilon}(x) = u(x) - \epsilon |x|^2, x \in \bar{\Omega}$. Then $\partial_j u_{\epsilon}(x) = \partial_j u(x) - 2 \epsilon x_j, \partial_j^2 u_{\epsilon}(x) = \partial_j^2 u(x) - 2 \epsilon$.  We sum over $j$ from 1 to $n, \Delta u_{\epsilon}(x) = \Delta u (x) - 2 n \epsilon \leq -2 n \epsilon < 0, x \in \Omega$. By case 1: $\min_{\bar{\Omega}} u_{\epsilon} = \min_{\partial \Omega} u_{\epsilon}$. Now, $\forall \;x \in \bar{\Omega}, u(x) \geq u_{\epsilon}(x)$ and $u_{\epsilon}(x)\geq u(x) - \epsilon |x|^2 \geq u(x) - \epsilon c^2$ where the const $c > 0$ has been chosen s.t. $|x| \leq c \; \forall \; x \in \Omega$ (recall that $\Omega$ is bdd).  Then $\min_{\bar{\Omega}} u \geq \min_{\bar{\Omega}} u_{\epsilon} = \min_{\partial \Omega} u_{\epsilon} \geq \min_{\partial \Omega} u - \epsilon c^2$.  Since this holds for each $\epsilon > 0, \min_{\bar{\Omega}} u \geq  \min_{\partial \Omega} u$. The opposite inequality holds because $\partial \Omega \subseteq \bar{\Omega} \qed$

{\bf E6.3.2}. Let $g_1, g_2: \partial \Omega \ra \R$ be cont and $u_1, u_2: \bar{\Omega} \ra \R$ be cont on $\bar{\Omega}$ and twice diff on $\Omega$ and $L u_j = 0$ on $\Omega,  u_j = g_j $ on $ \partial \Omega, j = 1, 2.$ Show $\max_{\bar{\Omega}} |u_1 - u_2| = \max_{\partial \Omega} |g_1 - g_2|.$ {\it Prf}. Set $u = u_1 - u_2$. Then $u: \bar{\Omega} \ra \R$ is cont and twice PD on $\Omega$ and satisfies $Lu=0$ on $\Omega$. Further $u = g_1 - g_2$ on $\partial \Omega$. By C 6.9, $\max_{\bar{\Omega}} |u| = \max_{\partial \Omega} |u|.$ So $\max_{\bar{\Omega}} |u_1 - u_2| = \max_{\partial \Omega} |g_1 - g_2| \qed$

{\bf E6.3.5}. Let $\Omega$ be an open bdd subset of $\R^2$. Let $u: \bar{\Omega}\ra \R$ be cont and twice contly diff on $\Omega$ and satisfy $(\partial_x^2 + \partial_y^2)u - \partial_x u + \partial_y u \geq 0,  (x,y) \in \Omega.$ Prove from scratch that $\max_{\bar{\Omega}} u = \max_{\partial \Omega} u$. {\it Prf}. Case 1:  Assume $(\partial_x^2 + \partial_y^2)u - \partial_x u + \partial_y u > 0, (x,y) \in \Omega$. Since $u$ is cont, $\exists$ a point $(x,y) \in \bar{\Omega}$ s.t. $u(x) = \max_{\bar{\Omega}} u$. If $(x,y) \in \Omega, \partial_x u(x,y) = 0 = \partial_y u(x,y)$ and $\partial_x^2 u(x,y) \leq 0$ and $\partial_y^2 u(x,y) \leq 0$. So $(\partial_x^2 + \partial_y^2)u - \partial_x u + \partial_y u \leq  0,$ a contradiction.  So $(x,y) \in \partial \Omega$ and the assertion follows. Case 2:  Assume 
$(\partial_x^2 + \partial_y^2)u - \partial_x u + \partial_y u \geq 0,  (x,y) \in \Omega.\; \forall \;  \epsilon > 0$ set $u_{\epsilon}(x,y) = u(x,y) + \epsilon(x-y)$. Then $(\partial_x^2 + \partial_y^2)u_{\epsilon} - \partial_x u_{\epsilon} + \partial_y u_{\epsilon} = (\partial_x^2 + \partial_y^2)u - \partial_x u + \partial_y u + 2 \epsilon > 0.$ By case 1, $\max_{\bar{\Omega}} u_{\epsilon} = \max_{\partial \Omega} u_{\epsilon}$. Since $\bar{\Omega}$ is bdd, $\exists \; c >0$ s.t. $|x| + |y| \leq c \; \forall \; (x,y) \in \Omega$. So $\max_{\bar{\Omega}} u \leq \max_{\bar{\Omega}} u_{\epsilon} + \epsilon c \leq \max_{\partial \Omega} u_{\epsilon} + \epsilon c \leq \max_{\partial \Omega} u + 2 \epsilon c.$
Since this holds for any $\epsilon > 0, \max_{\bar{\Omega}} u \leq \max_{\partial \Omega} u$. Since the opposite inequality is trivially true, equality holds $\qed$
% 6.4 The Laplace equation on a rectangle once more
Consider (PDE) $(\partial_x^2 + \partial_y^2)u(x,y) = 0, 0 < x < L, 0 < y < H$, (BC) $u(0,y) = g_1(y), u(L,y)=0, 0 < y < H, u(x,0)=0, u(x,H) = 0, 0 < x < L$ (6.28). This time we only assume that $g_1$ is cont and $g_1(0) = 0 = g_1(H)$. As in the proof of T5.16, we find a sequence of Lip cont fctns which are zero at 0 and $H$ that converges to $g_1$ unif on $[0, H]$. Every Lip cont fctn that is zero at 0 and $H$ can be unif approximated by its Fourier sine series.  This implies that $\exists$ a seq of inf often diff fctns $\tilde{g}_n : [0, H] \ra \R$ such that $\tilde{g}_n \ra g_1$ as $n \ra \infty$ unif on $[0, H], \tilde{g}_n(0) = 0 = \tilde{g}_n (H), \tilde{g}_n''(0) = 0 = \tilde{g}_n''(H)$. Let $u_n$ be the solution of the BVP (6.28) with $\tilde{g}_n$ replacing $g_1, \Omega = (0,L) \times (0, H)$. These solutions exist by T6.1.  By E 6.3.2, $\max_{\bar{\Omega}} |u_n - u_m| = \max_{\partial \Omega} |u_n - u_m| = \max_{[0,L]} |\tilde{g}_n - \tilde{g}_m|$. This implies that, for each $x \in [0, L], y \in [0, H], (u_n(x,y))$ is a (unif) Cauchy sequence.  Let $u(x,y) = \lim_{n \ra \infty} u_n(x,y)$. Then $u_n \ra u$ as $n \ra \infty$ unif on $\bar{\Omega} = [0, L] \times [0, H]$ and $u$ is cont, and satisfies the BC in (6.28).  In order to show that $\Delta u = 0$ on $\Omega$, let $z_0 \in \Omega$. Since $\Omega$ is open, there is an open disk $D$ with center $z_0$ and radius $a$ such that $\bar{D}$ is contained in $\Omega$. Let $D_0$ be the open disk with center $(0, 0)$ and radius $a$. Set $\tilde{u}_n(z) = u_n(z + z_0), \tilde{u}(z) = u(z + z_0), z \in \tilde{D}_0, f_n(\theta) = u_n (z_0 + a (\cos \theta, \sin \theta)), f(\theta) = u(z_0 + a (\cos \theta, \sin \theta)), \theta \in \R$. Then $\Delta \tilde{u}_n = 0$ on $D_0$ and $\tilde{u}_n (a \cos \theta, a \sin \theta) = f_n(\theta)$ for $\theta \in \R$. By (6.24), $\tilde{u}_n(x,y) = 1/(2 \pi) \int_{-\pi}^{\pi} f_n (\eta) (a^2 - x^2 - y^2)/(a^2 - 2 a x \cos \eta - 2 a y \sin \eta + x^2 + y^2)d \eta, x^2 + y^2 < a^2$. Since $\tilde{u}_n \ra \tilde{u}$ unif on $\bar{D}_0$ and $f_n \ra f$ unif on $\R$, we can take the limit as $n \ra \infty, \tilde{u}(x,y) = 1/(2 \pi) \int_{-\pi}^{\pi} f (\eta) (a^2 - x^2 - y^2)/(a^2 - 2 a x \cos \eta - 2 a y \sin \eta + x^2 + y^2)d \eta, x^2 + y^2 < a^2$. Then $\tilde{u}$ is infinitely often diff on $D_0$ and satisfies LE there. So $u$ is infinitely often diff on $D$ and satisfies LE on $D$. Since we can find such a disk around any $z_0 \in \Omega, u$ is infinitely often diff on $\Omega$ and $\Delta u = 0$ on $\Omega$.
{\bf T6.11}. Let $\Omega=(0,L) \times (0,H)$ and $g: \partial \Omega \ra \R$ cont.  Then $\exists$ a cont fctn $u: \bar{\Omega} \ra \R$ that is infinitely often diff on $\Omega$ and satisfies $\Delta u = 0$ on $\Omega$ and $u=g$ on $\partial \Omega$. {\it Prf}. Splitting the BVP into four parts and adding the four solutions yields a solution of the original problem provided that $g$ is zero at the four corners of the rectangle. Suppose that this is not the case.  Let $\phi(x,y) = a_0 + a_1 x + a_2 y + a_3 xy$. Then $\Delta \phi = 0$. We determine the coeff in such a way that $\phi$ equals $g$ at the corners of the rectangle.  For the origin, we get $a_0 = g(0,0)$. For $(0,L), g(L,0) = a_0 + a_1L$. and so $a_1 = (g(L,0)-a_0)/L$. Similarly $a_2 = (g(0,H)-a_0)/H$. Finally $g(L, H)=a_0+a_1 L + a_2 H + a_3 L H$, from which we determine $a_3$. By our previous consideration, $\exists$ a cont fctn $\tilde{u}$ which is infinitely often diff on $\Omega$ and satisfies $\Delta \tilde{u} = 0$ and $\tilde{u} = g - \phi$ on $\partial \Omega$. We set $u = \tilde{u} + \phi$. Then $u$ has all the required properties$\qed$
% Chapter 7 Gauss' thm and Green's formulas
Let $\Omega$ be an open bdd subset of $\R^n$ and $f: \Omega \ra \R^n$ be differentiable, $f(x) = (f_1(x), \dots, f_n(x))$. Then div$f(x) := \sum_{j=1}^n \partial_j f_j(x), x \in \Omega$ (7.1). $\Omega$ is called normal if the divergence thm (Gau$\ss$' integral thm) holds for every cont fctn $f: \bar{\Omega} \ra \R^n$ with cont bdd derivative on $\Omega: \int_{\Omega}$div$f(x)dx = \int_{\partial \Omega}f(x) \cdot \nu(x) d \sigma(x)$(7.2). Here $\nu(x)$ is the outer unit normal vector at $x \in \partial \Omega: \exists \; \epsilon > 0$ s.t. $x + \xi \nu (x) \ni \bar{\Omega}, x - \xi \nu (x) \in \Omega, 0 < \xi < \epsilon$. 
The notation $d \sigma (x)$ signalized that we take a surface integral. $f(x) \cdot \nu (x) = \sum_{j=1}^n f_j(x)\nu_j(x)$ is the Euclidean inner product in $\R^n$. Balls with respect to the three standard norms and Cartesian products of intervals are normal.  For $\Omega$ to be normal, $\partial \Omega$ must allow surface integration and have a cont outer normal, but additional assumptions must be satisfied.  This is an equivalent componentwise formulation of Gau$\ss$' thm.
{\bf T7.1}. $\Omega$ is normal iff $\int_{\Omega} \partial_j g(x) dx = \int_{\partial \Omega} g(x) \nu_j (x) d \sigma(x), j = 1, \dots, n$. for every cont $g: \bar{\Omega} \ra \R$ with cont and bdd derivative on $\Omega$. {\it Prf}. $\Rightarrow$ Let $j \in \{1, \dots, n\}$ and set $f = (0, \dots, 0, g, 0, \dots, 0)$ such that $f_j = g. \Leftarrow$ Apply this formula to $g = f_j, j = 1, \dots, n$ and add over $j \qed$ The following result generalizes IBP. 

{\bf T7.2} (Green's thm). Let $\Omega$ be normal and consider 2 cont fctns $u, v: \bar{\Omega} \ra \R$ with cont bdd derivatives on $\Omega$. Then $\int_{\Omega} (u \partial_j v + v \partial_j u)dx = \int_{\partial \Omega} uv \nu_j d\sigma, j = 1, \dots, n$. {\it Prf}. Set $g = uv$ in T7.1$\qed$

{\bf T7.3} (GFs). Let $\Omega, u$, and $v$ be as in the previous thm. Assume that the derivative of $v$ can and has been contly extended to $\bar{\Omega}$. If $v$ has a cont bdd second derivative on $\Omega, \int_{\Omega} u \Delta v dx + \int_{\Omega} \nabla u \cdot \nabla v dx = \int_{\partial \Omega} u \partial_{\nu} v d \sigma$. If both $u$ and $v$ have bdd cont second derivatives on $\Omega$ and if the first derivatives of $u$ and $v$ can and have been contly extended to $\bar{\Omega}, \int_{\Omega} (u \Delta v - v \Delta u)dx = \int_{\partial \Omega} (u \partial_{\nu} v - v \partial_{\nu} u) d \sigma$. Here $\nabla u \cdot \nabla v = \sum_{j=1}^n \partial_j u \partial_j v$ and $\partial_{\nu}v = \nu \cdot \nabla v = \sum_{j=1}^n \nu_j \partial_j v$. We mention that $\partial_{\nu} v$ is called the normal derivative of $v$. {\it Prf}. In Green's thm, replace $v$ by $\partial_j v, \int_{\Omega} (u \partial_j^2 v + \partial_j v \partial_j u)dx = \int_{\partial \Omega} u \partial_j v \nu_j d \sigma$, and add over $j$. For the 2nd formula, use the symmetry in $u$ and $v$ and subtract$\qed$

 %The following result will be used a couple of times. 
{\bf P7.4}. Let $\Omega \subseteq \R^n$ be open and $u: \Omega \ra \R$ be diff and $\nabla u \equiv 0$ on $\Omega$. Then $u$ is locally const on $\Omega$: for each $x \in \Omega \; \exists$  some open nbhood $U$ of $x$ s.t. $u$ is cons on $U$. If $\Omega$ is path-connected, then $u$ is const on $\Omega$.
\textbf{Recall} that $\Omega$ is path-connected if, for any $x, y \in \Omega$, there is an interval $[a,b]$ and a cont "path" $\gamma: [a,b] \ra \Omega$ s.t. $\gamma(a) = x$ and $\gamma(b)=y$. 
{\it Prf}. Let $x \in \Omega$. Since $\Omega$ is open, $\exists \; r > 0$ s.t. $U_r(x) = \{z \in \R^n; |z-x|<r \} \subseteq \Omega$.
%$ \; U_r(x)$ is the open ball with center $x$ and radius $r$. 
Let $z \in U_r(x)$. Then, $\forall \; \xi \in [0,1], |\xi z + (1-\xi)x - x|=|\xi(z-x)| \leq |z-x| < r$ and $\xi z + (1 - \xi)x \in U_r(x) \subseteq \Omega$. Thus $\phi(\xi) = u(\xi z + ( 1- \xi)x)$ is defined $\forall \; \xi \in [0,1]$ and diff, $\phi'(\xi) = \nabla u (\xi z + (1-\xi)x) \cdot (z - x) = 0$. So $u(z) = \phi(1) = \phi(0) = u(x), z \in U_r(x)$. Now assume that $\Omega$ is path-connected. Let $x,y \in \Omega$. Assume that $u(x) \neq u(y)$. There is an interval $[a,b]$ and a cont path $\gamma : [a,b] \ra \Omega$ such that $\gamma(a) = x$ and $\gamma(b)=y$ and so $\gamma(a) \neq \gamma(b)$. Let $t = \sup\{s \in [a,b]; u(\gamma(s))=u(\gamma(a))\}$. Then $a \leq t < b$ and $u(\gamma(t))=u(\gamma(a))$ because $u$ is cont. Since $u$ is locally const, $\exists$ an open ball $U$ with center $u(\gamma(t))$ such that $u(z) = u(\gamma(t))\; \forall \; z \in U$. Since $t < b$ and $\gamma$ is continous, $\exists \; s >t$ s.t. $\gamma(s) \in U$ and so $u(\gamma(s)) = u(\gamma(t)) = u(\gamma(a))$, a contradiction to the definition of $t$.  So $u(x) = u(y)$ for any two points $x$ and $y$ in $\Omega$, i.e. $u$ is const on $\Omega \qed$

% Boundary value problems for the Laplace equation
Consider  $\Omega \subseteq \R^n$. $\Delta u = f$ on $\Omega, \beta(x) \partial_{\nu} u(x) + \alpha(x) u(x) = g(x), x \in \partial \Omega$ (7.3). Here $\alpha$ and $\beta$ are nonneg cont real-valued fctns on $\partial \Omega, \alpha(x) + \beta(x) > 0 \; \forall \; x \in \partial \Omega$. Special cases:  $u = g$ on $\partial \Omega$: Dirichlet BCs, $\partial_{\nu}u = g$ on $\partial \Omega$: Neumann BCs, $\partial_{\nu} u + \alpha(x)u = g$ on $\partial \Omega$: Robin BCs.
{\bf T7.5}.  Any 2 solns of (7.3) that have cont bdd 2nd partial derivatives on $\Omega$ and whose 1st derivatives and themselves can be contly extended to $\bar{\Omega}$ have identical gradients. If $\Omega$ is path-connected, they are equal up to a const. If, in addition, $\alpha(x) \neq 0$ for some $x \in \partial \Omega$, there is at most one soln.  {\it Prf}. Let $u_1$ and $u_2$ be two solns.  Then $u = u_1 - u_2$ is cont on $\bar{\Omega}$, its 1st derivative can be contly extended to $\bar{\Omega}$ and $u$ is twice contly diff on $\Omega$.  Further $\Delta u = 0$ on $\Omega, \beta(x) \partial_{\nu} u(x) + \alpha(x) u(x) = 0, x \in \partial \Omega$. By GF1, $0 \leq \int_{\Omega} |\nabla u|^2 dx = \int_{\Omega} \nabla u \cdot \nabla u dx + \int_{\Omega} u \Delta u dx = \int_{\partial \Omega}u \partial_{\nu}u d \sigma$. If $x \in \partial \Omega$, there are two cases:  either $\beta(x) = 0$, then $\alpha(x) u(x) = 0$ and $\alpha(x) > 0$ and so $u(x) = 0$, or $\beta (x) > 0$, then $\partial_{\nu} u(x) = - \alpha(x)/\beta(x) u(x)$. So $0 \leq \int_{\Omega} | \nabla u|^2 dx = - \int_{\partial \Omega \cap \{\beta > 0 \}} \alpha(x)/\beta(x) u^2(x)d \sigma(x) \leq 0$. This implies that $0 = \int_{\Omega}|\nabla u|^2 dx$. Since $\nabla u$ is cont, $\nabla u \equiv 0$ on $\Omega$. If $\Omega$ is path-connected, $u$ is const by P 7.4. Then $0 \equiv \alpha u$ on $\partial \Omega$. So, if $\alpha(x) > 0$ for some $x \in \partial \Omega, u \equiv 0$ on $\Omega \qed$  Neumann BVP $\Delta u = f$ on $\Omega, \partial_{\nu} u(x) = g(x), x \in \partial \Omega$ (7.4). From GF1, for sufficiently smooth $v, \int_{\Omega} \nabla v \cdot \nabla u dx + \int_{\Omega} v \Delta u dx = \int_{\partial \Omega}v \partial_{\nu} u d \sigma$. So, for $v \equiv 1, \int_{\Omega} \Delta u dx = \int_{\partial \Omega} \partial_{\nu} u d \sigma$ and so $\int_{\Omega} f dx = \int_{\partial \Omega}  g d \sigma$. 
{\bf T7.6}. The Neumann BVP $\Delta u = f$ on $\Omega, \partial_{\nu} u = g$ on $\partial \Omega$ only has a soln if $\int_{\Omega} f dx = \int_{\partial \Omega}  g d \sigma$. If $u$ is a soln, then $\tilde{u}(x) = u(x) + c$ with a const $c$ is also a soln. 
% Boundary value problems for the heat equation
HE in several space dims, $(\partial_t - a \Delta_x - c(x,t))u = f(x,t), x \in \Omega, t \in (0, T), \beta(x,t)\partial_{\nu} u(x,t) + \alpha(x,t)u(x,t)=0, x \in \partial \Omega, t \in (0,T)$ (7.5). $\Omega$ is a normal subset of $\R^n, \; a$ is a pos const, $\Delta_x = \sum_{j=1}^n \partial^2/\partial x_j^2, \alpha$ and $\beta$ are cont non-neg fctns on $\partial \Omega \times (0, T)$ and $\alpha + \beta$ is strictly pos.  Further $c$ and $f$ are cont bdd fctns on $\Omega \times (0, T)$. We derive an energy estimate for $\int_{\Omega}u^2(x,t)dx$. Assume that $u$ is cont on $\bar{\Omega}\times [0,T]$ and that the PDs $\partial_t u$ exist and are cont on $\bar{\Omega} \times (0, T)$. Assume that $u(x,t)$ is twice contly diff in $x$ and that $\nabla_x u$ can be contly extended to $\bar{\Omega} \times (0, T)$. Then the derivative $\partial_t u^2 = 2 u \partial_t u$ exists and is cont on $\bar{\Omega} \times (0,T)$. So $\int_{\Omega} u^2(x,t) dx$ is diff in $t \in (0, T)$ and $d/dt \int_{\Omega} u^2(x,t) dx= \int_{\Omega} \partial_t u^2(x,t) dx = \int_{\Omega} 2 u(x,t) \partial_t u(x,t) dx= \int_{\Omega} 2 u(x,t) ((a \Delta_x + c(x,t))u(x,t)+ f(x,t)) dx= 2a \int_{\Omega}u(x,t) \Delta_xu(x,t)dx+2 \int_{\Omega}c(x,t)u^2(x,t)dx + 2 \int_{\Omega} u(x,t)f(x,t)dx$. By GF1, $\int_{\Omega} u(x,t) \Delta_x u(x,t) dx$ = $- \int_{\Omega} \nabla_x u(x,t) \cdot \nabla_x u(x,t)dx + \int_{\partial \Omega} u(x,t) \partial_{\nu} u(x,t) d \sigma(x)$ = $- \int_{\Omega}|\nabla u(x,t)|^2 dx - \int_{\partial \Omega \cap \{\beta > 0\}} \alpha(x,t)/\beta(x,t) u^2(x,t) d\sigma(x) \leq 0$. So $\partial_t \int_{\Omega} u^2(x,t)$
$dx \leq 2 \int_{\Omega} c(x,t) u^2(x,t) dx + 2 \int_{\Omega} u(x,t)f(x,t)dx$. Let $\bar{c} = \sup_{\Omega \times (0,T)}c$. Choose some $\epsilon > 0$. By the Cauchy-Schwarz ineq, $\partial_t \int_{\Omega}u^2(x,t)dx \leq 2 \bar c \int_{\Omega} u^2(x,t) dx + 2(\int_{\Omega}u^2(x,t)dx)^{1/2}(\int_{\Omega} f^2(x,t)dx)^{1/2} \leq (2 \bar c + \epsilon)\int_{\Omega}u^2(x,t)dx+ 1/\epsilon \int_{\Omega} f^2 (x,t)dx$. Here we have used the ineq $2rs \leq \epsilon s^2 +(1/\epsilon)r^2$. Set $\kappa = 2 \bar c + \epsilon$. Using an integrating factor, we obtain $\int_{\Omega} u^2(x,t) dx \leq e^{\kappa t} \int_{\Omega} u^2 (x,0)dx + 1/\epsilon \int_0^t e^{k(t-s)}(\int_{\Omega}f^2(x,s)dx)ds$. Assume $\bar c < 0$. Then choose $\epsilon > 0$ s.t. $\kappa < 0 \implies e^{\kappa t} \int_{\Omega} u^2 (x,0)dx  \ra 0, t \ra \infty$ and $1/\epsilon \int_0^t e^{k(t-s)}(\int_{\Omega}f^2(x,s)dx)ds \leq \sup_{s \in (0,t)} \int_{\Omega}f^2(s,x) dx 1/(-\kappa \epsilon)$. Notice that $-\kappa \epsilon = -(2\bar c + \epsilon)\epsilon = (2|\bar c|-\epsilon)\epsilon$ takes its maximum at $\epsilon = |\bar c|$ where it is $\bar c^2$. So we pick the estimate, $\int_{\Omega} u^2(x,t)dx \leq e^{\bar c t} \int_{\Omega}u^2(x,0)dx + 1/ \bar c^2 \sup_{0<s<t} \int_{\Omega} f^2$
$(x, s)dx$. 

%7.3 Boundary value problems for the wave equation
Consider the WE on a normal subset $\Omega$ of $\R^n. (\partial_t^2 - c^2 \Delta_x)u = 0, x \in \Omega, t \in (0, T), u(x,t) = g(x), x \in \partial \Omega, t \in (0, T), u(x,0) = \phi(x), x \in \Omega, \partial_tu(x,0) = \psi(x), x \in \Omega$ (7.6). Define $E(t) = 1/2 \int_{\Omega}((\partial_tu(x,t))^2 + c^2|\nabla_xu(x,t)|^2)dx$ (7.7). 
%This expression is often interpreted as the energy of the solution. Here $\nabla_x u$ is to be understood as the gradient with respect to the spatial variable $x$. 
Assume that $u$ is twice contly diff on $\Omega \times (0, T)$ and that the 1st and 2nd PDs are bdd.  Assume that $\nabla_x u$ and $\partial_t u$ can be contly extended to $\Omega \times (0, T)$. Then $\partial_t(\partial_tu)^2 = 2 \partial_tu \partial_t^2 u$ exists and is cont and bdd on $\Omega \times (0, T)$. Moreover $\partial_t |\nabla_xu|^2 = 2 \nabla_x u \cdot \partial_t \nabla_x u = 2 \nabla_x u \cdot \nabla_x \partial_t u$ exists and is cont and bdd on $\bar{\Omega} \times (0, T)$. This implies that $E$ is diff and diff and int can be interchanged and $E'(t) = \int_{\Omega}(\partial_tu(x,t)\partial_t^2u(x,t) + c^2 \nabla_x u(x,t)\cdot \partial_t \nabla_x u(x,t))dx = c^2(\int_{\Omega} \partial_t u(x,t) \Delta_x u(x,t)dx + \int_{\Omega} \nabla_xu(x,t)\cdot \nabla_x \partial_tu(x,t))dx)$. By GF1, $E'(t) = c^2 \int_{\partial \Omega}\partial_t u(x,t) \partial_{\nu} u(x,t) d\sigma(x)$. Since $u(x,t) = g(x)$ for $x \in \partial \Omega, t \in (0, T), \partial_t u(x,t)=0$ for $x \in \partial \Omega, t \in (0,T)$. So $E'(t) = 0\; \forall \; t \in (0, T)$. Thus $E(t) = E(0) = 1/2 \int_{\Omega} (\psi^2(x) + c^2|\nabla \phi(x)|^2)dx$. 
% Exercises
In the following, $\Omega$ is always a normal set contained in $\R^n$. 
%{\bf E7.3.1}. Consider a cont fctns $u: \bar{\Omega} \ra \R$ with cont bdd derivatives on $\Omega$. Show $\int_{\Omega} u \partial_j u dx = 1/2 \int_{\partial \Omega}u^2 \nu_j d \sigma, j = 1, \dots, n$. 
{ \bf E7.3.3}. Consider the following version of the Neumann boundary problem for the LE. $ \Delta u(x) + c(x) u(x) = f(x), \quad x \in \Omega, \partial_{\nu} u(x) = g(x), \quad x \in \partial \Omega$. Find a sign cond for the fctn $c$ that guarantees that there is at most one soln $u$ even if $\Omega$ is not path-connected (with prf). {\it Prf}.  Let $u_1$ and $u_2$ be two solns and $v = u_1 - u_2$. Then $\Delta v(x) + c(x) v(x) = 0, \quad x \in \Omega, \partial_{\nu} v(x) = 0, \quad x \in \partial \Omega$. By GF1, $0 = \int_{\Omega} v(\Delta v(x) + c(x) v(x))dx = - \int_{\Omega} \Delta v \cdot \Delta v dx +\int_{\Omega} v \partial_{\nu} v \partial \sigma +\int_{\Omega} c(x) v^2(x) dx  \leq \int_{\Omega} c(x) v^2(x) dx.$ Suppose that $c(x) < 0\; \forall \; x \in \Omega$. Then $0 \leq \int_{\Omega} c(x) v^2(x) dx \leq 0 \quad \text{ and } \quad \int_{\Omega} c(x) v^2(x) dx = 0.$
Assume that $c$ is cont. Then $c(x) v(x) = 0\; \forall \; x \in \Omega$ and $v(x) = 0\; \forall \; x \in \Omega \qed$. 

{\bf E7.3.4}.  Consider the Newmann boundary problem for the LE. $\Delta u(x) + c(x) u(x) = f(x), \quad x \in \Omega,  \partial_{\nu} u(x) = g(x), \quad x \in \partial \Omega$.  Assume that $\Omega$ is path-connected, $c: \Omega \ra \R$ is nonpos and cont and $c(x) < 0$ for some $x \in \Omega$. Show that there is at most one soln $u$. {\it Prf}. Let $u_1$ and $u_2$ be two solns and $u=u_1 - u_2$.  Then $ \Delta u(x) + c(x) u(x) = 0, \quad x \in \Omega, \partial_{\nu} u(x) = 0, \quad x \in \partial \Omega$. By GF1, $0 \leq \int_{\Omega} | \nabla u(x)|^2 = \int_{\Omega} \nabla u(x) \cdot \nabla u(x) dx + \int_{\Omega}(\Delta u(x) + c(x) u(x)) u(x) dx = \int_{\Omega} \nabla u(x) \cdot \nabla u(x) dx +\int_{\Omega}\Delta u(x) u(x) +  \int_{\Omega}  c(x) u(x)^2 dx = \int_{\partial_\Omega}u(x) \partial_{\nu}u(x) d \sigma + \int_{\Omega}  c(x) u(x)^2 dx=\int_{\Omega}  c(x) u(x)^2 dx$. Because $c(x) < 0$ we have that $0 \leq \int_{\Omega} | \nabla u(x)|^2 =\int_{\Omega}  c(x) u(x)^2 dx \leq 0$. And so by P 7.4, $\nabla u(x) = 0$ on $\Omega$ and so $u(x)$ is const.  Also $0 = \int_{\Omega}  c(x) u(x)^2 dx = u(x)^2 \int_{\Omega}  c(x) dx .$ Because $c(x) \neq 0, \int_{\Omega}  c(x) dx \neq 0$ and therefore  $u(x)^2 = 0$ and so $u(x) = 0$ which shows that $u_1 = u_2$ and so there is at most one soln $u \qed$ 

{\bf E7.3.5}. Consider the HE with mixed BC $(\partial_t - a \Delta_x - c(t,x))u=f(t,x),  x \in \Omega, \; t \in (0, T), \beta(x,t)\partial_{\nu}u(x,t)+\alpha(x,t)u(x,t) = g(x,t),  x \in \partial \Omega, \; t \in (0, T), u(x,0) =\phi(x),  x \in \Omega,$ with a bdd cont fctn $c: \Omega \times (0, T) \ra \R$ and $\alpha$ and $\beta$ as before. Show:  Given $f, g, \phi$, there is at most one soln u that satisfies smoothness assumptions of Section 7.2. {\it Prf}. Let $u_1$ and $u_2$  be 2 solutions. Set $v = u_1 - u_2$. Then  $(\partial_t - a \Delta_x - c(t,x))v=0,  x \in \Omega, \; t \in (0, T), \beta(x,t)\partial_{\nu}v(x,t)+\alpha(x,t)v(x,t) = 0, x \in \partial \Omega, \; t \in (0, T), v(x,0) =0,  x \in \Omega,$ By the considerations in Section 7.2, $\int_{\Omega} v^2(x,t) dx \leq 0.$ Since this integral is non-neg, it is 0. Since $v^2$ is cont and non-neg, $v^2(x,t)=0$ and so we have that  $u_1(x,t) = u_2(x,t)\; \forall \; t \in (0,T), x \in \Omega \qed$ 
{\bf E7.3.8}. Consider the WE with Dirichlet BCs $ (\partial_t^2 - c\Delta_x)u = f(x,t), \quad x \in \Omega, \; t \in (0, T), u(x,t) = g(x, t),  \quad x \in \partial \Omega, \; t \in (0, T), u(x,0) = \phi(x), \quad x \in \Omega, \partial_t u(x,0) = \psi(x), \quad x \in \Omega$. Show: there exists at most one solution $u$ that satisfies the smoothness assupmtions of section 7.3. {\it Prf}. Let $u_1$ and $u_2$  be 2. Set $v = u_1 - u_2.  (\partial_t^2 - c\Delta_x)v =0, \; x \in \Omega, \; t \in (0, T), v(x,t) = 0,  \; x \in \partial \Omega, \; t \in (0, T), v(x,0) = 0, \; x \in \Omega, \partial_t v(x,0) = 0, \; x \in \Omega$.  By the considerations in Section 7.3, $0 = E(0) = E(t) = 1/2 \int_{\Omega} ((\partial_tv(x,t))^2 + c^2|\nabla v(x,t)|^2)dx$.  Since the integrand is cont and non-neg, $(\partial_tv(x,t))^2 + c^2|\nabla v(x,t)|^2 = 0.$  In particular, $\partial_t v(x,t) = 0$ and $v(x,t) = v(x,0) = 0$. So $u_1(x,t) = u_2(x,t)\; \forall \; t \in (0,T), x \in \Omega \qed$
{\bf E7.3.10}. Let $X$ be a Hilbert space with inner product $\langle \cdot, \cdot \rangle$. Let $I$ be an interval with end points $a$ and $b, a < b$, and $f, g: I \ra X$ two functions that are diff at some point $t \in I$. Define $\phi: I \ra \C$ by $\phi(t) = \langle f(t), g(t) \rangle$. Show that $\phi$ is diff at $t$ and derive a formula for $\phi'(t)$. {\it Prf}. Let $t, s \in I$.  By the properties of the inner product, $\phi(s) - \phi(t) = \langle f(s) - f(t), g(s) \rangle + \langle f(t), g(s) - g(t) \rangle$, and $(\phi(s) - \phi(t))/(s-t) = \langle (f(s) - f(t))/(s-t), g(s) \rangle + \langle f(t), (g(s) - g(t))/(s-t) \rangle$. Since the inner product is a continuous function from $X \times X \ra \C$ as a consequence of the Cauchy Schwarz inequality and $(f(s) - f(t))/(s-t) \ra f'(t), (g(s) - g(t))/(s-t) \ra g'(t), g(s) \ra g(t), s \ra t$, we have $\phi(s) - \phi(t))/(s-t) \ra  \langle f'(t), g(t) \rangle + \langle f(t), g'(t) \rangle \qed$
% Chapter 9 The heat equation on the whole real line
Consider $u: \R \times \R_+ \ra \R$ solving $\partial_t u(x,t) = \partial_x^2 u(x,t), x \in \R, t > 0, u(x,0) = u_0(x), x \in \R$ (9.1), for $u_0: \R \ra \R$. 
% 9.1 The Gaussian kernel
Let $\Gamma (x,t) := (4 \pi t)^{-1/2}e^{-x^2(4t)^{-1}}, x \in \R, t > 0$ (9.2). $\Gamma$ is strictly pos and inf often diff, $\partial_x \Gamma(x,t) = - x/(2t)^{-1}\Gamma(x,t), \partial_x^2 \Gamma(x,t) = x^2(2t)^{-2} \Gamma(x,t) - (2t)^{-1}\Gamma (x,t), \partial_t \Gamma(x,t) = -2 \pi(4\pi t)^{-3/2}e^{-x^2(4t)^{-1}} + x^2 4 /(4t)^{-2} \Gamma (x,t) = \partial_x^2 \Gamma(x,t)$ (9.3). Notice $\Gamma$ and these PDs are bdd and integrable on $\R \times [\epsilon, c]$ for any $c > \epsilon > 0$ as are all higher derivatives. We combine the 2nd and 3rd eq and reorg, $\partial_t \Gamma(x,t) = [x^2 - (2t)](2t)^{-2}\Gamma(x,t)$. 
{\bf L9.1}. (a) for each $x \in \R, \Gamma(x, \cdot)$ is strictly increasing on $(0, x^2/2)$ and strictly decreasing on $(x^2/2, \infty)$. (b) For each $t > 0, \Gamma(\cdot, t)$ is strictly increasing on $(-\infty, 0)$, and strictly decreasing on $(0, \infty)$. 
Also, by the change of variables $x = y(4t)^{1/2}, \int_{\R} \Gamma(x,t)dx = (\pi)^{-1/2} \int_{\R}e^{-y^2}dx = 1$ (9.4). Notice that, for any $t > 0, \Gamma (\cdot, t)$ is the probability density of a normal (or Gau$\ss$) distribution with mean 0 and variance $2t$ (E 9.1.2). By the same change of variables, $\int_{\R \setminus[-\epsilon, \epsilon]} \Gamma(x,t)dx = 2 \int_{\epsilon}^{\infty} (4 \pi t)^{-1/2} e^{-x^2(4t)^{-1}} dx = 2 \int_{\epsilon/(4t)^{-1/2}}^{\infty}(\pi)^{-1/2}e^{-y^2} dy \ra 0, t \ra 0$ (9.5). Define $u(x,t) = \int_{\R} \Gamma(x-y,t)u_0(y)dy, x \in \R, t>0$ (9.6). We substitute $y=x-z, u(x,t)= \int_{\R} \Gamma (z,t)u_0(x-z)dz$ (9.7). Assume that $u_0$ is measurable and either integrable or bdd.  Because of the props of $\Gamma$, we can interchange diff and int in (9.6) and $\partial_t u(x,t) = \int_{\R} \partial_t \Gamma (x-y,t) u_0(y)dy = \int_{\R} \partial_x^2 \Gamma(x-y,t)u_0(y)dy = \partial_x^2 u(x,t)$. 

{\bf P9.2}. Let $u_0: \R \ra \R$ be bdd and cont. Then the fctn $u: \R \times \R_+ \ra \R$ defined by $u(x,t)=\int_{\R} \Gamma(t,x-y) u_0(y) dy, t > 0, x \in \R, u(x,0) = u_0(x), x \in \R$, is cont and bdd.  Hint:  Show $u(x,t) \ra u_0(x)$ as $t \ra 0$ unif on each bdd subset of $\R$. {\it Prf}. Since $u_0$ is bdd, its abs value has a sup in $\R$ and $|u(x,t)| \leq \int_{\R} \Gamma(t,x-y) \sup_{z \in \R}|u_0(z)| dy = \int_{\R} \Gamma(t,z)dz \sup_{z \in \R}|u_0(z)| = \sup_{z \in \R}|u_0(z)|$. By (9.4).  By (9,7) and (9.4), $u(x,t) - u_0(x) = \int_{\R} \Gamma(z,t)(u_0(x-z)-u_0(x))dz$. Hence $|u(x,t) - u_0(x)| \leq \int_{\R} \Gamma(z,t)|u_0(x-z)-u_0(x)|dz$. Recall that $u_0$ is bdd and cont.  Let $\delta \in (0,1)$. Split up the int; $\forall x \in \R, |u(x,t)-u_0(x)| \leq \int_{\R \setminus[-\delta, \delta]} \Gamma(z,t)|u_0(x-z)-u_0(x)| dz + \int_{-\delta}^{\delta} \Gamma (z,t) |u_0(x-z)-u_0(x)| dz \leq 2 \sup|u_0| \int_{\R \setminus[-\delta, \delta]} \Gamma(z,t) dz + \int_{-\delta}^{\delta} \Gamma (z,t)dz\sup_{|z|<\delta}|u_0(x-z)-u_0(x)| \leq 2 \sup|u_0|\int_{\R \setminus[-\delta, \delta]} \Gamma(z,t) dz +\sup_{|z|<\delta}|u_0(x-z)-u_0(x)|$. In the last inequality we used $\int_{-\delta}^{\delta} \Gamma (z,t)dz \leq \int_{\R} \Gamma (z,t)dz=1$. Let $B$ be a bdd subset of $\R$. Then  $\exists \; n \in \N$ s.t. $B \subseteq [-n, n]$. Let $\epsilon > 0$. Since $u_0$ is unif cont on $[-(n+1), n+1], \exists \; \delta \in (0,1)$ s.t. $|u_0(y)-u_0(x)| < \epsilon/3 \; \forall y, x \in [-(n+1), n+1]$ with $|y-x| < \delta$. Let $x \in [-n, n]$. If $|z| < \delta \leq 1$, then $x-z \in [-(n+1), n+1]$ and $|(x-z) - x| = |z| < \delta$, hence $|u_0(x-z)-u_0(x)| < \epsilon/3$. This implies $\sup_{|z| < \delta}|u_0(x-z)-u_0(x)| \leq \epsilon/3 < \epsilon/2, x \in B \subseteq [-n, n]$. By (9.5), $\exists \; \eta > 0$ s.t. $2 \sup|u_0|\int_{\R \setminus[-\delta, \delta]} \Gamma(z,t) dz < \epsilon/2, 0 < t < \eta$. So $|u(x,t)-u_0(x)| < \epsilon\; \forall \; x\in B$ if $0 < t<\eta$. Continuity of $u$ at any point $(x,t)$ with $t > 0$ follows from the properties of $\Gamma$. Let $x \in \R$. Let $\epsilon > 0$. Since $u(t,y) \ra u_0(y)$ as $t \ra 0$ unif for $y \in [x-1, x+1], \exists \; \eta > 0$ s.t. $|u(t,y)-u_0(y)| < \epsilon/2$ whenever $t \in [0, \eta)$ and $y \in [x-1, x+1]$. Since $u_0$ is cont, $\exists \; \delta_1 > 0$ s.t. $|u_0(x)-u_0(y)| < \epsilon/2 \; \forall y \in \R$ with $|y-x| < \delta_1$. Set $\delta = \min \{\delta_1, \eta, 1\}$. Let $y \in \R$ and $|y-x|+t<\delta$. Then $|y-x| < \delta$ and $|y-x| < 1$ and so $y \in [x-1, x+1]$. Further $t < \eta$. By the TI, $|u(y,t)-u(x,0)| \leq |u(y,t)-u_0(y)|+|u_0(y)-u_0(x)| < \epsilon/2 + \epsilon/2 = \epsilon \qed$
{\bf T9.3}. If $u_0$ is measurable and either integrable or bdd, formula (9.6) provides a solution of the HE on $\R \times (0, \infty)$.  If $u_0$ is bdd and cont on $\R$ and $u(x,0):=u_0(x)$ for $x \in \R$, then $u$ is bdd and cont on $\R \times [0, \infty) \qed$. We investigate under which conditions on $u$, a soln $u$ of (9.1) would  necessarily be given by (9.6). To start, we assume that $u: \R \times \R_+ \ra \R$ is cont and bdd on $\R \times [0, T]$ for every $T \in (0, \infty)$. Fix $t > 0$ and $x \in \R$. Define $v(s) = \int_{\R} \Gamma(t-s, x-y) u(y,s)dy, 0 \leq s < t$, and, for each $n \in \N, v_n(s) = \int_{-n}^n \Gamma(t-s, x-y) u(y,s)dy, 0 \leq s < t$.  Choose $\epsilon \in (0, t/2)$ and assume that $u$ and $\partial_t u, \partial_x u$ and $\partial_x^2 u$ are bdd on $[\epsilon, t]$. Then by the properties of $\Gamma, v_n$ is diff on $(\epsilon, t)$ and $v_n'(s) = \int_{-n}^n \partial_s [ \Gamma(t-s, x-y) u(y,s)]dy$. By the product rule, $v_n'(s) = \int_{-n}^n [\partial_s  \Gamma(t-s, x-y)] u(y,s)dy + \int_{-n}^n  \Gamma(t-s, x-y) \partial_s u(y,s)dy $. Using the respective PDEs, $v_n'(s) = \int_{-n}^n [-\partial_y^2  \Gamma(t-s, x-y)] u(y,s)dy + \int_{-n}^n  \Gamma(t-s, x-y) \partial_y^2 u(y,s)dy $.  We IBP, $v_n'(s) = \partial_y  \Gamma(t-s, x+n) u(-n,s) - \partial_y  \Gamma(t-s, x-n) u(n,s) +   \Gamma(t-s, x-n) \partial_y u(n,s)  - \Gamma(t-s, x+n) \partial_y u(-n,s)$. By the props of $\Gamma$ and $u$, the rhs converges to 0 as $n \ra \infty$ unif for $s \in (\epsilon, t-\epsilon)$. Further $v_n(s) \ra v(s)$ as $n \ra \infty, s \in (\epsilon, t-\epsilon)$. This implies that $v$ is diff on $(\epsilon, t-\epsilon)$ and $v'(s) = 0$. So $v$ is const on $[\epsilon, t-\epsilon]$ and $\int_{\R} \Gamma(\epsilon, x-y)u(y, t-\epsilon)dy = \int_{\R} \Gamma(t- \epsilon, x-y)u(y, \epsilon)dy$ (9.8). Let us assume that $u, \partial_t u, \partial_x u, \partial_x^2 u$ are bdd on $\R \times [\delta, 1/\delta]$ for every $\delta \in (0,1)$. Then $u(y, t-\epsilon) \ra u(y,t)$ as $\epsilon \ra 0$ unif in $x \in \R$ and $u(\cdot, t)$ is unif cont.  So $|\int_{\R} \Gamma(\epsilon, x-y)u(y, t-\epsilon)dy - u(x,t)| \leq |\int_{\R} \Gamma(\epsilon, x-y)u(y, t-\epsilon)dy - \int_{\R} \Gamma(\epsilon, x-y)u(y, t)dy| + | \int_{\R} \Gamma(\epsilon, x-y)u(y, t)dy - u(x,t)|$. Since the last term tends to 0 as $\epsilon \ra 0$, we only need to deal with the last but one term estimated by $\int_{\R} \Gamma(\epsilon, x-y)|u(y, t- \epsilon)- u(y,t)|dy \leq \int_{\R} \Gamma(\epsilon, x-y) dy \sup_y|u(y, t- \epsilon)- u(y,t)| \ra 0, \epsilon \ra 0$. Further, by Lebesque' thm of dominated convergence, $\int_{\R} \Gamma(t- \epsilon, x-y) u(y, \epsilon)dy \ra \int_{\R} \Gamma (t, x-y) u(y,0) dy$. Notice that, for $0< \epsilon < t/2, \Gamma( t- \epsilon, x) \leq (2 \pi t)^{-1/2}e^{-x^2(4t)^{-1}} \leq \sqrt{2} \Gamma(x,t)$ (9,9). Take the lim of (9.8) as $\epsilon \ra 0$ and obtain $u(x,t) = \int_{\R} \Gamma(t, x-y) u(y,0)dy$. 
{\bf T9.4}. Let $u:\R \times \R_+ \ra \R$ be cont and bdd on $\R \times [0, T]$ for every $T \in (0, \infty)$. Further, let $\partial_t u, \partial_x u, \partial_x^2 u$ be bdd on $\R \times [\epsilon, 1/\epsilon]$ for every $\epsilon \in (0, 1)$. Then, if $u$ solves (9.1), $u(x,t) = \int_{\R} \Gamma(t, x-y) u_0(y) dy, t > 0$. Further, if $u_0$ is unif cont and bdd, this formula gives a sol of (9.1) with the props above. Fix $r >0$ and define $u(x,t) = \Gamma(t+r,x), x \in \R, t \geq 0$. Then $\partial_t u = \partial_x^2 u$ and $u$ satisfies all the assumptions we made before.  This yields $\Gamma(t+r, x) = \int_{\R} \Gamma(t, x-y) u(y,0)dy = \int_{\R}\Gamma(t, x-y) \Gamma (r,y)dy$ (9.10). This means that $\Gamma$ satisfies the Chapman-Kolmogorov equation. Consider the HE in  the space of integrable fctns $L^1(\R)$. Similarly as before, $u$ given by (9,6) solves the PDE on $\R \times (0, \infty)$. We assume that $u_0$ is non-neg and measurable with finite integral $\int_{\R}u_0(y)dy$ and $u$ given by (9.6). Then $u$ is non-neg and, by Tonelli's thrm, $\int_{\R}u(x,t)dx = \int_{\R}(\int_{\R} \Gamma(t,x-y) dx) u_0(y) dy=\int_{\R} u_0(y)dy, t > 0$. But, by (9.2), $u(t,x) \leq (4 \pi t)^{-1/2} \int_{\R}u_0(y)dy \ra 0, t \ra \infty$, unif for $x \in \R$. In what sense does $u$ given by (9.6) satisfy the initial condition?
{\bf T9.5}. Let $u_0$ be measurable with finite integral $\int_{\R}|u_0(y)|dy$ and $u$ be given by (9.6). Then $\int_{\R} |u(x,t) - u_0(x)|dx \ra 0$ as $t \ra 0$. We use a result from integration theory, namely, that $C_0(\R)$, the space of cont fctns with compact support, is dense in $L^1(\Omega)$. Let $\epsilon >0$. Then $\exists \; f \in C_0(\R)$ s.t. $\int_{\R}|u_0(x) - f(x)|dx < \epsilon/6$. By the TI, $\int_{\R}|u(x,t) - u_0(x)|dx \leq \int_{\R}|\int_{\R}\Gamma(t,x-y)u_0(y)dy - f(x)|dx + \int_{\R}|f(x)-u_0(x)|dx \leq \int_{\R}|\int_{\R}\Gamma(t,x-y)[u_0(y) - f(x)]dy|dx +\int_{\R}|\int_{\R}\Gamma(t,x-y)f(y)dy - f(x)|dx + \int_{\R}|f(x)-u_0(x)|dx  \leq \int_{\R}(\int_{\R}\Gamma(t,x-y)|u_0(y) - f(y)|dy)dx +\int_{\R}|\int_{\R}\Gamma(t,x-y)f(y)dy - f(x)|dx + \int_{\R}|f(x)-u_0(x)|dx$. By Fubini's thrm, $\int_{\R}(\int_{\R}\Gamma(t,x-y)|u_0(y) - f(y)|dy)dx =\int_{\R}(\int_{\R}\Gamma(t,x-y)dx) |u_0(y) - f(y)|dy= \int_{\R} |u_0(y) - f(y)|dy$. So $\int_{\R} |u(x,t) - u_0(x)|dx \leq \int_{\R}|\int_{\R}\Gamma(t,x-y) f(y)dy - f(x)|dx +2\int_{\R}|f(x) - u_0(x)|dx$ (9.11). Since $f \in C_0(\R)$, by E 9.1.1, $|\int_{\R}\Gamma(t,x-y) f(y)dy - f(x)| \ra 0, t \ra 0$, unif for $x \in \R$. Further $\exists \; b >0$ s.t. $f(x) =0$ for $x \in \R \setminus (-b,b)$. We split up the 1st integral on the rhs of inequality (9.11), $\int_{\R}|\int_{\R}\Gamma(t,x-y) f(y)dy - f(x)|dx \leq \int_{-2b}^{2b}|\int_{\R}\Gamma(t,x-y) f(y)dy - f(x)|dx+ \int_{2b}^{\infty}(\int_{\R}\Gamma(t,x-y) |f(y)|dy)dx + \int_{-\infty}^{-2b}(\int_{\R}\Gamma(t,x-y) |f(y)|dy)dx$ (9.12). $\exists \; \delta_1 > 0$ s.t. $|\int_{\R}\Gamma(t,x-y) f(y)dy - f(x)|<\epsilon/(36 b), t \in (0, \delta_1)$, and so $\int_{-2b}^{2b}|\int_{\R}\Gamma(t,x-y) f(y)dy - f(x)|dx < \epsilon/(9b), t \in (0, \delta_1)$ (9.13). Now, with $c = \sup|f|$, by Funbini's thm, $\int_{2b}^{\infty}(\int_{\R}\Gamma(t,x-y) |f(y)|dy)dx\leq \int_{2b}^{\infty}(\int_{-b}^b \Gamma(t,x-y) c dy)dx =c \int_{-b}^b(\int_{2b}^{\infty} \Gamma(t,x-y) dx)dy$.  By a change of variables, $\int_{2b}^{\infty} \int_{\R}\Gamma(t,x-y) f(y) dy dx \leq c \int_{-b}^b (\int_{2b-y}^{\infty} \Gamma(t,x) dx)dy \leq c \int_{-b}^b(\int_b^{\infty} \Gamma(t,x) dx)dy \leq 2cb \int_b^{\infty}(4 \pi t)^{-1/2}e^{-x^2 (4t)^{-1}} dx$. By a change of variables, $\int_{2b}^{\infty} \int_{\R}\Gamma(t,x-y) f(y) dy dx \leq 2 c b \pi^{-1/2}\int_{b(4\pi t)^{-1/2}}^{\infty} e^{-x^2}dx \ra 0$ as $t \ra 0$. So $\exists \; \delta_2 >0$ s.t. $\int_{2b}^{\infty} \int_{\R}\Gamma(t,x-y) f(y)dydx < \epsilon/9, t \in (0, \delta_2)$ (9.14). Similarly, $\exists \; \delta_3 > 0$ s.t. $\int_{-\infty}^{-2b} \int_{\R}\Gamma(t,x-y) |f(y)|dydx < \epsilon/9, t \in (0, \delta_3)$ (9.15). We set $\delta = \min_{j=1}^3 \delta_j$ and combine the inequalities (9.12) - (9.15), $\int_{\R}|\int_{\R}\Gamma(t,x-y) f(y)dy - f(x)|dx < \epsilon/3, t \in (0, \delta)$. This, combined with (9.11) yields $\int_{\R}|u(x,t) - u_0(x)|dx < \epsilon, t \in (0, \delta)$.
% Exercises
{\bf E9.1.1}. Let $u_0: \R \ra \R$ be bdd and unif cont. Define $u: \R \times \R_+ \ra \R$ by $u(x,t) = \int_{\R} \Gamma(x-y, t)u_0(y)dy, t > 0, x \in \R, u(x,0) = u_0(x), x \in \R.$  Show (a) $u(x,t) \ra u(x,0)$ as $t \ra 0$, unif for $x \in \R$. (b) For any $t >0, u(x,t)$ is a unif cont fctn of $x \in \R$.
{\it Prf}. (a) By (9.7) and (9.4) $u(x,t) - u_0(x) = \int_{\R} \Gamma(z,t)(u_0(x-z)-u_0(x))dx$. Hence $|u(x,t) - u_0(x)| \leq  \int_{\R} \Gamma(z,t) |(u_0(x-z)-u_0(x))|dx$. Assume that $u_0$ is bdd and unif cont. Let $\epsilon > 0. \; \exists \delta > 0$ s.t. $|u_0(x-z) - u_0(x)| < \epsilon/2$ if $z, x \in \R$ and $|z| < \delta$. Hence $|u(x,t) - u_0(x)| \leq  \int_{\R \setminus [-\delta, \delta]} \Gamma(z,t) |u_0(x-z)-u_0(x)|dz + \int_{-\delta}^{\delta} \Gamma(z,t) |u_0(x-z)-u_0(x)|dz \leq 2$ sup$|u_0| \int_{\R \setminus [-\delta, \delta]} \Gamma(z,t) dz + \epsilon/2 \int_{-\delta}^{\delta} \Gamma(z,t) dz \leq 2$ sup$|u_0| \int_{\R \setminus [-\delta, \delta]} \Gamma(z,t) dz + \epsilon/2$. Now $\exists \; \eta > 0$ s.t. 2 sup$|u_0| \int_{\R \setminus [-\delta, \delta]} \Gamma(z,t) dz < \epsilon/2, 0 < t < \eta$. So $|u(x,t) - u_0(x)| < \epsilon \; \forall \; x \in \R$ if $0 < t < \eta$. (b) Choose $\delta > 0$ s.t. $|u_0(x) - u_0(\tilde{x})| < \epsilon \; \forall \; x, \tilde{x} \in \R$ with $|x - \tilde x | < \delta$. Let $x, \tilde x \in \R$ with $|x - \tilde x | < \delta$. Then $|(x-z) - (\tilde x - z) | < \delta \; \forall \; z \in \R$ and, by (9.7) and (9.4), $|u(x,t) - u( \tilde x, t))| \leq \int_{\R} \Gamma(t,z)|u_0(x-z)- u_0(\tilde x - z)|dz \leq  \int_{\R} \Gamma(t,z) \epsilon dz = \epsilon \qed$
{\bf E9.1.5} Let $\Gamma$ be the fund soln of the HE in 1 space dim defined by (9.2). Let $b: \R_+ \ra \R$ and $a: \R_+ \ra \R$ be cont. Let $B(t) = \int_0^t b(s) ds$ and $A(t) = \int_0^t a(s) ds$. Assume $A(t) > 0\; \forall \; t>0$. Define $\tilde{\Gamma} (t,x) = \Gamma(x+B(t), A(t)), x \in \R, t>0$. (a) Show that $\partial_t \tilde{\Gamma}(t,x) = b(t)\partial_x\tilde{\Gamma}(t,x) + a(t) \partial_x^2 \tilde{\Gamma}(t,x)$. (b) Show that $\int_{\R}\tilde{\Gamma}(t,x)dx=1, \int_{\R}x \tilde{\Gamma}(t,x)dx=-B(t)$ and $\int_{\R} (x+ B(t))^2\tilde{\Gamma}(t,x)dx=2 A(t)$. 
%(This means that, for any $t>0, \tilde{\Gamma}(t, \cdot)$ is the probability density of a normal distribution with mean $-B(t)$ and variance $2A(t)$.) Remark:  As you notice, notation for the fundamental solution is not consistent as to whether the space variable come first or the time variable. 
{\it Prf}. (a) By the chain rule and the fund thm of calc, $\partial_t \tilde{\Gamma}(t,x) = \partial_1 \Gamma(x + B(t), A(t))b(t) + \partial_2 \Gamma(x + B(t), A(t))a(t)$. Again, by the chain rule, and by the PDE for $\Gamma, \partial_t \tilde{\Gamma}(t,x) = b(t) \partial_x \Gamma(x+B(t), A(t)) + a(t) \partial_1^2 \Gamma(x+B(t), A(t))$. Use chain rule, $\partial_t \tilde{\Gamma}(t,x) = b(t)\partial_x \tilde{\Gamma}(t,x) + a(t) \partial_x^2 \Gamma(x+B(t), A(t)) = b(t)\partial_x \tilde{\Gamma}(t,x) + a(t) \partial_x^2 \tilde{\Gamma}(t, x)$. (b) Subst $x = y - B(t), \int_{\R} \tilde{\Gamma}(t,x) dx$ = $\int_{\R} \Gamma (x + B(t), A(t)) dx$ = $\int_{\R}\Gamma(y, A(t))dy = 1, \int_{\R} x \tilde{\Gamma}(t,x) dx$ = $ \int_{\R} x \Gamma (x + B(t), A(t)) dx$ = $\int_{\R}(y-B(t))\Gamma(y, A(t))dy$ = $\int_0^{\infty} y \Gamma (y, A(t))dy - \int_0^{\infty} y \Gamma (-y, A(t))dy - B(t) \int_{\R} \Gamma(y, A(t))dy$. Since $\Gamma( -y, A(t)) = \Gamma (y, A(t))$ and $\int_{\R} \Gamma(y, A(t))dy = 1$, the assertion follows.  Finally, $V(t) := \int_{\R} (x + B(t))^2 \tilde{\Gamma} (t,x) dx$ = $\int_{\R} (x + B(t))^2 \Gamma (x + B(t), A(t)) dx$ = $\int_{\R} y^2 \Gamma (y, A(t)) dy = \int_{\R} y^2 (4 \pi A(t))^{-1/2} e^{-y^2(4A(t))^{-1}} dy$.  We substitute $y = (4 A(t))^{1/2} z$ and IBP, $V(t) = (\pi)^{-1/2} 4 A(t) \int_{\R} z^2 e^{-z^2}dz = - (\pi)^{-1/2} 2 A(t) \int_{\R} z (d/dz) e^{-z^2}dz$ = $- (\pi)^{-1/2} 2 A(t) \int_{\R} z  e^{-z^2}dz]_{z = - \infty}^{z = \infty} +  (\pi)^{-1/2} 2 A(t) \int_{\R} z  e^{-z^2}dz = 2 A(t) \qed$

{\bf Higher space dimension}. We look for a function $u: \R^n \times \R_+ \ra \R$ solving $\partial_t u(x,t) = \Delta_x u(x,t), x \in \R^n, t > 0, u(x,0) = u_0 (x), x \in \R^n$ (9.19), for a given function $u_0: \R^n \ra \R$. 
{\bf P9.6}. Define $G(x,t) = \Pi_{i=1}^n \Gamma(x_i, t), t >0, x = (x_1, \dots, x_n) \in \R^n$. Then: (a) $\partial_t G(x,t) = \Delta_x G(x,t), t > 0, x \in \R^n$.  (b) $\int_{\R^n} G(x,t) dx = 1\; \forall \; t > 0$. (c) $G(x,t) = (4 \pi t)^{-n/2} e^{-||x||^2(4t)^{-1}}$, where $|| \cdot ||$ is the Euclidean norm on $\R^n$. (d) For every $\epsilon > 0, \int_{||x|| \geq \epsilon} G(x,t) dx \ra 0$ as $t \ra 0$. (e) (Chapman-Kolmogorov eq) $\forall \; t,r \in [0, \infty), x \in \R^n, G(x,t+r) = \int_{\R^n} G(x-y, t) G(y,r)dy$. {\it Prf}. (a) By the product rule, $\partial_t G(x,t) = \sum_{j=1}^n ( \Pi_{i \neq j} \Gamma (x_i, t))\partial_t \Gamma(x_j,t)$ = $\sum_{j=1}^n ( \Pi_{i \neq j} \Gamma (x_i, t))\partial_{x_j}^2 \Gamma(x_j,t)$ = $ \sum_{j=1}^n \partial_{x_j}^2 ( \Pi_{i \neq j} \Gamma (x_i, t))\Gamma(x_j,t)$  = $\Delta_x G(x,t)$. (b) We use induction over $n$. Write $G_n$ for $G$. We know the statement to be true for $n =1$. Let $n \in \N$ be arbitrary and assume that the statement is true for $n$. Notice that, for $y \in \R^n$ and $z \in \R, G_{n+1}((y,z),t) = G_n(y,t)\Gamma(z,t)$. Now, by the thms of Fubini or Tonelli, $\int_{\R^{n+1}} G_{n+1}(x, t) dx$ = $ \int_{\R^n} \int_{\R} G_n (y, t) \Gamma (z, t) dy dz$ = $( \int_{\R} G_n (y, t) dy)(\int_{\R} \Gamma (z, t) dz) = 1 \cdot 1 = 1$. (c) $G(x, t) = \Pi_{i=1}^n ((4 \pi t)^{-1/2} e^{-x_i^2(4t)^{-1}})$ = $(4 \pi t)^{-n/2} \Pi_{i=1}^n e^{-x_i^2(4t)^{-1}}$ = $(4 \pi t)^{-n/2} \exp ( - \sum_{i=1}^n x_i^2 (4t)^{-1})$ = $(4 \pi t)^{-n/2} e^{-||x||^2(4t)^{-1}}$. (d) Let $\epsilon > 0$. We substitute $x = t^{1/2}y, \int_{||x||\geq \epsilon} G(x,t)dx$ = $(4 \pi )^{-n/2} \int_{||y||\geq \epsilon t^{-1/2}} e^{-||y||^2 /4}$ = $\int_{||y||\geq \epsilon t^{-1/2}} G(y,1) dy \ra 0$, as $t \ra 0$. (e) Let $t, r \in [0, \infty), x \in \R^n. \int_{\R^n} G(x-y, t) G(y,r) dy$ = $ \int_{\R^n} \Pi_{j=1}^n \Gamma(x_j - y_j, t) \Pi_{j=1}^n \Gamma(y_j, r) dy$ = $\int_{\R^n} \Pi_{j=1}^n [\Gamma(x_j - y_j, t)  \Gamma(y_j, r) ]dy$ = $  \Pi_{j=1}^n \int_{\R} [\Gamma(x_j - y_j, t)  \Gamma(y_j, r) ]dy_j$ = $ \Pi_{j=1}^n \Gamma(x_j. t+r) = G(x,t+r) \qed$
{\bf R9.7} $\int_{||x|| \geq \epsilon} G(x,t) dx = \pi^{-2/n} \int_{||z||\geq \epsilon (4t)^{-1/2}}e^{-||z||^2}dz$ = $\pi^{-2/n} $vol$(U_1(0))\int_{ \epsilon (4t)^{-1/2} }^{\infty} n r^{n-1} e^{-r^2} dr$. For $n = 2$, using polar coords, $\int_{||x|| \geq \epsilon} G(x,t) dx$ = $e^{-\epsilon^2(4t)^{-1}}$.  For $n = 4, \int_{||x|| \geq \epsilon} G(x,t) dx$ = $\pi^{-1/2}$vol$(U_1^4(0)) 2 ( \int_{ \epsilon (4t)^{-1/2} }^{\infty} r^2 (d/dr) e^{-r^2} dr)$ = $\pi^{-1/2}$vol$(U_1^4(0)) 2 ( \epsilon^2 (4t)^{-1} e^{-\epsilon^2 (4t)^{-1}} + \int_{ \epsilon (4t)^{-1/2} }^{\infty}2 r  e^{-r^2} dr)$ = $\pi^{-1/2}$vol$(U_1^4(0)) 2 (\epsilon^2 (4t)^{-1} e^{-\epsilon^2 (4t)^{-1}}$ + $e^{-\epsilon^2 (4t)^{-1}} = \pi^{-1/2} $vol$(U_1^4 (0)) 2 (\epsilon^2 (4t)^{-1} + 1) e^{-\epsilon^2 (4t)^{-1}}$.

{\bf T9.8}. Let $u_0: \R^n \ra \R$ be bdd and unif cont.  Define the fctn $u: \R^n \times (0, \infty) \ra \R$ by $u(x,t) = \int_{\R^n} G(x-y, t) u_0(y) dy, t>0, x \in \R^n$. Then (a) $u(x,t) \ra u(x,0)$ as $t \ra 0$, unif for $x \in \R^n$. (b) For any $t > 0, u(x,t)$ is a bdd unif cont fctn of $x \in \R$. {\it Prf}. By a change of variables $y \mapsto x - z, u(x,t) = \int_{\R^n} G(z, t) u_0(x-z) dz, x \in \R^n, t \in (0, \infty)$ (9.20). (a) By (9.20) and P 9.6 (b) and the nonneg of $G, |u(x,t)| \leq \int_{\R^n} G(z, t) |u_0(x-z) |dz \leq \sup_{\R^n} |u_0|$ (9.21). So $u$ is bdd and $\sup_{x \in \R^n, t > 0} |u(x,t)| \leq \sup_{\R^n} |u_0|$. Again by (9.20), P 9.6 (b), $u(x,t) - u_0(x) = \int_{\R^n} G(z,t)(u_0(x-z)-u_0 (x)) dx$. Since $G$ is non-neg, $|u(x,t) - u_0(x)| = \int_{\R^n} G(z,t)|(u_0(x-z)-u_0 (x)) |dx$. Assume that $u_0$ is bdd and unif cont. Let $\epsilon > 0$. Then $ \exists \; \delta >0$ s.t. $|u_0(x-z)-u_0 (x)| < \epsilon/2$ if $z, x \in \R^n$ and $||z|| < \delta$. We split up the integral accordingly, $|u(x,t) - u_0(x)| \leq  \int_{||z|| \geq \delta} G(z,t)|u_0(x-z)-u_0 (x)|dz +  \int_{||z|| < \delta} G(z,t)|u_0(x-z)-u_0 (x)|dz \leq 2 \sup|u_0|\int_{||z|| \geq \delta} G(z,t) dz + \epsilon/2 \int_{||z|| < \delta} G(z,t)dz \leq 2 \sup|u_0|\int_{||z|| \geq \delta} G(z,t) dz + \epsilon/2$. By P 9.6 (d), $\exists \; \eta > 0$ s.t. $2 \sup|u_0|\int_{||z|| \geq \delta} G(z,t) dz < \epsilon/2, 0 < t < \eta$. So $|u(x,t) - u_0(x)| < \epsilon \; \forall \; x \in \R^n$ if $0<t< \eta$. (b) Choose $\delta > 0$ s.t. $|u_0(x) - u_0(\tilde{x})|  < \epsilon \; \forall \; x, \tilde{x}, \in \R^n$ with $||x - \tilde{x}|| < \delta$. Let $x, \tilde{x} \in \R^n$ with $||x - \tilde x||< \delta$. Then $||(x - z) - (\tilde{x} -z)|| < \delta \; \forall \; z \in \R$ and, by (9.20) and P 9.6 (b),  $|u(x,t) - u(\tilde{x}, t)| \leq \int_{\R^n} G (t,z)|u_0(x - z) - u_0(\tilde{x} - z)| dz \leq \int_{\R^n} G (t,z)\epsilon dz = \epsilon \qed$  
Let $X = BUC(\R^n)$ be the Banach space of unif cont bdd fctns from $\R^n$ to $\R$ with the sup norm.  For $t>0$, define $(S(t)f)(x) = \int_{R^n} G(x-y,t)f(y)dy, f \in X$. Then $S(t)$ is a bdd linear operator on $X$ with $||S(t)||=1, S(t)S(r) = S(t+r), r,t > 0$ (9.22) and $S(t)f \ra f, t \ra 0, f \in X$ (9.23). Families of operators with these properties are called $C_0$-(operator-)semi-groups. That $S(t)$ maps $X$ into $X$ follows from part (b) of the last thm while (9.23) follows from part (a). (9.22) follows from the Chapman-Kolmogorov equations by switching the order of integration.  Let $f \in BUC(\R^n), t,r > 0, x \in \R^n$, and $g=S(r)f, [S(t)S(r)f](x) = [S(t)g](x) = \int_{\R^n}G(x-y,t) g(y)dy = \int_{\R^n}G(x-y,t) (\int_{\R^n}G(y-z,t) f(z)dz)dy$ = $ \int_{\R^n}(\int_{\R^n}G(x-y,t) G(y-z,t) dy) f(z)dz$. We substitute $y = \tilde y + z$ and use the Chapman-Kolmogorov equations, $[S(t)S(r)f](x) =\int_{\R^n}(\int_{\R^n} G(x-\tilde{y}-z,t) G(\tilde{y},t) d\tilde{y}) f(z)dz$  = $\int_{\R^n} G(x-z,t+r) f(z)dz = [S(t+r)f]{x}$. Since this holds $\forall \; x \in \R^n$, we have $S(t)S(r)f = S(t+r)f$. The linearity of $S(t)$ follows from the linearity of the integral.  The boundedness of $S(t)$ and $||S(t)|| = 1$ follows from $\int_{\R^n} G(xt)dx = 1, ||S(t)f||_{\infty} \leq ||f||_{\infty}$, with $||f||_{\infty} = \sup_{x \in \R^n} |f(x)|$. See (9.21). Notice that $BUC(\R^n)$ contains the const functions and $S(t)f = f$ for any const function $f: \R^n \ra \R$. 

% Identities
{\bf Identities} $\sin^2x = (1-\cos 2x)/2, \cos^2x = (1+\cos 2x)/2$, $e^{ix}=\cos x + i \sin x, e^{-ix}=\cos x - i \sin x$,  $2\cos x=(e^{ix}+e^{-ix}), 2i\sin x=(e^{ix}-e^{-ix}), 2 \cosh x=(e^x+e^{-x}), 2\sinh x=(e^x-e^{-x}). \cosh^2 x-\sinh^2x=1$. 
{\bf Sum and Difference Formula} $\sin(A\pm B)=\sin A \cos B\pm \cos A \sin B$. $\cos(A\mp B)=\cos A \cos B\pm \sin A \sin B$. $\tan(A \pm B)=(\tan A\pm \tan B)/(1\mp \tan A \tan B)$.
{\bf Double Angle Formula}  $\sin(2A)=2 \sin A \cos A$. $\cos(2A)=\cos^2 A-\sin^2 A=2 \cos^2 A-1=1-2\sin^2 A$. $\tan(2A)=(2\tan A)/(1-\tan^2 A)$. 
{\bf Half Angle Formula} $\sin(A/2)=\pm \sqrt{(1-\cos A)/2}$. $\cos(A/2)=\pm \sqrt{(1+\cos A)/2}$. $\tan(A/2)=(1-\cos A)/(\sin A) = (\sin A)/(1+\cos A)$. 
{\bf Product to Sum} $\cos A \cos B=(1/2)(\cos(A+B)+\cos(A-B))$. $\sin A \sin B=(1/2)(\cos(A-B)-\cos(A+B)$. $\sin A \cos B=(1/2)(\sin(A+B)+\sin(A-B)$. 
{\bf Sum to Product} $\sin A\pm \sin B=2\sin((A\pm B)/2)\cos((A\mp B)/2)$. $\cos A - \cos B=-2\sin((A+B)/2)\sin((A-B)/2)$. $\cos A + \cos B=2\cos((A+B)/2)\cos((A-B)/2)$. 
{\bf Geometric Sum} $\sum_{k=1}^{\infty}q^k=q/(1-q)$. $\sum_{k=1}^n q^k=(q-q^{n-1})/(1-q)$. 
{\bf General ODE Solutions}  $y''=y(t)\implies y=c_1e^{-t}+c_2e^t \qed \; dy/dt+p(t)y=g(t) \implies y=(\int u(t)g(t))/u(t) + c$ where $u(t)=$exp$(\int p(t)dt) \qed \; y'=x; x'=y \implies x=c_1 \cosh t + c_2 \sinh t, y=c_1\sinh t+c_2 \cosh t$ or $x=c_1e^t+c_2e^{-t}, y=c_1e^t-c_2e^{-t} \qed \; y'=-x; x'=y \implies y=c_1 \cos t + c_2 \sin t, x=c_1\sin t-c_2 \cos t \qed \; x'=x+y; y'=-x+y \implies x=e^t(c_1 \cos t+c_2 \sin t); y=e^t(-c_1\sin t +c_2 \cos t) \qed \; v'=\gamma v, v(z,0)=u_0 \implies v=u_0e^{\gamma t} \qed$ Variation of consts formula $x' = \alpha x, x(0) = x_0, x(t)=x_0 e^{\alpha t}, y' = \alpha y + F(t) \implies y(t) = y(0) e^{\alpha t} + \int_0^t e^{\alpha s} F(t-s) ds$.

\end{multicols}
\end{document}