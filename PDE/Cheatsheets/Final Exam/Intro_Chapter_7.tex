Let $\Omega$ be an open bdd subset of $\R^n$ and $f: \Omega \ra \R^n$ be differentiable, $f(x) = (f_1(x), \dots, f_n(x))$. Then div$f(x) := \sum_{j=1}^n \partial_j f_j(x), x \in \Omega$ (7.1). $\Omega$ is called normal if the divergence thm (Gau$\ss$' integral thm) holds for every cont fctn $f: \bar{\Omega} \ra \R^n$ with cont bdd derivative on $\Omega: \int_{\Omega}$div$f(x)dx = \int_{\partial \Omega}f(x) \cdot \nu(x) d \sigma(x)$(7.2). Here $\nu(x)$ is the outer unit normal vector at $x \in \partial \Omega: \exists \; \epsilon > 0$ s.t. $x + \xi \nu (x) \ni \bar{\Omega}, x - \xi \nu (x) \in \Omega, 0 < \xi < \epsilon$. 
The notation $d \sigma (x)$ signalized that we take a surface integral. $f(x) \cdot \nu (x) = \sum_{j=1}^n f_j(x)\nu_j(x)$ is the Euclidean inner product in $\R^n$. Balls with respect to the three standard norms and Cartesian products of intervals are normal.  For $\Omega$ to be normal, $\partial \Omega$ must allow surface integration and have a cont outer normal, but additional assumptions must be satisfied.  This is an equivalent componentwise formulation of Gau$\ss$' thm.