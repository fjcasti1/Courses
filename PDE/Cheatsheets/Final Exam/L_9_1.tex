{\bf L9.1}. (a) for each $x \in \R, \Gamma(x, \cdot)$ is strictly increasing on $(0, x^2/2)$ and strictly decreasing on $(x^2/2, \infty)$. (b) For each $t > 0, \Gamma(\cdot, t)$ is strictly increasing on $(-\infty, 0)$, and strictly decreasing on $(0, \infty)$. 
Also, by the change of variables $x = y(4t)^{1/2}, \int_{\R} \Gamma(x,t)dx = (\pi)^{-1/2} \int_{\R}e^{-y^2}dx = 1$ (9.4). Notice that, for any $t > 0, \Gamma (\cdot, t)$ is the probability density of a normal (or Gau$\ss$) distribution with mean 0 and variance $2t$ (E 9.1.2). By the same change of variables, $\int_{\R \setminus[-\epsilon, \epsilon]} \Gamma(x,t)dx = 2 \int_{\epsilon}^{\infty} (4 \pi t)^{-1/2} e^{-x^2(4t)^{-1}} dx = 2 \int_{\epsilon/(4t)^{-1/2}}^{\infty}(\pi)^{-1/2}e^{-y^2} dy \ra 0, t \ra 0$ (9.5). Define $u(x,t) = \int_{\R} \Gamma(x-y,t)u_0(y)dy, x \in \R, t>0$ (9.6). We substitute $y=x-z, u(x,t)= \int_{\R} \Gamma (z,t)u_0(x-z)dz$ (9.7). Assume that $u_0$ is measurable and either integrable or bdd.  Because of the props of $\Gamma$, we can interchange diff and int in (9.6) and $\partial_t u(x,t) = \int_{\R} \partial_t \Gamma (x-y,t) u_0(y)dy = \int_{\R} \partial_x^2 \Gamma(x-y,t)u_0(y)dy = \partial_x^2 u(x,t)$. 
