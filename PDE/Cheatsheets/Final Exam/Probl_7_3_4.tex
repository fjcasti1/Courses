{\bf E7.3.4}.  Consider the Newmann boundary problem for the LE. $\Delta u(x) + c(x) u(x) = f(x), \quad x \in \Omega,  \partial_{\nu} u(x) = g(x), \quad x \in \partial \Omega$.  Assume that $\Omega$ is path-connected, $c: \Omega \ra \R$ is nonpos and cont and $c(x) < 0$ for some $x \in \Omega$. Show that there is at most one soln $u$. {\it Prf}. Let $u_1$ and $u_2$ be two solns and $u=u_1 - u_2$.  Then $ \Delta u(x) + c(x) u(x) = 0, \quad x \in \Omega, \partial_{\nu} u(x) = 0, \quad x \in \partial \Omega$. By GF1, $0 \leq \int_{\Omega} | \nabla u(x)|^2 = \int_{\Omega} \nabla u(x) \cdot \nabla u(x) dx + \int_{\Omega}(\Delta u(x) + c(x) u(x)) u(x) dx = \int_{\Omega} \nabla u(x) \cdot \nabla u(x) dx +\int_{\Omega}\Delta u(x) u(x) +  \int_{\Omega}  c(x) u(x)^2 dx = \int_{\partial_\Omega}u(x) \partial_{\nu}u(x) d \sigma + \int_{\Omega}  c(x) u(x)^2 dx=\int_{\Omega}  c(x) u(x)^2 dx$. Because $c(x) < 0$ we have that $0 \leq \int_{\Omega} | \nabla u(x)|^2 =\int_{\Omega}  c(x) u(x)^2 dx \leq 0$. And so by P 7.4, $\nabla u(x) = 0$ on $\Omega$ and so $u(x)$ is const.  Also $0 = \int_{\Omega}  c(x) u(x)^2 dx = u(x)^2 \int_{\Omega}  c(x) dx .$ Because $c(x) \neq 0, \int_{\Omega}  c(x) dx \neq 0$ and therefore  $u(x)^2 = 0$ and so $u(x) = 0$ which shows that $u_1 = u_2$ and so there is at most one soln $u \qed$ 
