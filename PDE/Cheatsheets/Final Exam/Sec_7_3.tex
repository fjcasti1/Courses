Consider the WE on a normal subset $\Omega$ of $\R^n. (\partial_t^2 - c^2 \Delta_x)u = 0, x \in \Omega, t \in (0, T), u(x,t) = g(x), x \in \partial \Omega, t \in (0, T), u(x,0) = \phi(x), x \in \Omega, \partial_tu(x,0) = \psi(x), x \in \Omega$ (7.6). Define $E(t) = 1/2 \int_{\Omega}((\partial_tu(x,t))^2 + c^2|\nabla_xu(x,t)|^2)dx$ (7.7). 
%This expression is often interpreted as the energy of the solution. Here $\nabla_x u$ is to be understood as the gradient with respect to the spatial variable $x$. 
Assume that $u$ is twice contly diff on $\Omega \times (0, T)$ and that the 1st and 2nd PDs are bdd.  Assume that $\nabla_x u$ and $\partial_t u$ can be contly extended to $\Omega \times (0, T)$. Then $\partial_t(\partial_tu)^2 = 2 \partial_tu \partial_t^2 u$ exists and is cont and bdd on $\Omega \times (0, T)$. Moreover $\partial_t |\nabla_xu|^2 = 2 \nabla_x u \cdot \partial_t \nabla_x u = 2 \nabla_x u \cdot \nabla_x \partial_t u$ exists and is cont and bdd on $\bar{\Omega} \times (0, T)$. This implies that $E$ is diff and diff and int can be interchanged and $E'(t) = \int_{\Omega}(\partial_tu(x,t)\partial_t^2u(x,t) + c^2 \nabla_x u(x,t)\cdot \partial_t \nabla_x u(x,t))dx = c^2(\int_{\Omega} \partial_t u(x,t) \Delta_x u(x,t)dx + \int_{\Omega} \nabla_xu(x,t)\cdot \nabla_x \partial_tu(x,t))dx)$. By GF1, $E'(t) = c^2 \int_{\partial \Omega}\partial_t u(x,t) \partial_{\nu} u(x,t) d\sigma(x)$. Since $u(x,t) = g(x)$ for $x \in \partial \Omega, t \in (0, T), \partial_t u(x,t)=0$ for $x \in \partial \Omega, t \in (0,T)$. So $E'(t) = 0\; \forall \; t \in (0, T)$. Thus $E(t) = E(0) = 1/2 \int_{\Omega} (\psi^2(x) + c^2|\nabla \phi(x)|^2)dx$. 