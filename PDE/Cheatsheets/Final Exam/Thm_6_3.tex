{\bf T 6.3} Let $\Omega$ be a disk in $\R^2$ with the origin as center and $f: \partial \Omega \ra \R$ be Lipschitz cont. Then $\exists$ a fctn $u: \bar{\Omega} \ra \R$ s.t. $u$ is cont on $\bar{\Omega} \setminus \{(0,0)\}, \; u$ is infinitely often diff in $\Omega \setminus \{(0,0)\}$ and $\Delta u = 0 $ on $\Omega \setminus \{(0,0)\}$ and $u = f$ on $ \partial \Omega$. We do not obtain diff at the origin right away because the transformation from polar to rectangular coords is not invertible at the origin. Similarly as for the HE, we have a representation of $v$ via a Green's type fctn.  It holds, if $f$ is just cont. We use the definition of $\hat{f}_j$ in (6.17) and interchange series and int in (6.19), $v(r, \theta) = \int_{-\pi}^{\pi} f(\eta) G(r, \eta - \theta) d \eta, 2 \pi G(r, \theta) = \sum_{j=-\infty}^{\infty}(r/a)^{|j|}e^{-ij \theta}, 0 \leq r < a, \theta \in \R$ (6.20). Similarly as before, it can be shown that $G$ is infinitely often diff for $0 \leq r < a$ and $(r^2 \partial_r^2 + r \partial_r + \partial_{\theta}^2)G(r, \theta) = 0, 0 \leq r < a, \theta \in \R$. Notice that, if $f \equiv 1, v \equiv 1$ is a soln of (6.16). Certainly $v$ is sufficiently smooth to allow all the operations we have done before.  We set $v \equiv f \equiv 1$ in (6.20), $\int_{-\pi}^{\pi} G(r, \eta - \theta) d \eta = 1, 0 \leq r < a, \theta \in \R$ (6.21). Differently from the HE, we can obtain an explicit expression for the GF. $G$ can be rewritten as $2 \pi G(r, \theta)=\sum_{j=0}^{\infty}[(r/a)e^{i \theta}]^j + \sum_{j=0}^{\infty}[(r/a)e^{-i \theta}]^j -1$. The geometric series converge if $r < a, 2 \pi G(r, \theta) = 1/(1-(r/a)e^{i \theta})+ 1/(1-(r/a)e^{-i \theta})-1$. We bring the expression in parentheses into a common denominator $2 \pi G(r, \theta) = (1-(r/a)^2)/(1-(r/a)(e^{i \theta}+e^{-i \theta}) + (r/a)^2)$. This simplifies to $2 \pi G(r, \theta) = (a^2-r^2)/(a^2-2ra \cos \theta +r^2) > 0, 0 \leq r < a, \theta \in \R$ (6.22). We obtain Poisson's formula for the solution of the LE in polar coords, $v (r, \theta) = (1/(2 \pi))\int_{-\pi}^{\pi} f(\eta) (a^2-r^2)/(a^2-2ra \cos ( \eta - \theta) +r^2) d \eta, 0 \leq r < a$ (6.23). Since $r \cos(\eta - \theta) = r(\cos \eta \cos \theta + \sin \eta \sin \theta) = x \cos \eta + y \sin \eta$, we can express the soln in rect coords, $u(x,y) =  (1/(2 \pi))\int_{-\pi}^{\pi} f(\eta) (a^2-x^2-y^2)/(a^2-2ax \cos \eta - 2 a y \sin \eta + x^2 + y^2) d \eta, x^2 + y^2 < a^2$ (6.24). This shows that $u$ is inf often diff at the origin as well. By continuity, it also satisfies the LE at the origin.  Using the properties of the GF, we can extend T6.3 to cont boundary data. 