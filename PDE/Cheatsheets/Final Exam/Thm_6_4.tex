{\bf T 6.4} Let $\Omega$ be a disk in $\R^2$ and $f: \partial \Omega \ra \R$ be cont. Then there exists a cont fctn $u: \bar{\Omega} \ra \R$ such that $u$ is infinitely often diff in $\Omega$ and $\Delta u = 0$ on $\Omega$ and $u = f$ on $\partial \Omega$. See E 6.2.1. Poisson's formula in rectangular coords can be rewritten in a form that can be generalized to the ball $\Omega$ in $\R^2$ with radius $a$ and arbitrary center, $u(x)=(1/(A(\partial \Omega))) \int_{\partial \Omega} f(y) (a^2 - |x|^2)/(a^2 - 2 \langle x, y \rangle + |x|^2) d \sigma(y) = (1/(A(\partial \Omega))) \int_{\partial \Omega} f(y) (|y|^2 - |x|^2)/(|x-y|^2) d \sigma(y), x \in \Omega$ (6.25). Notice that $x$ and $y$ are now vectors in $\R^2, \; |x|$ is the Euclidean norm of $x$, and $|y| = a$ for $y \in \partial \Omega$. The symbol $d \sigma$ signalized that we take the survace integral over the sphere $\partial \Omega$ in $\R^2$ with radius $a. \; A(\partial \Omega)$ is the surface area of this sphere. This formula generalizes to $\R^n, u(x)=(1/(A(\partial \Omega))) \int_{\partial \Omega} f(y)(|y|^2 - |x|^2)/(|x-y|^n) d \sigma(y), x \in \Omega$ (6.26). See E 6.2.2.
