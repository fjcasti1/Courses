{\bf T6.7}. Assume that $u: \bar{\Omega} \ra \R$ is cont and twice PD on $\Omega$ as above and satisfies $(Lu)(x) \geq 0, x \in \Omega$. Then $\max_{\bar{\Omega}} u = \max_{\partial \Omega} u$. {\it Prf}. Case 1: $(Lu)(x) > 0\; \forall \; x \in \Omega$. Since $u$ is cont on $\bar{\Omega}, \exists \; x \in \bar{\Omega}$ s.t. $u(x) = \max_{\bar{\Omega}} u$. If $x \in \Omega$, then $\partial_j u(x) = 0$ and $\partial_j^2 u(x) \leq 0, j = 1, \dots, n$. So $(Lu)(x) \leq 0$, a contradiction. So $x \in \partial \Omega$ and the assertion follows. Case 2: $(Lu)(x) \geq 0\; \forall \; x \in \Omega$. For $\epsilon > 0$, set $u_{\epsilon} (x) = u(x) + \epsilon c \sum_{j=1}^n \xi_j x_j + \epsilon |x|^2, x \in \bar{\Omega}$, where $\xi_j \in \{0, 1, -1\}$ have the sign of $b_j, |x|$ denotes the Euclidean norm and $c > 0$ will be determined.  Then $\partial_j u_{\epsilon}(x) = \partial_j u(x) + \epsilon c \xi_j + 2 c x_j$ and $\partial_j^2 u_{\epsilon}(x) = \partial_j^2 u(x) + 2 \epsilon$. By (6.27) $(Lu_{\epsilon})(x) = (Lu)(x) + \epsilon \sum_{j=1}^n c|b_j| + 2 \epsilon \sum_{j=1}^n b_j x_j + 2 \epsilon \sum_{j=1}^n a_j$. By the Cauchy-Schwarz inequality in $\R^n$ and $(Lu)(x) \geq 0$, with $b = (b_1, \dots, b_n), (Lu_{\epsilon})(x) \geq \epsilon(c \sum_{j=1}^n|b_j|-2|b||x| + 2 \sum_{j=1}^n a_j)$. Recall that $|b| \leq \sum_{j=1}^n|b_j|$. Since $\Omega$ is bdd, $\exists \; c > 0$ s.t. $|x| \leq c/3 \; \forall \; x \in \Omega$. So $(Lu_{\epsilon})(x) \geq \epsilon \sum_{j=1}^n[|b_j|(c-2|x|) + 2 a_j] \geq \epsilon \sum_{j=1}^n[(c/3)|b_j| + 2 a_j] >0$. By case 1, $\max_{\bar{\Omega}} u_{\epsilon} = \max_{\partial \Omega} u_{\epsilon}$. Now, for $x \in \bar{\Omega}$, by the Cauchy-Schwartz inequality in $\R^n, u(x) \leq u_{\epsilon}(x) + \epsilon c \sqrt{n} |x| \leq u_{\epsilon}(x) + \epsilon c^2 \sqrt{n}$ and $u_{\epsilon}(x) \leq u(x) + \epsilon c^2(\sqrt{n}+1)$. So $\max_{\bar{\Omega}} u \leq \max_{\bar{\Omega}} u_{\epsilon} + \epsilon c^2 \sqrt{n} \leq \max_{\partial \Omega} u_{\epsilon} + \epsilon c^2 \sqrt{n} \leq \max_{\partial \Omega} u + \epsilon c^2(\sqrt{n}+1)$. Since this holds for any $\epsilon > 0, \max_{\bar{\Omega}}u \leq \max_{\partial \Omega} u$. The opposite inequality is trivial $\qed$