{\bf T7.5}.  Any 2 solns of (7.3) that have cont bdd 2nd partial derivatives on $\Omega$ and whose 1st derivatives and themselves can be contly extended to $\bar{\Omega}$ have identical gradients. If $\Omega$ is path-connected, they are equal up to a const. If, in addition, $\alpha(x) \neq 0$ for some $x \in \partial \Omega$, there is at most one soln.  {\it Prf}. Let $u_1$ and $u_2$ be two solns.  Then $u = u_1 - u_2$ is cont on $\bar{\Omega}$, its 1st derivative can be contly extended to $\bar{\Omega}$ and $u$ is twice contly diff on $\Omega$.  Further $\Delta u = 0$ on $\Omega, \beta(x) \partial_{\nu} u(x) + \alpha(x) u(x) = 0, x \in \partial \Omega$. By GF1, $0 \leq \int_{\Omega} |\nabla u|^2 dx = \int_{\Omega} \nabla u \cdot \nabla u dx + \int_{\Omega} u \Delta u dx = \int_{\partial \Omega}u \partial_{\nu}u d \sigma$. If $x \in \partial \Omega$, there are two cases:  either $\beta(x) = 0$, then $\alpha(x) u(x) = 0$ and $\alpha(x) > 0$ and so $u(x) = 0$, or $\beta (x) > 0$, then $\partial_{\nu} u(x) = - \alpha(x)/\beta(x) u(x)$. So $0 \leq \int_{\Omega} | \nabla u|^2 dx = - \int_{\partial \Omega \cap \{\beta > 0 \}} \alpha(x)/\beta(x) u^2(x)d \sigma(x) \leq 0$. This implies that $0 = \int_{\Omega}|\nabla u|^2 dx$. Since $\nabla u$ is cont, $\nabla u \equiv 0$ on $\Omega$. If $\Omega$ is path-connected, $u$ is const by P 7.4. Then $0 \equiv \alpha u$ on $\partial \Omega$. So, if $\alpha(x) > 0$ for some $x \in \partial \Omega, u \equiv 0$ on $\Omega \qed$  Neumann BVP $\Delta u = f$ on $\Omega, \partial_{\nu} u(x) = g(x), x \in \partial \Omega$ (7.4). From GF1, for sufficiently smooth $v, \int_{\Omega} \nabla v \cdot \nabla u dx + \int_{\Omega} v \Delta u dx = \int_{\partial \Omega}v \partial_{\nu} u d \sigma$. So, for $v \equiv 1, \int_{\Omega} \Delta u dx = \int_{\partial \Omega} \partial_{\nu} u d \sigma$ and so $\int_{\Omega} f dx = \int_{\partial \Omega}  g d \sigma$. 