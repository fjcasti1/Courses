{\bf T9.4}. Let $u:\R \times \R_+ \ra \R$ be cont and bdd on $\R \times [0, T]$ for every $T \in (0, \infty)$. Further, let $\partial_t u, \partial_x u, \partial_x^2 u$ be bdd on $\R \times [\epsilon, 1/\epsilon]$ for every $\epsilon \in (0, 1)$. Then, if $u$ solves (9.1), $u(x,t) = \int_{\R} \Gamma(t, x-y) u_0(y) dy, t > 0$. Further, if $u_0$ is unif cont and bdd, this formula gives a sol of (9.1) with the props above. Fix $r >0$ and define $u(x,t) = \Gamma(t+r,x), x \in \R, t \geq 0$. Then $\partial_t u = \partial_x^2 u$ and $u$ satisfies all the assumptions we made before.  This yields $\Gamma(t+r, x) = \int_{\R} \Gamma(t, x-y) u(y,0)dy = \int_{\R}\Gamma(t, x-y) \Gamma (r,y)dy$ (9.10). This means that $\Gamma$ satisfies the Chapman-Kolmogorov equation. Consider the HE in  the space of integrable fctns $L^1(\R)$. Similarly as before, $u$ given by (9,6) solves the PDE on $\R \times (0, \infty)$. We assume that $u_0$ is non-neg and measurable with finite integral $\int_{\R}u_0(y)dy$ and $u$ given by (9.6). Then $u$ is non-neg and, by Tonelli's thrm, $\int_{\R}u(x,t)dx = \int_{\R}(\int_{\R} \Gamma(t,x-y) dx) u_0(y) dy=\int_{\R} u_0(y)dy, t > 0$. But, by (9.2), $u(t,x) \leq (4 \pi t)^{-1/2} \int_{\R}u_0(y)dy \ra 0, t \ra \infty$, unif for $x \in \R$. In what sense does $u$ given by (9.6) satisfy the initial condition?