\begin{questions}

\question{Solve the Cauchy problem
\begin{align*}
&u\partial_1u+\partial_2u=2~,\\
&u(x_1,x_2)=0~\text{on the line}~x_1=x_2.
\end{align*}
Where is the solution defined and where is it differentiable?
}

\begin{solution}
The PDE above has the form of the equation $(3.2)$ from the notes. Since $c$ and $a_j$ are partially differentiable in all variables and these partial derivatives are continuous, the solution $u$ must be unique acording to \textsl{Theorem 3.5}. Looking at the initial condition we identify the hypersurface $S={(z,z);z\in\R}$. So $S=g(\R)$ with $g(z)=(z,z)$. Now we can write our \textit{Characteristic System} with the \textit{Initial Conditions}:
\begin{align*}
\partial_t\xi_1(z,t)=v(z,t),~~&~~\xi_1(z,0)=z,\\
\partial_t\xi_2(z,t)=1,~~&~~\xi_2(z,0)=z,\\
\partial_tv(z,t)=2,~~&~~v(z,0)=0.
\end{align*}
Let's first integrate for $\xi_2$,
\begin{align*}
\partial_t\xi_2(z,t)=1~~\Rightarrow~~ \xi_2(z,t)=t+f_2(z).
\end{align*}
Imposing the initial condition,
\begin{align*}
\xi_2(z,0)=f_2(z)=z,
\end{align*}
we obtain
\begin{align*}
\xi_2(z,t)=z+t.
\end{align*}
Now we solve for $v$,
\begin{align*}
\partial_tv(z,t)=2~~\Rightarrow~~ v(z,t)=2t+f_3(z).
\end{align*}
Imposing the initial condition,
\begin{align*}
v(z,0)=f_3(z)=0,
\end{align*}
we obtain
\begin{align*}
v(z,t)=2t=v(t).
\end{align*}
Finally we can integrate $\xi_1$,
\begin{align*}
\partial_t\xi_1(z,t)=v(t)=2t~~\Rightarrow~~ \xi_1(z,t)=t^2+f_1(z).
\end{align*}
Imposing the initial condition,
\begin{align*}
\xi_1(z,0)=f_1(z)=z,
\end{align*}
we obtain
\begin{align*}
\xi_1(z,t)=z+t^2.
\end{align*}
Again, to find the solution we need to find $t(x_1,x_2)$ using that
\begin{align*}
x_1&=\xi_1=z+t^2,\\
x_2&=\xi_2=z+t.
\end{align*}
Substracting the two previous equations we find
\begin{align*}
x_1-x_2=t^2-t~~\Rightarrow~~t^2-t+x_2-x_1=0,
\end{align*}
which we can solve for $t$,
\begin{align*}
t=\frac{1\pm\sqrt{1-4(x_2-x_1)}}{2}.
\end{align*}
Now we can obtain the solution
\begin{align*}
u(x_1,x_2)=v(t(x_1,x_2))=2t(x_1,x_2)=1\pm\sqrt{1-4(x_2-x_1)}.
\end{align*}
The solution must be unique so, to get rid of one sign, we impose the initial condition for $u$,
\begin{align*}
u(x_1,x_1)=1\pm 1=0,
\end{align*}
which implies that
\begin{align*}
u(x_1,x_2)=1-\sqrt{1-4(x_2-x_1)}.
\end{align*}
The solution is defined for all $\R^2$ except for those points that don't satisfy $x_2-x_1\leq 1/4$,\\ i.e.,$\R^2\backslash\left\lbrace (x_1,x_2);x_2-x_1>1/4\right\rbrace$. To see when the solution is differentiable we calculate
\begin{align*}
\partial_1u=\frac{-2}{\sqrt{1-4(x_2-x_1)}},~~~~\partial_2u=\frac{2}{\sqrt{1-4(x_2-x_1)}}.
\end{align*}
Clearly, $u$ is differentiable in $\R^2\backslash\left\lbrace (x_1,x_2);x_2-x_1\geq 1/4\right\rbrace$, since $x_2-x_1=1/4$ would make the derivatives to go to infinity.
\end{solution}
\end{questions}