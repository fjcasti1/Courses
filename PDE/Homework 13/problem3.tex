\begin{questions}
\question{Consider the following version of the Neumann boundary problem for
the Laplace equation.
\begin{align*}
\Delta u(x)+c(x)u(x)&=f(x),\quad x\in\Omega,\\
\partial_{\nu} u(x)&=g(x),\quad x\in\partial\Omega.
\end{align*}

Find a sign condition for the function $c$ that guarantees that there is at most
one solution $u$ even if $\Omega$ is not path-connected.
}
\begin{solution}
Let $c$ be continuous and negative for all $x\in\Omega$. Let $u$, $v$ be solutions to he previous PDE and boundary conditions. Define $w:\overline{\Omega}\rightarrow\R$ by $w(x)=u(x)-v(x)$. Then,
\begin{align*}
\Delta w(x)+c(x)w(x)&=0,\quad x\in\Omega,\\
\partial\nu w(x)&=0,\quad x\in\partial\Omega.
\end{align*}
Making use of \textsc{Theorem 7.3} for $w$ we get
\begin{align*}
\int_{\Omega}w\Delta wdx+\int_{\Omega}\nabla w\cdot\nabla w dx=\int_{\partial\Omega}w\partial_{\nu}w d\sigma.
\end{align*}
Using the PDE into the previous equation,
\begin{align*}
-\int_{\Omega}c(x)w^2(x)dx+\int_{\Omega}\nabla w\cdot\nabla w dx=0.
\end{align*}
Multiplying the equation by $-1$ and noting that the inner product is always non-negative,
\begin{align*}
0=\int_{\Omega}c(x)w^2(x)dx-\int_{\Omega}\nabla w\cdot\nabla w dx\leq\int_{\Omega}c(x)w^2(x)dx\leq 0,
\end{align*}
since $c(x)<0$ for all $x\in\Omega$. Therefore,
\begin{align*}
\int_{\Omega}c(x)w^2(x)dx=0,
\end{align*}
which implies that $w(x)=0$ for all $x\in\Omega$ since $c(x)w^2(x)$ is continuous on the domain. Finally, since $c(x)\neq 0$, $w(x)=0$ for all $x\in\Omega$. Hence, we have that $u=v$ for all $x\in\Omega$, proving that there exists at most one solution for the PDE and boundary conditions given provided that $c(x)<0$ for all $x\in\Omega$.
\end{solution}
\end{questions}