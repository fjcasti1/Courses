\begin{questions}

\question{
Find the solution $u$ of two variables $x,y\in\R$ of 
\begin{align*}
(y+x)u_x+(y-x)u_y=u^2,\\
u=1\text{ on the circle }x^2+y^2=1.
\end{align*}
Where is the solution defined?
}

\begin{solution}
We can rewrite the PDE in terms of $x_1,x_2$ by making $x_1=x$ and $x_2=y$.
\begin{align*}
(x_2+x_1)\partial_1u+(x_2-x_1)\partial_2u=u^2,\\
u=1\text{ on the circle }x_1^2+x_2^2=1.
\end{align*}
We can also parametrize the initial condition using sine and cosine functions,\begin{align*}
u(\cos{z},\sin{z})=1.
\end{align*}
Therefore we get
\begin{align*}
g(z)=\begin{pmatrix*}[r]
\cos{z}\\
\sin{z}
\end{pmatrix*}.
\end{align*}
We can now rewrite the PDE as a system of ODE, the \textit{Characteristic System},
\begin{align*}
&\partial_t\xi_1(z,t)=\xi_2+\xi_1,~~~~\xi_1(z,0)=\cos{z},\\
&\partial_t\xi_2(z,t)=\xi_2-\xi_1,~~~~\xi_1(z,0)=\sin{z},\\
&\partial_tv(z,t)=v^2,~~~~~~~~~~~v(z,0)=1.
\end{align*}
We solve first for $\xi_1,\xi_2$, which form a linear system of differential equations, we can solve it using the matrix exponential. We can rewrite the system in matrix form,
\begin{align*}
\xi'=A\xi,
\end{align*}
\begin{align*}
\begin{pmatrix*}[r]
\partial_t\xi_1\\
\partial_t\xi_2
\end{pmatrix*}=\begin{pmatrix*}[r]
1 & 1\\
-1 & 1
\end{pmatrix*}
\begin{pmatrix*}[r]
\xi_1\\
\xi_2
\end{pmatrix*}.
\end{align*}
We can easily calculate the eigenvalues of the matrix $A$, $\lambda=1\pm i$, and its eigenvectors
\begin{align*}
V_1=\begin{pmatrix*}[r]
1\\
i
\end{pmatrix*},~~~~~~
V_2=\begin{pmatrix*}[r]
1\\
-i
\end{pmatrix*}
\end{align*}
Thus, the solution of the system is
\begin{align*}
\xi(z,t)=f_1(z)e^{(1+i)t}\begin{pmatrix*}[r]
1\\
i
\end{pmatrix*}+f_2(z)e^{(1-i)t}\begin{pmatrix*}[r]
1\\
-i
\end{pmatrix*},
\end{align*}
which we can express in terms of sines and cosines,
\begin{align*}
\xi(z,t)&=f_1(z)e^t\left[\cos{t}+i\sin{t} \right]\begin{pmatrix*}[r]
1\\
i
\end{pmatrix*}+f_2(z)e^t\left[\cos{t}-i\sin{t} \right]\begin{pmatrix*}[r]
1\\
-i
\end{pmatrix*}\\
&=\left[f_1(z)+f_2(z)\right]e^t\begin{pmatrix*}[r]
\cos{t}\\
-\sin{t}
\end{pmatrix*}+i\left[f_1(z)-f_2(z)\right]e^t\begin{pmatrix*}[r]
\sin{t}\\
\cos{t}
\end{pmatrix*}\\
&=K_1(z)e^t\begin{pmatrix*}[r]
\cos{t}\\
-\sin{t}
\end{pmatrix*}+iK_2(z)e^t\begin{pmatrix*}[r]
\sin{t}\\
\cos{t}
\end{pmatrix*}.
\end{align*}
Imposing the initial conditions
\begin{align*}
\xi(z,0)&=K_1(z)\begin{pmatrix*}[r]
1\\
0
\end{pmatrix*}+iK_2(z)\begin{pmatrix*}[r]
0\\
1
\end{pmatrix*}\\
&=\begin{pmatrix*}[r]
K_1(z)\\
iK_2(z)
\end{pmatrix*}=\begin{pmatrix*}[r]
\cos{z}\\
\sin{z}
\end{pmatrix*},
\end{align*}
we can find $K_1$ and $K_2$
\begin{align*}
K_1(z)=\cos{z},~~~~K_2(z)=-i\sin{z}.
\end{align*}
Now we can write the solution for $\xi$
\begin{align*}
\xi(z,t)&=\cos{(z)}e^t\begin{pmatrix*}[r]
\cos{t}\\
-\sin{t}
\end{pmatrix*}+\sin{(z)}e^t\begin{pmatrix*}[r]
\sin{t}\\
\cos{t}
\end{pmatrix*}.
\end{align*}
Now we solve for $v$
\begin{align*}
v(z,t)=\frac{-1}{t+f_3(z)},
\end{align*}
with initial condition
\begin{align*}
v(z,0)=\frac{-1}{f_3(z)}=1,
\end{align*}
giving us $f_3(z)=-1$. Therefore
\begin{align*}
v(z,t)=\frac{1}{1-t}=v(t),
\end{align*}
and we only have to find $t(x_1,x_2)$ to obtain the solution $u(x_1,x_2)=v(t(x_1,x_2))$. In order to find $t(x_1,x_2)$ we need to solve
\begin{align*}
x_1=e^t\left(\cos{z}\cos{t}+\sin{z}\sin{t}\right),\\
x_2=e^t\left(\sin{z}\cos{t}-\cos{z}\sin{t}\right).
\end{align*}
Let us compute $x_1^2+x_2^2$,
\begin{align*}
x_1^2+x_2^2=e^{2t}(&\cos^2{z}\cos^2{t}+\sin^2{z}\sin^2{t}+2\sin{z}\cos{z}\sin{t}\cos{t}\\
&+\sin^2{z}\cos^2{t}+\cos^2{z}\sin^2{t}-2\sin{z}\cos{z}\sin{t}\cos{t}),
\end{align*}
\begin{align*}
x_1^2+x_2^2&=e^{2t}\left[\left(\sin^2{z}+\cos^2{z}\right)\cos^2{t}+\left(\sin^2{z}+\cos^2{z}\right)\sin^2{t}\right]\\
&=e^{2t}.
\end{align*}
Now we can isolate $t(x_1,x_2)$ from the previous equation
\begin{align*}
t(x_1,x_2)=\frac{1}{2}\ln\left(x_1^2+x_2^2\right),
\end{align*}
which gives us the solution to the PDE
\begin{align*}
u(x_1,x_2)=\frac{1}{1-\frac{1}{2}\ln\left(x_1^2+x_2^2\right)}.
\end{align*}
It is clear that the solution is not defined at the origin since $\ln{(0)}$ is not defined. In addition we know that the denominator cannot be zero, therefore the solution is not defined in points that satisfy $x_1^2+x_2^2=e^2$ which describes the circle of radius $e>1$. Since the domain of existence has to be connected, the solution exists on the annulus $0<x_1^2+x_2^2<e^2$.
\end{solution}
\end{questions}