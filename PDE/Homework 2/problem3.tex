\begin{questions}

\question{Determine all solutions $u=u(x_1,x_2)$ of
\begin{align*}
(1-u)\frac{\partial u}{\partial x_1}+(1+u)\frac{\partial u}{\partial x_2}&=1,~~x_1,x_2\in\R,\\
u(x_1,x_2)&=0,~~x_1=x_2.
\end{align*}
Where are the solutions defined? Interpret your results in the light of the genereal local existence theorem.
}
\begin{solution}
We can rewrite the PDE as
\begin{align*}
(1-u)\partial_1u+(1+u)\partial_2u=1,
\end{align*}
with a parametrized initial condition 
\begin{align*}
u(z,z)=0.
\end{align*}
Therefore we can obtain
\begin{align*}
g(z)=\begin{pmatrix*}[r]
z\\
z
\end{pmatrix*},
\end{align*}
and
\begin{align*}
a(x,u)=g(z)=\begin{pmatrix*}[r]
1-u\\
1+u
\end{pmatrix*}~\Rightarrow~
a(g(z),u_0(g(z)))=\begin{pmatrix*}[r]
1\\
1
\end{pmatrix*}.
\end{align*}
We now rewrite the PDE as a system of ODE, the \textit{Characteristic System},
\begin{align*}
&\partial_t\xi_1=1-v,~~~~\xi_1(z,0)=z\\
&\partial_t\xi_2=1+v,~~~~\xi_2(z,0)=z\\
&\partial_tv=1,~~~~~~~~~~v(z,0)=0.\end{align*}
Given the previous system, we have to start by solving for $v$
\begin{align*}
v(z,t)=t+f_3(z).
\end{align*}
By imposing the initial condition we find that $f_3(z)=0$ and
\begin{align*}
v(z,t)=v(t)=t.
\end{align*}
Now we solve for $\xi_1$
\begin{align*}
\partial_t\xi_1=1-t~\Rightarrow~\xi_1(z,t)=t-\frac{1}{2}t^2+f_1(z),
\end{align*}
and imposing the initial condition we find that $f_1(z)=z$ and
\begin{align*}
\xi_1(z,t)=z+t-\frac{1}{2}t^2.
\end{align*}
Now we solve for $\xi_2$
\begin{align*}
\partial_t\xi_2=1+t~\Rightarrow~\xi_2(z,t)=t+\frac{1}{2}t^2+f_2(z),
\end{align*}
and imposing the initial condition we find that $f_2(z)=z$ and
\begin{align*}
\xi_2(z,t)=z+t+\frac{1}{2}t^2.
\end{align*}
To obtain the solution $u(x_1,x_2)=v(t(x_1,x_2))$ we need to find $t(x_1,x_2)$ by solving
\begin{align*}
x_1(z,t)=z+t-\frac{1}{2}t^2,
x_2(z,t)=z+t+\frac{1}{2}t^2.
\end{align*}
Let us compute $x_2-x_1$
\begin{align*}
x_2-x_1=t^2~\Rightarrow~t=\pm\sqrt{x_2-x_1},
\end{align*}
and finally
\begin{align*}
u(x_1,x_2)=\pm\sqrt{x_2-x_1}
\end{align*}
and we see that by applying the initial condition we cannot get rid of any of the two signs,
\begin{align*}
u(x_1,x_1)=\pm\sqrt{x_1-x_1}=0.
\end{align*}
Therefore we see that there is not an unique solution. In order to relate this to the general local existence theorem, we compute first
\begin{align*}
g(z)=\begin{pmatrix*}[r]
1\\
1
\end{pmatrix*},
\end{align*}
and now the determinant
\begin{align*}
det(g'(z),a(g(z),u_0(g(z))))=det\left(\begin{pmatrix*}[r]
1 1\\
1 1
\end{pmatrix*}\right)=0.
\end{align*}
Thus, according to the \textsl{Theorem 3.7}, we cannot guarantee that there exists an open neighborhood $U$ and an uniquely determined function $u:U\rightarrow\R$ that is solution to the Cauchy problem, which is in agreement with the results obtained.
\end{solution}

\end{questions}