\begin{questions}

\question{Solve the linear size-structured population problem
\begin{align*}
\partial_tu+\gamma(t)\partial_xu+\mu u=f(x,t),~~~~x,t\in\R,\\
u(x,0)=u_0(x),~~~~~~~x\in\R.
\end{align*}

Assume that $f : \R^2\rightarrow\R$ and $u_0 : \R\rightarrow\R$ are continuously differentiable. Further $\gamma : \R_+\rightarrow\R_+$ is continuous and $\mu\geq 0$ is a constant.
Show also: If $u_0(x) = 0$ for $x\leq 0$ and $f(x,t) = 0$ for all $x\leq 0$, $t\geq 0$,
then $u(x, t) = 0$ for all $x\leq 0$, $t\geq 0$.

}
\begin{solution}
We start by writing the \textit{Characteristic System} corresponding to this problem,
\begin{align*}
\partial_t\xi_1(z,t)=\gamma(t),~~~~~~\xi_1(z,0)=z,\\
\partial_t\xi_2(z,t)=1,~~~~~~\xi_2(z,0)=0,\\
\partial_tv(z,t)=f(x,t)-\mu v,~~~~~~v(z,0)=u_0(z).
\end{align*}
We start by solving for $\xi_2$,
\begin{align*}
\xi_2(z,t)=t+h_2(z).
\end{align*}
Imposing the initial condition
\begin{align*}
\xi_2(z,0)=h_2(z)=0,
\end{align*}
we get
\begin{align*}
\xi_2(z,t)=t.
\end{align*}
Now we solve for $\xi_1$,
\begin{align*}
\xi_1(z,t)=\int_0^t\gamma(s)ds+h_1(z).
\end{align*}
Imposing the initial condition
\begin{align*}
\xi_1(z,0)=h_1(z)=z,
\end{align*}
we get
\begin{align*}
\xi_1(z,t)=z+g(t),
\end{align*}
where we have defined
\begin{align*}
g(t)=\int_0^t\gamma(s)ds.
\end{align*}
Observe that $g(0)=0$ and, since $\gamma(s)>0$ for all $s$, $g(t_2)>g(t_1)$ if $t_2>t_1$. Finally, we solve for $v$ retaking its ordinary differential equation
\begin{align*}
\partial_tv(z,t)=-\mu v(z,t)+f(z+g(t),t).
\end{align*}
By Duhamel's formula we have
\begin{align*}
v(z,t)=u_0(z)e^{-\mu t}+\int_0^te^{-\mu(t-s)}f(z+g(s),s)ds.
\end{align*}
Now we can obtain the solution $u(x,t)$ by plugging in $z=x-g(t)$,
\begin{align*}
u(x,t)=u_0(x-g(t))e^{-\mu t}+\int_0^te^{-\mu(t-s)}f(x-\left(g(t)-g(s)\right),s)ds.
\end{align*}
Now we analyze the argument of the function $u_0$ and the first argument of $f$. Since $g(t)$ is nonnegative for all $t$, if $x$ is negative, then $x-g(t)$ is also negative and $u(x-g(t)=0$. On the other hand,
\begin{align*}
g(t)-g(s)=\int_0^t\gamma(r)dr-\int_0^s\gamma(r)dr,
\end{align*}
and, since $t\geq s$, it is nonnegative. Therefore, if $x$ is negative and $t$ nonnegative, then $x-(g(t)-g(s))$ is also negative and $f(x-(g(t)-g(s)),s)=0$. Thus, for all $x\leq 0$ and $t\geq 0$, $u(x,t)=0$.
\end{solution}
\end{questions}