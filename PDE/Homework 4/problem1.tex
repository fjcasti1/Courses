\begin{questions}

\question{Solve the wave equation
\begin{eqnarray*}
\partial_t^2u - \partial_x^2u = 0,& x,t \in \mathbb{R} \\
u(x,x) = f(x),& \partial_tu(x,0) = g(x), \quad x \in \mathbb{R},
\end{eqnarray*}
where $g, f : \mathbb{R} \to \mathbb{R}$.

State appropriate assumptions for $f$ and $g$ such that you really have a solution.

Notice that $c = 1$.
}
\begin{solution}
Given the previous PDE, the wave equation, we can express the solution $u(x,t)$ as
\begin{align*}
u(x,t)=F(x+t)+G(x-t),
\end{align*}
where we have already taken into account that $c=1$. Imposing the first initial condition,
\begin{align}\label{ec:A}
u(x,x)=F(2x)+G(0)=f(x).
\end{align}
For the second initial condition, we calculate first $\partial_tu$,
\begin{align*}
\partial_tu(x,t)=F'(x+t)-G'(x-t),
\end{align*}
and impose the initial condition,
\begin{align*}
\partial u(x,0)=F'(x)-G'(x)=g(x).
\end{align*}
Integrating the equation we get
\begin{align*}
F(x)-G(x)=F(0)-G(0)+\int_0^xg(y)dy,
\end{align*}
or
\begin{align*}
F(2x)-G(2x)=F(0)-G(0)+\int_0^{2x}g(y)dy.
\end{align*}
Substract now equation (\ref{ec:A}) from the previous equation and get
\begin{align*}
-G(2x)-G(0)=F(0)-G(0)-f(x)+\int_0^{2x}g(y)dy,\\
-G(2x)=F(0)-f(x)+\int_0^{2x}g(y)dy,
G(2x)=f(x)-F(0)-\int_0^{2x}g(y)dy,
\end{align*}
or
\begin{align*}
G(x)=f(\frac{x}{2})-F(0)-\int_0^{x}g(y)dy.
\end{align*}
From the equation above we obtain $G(x-t)$ needed for our solution,
\begin{align*}
G(x-t)=f(\frac{x-t}{2})-F(0)-\int_0^{x-t}g(y)dy.
\end{align*}
Now, we retake equation (\ref{ec:A}) and rewrite it as
\begin{align*}
F(x+t)=f(\frac{x+t}{2})-G(0).
\end{align*}
Thus, the solution is
\begin{align*}
u(x,t)=f(\frac{x+t}{2})+f(\frac{x-t}{2})-\left(F(0)+G(0)\right)-\int_0^{x-t}g(y)dy,
\end{align*}
and it is left to calculate $F(0)+G(0)$, which we do by evaluating equation (\ref{ec:A}) at $x=0$,
\begin{align*}
F(0)+G(0)=f(0).
\end{align*}
Hence,
\begin{align*}
u(x,t)=f(\frac{x+t}{2})+f(\frac{x-t}{2})-f(0)-\int_0^{x-t}g(y)dy,
\end{align*}
which satisfies the initial conditions:
\begin{itemize}
\item $u(x,x)=f(x).$
\begin{align*}
u(x,x)=f(\frac{2x}{2})+f(0)-f(0)-\int_0^{0}g(y)dy=f(x).
\end{align*}
\item $\partial_tu(x,0)=g(x).$
\begin{align*}
\partial_tu(x,0)=\frac{1}{2}f(\frac{x}{2})-\frac{1}{2}f(\frac{x}{2})+g(x)=g(x),
\end{align*}
where to evaluate the derivative of the integral we have used the Fundamental Theorem of Calculus.
\end{itemize}
Thus, since the $u(x,t)$ given satisfies the PDE and the initial conditions, it is the solution provided that $f$ is twice differentiable and $g$ is once differentiable.
\end{solution}

\end{questions}