\begin{questions}
\question{
Let $u$ solve
\begin{eqnarray*}
(\partial_t^2 - c^2 \partial_x^2)u(x,t) &=& \phi(x,t), \quad x,t \in \mathbb{R}. \\
u(x,0)&=&0, \quad x \in \mathbb{R}, \\
\partial_t u(x,0)&=&0, \quad x \in \mathbb{R}\\
\end{eqnarray*}
and $\tilde{u}$ solve
\begin{eqnarray*}
(\partial_t^2 - c^2 \partial_x^2)\tilde{u}(x,t) &=& 0, \quad x,t \in \mathbb{R}. \\
\tilde{u}(x,0)&=&f(x), \quad x \in \mathbb{R}, \\
\partial_t \tilde{u}(x,0)&=&g(x), \quad x \in \mathbb{R}\\
\end{eqnarray*}
Prove that $U = u + \tilde{u}$ solves
\begin{eqnarray*}
(\partial_t^2 - c^2 \partial_x^2)U(x,t) &=& \phi(x,t), \quad x,t \in \mathbb{R}. \\
U(x,0)&=&f(x), \quad x \in \mathbb{R}, \\
\partial_t U(x,0)&=&g(x), \quad x \in \mathbb{R}\\
\end{eqnarray*}
This is a special case of the so-called principle of superposition. It works here
because the problem is linear.

}
\begin{solution}
We start by proving that $U$ satisfies the PDE,
\begin{align*}
(\partial_t^2-c^2\partial_x^2)U&=(\partial_t^2-c^2\partial_x^2)(u+\tilde{u})\\
&=\partial_t^2(u+\tilde{u})-c^2\partial_x^2(u+\tilde{u})\\
&=\partial_t^2u+\partial_t^2\tilde{u}-c^2\partial_x^2u-c^2\partial_x^2\tilde{u}
&=(\partial_t^2-c^2\partial_x^2)u+(\partial_t^2-c^2\partial_x^2)\tilde{u}\\
&=\phi(x,t)+0
&=\phi(x,t)
\end{align*}
Now we check that it satisfies the first initial condition,
\begin{align*}
U(x,0)&=u(x,0)+\tilde{u}(x,0)\\
&=0+f(x)\\
&=f(x).
\end{align*}
To check the second initial condition we first calculate
\begin{align*}
\partial_tU(x,t)=\partial_t(u+\tilde{u})=\partial_tu(x,t)+\partial_t\tilde{u}(x,t),
\end{align*}
and make $t=0$,
\begin{align*}
\partial_tU(x,0)&=\partial_tu(x,0)+\partial_t\tilde{u}(x,0)\\
&=0+g(x)\\
&=g(x).
\end{align*}
Hence, since $U=u+\tilde{u}$ satisfies the PDE and the initial conditions, it is the solution.
\end{solution}

\end{questions}