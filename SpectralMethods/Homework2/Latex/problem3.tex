\textbf{Write a function that solves Burgers equation
\begin{align*}
&u_t+uu_x=\varepsilon u_{xx},~~~~~x\in(0,1),~~t\in(0,t_{max}]\\
&u(0,x)=sin^4(\pi x)
\end{align*}
and periodic boundary conditions. Use Fourier to compute derivatives in space and ode113 to advance in time. Solve this PDE for $\varepsilon= 0.1,0.01,$ and $0.001$. In each case, can you find solutions that areaccurate to three digits at $t= 1$?}
\newline

Since the domain is $1$-periodic, in order to use Fourier DFT comfortably we will change the space variable $x$ to work with a PDE and boundary conditions $2\pi$-periodic. The convenient change of variable is $y = 2\pi x$. Now $y$ is $2\pi$-periodic. Then, the PDE is changed as follows,
\begin{align*}
&u_t+2\pi uu_y=4\pi^2\varepsilon u_{yy},~~~~~y\in(0,2\pi),~~t\in(0,t_{max}]\\
&u(0,y)=sin^4(y/2).
\end{align*}
We can easily solve this PDE using Fourier Derivatives, see code below. The errors I have obtained, using the infinity norm, are:
\begin{center}
 \begin{tabular}{||c c c||} 
 \hline
 $\varepsilon$ & $N$ & Error \\ [0.5ex] 
 \hline\hline
 $0.1$  & $64$ & $0.4192\cdot 10^{-3}$ \\ 
 \hline
 $0.01$  & $64$ & $0.4839\cdot 10^{-3}$ \\
 \hline
 $0.001$ & $512$ & $0.9414\cdot 10^{-3}$ \\
 \hline
\end{tabular}
\end{center}
We can see how, the smaller the $\varepsilon$ (and therefore more insignificant the diffusion), the more difficult it is to obtain an accurate solution.

\subsection*{Matlab code for this problem}
\begin{verbatim}
%% Problem 3
tmax = 1;
eps = [1e-1 1e-2 1e-3]';
tol = 1e-3;
mesh = zeros(length(eps),1);
E = zeros(length(eps),1);
for m=1:length(eps)
    j = 5;
    N = 2^j;
    error = 1;
    u = PDE_solve(N,tmax,eps(m),false);
    while (error>tol && N<2048)
        j = j+1;
        N = 2^j;
        ufine = PDE_solve(N,tmax,eps(m),false);
        error = norm(u(end,:)-ufine(end,2:2:end),inf);
        u = ufine;
        j
    end
    m
    mesh(m) = N;
    E(m) = error;
end
mesh
E

function [u,x]=PDE_solve(N,tmax,eps,movie)
x0 = 0;
xf = 2*pi;
h = (xf-x0)/N;
x = x0 + h*(1:N)';
u0 = sin(x/2).^4;
 
tic;
[t,u] = ode113(@(t,u) rhs(u,N,eps), [0 tmax], u0);
toc
if movie
    figure
    for k = 1:2:length(t)
        plot(x,u(k,:));
        axis([x0 xf -.2 1.2]);
        drawnow
    end
end

end
 
function y = rhs(u,N,eps)
    vhat = fft(u);
    what = 1i*[0:N/2-1 0 -N/2+1:-1]' .* vhat;
    what2 = -([0:N/2-1 N/2 -N/2+1:-1].^2)' .* vhat;
    
    ux = real(ifft(what));
    uxx = real(ifft(what2));
    
    y = 4*pi^2*eps*uxx-2*pi*u.*ux;
end
\end{verbatim}