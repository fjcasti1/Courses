\textbf{Modify Program 20 to make use of matricves instead of FFT. Make sure to do this elegantly, using matrix-matrix multiplications rather than explicit loops. You will find that the code gets much shorter, and faster, too. How much faster is it? Does increasing $N$ from 24 to 48 tip the balance}
\newline

In the following table we have the times spent by the different codes to compute the first derivative (the plotting section was commented out). It can be seen that as $N$ gets large using matrices is more expensive.

\begin{table}[h]
\centering
 \begin{tabular}{||c c c||} 
 \hline
 $N$ & FFT & Matrices \\ [0.5ex] 
 \hline\hline
 24 & 0.1420280 & 0.0113390 \\ 
 \hline
 48 & 1.0236130 & 3.7638280 \\ [1ex] 
 \hline
\end{tabular}
\caption{Time spent in computing the first derivative of $f$ using FFTs and matrices.}
\end{table}

\subsection*{Matlab code for this problem}
\begin{verbatim}
format long
% Grid and initial data:
Nvector = [24,48];
time = zeros(length(Nvector),1);
for k = 1:length(Nvector)
    N=Nvector(k);
    [D,x] = cheb(N);  y = x;
    D2 = D^2;
    D2 = D2(2:end-1,2:end-1);
    L = kron(eye(N-1),D2)+kron(D2,eye(N-1));
    dt = 6/N^2;
    [xx,yy] = meshgrid(x(2:N),y(2:N));
    x = xx(:); y = yy(:);

    plotgap = round((1/3)/dt); dt = (1/3)/plotgap;
    u = exp(-40*((x-.4).^2 + y.^2));
    uold = u; 

    % Time-stepping by leap frog formula:
    [ay,ax] = meshgrid([.56 .06],[.1 .55]); clf
    tic
    for n = 0:3*plotgap
        t = n*dt;
%         if rem(n+.5,plotgap)<1     % plots at multiples of t=1/3
%           uu = reshape(u,N-1,N-1);
%           i = n/plotgap+1;
%           subplot('position',[ax(i) ay(i) .36 .36])
%           [xxx,yyy] = meshgrid(-1:1/16:1,-1:1/16:1);
%           uuu = interp2(xx,yy,uu,xxx,yyy,'cubic');
%           mesh(xxx,yyy,uuu), axis([-1 1 -1 1 -0.15 1])
%           colormap(1e-6*[1 1 1]); title(['t = ' num2str(t)]), drawnow
%         end
    % -------------------- %    
    % -- Using matrices -- %
    % -------------------- %    
        unew = 2*u - uold + dt^2*L*u; 
        uold = u; u = unew;
    % --------------- %
    % -- Using fft -- % Done in Program 20
    % --------------- %
    %     uxx = zeros(N+1,N+1); uyy = zeros(N+1,N+1);
    %     ii = 2:N;
    %     for i = 2:N                % 2nd derivs wrt x in each row
    %       v = vv(i,:); V = [v fliplr(v(ii))];
    %       U = real(fft(V));
    %       W1 = real(ifft(1i*[0:N-1 0 1-N:-1].*U)); % diff wrt theta
    %       W2 = real(ifft(-[0:N 1-N:-1].^2.*U));    % diff^2 wrt theta
    %       uxx(i,ii) = W2(ii)./(1-x(ii).^2) - x(ii).* ... 
    %                      W1(ii)./(1-x(ii).^2).^(3/2);
    %     end
    %     for j = 2:N                % 2nd derivs wrt y in each column
    %       v = vv(:,j); V = [v; flipud(v(ii))];
    %       U = real(fft(V));
    %       W1 = real(ifft(1i*[0:N-1 0 1-N:-1]'.*U));% diff wrt theta   
    %       W2 = real(ifft(-[0:N 1-N:-1]'.^2.*U));   % diff^2 wrt theta
    %       uyy(ii,j) = W2(ii)./(1-y(ii).^2) - y(ii).* ...
    %                      W1(ii)./(1-y(ii).^2).^(3/2);
    %     end
    end
    time(k) = toc;
end
\end{verbatim}