\textbf{Find a way to modify your program of Exercise 8.4, as in Exercise 3.8 but now for a second order problem, to make use of matrix exponentials rather than time discratization. What effect does this have on the computation time?}
\newline

In this case, just for $N=24$ it took $t=  25.3899480~s$ to complete the simulation, a severe increase for such a small number of grid points.

\subsection*{Matlab code for this problem}
\begin{verbatim}
%% Problem 5 - 8.5 Trefethen

% Grid and initial data:

Nvector = [24];
time = zeros(length(Nvector),1);
for k = 1:length(Nvector)
    N=Nvector(k);
    [D,x] = cheb(N);  y = x;
    D2 = D^2;
    D2 = D2(2:end-1,2:end-1);
    L = kron(eye(N-1),D2)+kron(D2,eye(N-1));
    A = [zeros(size(L)) eye(size(L)); L zeros(size(L))];
    dt = 6/N^2;
    [xx,yy] = meshgrid(x(2:N),y(2:N));
    x = xx(:); y = yy(:);

    plotgap = round((1/3)/dt); dt = (1/3)/plotgap;
    u0 = exp(-40*((x-.4).^2 + y.^2));
    u = u0;
    % Time-stepping by leap frog formula:
    [ay,ax] = meshgrid([.56 .06],[.1 .55]); clf
    tic
    for n = 0:3*plotgap
        t = n*dt;
%         if rem(n+.5,plotgap)<1     % plots at multiples of t=1/3
%           uu = reshape(u,N-1,N-1);
%           i = n/plotgap+1;
%           subplot('position',[ax(i) ay(i) .36 .36])
%           [xxx,yyy] = meshgrid(-1:1/16:1,-1:1/16:1);
%           uuu = interp2(xx,yy,uu,xxx,yyy,'cubic');
%           mesh(xxx,yyy,uuu), axis([-1 1 -1 1 -0.15 1])
%           colormap(1e-6*[1 1 1]); title(['t = ' num2str(t)]), drawnow
%         end
    % ------------------------------ %    
    % -- Using matrix exponential -- %
    % ------------------------------ %    
        v = expm(A*t)*[u0;zeros(size(u0))];
        u = v(1:length(u0));
    end
    time(k) = toc;
end
\end{verbatim}