% A (minimal) template for problem sets and solutions using the exam document class

% Organization:
%% Define new commands, macros, etc. in macros.tex
%% Anything that you would put before \begin{document} should go in prelude.tex

%% For multiple psets, each should get its own file to \input into main with a \section{}
\documentclass[answers]{exam}
\qformat{}

\usepackage{cancel}
\usepackage{amsmath}
\usepackage{amsthm}
\usepackage{amsfonts}
\usepackage{mathtools}
\usepackage{amssymb}
\usepackage{mathrsfs}
\usepackage{graphicx}
\usepackage{enumitem}
\renewcommand{\qedsymbol}{$\blacksquare$}

\newcommand{\R}{\mathbb{R}}
\newcommand{\C}{\mathbb{C}}
\newcommand{\Z}{\mathbb{Z}}
\newcommand{\N}{\mathbb{N}}
\newcommand\tab[1][1cm]{\hspace*{#1}}
\newcommand{\K}{\mathbb{K}}
\newcommand{\zero}{\mathbb{O}}
\begin{document}

\title{Computational Fluid Dynamics\\Homework 5}
\author{Francisco Jose Castillo Carrasco}
\date{\today}
\maketitle

%% \union - Example: \union{j \in J}{A_j}
\newcommand{\union}[2]{\underset{#1}\bigcup #2}

%% \inter - like \union, but with \bigcap
\newcommand{\inter}[2]{\underset{#1}\bigcap #2}

%% Content goes here
\section*{Introduction}
In this assignment I will be using the Crank-Nicholson method to solve the following PDE
\begin{align*}
\frac{\partial T}{\partial t}=\alpha\frac{\partial^2 T}{\partial x^2}+q(x,t),
\end{align*}
defined on the domain $x\in[-1,1]$ with boundary conditions 
\begin{align*}
T(-1,t)&=2-\sin\left(\frac{3\pi}{2}t\right),\\
\frac{\partial T}{\partial x}(1,t)&=0,
\end{align*}
and initial condition $T(x,0)=2$, with $\alpha=0.1$. The source term $q$ for $t>0$ is given by
\begin{align*}
q(x,t)=\frac{3\pi}{2\sqrt{t}}&\sin\left(\frac{\pi}{2}\sqrt{t}\right)\cos\left(\frac{\pi}{2}\sqrt{t}\right)\left(x^3-x^2-x+1\right)+\frac{3\pi}{2}\cos\left(\frac{3\pi}{2}t\right)\sin\left(\frac{\pi}{2}x\right)\\
&+\alpha\left(\sin^2\left(\frac{\pi}{2}\sqrt{t}\right)(6-18x)+\frac{\pi^2}{4}\sin\left(\frac{3\pi}{2}t\right)\sin\left(\frac{\pi}{2}x\right)\right),
\end{align*}
and $q(x,0)=0$.

This first part of the assignment (the one typed) includes the formulation of the problem using the Crank Nicholson method for a node based and a cell based mesh.
\subsection*{Node Based Mesh}
For simplicity, since I am using \textsc{Matlab}, I will discretize the domain starting at node 1. Since we are using $N$ elements, a node based mesh has $N+1$ nodes in the coordinate $x$. The number of nodes in time will depend on the size of our time step and for how long we want to run the simulation. The PDE discretized is
\begin{align*}
\left.\frac{\partial T}{\partial t}\right|_i^n=\alpha\left.\frac{\partial^2 T}{\partial x^2}\right|_i^n+q_i^n,
\end{align*}
where $n$ is a superscript and not a power. Note that, to follow \textsc{Matlab} indices, $i=1,2,...,N+1$. The boundary conditions in index form are expressed as
\begin{align*}
&T_1^n=2-\sin\left(\frac{3\pi}{2}t_n\right)=g(t_n)=g^n,\\
&\left.\frac{\partial T}{\partial x}\right|_{N+1}^n=0~~\Rightarrow~~T_{N+1}^n=T_{N}^n,\\
\end{align*}
and the initial condition as $T_i^0=2$. In the time coordinate we don't follow \textsc{Matlab} indices since we will be overwriting the solution, therefore there is no problem with the index $n=0$. Now we can take the discretized PDE and apply forward finite differences in time and central in space, keeping in mind that the Crank-Nicholson method average the right hand side between the present time step $n$ and the next one $n+1$. Thus, the PDE turns into
\begin{align*}
\frac{T_i^{n+1}-T_i^n}{\Delta t}=\frac{\alpha}{2}\left(\frac{T_{i+1}^{n+1}-2T_{i}^{n+1}+T_{i-1}^{n+1}}{h^2}+\frac{T_{i+1}^{n}-2T_{i}^{n}+T_{i-1}^{n}}{h^2}\right)+\frac{1}{2}\left(q_i^n+q_i^{n+1}\right).
\end{align*}
Multiplying by the time step and taking $1/h^2$ common factor,
\begin{align*}
T_i^{n+1}-T_i^n=\frac{\alpha\Delta t}{2h^2}\left(T_{i+1}^{n+1}-2T_{i}^{n+1}+T_{i-1}^{n+1}+T_{i+1}^{n}-2T_{i}^{n}+T_{i-1}^{n}\right)+\frac{\Delta t}{2}\left(q_i^n+q_i^{n+1}\right).
\end{align*}
Defining 
\begin{align*}
B=\frac{\alpha\Delta t}{2h^2},
\end{align*}
and gathering all the $n+1$ terms on the left hand side and the $n$ terms on the right hand side we obtain
\begin{align*}
-BT_{i-1}^{n+1}+(1+2B)T_i-BT_{i+1}^{n+1}=T_i^n+B\left(T_{i+1}^n-2T_{i}^n+T_{i-1}^n\right)+\frac{\Delta t}{2}\left(q_i^n+q_i^{n+1}\right),
\end{align*}
which has the tridiagonal form that we were looking form
\begin{align*}
a_iT_{i-1}^{n+1}+b_iT_i+c_iT_{i+1}^{n+1}=d_i, i=1,...,N+1.
\end{align*}
We only need to solve this tridiagonal system for the interior nodes, $i=2,...,N$, since we have the boundary conditions at $i=1$ and $i=N+1$. Therefore the vectors $a,b,c,d$ have dimension $N-1$. To account for this difference in size we modify the previous equation with a $+1$ shift in the $i$ index and the previous equations can be written as
\begin{align*}
-BT_{i}^{n+1}+(1+2B)T_{i+1}-BT_{i+2}^{n+1}=T_{i+1}^n+B\left(T_{i+2}^n-2T_{i+1}^n+T_{i}^n\right)+\frac{\Delta t}{2}\left(q_{i+1}^n+q_{i+1}^{n+1}\right),
\end{align*}
\begin{align*}
a_{i}T_{i}^{n+1}+b_{i}T_{i+1}+c_{i}T_{i+1}^{n+1}=d_{i},~~~~ i=1,...,N-1
\end{align*} 
Now there is a direct correspondance between the values of the temperature vector and the vectors $a,b,c,d$ needed to solve the tridiagonal system in the interior. Thus, we have
\begin{align*}
&a_i=-B,\\
&b_i=1+2B,\\
&c_i=-B,\\
d_i=T_{i+1}^n+&B\left(T_{i+2}^n-2T_{i+1}^n+T_{i}^n\right)+\frac{\Delta t}{2}\left(q_{i+1}^{n}+q_{i+1}^{n+1}\right),\\
\end{align*}
where $i=1,2,...,N-1$. However, the previous values of the tridiagonal vectors are only acceptable for the \textit{interior of the interior}, $i=2,...,N-2$. For the last nodes of the interior we need to impose the boundary conditions:
\begin{itemize}
\item Solve for $i=1$.
\begin{align*}
-BT_{1}^{n+1}+(1+2B)T_2^{n+1}-BT_{3}^{n+1}&=T_2^n+B\left(T_{3}^n-2T_{2}^n+T_{1}^n\right)+\frac{\Delta t}{2}\left(q_2^n+q_2^{n+1}\right),\\
(1+2B)T_2^{n+1}-BT_{3}^{n+1}&=T_2^n+B\left(T_{3}^n-2T_{2}^n+g^n+g^{n+1}\right)+\frac{\Delta t}{2}\left(q_2^n+q_2^{n+1}\right),
\end{align*}
where we have used that $T_1^n=g^n$ and $T_1^{n+1}=g^{n+1}$. Hence,
\begin{align*}
&a_1=0,\\
&b_1=1+2B,\\
&c_1=-B,\\
d_1=T_2^n+&B\left(T_{3}^n-2T_{2}^n+g^n+g^{n+1}\right)+\frac{\Delta t}{2}\left(q_2^n+q_2^{n+1}\right).
\end{align*}
However, since $a_1$ does not affect the solution, there is no need to change its value.
\item Solve for $i=N-1$.
\begin{align*}
-BT_{N-1}^{n+1}+(1+2B)T_N^{n+1}-BT_{N+1}^{n+1}=T_N^n+B\left(T_{N+1}^n-2T_{N}^n+T_{N-1}^n\right)+\frac{\Delta t}{2}\left(q_N^n+q_N^{n+1}\right),\\
-BT_{N-1}^{n+1}+(1+2B)T_N^{n+1}-BT_{N}^{n+1}=T_N^n+B\left(T_{N}^n-2T_{N}^n+T_{N-1}^n\right)+\frac{\Delta t}{2}\left(q_N^n+q_N^{n+1}\right),\\
-BT_{N-1}^{n+1}+(1+B)T_N^{n+1}=T_N^n+B\left(-T_{N}^n+T_{N-1}^n\right)+\frac{\Delta t}{2}\left(q_N^n+q_N^{n+1}\right),
\end{align*}
where we have used that $T_{N+1}^{n+1}=T_N^{n+1}$ and $T_{N+1}^n=T_N^n$. Hence,
\begin{align*}
&a_{N-1}=-B,\\
&b_{N-1}=1+B,\\
&c_{N-1}=0,\\
d_{N-1}=T_N^n+&B\left(-T_{N}^n+T_{N-1}^n\right)+\frac{\Delta t}{2}\left(q_N^n+q_N^{n+1}\right).
\end{align*}
However, since $c_{N-1}$ does not affect the solution, there is no need to change its value.
\end{itemize}

\subsection*{Cell Based Mesh}
In the case of a cell based mesh we use a similar approach. Since the points of study are the centers of the cells, on the domain $-1\leq x\leq 1$, the points are located at $-\frac{h}{2}-1,-1+\frac{h}{2},-1+\frac{3h}{2},...,1-\frac{h}{2},1+\frac{h}{2}$, where the first and last points correspond to the ghost cells. Therefore, having $N$ elements we will have $N+2$ cell centers. To follow \textsc{Matlab} indices, $x_1=-1-\frac{h}{2}$, $x_2=-1+\frac{h}{2}$, and so on. The temperature values $T_i$ are the ones corresponding to the cell centered positions $x_i$. Hence, for $N$ elements we have $T_i$ values of tempereature, $i=1,2,...,T_{N+2}$, where $T_1$ and $T_{N+2}$ correspond to ghost cells. Like before we only need to solve for the interior of our vector $T$. The discretized PDE is the same as before
\begin{align*}
\left.\frac{\partial T}{\partial t}\right|_i^n=\alpha\left.\frac{\partial^2 T}{\partial x^2}\right|_i^n+q_i^n,
\end{align*}
where $n$ is a superscript and not a power. Note that, to follow \textsc{Matlab} indices, $i=1,2,...,N+2$. The boundary conditions in index form are expressed as
\begin{align*}
&T_{1+1/2}^n=\frac{T_1^n+T_2^n}{2}=2-\sin\left(\frac{3\pi}{2}t_n\right)=g^n~~\Rightarrow~~T_1^n=2g^n-T_2^n,\\
&\left.\frac{\partial T}{\partial x}\right|_{N+1+1/2}^n=0~~\Rightarrow~~T_{N+2}^n=T_{N+1}^n,\\
\end{align*}
and the initial condition as $T_i^0=2$. In the time coordinate we don't follow \textsc{Matlab} indices since we are overwriting the solution, therefore there is no problem with the index $n=0$. Note that the point $T_{1+1/2}$ corresponds to the temperature value between the cells $1$ and $2$, between the ghost cell and the first cell of the domainat the boundary, i.e. at the boundary. Same thing for the other boundary. The code is going to work following this process: calculate tridiagonal vectors for the present time step, solve the tridiagonal system for the interior cells and obtain the solution for the interior in the next time step, update the ghost cells, recalculate the tridiagonal vectors and repeat. The Crank-Nicholson index equation, thanks to the index formulation chosen in both cases, is the same as in the previous section
\begin{align*}
-BT_{i-1}^{n+1}+(1+2B)T_i-BT_{i+1}^{n+1}=T_i^n+B\left(T_{i+1}^n-2T_{i}^n+T_{i-1}^n\right)+\frac{\Delta t}{2}\left(q_i^n+q_i^{n+1}\right),
\end{align*}
where $B$ is defined as above
\begin{align*}
B=\frac{\alpha\Delta t}{2h^2}.
\end{align*}
The index equation has the tridiagonal form that we were looking form
\begin{align*}
a_iT_{i-1}^{n+1}+b_iT_i+c_iT_{i+1}^{n+1}=d_i, i=1,...,N+2.
\end{align*}

We only need to solve this tridiagonal system for the interior nodes, $i=2,...,N+1$, since we have the ghost cells at $i = 1$ and $i = N + 2$. Therefore the vectors $a, b, c, d$ have dimension $N$, and will start at the first node of the interior. Hence, there is a $+1$ shift in the space index to account for this difference in size and the previous equation can be written as
\begin{align*}
-BT_{i}^{n+1}+(1+2B)T_{i+1}-BT_{i+2}^{n+1}=T_{i+1}^n+B\left(T_{i+2}^n-2T_{i+1}^n+T_{i}^n\right)+\frac{\Delta t}{2}\left(q_{i+1}^n+q_{i+1}^{n+1}\right),
\end{align*}
\begin{align*}
a_iT_{i}^{n+1}+b_iT_{i+1}+c_iT_{i+2}^{n+1}=d_i,~~~~i=1,...,N.
\end{align*}
The vectors $a,b,c,d$ have dimension $N$ (whereas they had dimension $N-1$ in the node based mesh). Hence,
\begin{align*}
&a_i=-B,\\
&b_i=1+2B,\\
&c_i=-B,\\
d_i=T_{i+1}^n+&B\left(T_{i+2}^n-2T_{i+1}^n+T_{i}^n\right)+\frac{\Delta t}{2}\left(q_{i+1}^n+q_{i+1}^{n+1}\right).
\end{align*}
for $i=2,...,N-1$. To obtain the values of the tridiagonal vector at the first and last cell we solve the equation at those locations.
\begin{itemize}
\item Solve for $i=1$.
\begin{align*}
-BT_{1}^{n+1}+(1+2B)T_2^{n+1}-BT_{3}^{n+1}&=T_2^n+B\left(T_{3}^n-2T_{2}^n+T_{1}^n\right)+\frac{\Delta t}{2}\left(q_2^n+q_2^{n+1}\right),\\
-2Bg^{n+1}+BT_2^{n+1}+(1+2B)T_2^{n+1}-BT_{3}^{n+1}&=T_2^n+B\left(T_{3}^n-2T_{2}^n+2g^n-T_2^n\right)+\frac{\Delta t}{2}\left(q_2^n+q_2^{n+1}\right),\\
(1+3B)T_2^{n+1}-BT_{3}^{n+1}&=T_2^n+B\left(T_{3}^n-3T_{2}^n+2g^n+2g^{n+1}\right)+\frac{\Delta t}{2}\left(q_2^n+q_2^{n+1}\right),
\end{align*}
where we have used that $T_1^n=2g^n-T_2^n$ and $T_1^{n+1}=2g^{n+1}-T_2{n+1}$. Hence,
\begin{align*}
&a_1=0,\\
&b_1=1+3B,\\
&c_1=-B,\\
d_1=T_2^n+&B\left(T_{3}^n-3T_{2}^n+2g^n+2g^{n+1}\right)+\frac{\Delta t}{2}\left(q_2^n+q_2^{n+1}\right).
\end{align*}
However, since $a_1$ does not affect the solution, there is no need to change its value.
\item Solve for $i=N$.
\begin{align*}
-BT_{N}^{n+1}+(1+2B)T_{N+1}^{n+1}-BT_{N+2}^{n+1}=T_{N+1}^n+B\left(T_{N+2}^n-2T_{N+1}^n+T_{N}^n\right)+\frac{\Delta t}{2}\left(q_{N+1}^n+q_{N+1}^{n+1}\right),\\
-BT_{N}^{n+1}+(1+2B)T_{N+1}^{n+1}-BT_{N+1}^{n+1}=T_{N+1}^n+B\left(T_{N+1}^n-2T_{N+1}^n+T_{N}^n\right)+\frac{\Delta t}{2}\left(q_{N+1}^n+q_{N+1}^{n+1}\right),\\
-BT_{N}^{n+1}+(1+B)T_{N+1}^{n+1}=T_{N+1}^n+B\left(-T_{N+1}^n+T_{N}^n\right)+\frac{\Delta t}{2}\left(q_{N+1}^n+q_{N+1}^{n+1}\right),
\end{align*}
where we have used that $T_{N+2}=T_{N+1}$. Hence,
\begin{align*}
&a_{N}=-B,\\
&b_{N}=1+B,\\
&c_{N}=0,\\
d_{N}=T_{N+1}^n+&B\left(-T_{N+1}^n+T_{N}^n\right)+\frac{\Delta t}{2}\left(q_{N+1}^n+q_{N+1}^{n+1}\right).
\end{align*}
However, since $c_{N}$ does not affect the solution, there is no need to change its value.
\end{itemize}
The values specified in both the node base mesh and the cell based mesh will be implemented in \textsc{Matlab} to obtain the solutions.
\section*{Maximum stable time step for the FTCS}
As we saw in class, the maximum stable time step size for the FTCS is
\begin{align*}
\Delta t=\frac{h^2}{2\alpha},
\end{align*}
and the time step used in the code is four time this,
\begin{align*}
\Delta t=2\frac{h^2}{\alpha}.
\end{align*}
There is no index form for this.
\end{document}