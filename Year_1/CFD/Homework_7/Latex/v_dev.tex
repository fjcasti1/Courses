\subsection{Step 1}
Let's start by rewriting our ADI equation for the Step 1 (given in class 13 slide 13), for an staggered mesh for the vertical velocity $v$:
\begin{align*}
-d_1v_{i-1,j+1/2}^{n+1/2}+(1+2d_1)v_{i,j+1/2}^{n+1/2}-d_1v_{i+1,j+1/2}^{n+1/2}=d_2v_{i,j+3/2}^{n}+(1-2d_2)v_{i,j+1/2}^{n}+d_2v_{i,j-1/2}^{n},
\end{align*}
which has the form 
\begin{align*}
av_{i-1,j+1/2}^{n}+bv_{i,j+1/2}^{n}+cv_{i+1,j+1/2}^{n}=d,
\end{align*}
a tridiagonal system. The staggered mesh for $v$ gives us a matrix $v$ that is $(M+2)\times(N+1)$. We only need to solve the previous equation for the interior with two $for$ loops in \textsl{Matlab},
\begin{verbatim}
for j=2:N
for i=2:M+1
		...
	end
end
\end{verbatim}.
We will now include the boundary conditions for each case.
\subsubsection*{Case $j=2$ and $i=2$}
In this case the general equation becomes
\begin{align*}
-d_1v_{1,5/2}^{n+1/2}+(1+2d_1)v_{2,5/2}^{n+1/2}-d_1v_{3,5/2}^{n+1/2}=d_2v_{2,7/2}^{n}+(1-2d_2)v_{2,5/2}^{n}+d_2v_{2,3/2}^{n},
\end{align*}
The value $v_{1,5/2}^{n+1/2}=-v_{2,5/2}^{n+1/2}$ by the boundary conditions and $v_{2,3/2}^{n}=0$ since it corresponds to the bottom wall. Thus the equation yields
\begin{align*}
(1+3d_1)v_{2,5/2}^{n+1/2}-d_1v_{3,5/2}^{n+1/2}=d_2v_{2,7/2}^{n}+(1-2d_2)v_{2,5/2}^{n},
\end{align*}
giving $a=0$ and $b=1+3d_1$ for this case, $d$ is given by the right hand side.
\subsubsection*{Case $j=2$ and $i\in [3,M]$}
In this case the general equation becomes
\begin{align*}
-d_1v_{i-1,5/2}^{n+1/2}+(1+2d_1)v_{i,5/2}^{n+1/2}-d_1v_{i+1,5/2}^{n+1/2}=d_2v_{i,7/2}^{n}+(1-2d_2)v_{i,5/2}^{n}+d_2v_{i,3/2}^{n},
\end{align*}
Like in the previous case, the value $v_{i,3/2}^{n}=0$ and $d$ is the right hand side of the previous equation.

\subsubsection*{Case $j=2$ and $i=M+1$}
In this case the general equation becomes
\begin{align*}
-d_1v_{M,5/2}^{n+1/2}+(1+2d_1)v_{M+1,5/2}^{n+1/2}-d_1v_{M+2,5/2}^{n+1/2}=d_2v_{M+1,7/2}^{n}+(1-2d_2)v_{M+1,5/2}^{n}+d_2v_{M+1,3/2}^{n},
\end{align*}
Like in the previous case, the value $v_{M+1,3/2}^{n}=0$. The value of $v_{M+2,5/2}^{n+1/2}=-v_{M+1,5/2}^{n+1/2}$ and the previous equation yields
\begin{align*}
-d_1v_{M,5/2}^{n+1/2}+(1+3d_1)v_{M+1,5/2}^{n+1/2}=d_2v_{M+1,7/2}^{n}+(1-2d_2)v_{M+1,5/2}^{n}.
\end{align*}
Thus, $c=0$, $b=1+3d_1$ and $d$ is the right hand side of the previous equation.

\subsubsection*{Case $j\in[3,N-1]$ and $i=2$}
In this case the general equation becomes
\begin{align*}
-d_1v_{1,j+1/2}^{n+1/2}+(1+2d_1)v_{2,j+1/2}^{n+1/2}-d_1v_{3,j+1/2}^{n+1/2}=d_2v_{2,j+3/2}^{n}+(1-2d_2)v_{2,j+1/2}^{n}+d_2v_{2,j-1/2}^{n}.
\end{align*}
The value $v_{1,j+1/2}^{n+1/2}=-v_{2,j+1/2}^{n+1/2}$ by the boundary condition. Thus the equation yields
\begin{align*}
(1+3d_1)v_{2,j+1/2}^{n+1/2}-d_1v_{3,j+1/2}^{n+1/2}=d_2v_{2,j+3/2}^{n}+(1-2d_2)v_{2,j+1/2}^{n}+d_2v_{2,j-1/2}^{n},
\end{align*}
giving $a=0$, $b=1+3d_1$ and $d$ given by the right hand side.

\subsubsection*{Case $j\in[3,N-1]$ and $i\in[3,M]$}
In this case the general equation does not include any boundary conditions, therefore it is unaltered.

\subsubsection*{Case $j\in[3,N-1]$ and $i=M+1$}
In this case the general equation becomes
\begin{align*}
-d_1v_{M,j+1/2}^{n+1/2}+(1+2d_1)v_{M+1,j+1/2}^{n+1/2}-d_1v_{M+2,j+1/2}^{n+1/2}=d_2v_{M+1,j+3/2}^{n}&+(1-2d_2)v_{M+1,j+1/2}^{n}\\&+d_2v_{M+1,j-1/2}^{n}.
\end{align*}
The value $v_{M+2,j+1/2}^{n+1/2}=-v_{M+1,j+1/2}^{n+1/2}$ by the boundary condition. Thus the equation yields
\begin{align*}
-d_1v_{M,j+1/2}^{n+1/2}+(1+3d_1)v_{M+1,j+1/2}^{n+1/2}=d_2v_{M+1,j+3/2}^{n}+(1-2d_2)v_{M+1,j+1/2}^{n}+d_2v_{M+1,j-1/2}^{n},
\end{align*}
giving $c=0$, $b=1+3d_1$ and $d$ given by the right hand side.

\subsubsection*{Case $j=N$ and $i=2$}
In this case the general equation becomes
\begin{align*}
-d_1v_{1,N+1/2}^{n+1/2}+(1+2d_1)v_{2,N+1/2}^{n+1/2}-d_1v_{3,N+1/2}^{n+1/2}=d_2v_{2,N+3/2}^{n}+(1-2d_2)v_{2,N+1/2}^{n}+d_2v_{2,N-1/2}^{n},
\end{align*}
The value $v_{1,N+1/2}^{n+1/2}=-v_{2,N+1/2}^{n+1/2}$ by the boundary conditions and $v_{2,N+3/2}^{n}=0$ since it corresponds to the top wall. Thus the equation yields
\begin{align*}
(1+3d_1)v_{2,N+1/2}^{n+1/2}-d_1v_{3,N+1/2}^{n+1/2}=(1-2d_2)v_{2,N+1/2}^{n}+d_2v_{2,N-1/2}^{n},
\end{align*}
giving $a=0$ and $b=1+3d_1$ for this case, $d$ is given by the right hand side.

\subsubsection*{Case $j=N$ and $i\in [3,M]$}
In this case the general equation becomes
\begin{align*}
-d_1v_{i-1,N+1/2}^{n+1/2}+(1+2d_1)v_{i,N+1/2}^{n+1/2}-d_1v_{i+1,N+1/2}^{n+1/2}=d_2v_{i,N+3/2}^{n}+(1-2d_2)v_{i,N+1/2}^{n}+d_2v_{i,N-1/2}^{n},
\end{align*}
Like in the previous case, the value $v_{i,N+3/2}^{n}=0$ and $d$ is the right hand side of the previous equation.

\subsubsection*{Case $j=N$ and $i=M+1$}
In this case the general equation becomes
\begin{align*}
-d_1v_{M,N+1/2}^{n+1/2}+(1+2d_1)v_{M+1,N+1/2}^{n+1/2}-d_1v_{M+2,N+1/2}^{n+1/2}=d_2v_{M+1,N+3/2}^{n}&+(1-2d_2)v_{M+1,N+1/2}^{n}\\&+d_2v_{M+1,N-1/2}^{n}.
\end{align*}
By the boundary conditions, $v_{M+2,N+1/2}^{n+1/2}=-v_{M+1,N+1/2}^{n+1/2}$ and $v_{M+1,N+3/2}^{n}=0$ and the previous equation yields
\begin{align*}
-d_1v_{M,N+1/2}^{n+1/2}+(1+3d_1)v_{M+1,N+1/2}^{n+1/2}=(1-2d_2)v_{M+1,N+1/2}^{n}+d_2v_{M+1,N-1/2}^{n},
\end{align*}
Thus, $c=0$, $b=1+3d_1$ and $d$ is the right hand side of the previous equation.

\subsection{Step 2}
We start by rewriting our ADI equation for the Step 1 (given in class 13 slide 14), for an staggered mesh for the vertical velocity $v$:
\begin{align*}
-d_2v_{i,j-1/2}^{n+1}+(1+2d_2)v_{i,j+1/2}^{n+1}-d_2v_{i,j+3/2}^{n+1}=d_1v_{i+1,j+1/2}^{n+1/2}+(1-2d_1)v_{i,j+1/2}^{n+1/2}+d_1v_{i-1,j+1/2}^{n+1/2}
\end{align*}
which has the form 
\begin{align*}
av_{i,j-1/2}^{n+1}+bv_{i,j+1/2}^{n+1}+cv_{i,j+3/2}^{n+1}=d,
\end{align*}
a tridiagonal system. The staggered mesh for $v$ gives vs a matrix $v$ that is $(M+2)\times(N+1)$. We only need to solve the previous equation for the interior with two $for$ loops in \textsl{Matlab},
\begin{verbatim}
for j=2:M+1
for i=2:N
		...
	end
end
\end{verbatim}.
We will now include the boundary conditions for each case.

\subsubsection*{Case $i=2$ and $j=2$}
In this case the general equation becomes
\begin{align*}
-d_2v_{2,3/2}^{n+1}+(1+2d_2)v_{2,5/2}^{n+1}-d_2v_{2,7/2}^{n+1}=d_1v_{3,5/2}^{n+1/2}+(1-2d_1)v_{2,5/2}^{n+1/2}+d_1v_{1,5/2}^{n+1/2}
\end{align*}
By the boundary conditions, $v_{2,3/2}^{n+1}=0$ and $v_{1,5/2}^{n+1/2}=-v_{2,5/2}^{n+1/2}$. Thus $a=0$ and $d$ is given by the right hand side.

\subsubsection*{Case $i=2$ and $j\in[3,N-1]$}
In this case the general equation becomes
\begin{align*}
-d_2v_{2,j-1/2}^{n+1}+(1+2d_2)v_{2,j+1/2}^{n+1}-d_2v_{2,j+3/2}^{n+1}=d_1v_{3,j+1/2}^{n+1/2}+(1-2d_1)v_{2,j+1/2}^{n+1/2}+d_1v_{1,j+1/2}^{n+1/2}
\end{align*}
By the boundary conditions, $v_{1,j+1/2}^{n+1/2}=-v_{2,j+1/2}^{n+1/2}$ and $d$ is given by the right hand side.

\subsubsection*{Case $i=2$ and $j=N$}
In this case the general equation becomes
\begin{align*}
-d_2v_{2,N-1/2}^{n+1}+(1+2d_2)v_{2,N+1/2}^{n+1}-d_2v_{2,N+3/2}^{n+1}=d_1v_{3,N+1/2}^{n+1/2}+(1-2d_1)v_{2,N+1/2}^{n+1/2}+d_1v_{1,N+1/2}^{n+1/2}
\end{align*}
By the boundary conditions, $v_{2,N+3/2}^{n+1}=0$ and $v_{1,N+1/2}^{n+1/2}=-v_{2,N+1/2}^{n+1/2}$. Thus $c=0$ and $d$ is given by the right hand side.

\subsubsection*{Case $i\in[3,M]$ and $j=2$}
In this case the general equation becomes
\begin{align*}
-d_2v_{i,3/2}^{n+1}+(1+2d_2)v_{i,5/2}^{n+1}-d_2v_{i,7/2}^{n+1}=d_1v_{i+1,5/2}^{n+1/2}+(1-2d_1)v_{i,5/2}^{n+1/2}+d_1v_{i-1,5/2}^{n+1/2}
\end{align*}
By the boundary conditions, $v_{i,3/2}^{n+1}$ will be zero at the walls and will depend on $v_{i,5/2}^{n+1}$ and $v_{i,7/2}^{n+1}$ for the points at the outlet. Thus, $a=0$ and the values of $b$ and $c$ depend on $i$, $d$ is given by the right hand side.

\subsubsection*{Case $i\in[3,M]$ and $j\in[3,N-1]$}
In this case the general equation does not include any boundary conditions, therefore it is unaltered.

\subsubsection*{Case $i\in[3,M]$ and $j=N$}
In this case the general equation becomes
\begin{align*}
-d_2v_{i,N-1/2}^{n+1}+(1+2d_2)v_{i,N+1/2}^{n+1}-d_2v_{i,N+3/2}^{n+1}=d_1v_{i+1,N+1/2}^{n+1/2}+(1-2d_1)v_{i,N+1/2}^{n+1/2}+d_1v_{i-1,N+1/2}^{n+1/2}
\end{align*}
By the boundary conditions, the value of $v_{i,N+3/2}^{n+1}$ will depend on $i$. We will pass it to the right hand side and have
\begin{align*}
-d_2v_{i,N-1/2}^{n+1}+(1+2d_2)v_{i,N+1/2}^{n+1}=d_1v_{i+1,N+1/2}^{n+1/2}+(1-2d_1)v_{i,N+1/2}^{n+1/2}+d_1v_{i-1,N+1/2}^{n+1/2}+d_2v_{i,N+3/2}^{n+1}.
\end{align*}
Thus, $c=0$ and $d$ is given by the right hand side.

\subsubsection*{Case $i=M+1$ and $j=2$}
In this case the general equation becomes
\begin{align*}
-d_2v_{M+1,3/2}^{n+1}+(1+2d_2)v_{M+1,5/2}^{n+1}-d_2v_{M+1,7/2}^{n+1}=d_1v_{M+2,5/2}^{n+1/2}+(1-2d_1)v_{M+1,5/2}^{n+1/2}+d_1v_{M,5/2}^{n+1/2}
\end{align*}
By the boundary conditions, $v_{M+1,3/2}^{n+1}=0$ and $v_{M+2,5/2}^{n+1/2}=-v_{M+1,5/2}^{n+1/2}$. Thus, $a=0$ and $d$ is given by the right hand side.

\subsubsection*{Case $i=M+1$ and $j\in[3,N-1]$}
In this case the general equation becomes
\begin{align*}
-d_2v_{M+1,j-1/2}^{n+1}+(1+2d_2)v_{M+1,j+1/2}^{n+1}-d_2v_{M+1,j+3/2}^{n+1}=d_1v_{M+2,j+1/2}^{n+1/2}&+(1-2d_1)v_{M+1,j+1/2}^{n+1/2}\\&+d_1v_{M,j+1/2}^{n+1/2}
\end{align*}
By the boundary conditions, $v_{M+2,j+1/2}^{n+1/2}=-v_{M+1,j+1/2}^{n+1/2}$ and $d$ is given by the right hand side.

\subsubsection*{Case $i=M+1$ and $j=N$}
In this case the general equation becomes
\begin{align*}
-d_2v_{M+1,N-1/2}^{n+1}+(1+2d_2)v_{M+1,N+1/2}^{n+1}-d_2v_{M+1,N+3/2}^{n+1}=d_1v_{M+2,N+1/2}^{n+1/2}&+(1-2d_1)v_{M+1,N+1/2}^{n+1/2}\\+d_1v_{M,N+1/2}^{n+1/2}
\end{align*}
By the boundary conditions, $v_{M+1,N+3/2}^{n+1}=0$ and $v_{M+2,N+1/2}^{n+1/2}=-v_{M+1,N+1/2}^{n+1/2}$. Thus, $c=0$ and $d$ is given by the right hand side.