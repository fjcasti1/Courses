\begin{enumerate}[label=(\alph*)]
\item Prove that $x_1-x_0=\mathcal{O}(1/N^2)$.
\begin{align*}
x_1-x_0&=-\cos\left(\frac{\pi}{N}\right)+\cos(0)\\
&=1-\cos\left(\frac{\pi}{N}\right)\\
&=1-\left(1+\frac{1}{2}\left(\frac{\pi}{N}\right)^2+...\right)\\
&=-\frac{1}{2}\left(\frac{\pi}{N}\right)^2+...\\
&=\mathcal{O}(1/N^2),
\end{align*}
where we have expanded the cosine in Taylor series around zero since $\frac{\pi}{N}\rightarrow 0$ as $N\rightarrow\infty$.
\item Prove that $x_{N/2}-x_{N/2-1}=\mathcal{O}(1/N)$.
\begin{align*}
x_{N/2}-x_{N/2-1}&=-\cos\left(\frac{\pi}{2}\right)+\cos\left(\frac{\left(\frac{N}{2}-1\right)\pi}{N}\right)\\
&=\cos\left(\frac{\pi}{2}-\frac{\pi}{N}\right)\\
&=\cos\left(\frac{\pi}{2}\right)\cos\left(-\frac{\pi}{N}\right)-\sin\left(\frac{\pi}{2}\right)\sin\left(-\frac{\pi}{N}\right)\\
&=\sin\left(\frac{\pi}{N}\right)\\
&=\left(\frac{\pi}{N}-\frac{1}{6}\left(\frac{\pi}{N}\right)^3+...\right)\\
&=\mathcal{O}(1/N),
\end{align*}
where we have expanded the sine in Taylor series around zero since $\frac{\pi}{N}\rightarrow 0$ as $N\rightarrow\infty$.
\end{enumerate}
