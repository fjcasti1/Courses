\textbf{Modify the code diffusion} \verb+eq1_Neumann.m+ \textbf{to solve the diffusion equation with Neumann boundary conditions using finite differences.}

In this problem we use second order centered finite differences to calculate the second derivative in the interior
\begin{align*}
u''_j=\frac{u_{j-1}-2u_{j}+u_{j+1}}{h^2}.
\end{align*}
This discretization, if expressed in matrix form, gives us a tridiagonal matrix with $1,-2,1$ as lower diagonal, diagonal and upper diagonal, respectively. We only need to solve the following matrix equation
\begin{align*}
\frac{d}{dt}\textbf{u}=D_2\textbf{u},
\end{align*}
for the interior since we have the boundary conditions. They have been taking into account using a first order sided finite differences
\begin{align*}
u'_1=\frac{u_2-u_1}{h}=g_1(t)~\Rightarrow~u_1=u_2-hg_1(t),
\end{align*}
\begin{align*}
u'_{N+1}=\frac{u_{N+1}-u_N}{h}=g_2(t)~\Rightarrow~u_{N+1}=hg_2(t)+u_N,
\end{align*}
We will get rid of the first and last row and column the matrix $D_2$, obtaining $\tilde{D}_2$ and add another term to account for the boundary conditions of the Neumann kind in this case. The system obtained is the following
\begin{align*}
\frac{d}{dt}\tilde{\textbf{u}}=\tilde{D}_2\tilde{\textbf{u}}+\frac{1}{h}\begin{pmatrix} -1 & 1 \end{pmatrix}\begin{pmatrix} g_1(t) \\ g_2(t) \end{pmatrix},
\end{align*}
where $g_1(t)$ and $g_2(t)$ are the Neumann boundary condition functions at $x=-1$ and $x=1$, respectively. We defined them generally althought in this problem $g_1(t)=g_2(t)=1$. When solving for the interior, we will need the values at the boundaries that will be substituted by the two equations above. As a result, including the boundary conditions hasn't just altered the system adding that extra term, but it also changes the first and last entries of the matrix $\tilde{D}_2$. Computing the first and last nodes of the interior, and including the boundary conditions, we have
\begin{align*}
u''_2=\frac{u_1-2u_2+u_3}{h^2}&=\frac{u_2-hg_1(t)-2u_2+u_3}{h^2}\\
&=\frac{-u_2+u_3}{h^2}-\frac{1}{h} g_1(t),
\end{align*}
and
\begin{align*}
u''_N=\frac{u_{N-1}-2u_N+u_{N+1}}{h^2}&=\frac{u_{N-1}-2u_N+hg_2(t)+u_N}{h^2}\\
&=\frac{u_{N-1}-u_N}{h^2}+\frac{1}{h} g_2(t).
\end{align*}
This implies that the system is modified as showed above and the first and second entries of the matrix $\tilde{D}_2$ are
\begin{align*}
\tilde{D}_2(1,1)=\tilde{D}_2(N-1,N-1)=-\frac{1}{h^2}.
\end{align*}
Note that the $1/h^2$ factor is already included in $\tilde{D}_2$. The solution for the PDE is shown in the next figure.
\begin{figure}[H]
\centering     %%% not \center
{\includegraphics[scale=0.75]{P3_surf.eps}}
\caption{Solution of the diffusion equation, $N=50$.}
\end{figure}
We can see how at time zero we have the initial condition and in two seconds we reach the steady state where the whole domain has a slope 1, as imposed by the boundary conditions in the first and last nodes.

\subsection*{Matlab code for this problem}
\begin{verbatim}
%% Problem 3
% Simple code to solve the diffusion equation
% u_t = u_xx -1<x<1, with Neumann boundary conditions
clear all; close all; clc
legendfontsize=14;
axisfontsize=14;
labelfontsize=16;
N = 50;
L=2;
h=L/N;
x=-1:h:1;
D2 = gallery('tridiag',N+1,1,-2,1); % In sparse form.
%%

% initial condition
u0 = cos(pi*x)+0.2*cos(5*pi*x)+1;
% boundary conditions
g1 = @(t) 0*t+1;
g2 = @(t) 0*t+1;

A = D2(2:N,2:N)/h^2;
A(1,1)=-1/h^2;
A(end,end)=-1/h^2;
B = zeros(N-1,2);
B(1,1) = -1/h;
B(end,end) = 1/h;

t = 0:0.001:2;
%%
[T,U] = ode45(@(t,u) A*u+B*[g1(t);g2(t)], t, u0(2:end-1));
for k = 2:length(t)
    plot(x(2:end-1),U(k,:)','*-')
    ylim([-1 2])
    shg
    drawnow
end
[T,X]= meshgrid(t,x(2:end-1));
surf(T,X,U','edgecolor','none')
xlabel('$t$','fontsize',labelfontsize,...
            'interpreter','latex')
ylabel('$x$','fontsize',labelfontsize,...
            'interpreter','latex')
zlabel('$u(x,t)$','fontsize',labelfontsize,...
            'interpreter','latex')
set(gca,'fontsize',axisfontsize)
txt='Latex/FIGURES/P3_surf';
saveas(gcf,txt,'epsc')
\end{verbatim}