{\bf Exmp 5.15}. Consider the problem $(\partial_t - \partial_x^2)u=0$ on $(0,1) \times (0, \infty), u(0,t)=0=u(1,t), t \geq 0, u(x,0)=x(1-x)g(x), 0 \leq x \leq 1$ with some cont nonneg func $g: [0,1] \ra \R$. Show: $ \exists \alpha, \beta >0$ such that the cont sol $u: [0,L] \times [0,\infty) \ra \R$ ($\exists$ by T 5.16) satisfies $0 \leq u(x,t) \leq \beta x(1-x)e^{-\alpha t}, 0 \leq x \leq 1, t \geq 0$. We can choose $\alpha = 8$ and $\beta = \sup g$. {\it Proof}. Set $v(x,t)=\beta x (1-x)e^{-\alpha t}$ with $\alpha$ and $\beta$ tbd later.  Then $(\partial_t - \partial_x^2)v(x,t) = - \alpha \beta x(1-x)e^{-\alpha t} + \beta 2 e^{-\alpha t} = \beta e^{-\alpha t}(2-\alpha x(1-x))$. The local and global max of the func $x(1-x)$ is taken at $x = 1/2$ and equals $1/4$. So $(\partial_t - \partial_x^2)v(x,t) \geq \beta e^{-\alpha t}(2-\alpha/4)$. We choose $\alpha = 8$ and have $(\partial_t - \partial_x^2)v \geq 0$. In order to achieve $u(x,0)\leq v(x,0)$ we choose $\beta \geq g(x)$ for all $x \in [0,1]$. This is possible because $g$ is cont and thus bounded.  Define $ w = u - v$. Then $w$ is cont on $[0,1] \times [0, \infty), (\partial_t - \partial_x^2)w \leq 0$ on $[0,1] \times (0, \infty), w(x,t) = 0$ for $x = 0, x = 1, t \geq 0, w(x,0) \leq 0$ for all $x \in [0,1]$. So $w \leq 0$ by Theorem 5.11. The nonneg of $u$ follows by applying T 5.11 to $-u \qed$