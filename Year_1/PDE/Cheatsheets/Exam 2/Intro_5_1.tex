The sol, if $\exists$, can be written as a Fourier sine series (Lemma 4.19, Ex 4.3.5) $u(x,t)=\sum_{j=1}^{\infty}B_j(t)\sin(\lambda_jx), \lambda_j=j\pi/L=j\lambda_1$, with $B_j(t) = (2/L)\int_0^L u(y,t) \sin(\lambda_j y)dy$ (5.2), where, for fixed $t$, the series converges in the $L^2$-sense in $x$. If $u$ is a sol, it is suff smooth that we can diff under the int, $B_j'(t) = (2/L) \int_0^L \partial_tu(y,t)\sin(\lambda_jy)dy=(2/L)\int_0^L a\partial_y^2 u(y,t)\sin(\lambda_jy)dy$. Since the sines and $u$ satisfy zero boundary cond, we can int by parts twice and obtain the diff eq $B_j'(t)=-a \lambda_j^2B_j(t)$. The IC yields $B_j(0)=(2/L)\int_0^L f(y) \sin(\lambda_jy)dy$ (5.3). Solutions $B_j(t)=B_j(0)e^{-a\lambda_j^2t}$ (5.4). $\implies$ sol of (5.1) is uniquely determined. 
{\bf Existence} 