{\bf Exercise 4.2.1} (Riesz-Fisher Theorem). Let $\{v_m;m\in \N\}$ be an orthonormal set in a Hilbert space $H$ over $\K$ and $(\alpha_m)$ a sequence in $\K$. Show: The series $\sum_{m=1}^{\infty} \alpha_m v_m$ exists in $H$ iff $\sum_{m=1}^{\infty} |\alpha_m|^2 < \infty$. Further, if one and then both of these statements hold, $||\sum_{m=1}^{\infty} \alpha_m v_m||^2=\sum_{m=1}^{\infty}|\alpha_m|^2$. {\it Proof}. We define partial sums in $H$, $x_n=\sum_{m=1}^n \alpha_m v_m$, and in $\K$, $\beta_n = \sum_{m=1}^n |\alpha_m|^2$.  By the properties of the inner product and orthonormality, $||x_n - x_k||^2 = ||\sum_{m=k+1}^n \alpha_m v_m||^2=\langle\sum_{m=k+1}^n \alpha_m v_m, \sum_{j=k+1}^n \alpha_j v_j \rangle = \sum_{m=k+1}^n \sum_{j=k+1}^n \alpha_m \bar{\alpha}_j \langle  v_m,  v_j \rangle =\sum_{m=k+1}^n|\alpha_m|^2 = |\beta_n-\beta_k|$. This shows that $(x_n)$ is a Cauchy sequence in $H$ iff $(\beta_n)$ is a Cauchy sequence in $\R$. Assume that $\sum_{m=1}^{\infty}|\alpha_m|^2$ converges.  Then $(\beta_n)$ is a Cauchy sequence in $\R$ and $(x_n)$ is a Cauchy sequence in $H$.  Since $H$ is complete, $(x_n)$ converges, i.e., $\sum_{n=1}^{\infty} \alpha_n x_n$ converges. The other direction follows similarly.  Finally, by continuity of the norm and orthonormality, $||\sum_{m=1}^{\infty} \alpha_m v_m||^2=\lim_{n\ra \infty}||x_n||^2=\lim_{n\ra \infty}\sum_{m=1}^{\infty}|\alpha_m|^2=\sum_{m=1}^{\infty}|\alpha_m|^2 \qed$