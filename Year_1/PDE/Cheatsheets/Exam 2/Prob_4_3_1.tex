{\bf Exercise 4.3.1}. Let $B= \{\cos(jx);j\in \N\} \cup\{\sin(jx); j \in \N\} \cup \{1/\sqrt{2}\}$. Show that $B$ is an orthonormal basis of $L^2([-\pi,\pi],\R)$ with inner product $\langle f,g \rangle = (1/\pi) \int_{-\pi}^{\pi}fg$. Hint:  Use that $\{e^{ijx};j \in \Z\}$ is an orthonormal basis of $L^2([-\pi,\pi],\C)$ and express $\sin x$ and $\cos x$ in terms of $e^{ix}$ and $e^{-ix}$. {\it Proof}. Recall that $\cos(jx)=(1/2)(e^{ijx}+e^{-ijx}), \sin(jx)=(1/2i)(e^{ijx}-e^{-ijx})$. Since $e^{ijx}$ and $e^{ikx}$ are orthogonal to each other for $j \neq k$, so are $\cos jx$ and $\cos kx$, and $\cos jx$ and $\sin kx$, and $\sin jx$ and $\sin kx$ for $j \neq k$. $\sin jx$ and $\cos jx$ are orthogonal to each other because their product is odd about 0 and the integral yields 0. Further $(1/\pi)\int_{-\pi}^{\pi}\cos jx \cos jx dx = (1/\pi)\int_{-\pi}^{\pi}(1/4)(e^{ijx}+e^{-ijx})(e^{ijx}+e^{-ijx}) dx $. Notice that $\int_{-\pi}^{\pi}e^{ijx}e^{ijx}=2\pi\langle e^{ij\cdot},e^{-ij\cdot}\rangle_{\C}=0$ and $\int_{-\pi}^{\pi}e^{-ijx}e^{-ijx}=2\pi\langle e^{-ij\cdot},e^{ij\cdot}\rangle_{\C}=0$. Here $\langle \cdot, \cdot \rangle_{\C}$ denotes the inner product for $C([-\pi,\pi], \C)$. So $(1/\pi)\int_{-\pi}^{\pi}\cos(jx) \cos (jx) dx =1$. Similarly for the sines. In order to show that this is an orthonormal basis, we use Exercise 4.2.3. Let $f \in L^2([-\pi,\pi], \R)$ and $(1/\pi)\int_{-\pi}^{\pi}f(x)\sin(jx) dx =0, (1/\pi)\int_{-\pi}^{\pi}f(x)\cos(jx) dx =0, j\in \N, (1/\pi)\int_{-\pi}^{\pi}f(x)(1/\sqrt{2})=0$. By Euler's formula, for all $j \in \Z, (1/2\pi)\int_{-\pi}^{\pi}f(x)e^{ijx} dx =(1/2\pi)\int_{-\pi}^{\pi}f(x)\cos(jx) dx +i(1/2\pi)\int_{-\pi}^{\pi}f(x)\sin(jx) dx =0$. Since $\{e^{ijx}; j \in \Z\}$ is an orthonormal basis, $f=0$ by Exercise 4.2.3(a). Exercise 4.2.3 (b) implies that $B$ is an orthonormal basis for $L^2([-\pi,\pi],\R) \qed$