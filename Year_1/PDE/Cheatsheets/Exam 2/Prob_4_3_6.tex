{\bf Exercise 4.3.6}. Show that $\{v_j;j\in\Z_+\}$ with $v_j(x) = \cos(jx)$ and $v_0=\sqrt{1/2}$ is an orthonormal basis of $L^2([0,\pi],\R)$ with the inner product $\langle f,g \rangle = \int_0^{\pi}f(x)g(x)dx$. Conclude that, for $f \in L^2([0,\pi],\R), \int_0^{\pi}|f(x)-\sum_{j=0}^m B_j\cos(jx)dx|^2dx \ra 0, m \ra \infty, B_j=(2/\pi)\int_0^{\pi}f(x)\cos(jx)dx, j \in \N, B_0=(\sqrt{2}/\pi)\int_0^{\pi}f(x)dx$. {\it Proof}. By Exercise 4.3.1, $B=\{\cos(jx);j\in\N\}\cup \{\sin(jx);j\in\N\}\cup \{1/\sqrt{2}\}$ is an orthonormal basis of $L^2([-\pi,\pi],\R)$, with inner product $\langle f,g \rangle = (1/\pi)\int_{-\pi}^{\pi}fg$. In particular $\tilde{B} = \{v_j;j\in\N\}$ is an orthonormal subset of $L^2([-\pi,\pi],\R)$. So, for $j \neq k, 0 = \int_{-\pi}^{\pi}v_j(x) v_k(x) dx=2\int_0^{\pi}v_j(x) v_k(x)dx$, because $v_j v_k$ is an even function. By the same token, for $j\in \N, 1 = (1/\pi)\int_{-\pi}^{\pi}v_j(x)^2 dx= (2/\pi)\int_0^{\pi}v_j(x)^2 dx$. For $j = 0$ this property easily is directly verified. So $\tilde{B}$ is an orthonormal subset of $L^2([0,\pi], \R)$. To show that it is an orthonormal basis, we use Exercise 4.2.3:  We let $f=L^2([0,\pi], \R)$ with $\int_0^{\pi}f(x)v_j(x)dx=0$ for all $j \in \Z_+$ and show that $f=0$. Extend $f$ to an even function on $[-\pi,\pi]$ by setting $f(-x)=f(x)$ for $x \in (0,\pi]$. Then, for all $j \in \Z, \int_{-\pi}^{\pi}f(x)\sin(jx)dx=0$, because $f(x)\sin(jx)$ is an odd function of $x$. For all $j \in \Z_+, \int_{-\pi}^{\pi}f(x)v_j(x)dx=2\int_0^{\pi}f(x)v_j(x)dx=0$, because $f(x)v_j(x)$ is an even function of $x$. Since $B$ is an orthonormal basis of $L^2([-\pi,\pi],\R), f=0$ by Exercise 4.2.3 (a). So $\tilde{B}$ is an orthonormal basis by Exercise 4.2.3 (b).  $\qed$