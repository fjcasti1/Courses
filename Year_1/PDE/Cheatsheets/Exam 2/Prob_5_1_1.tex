{\bf Exercise 5.1.1} Let $L, a >0$. Consider the problem (PDE) $(\partial_t - a \partial_x^2)u = 0, 0 \leq x \leq L, t > 0$, (IC) $u(x,0)=f(x), 0 \leq x \leq L$, (BC) $\partial_xu(0,t)=0=\partial_xu(L,t), t > 0$ (5.13). This equation is a model for heat diffusion in a (finite) rod of length $L$. The no flux boundary condition means that both ends of the rod are insulated.  (a) Use Fourier cosine series to solve (5.13), at least as far as (PDE) and (BC) are concerned, under an appropriate condition for $f$. (b) Explore two assumptions for $f$ under which (IC) is satisfied in meaningful though not necessarily literal ways. (c) Show that $\int_0^L u(x,t)dx = \int_0^L u(x,0)dx$ for all $t \geq 0$.  Hint: These integrals are related to the Fourier cosine coefficient of index zero.  (d) Show that $u(x,t) \ra (1/L)\int_0^L f(x)dx$ as $t \ra \infty$, uniformly in $x \in [0,L]$. {\it Solution}. (a) Assume that $f: [0,L] \ra \R$ is integrable and $\int_0^L|f(y)|dy < \infty$. Define $u_m(x,t) = A_m \cos(\lambda_mx)e^{-a \lambda_m^2 t}, m \in \N \cup \{0\} = \Z_+$, with $\lambda_m = m\pi/L, A_m=2/L\int_0^L f(y) \cos(\lambda_my)dy, m \in \N, A_0=1/L\int_0^Lf(y)dy$. Then $u_m$ is infinitely often differentiable, $\partial_t u_m(x,t) = -a \lambda_m^2 u_m(x,t) = a \partial_x^2 u_m(x,t), x \in [0,L], t \geq 0$ and $\partial_x u_m(x,t) = -A_m \sin(\lambda_mx) e^{-\lambda_m^2 t} = 0$ for $x = 0, L$ and $ t \geq 0$. For all $k, \ell \in \Z_+$, $\partial_x^k \partial_t^{\ell} u_m(x,t) = A_m(d^k/dx^k)\cos(\lambda_mx)(d^{\ell}/dt^{\ell})e^{-a \lambda_m t}$. Notice that $|A_m| \leq 2/L \int_0^L |f(x)|dx =: A$. Let $t \geq \epsilon >0$. Then, by the form of $\lambda_m, |\partial_x^k \partial_t^{\ell} u_m(x,t)| \leq |A_m|\lambda_m^{k+2\ell}a^{\ell}e^{-\lambda_m^2 \epsilon} \leq A c m^{k+2 \ell} \eta^m$ for $\eta = e^{-a(\pi/L)^2 \epsilon} \in (0,1)$ and some $c >0$ that only depends on $k$ and $\ell$. It follows from the ratio test that the series $\sum_{m=1}^{\infty} A c m^{k+2 \ell} \eta^{2m}$ converges. By Theorem 5.1, each series $\sum_{m=0}^{\infty}\partial_x^k \partial_t^{\ell} u_m(x,t)$ (in particular the series for $u$) converges uniformly on $[0,L] \times [ \epsilon, \infty)$. Applying Theorem 5.3 repeatedly (or use induction) implies that $u = \sum_{m=1}^{\infty} u_m$ has partial derivatives of all order and can be differentiated term by term.  This holds on $[0,L] \times [\epsilon, \infty)$ for every $\epsilon >0$. So $u$ is infinitely often differentiable on $[0,L] \times (0, \infty)$ and can be differentiated term by term. Since each $u_m$ satisfies (PDE) and (BC), so does $u$ on $[0,L] \times (0, \infty)$. (b) First condition:  Let $f: [0,L] \ra \R$ be Lipschitz continuous. Then $u$ is continuous on $[0,L] \times [0, \infty)$ and $u(x,0) = f(x)$ for all $x \in [0,L]$. In particular, $u(x,t) \ra f(x)$ as $t \ra 0$, uniformly in $x \in [0,L]$. Notice that $|u_m(x,t)| \leq |A_m|, x \in [0,L], t \geq 0$ with $A_m$ being the appropriate Fourier cosine coefficient. Since $f$ is Lipschitz continuous, $\sum_{m=0}^{\infty} |A_m| < \infty$ by Exercise 4.3.3 and even extension of $f$. By Thorem 5.1, $u = \sum_{m=0}^{\infty}u_m$ converges uniformly and is continuous on $[0,L] \times [0, \infty)$. Further $u(x,0)= \sum_{m=0}^{\infty} A_m \cos(\lambda_mx)=f(x)$ by Exercise 4.3.4. The last statement follows as in Theorem 5.6.  Second condition:  Assume that $f: [0,L] \ra \R$ is integrable and $\int_0^L|f(x)|^2 dx < \infty$. Then the series $u$ in (5.2) with (5.4) and (5.3) satisfies $|| u(\cdot, t) - f||^2 = 2/L \int_0^L |u(x,t)-f(x)|^2 dx \ra 0, t \ra 0$. Let $\langle \phi, \psi \rangle = 2/L \int_0^L \phi(x) \psi(x) dx$ be the inner product of choice on $L^2([0,L], \R)$, the space of square integrable functions.  Then $\{v_m;m\in \Z_+\}$ with $v_m=\cos(\lambda_mx)$ for $m \in \N$ and $v_0 = 2^{-1/2}$ is an orthonormal basis, $U(t) := u(\cdot, t) =  \sum_{m=0}^{\infty} \langle f, v_m \rangle e^{-a \lambda_m^2 t} v_m$, with convergence holding in mean-square norm for $t \geq 0$ and uniformly on $[0,L]$ for $t>0$. Notice that $\langle U(t), v_m \rangle = \langle f, v_m \rangle e^{-a \lambda_m^2 t}$ is uniformly continuous on $\R_+$, $|\langle U(t), v_m \rangle| \leq \langle f, v_m \rangle$ and, by Parseval's relation, $\sum_{n=0}^{\infty} |\langle f, v_m \rangle|^2 = ||f||^2 < \infty$. By Theorem 4.11, the function $U: \R_+ \ra X, X = L^2([0,L]), U(t)=u(\cdot)$, is continuous on $\R_+ = [0,\infty)$ and $U(0)=f$. (c) By orthonormality and $\lambda_0 = 0$, for all $t \geq 0, \langle u(\cdot, t), v_0 \rangle = \langle f, v_0 \rangle = (2 \cdot 2^{-1/2})/L\int_0^L f(x)dx$. This implies the assertion. (d) Notice that $u(x,t)=-1/L\int_0^Lf(x)dx=u(x,t)-A_0 = \sum_{m=1}^{\infty}A_m \cos(\lambda_mx)e^{-a \lambda_m^2 t}$. So $|u(x,t)-1/L\int_0^Lf(x)dx\leq \sum_{m=1}^{\infty} |A_m |e^{-a \lambda_m^2 t} \leq A \sum_{m=1}^{\infty} e^{-a \lambda_m^2 t}$.  The proof now continues as the proof of Theorem 5.9 $\qed$