{\bf Exercise 5.1.2}. Let $f: [0,L] \ra \R$ be integrable and $\int_0^L |f(x)|dx < \infty$, i.e. $f \in L^1([0,L],\R)$. Consider the heat equation with zero boundary condition and initial data $f$. Show:  there exists a function $u: [0,L] \times (0,\infty) \ra \R$ that solves (PDE) and (BC) and satisfies the initial condition in the following weak sense:  If $\phi: [0,L] \ra \R$ is Lipshitz continuous and $\phi(0) = 0 = \phi(L)$, then $\int_0^L \phi(x)u(x,t)dx \ra \int_0^L\phi(x)f(x)dx, t \ra 0$. Hint: Notice (and prove) that  $\int_0^L \phi(x)u(x,t)dx =\int_0^L f(x)v(x,t)dx$ where $v$ is the solution of the heat equation with initial data $\phi$. {\it Proof}. Let $\langle \phi, \psi \rangle = 2/L \int_0^L \phi(x) \psi(x) dx$ be the inner product of choice on $L^2([0,L], \R)$, the space of square integrable functions.  Then $\{v_j; j \in \N\}$ with $v_j(x)=\sin(\lambda_jx)$ is an orthonormal basis. For $t > 0, u(\cdot, t)=\sum_{m=1}^{\infty} \langle f, v_m \rangle v_m e^{-a \lambda_m^2 t}$ (5.14) converges uniformly on $[0,L]$ (equivalently in the supremum norm) (Theorem 5.4).  Here, abusing the notation ($f$ not necessarily in $L^2$), we have written $\langle f, v_m \rangle$ for $1/(2L) \int_0^L f(x) v_m(x)dx$ which is defined because $v_m$ is continuous. Then, for $t >0, \langle \phi, u(\cdot, t)\rangle = \sum_{m=1}^{\infty}\langle f, v_m \rangle \langle \phi, v_m \rangle e^{-a \lambda_m^2 t}$. By Theorem 5.6, $w(\cdot, t) =  \sum_{m=1}^{\infty}\langle \phi, v_m \rangle e^{-a \lambda_m^2 t}$ converges uniformly on $[0,L]$, uniformly in $t \in \R_+$, and $w$ is continuous on $[0,L] \times \R_+$. So $1/(2L) \int_0^L f(x)w(x,t)dx=\sum_{m=1}^{\infty}\langle \phi, v_m \rangle \langle f, v_m \rangle e^{-a \lambda_m^2 t} = \langle \phi, u(\cdot, t)\rangle $. Since $w(\cdot, t) \ra \phi$ as $t \ra 0$, uniformly on $[0,L], 1/(2L)\int_0^L \phi(x) u(x,t) dx = \langle \phi, u(\cdot, t)\rangle \ra 1/(2L)\int_0^L \phi(x) f(x) dx, t \ra 0$. This implies the assertion. $\qed$