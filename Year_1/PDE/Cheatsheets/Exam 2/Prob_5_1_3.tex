{\bf Exercise 5.1.3}. Let $L, a > 0$. Consider the problem (PDE) $(\partial_t - a \partial_x^2)u = 0, -\pi \leq x \leq \pi, t > 0$, (IC) $u(x,0)=f(x), -\pi \leq x \leq \pi$, (BC) $u(-\pi,t)=u(\pi,t), \partial_x u(-\pi, t) =\partial_x u(\pi, t), t > 0$ (5.15).  (a) Use complex Fourier series to solve (5.15), at least as far as (PDE) and (BC) are concerned, under an appropriate condition for $f$. (b) Explore two assumptions for $f$ under which (IC) is satisfied in meaningful though not necessarily literal ways. Cf. Theorem 5.6 and 5.7. (c) Show that $u$ is real-valued if $f$ is real-valued. (d) Show that $\int_{-\pi}^{\pi} u(x,t)dx = \int_{-\pi}^{\pi} u(x,0)dx$ for all $t \geq 0$.  Hint: These integrals are related to the Fourier cosine coefficient of index zero.  (e) Show that $u(x,t) \ra (1/2 \pi)\int_{-\pi}^{\pi} f(x)dx$ as $t \ra \infty$, uniformly in $x \in [-\pi,\pi]$ provided that $\int_{-\pi}^{\pi}|f(x)|dx < \infty$. {\it Proof}. (a) We try to find $u$ as a complex Fourier series $u(x,t)=\sum_{m\in \Z} C_m(t)e^{imx}, C_m(t) = 1/(2 \pi)\int_{-\pi}^{\pi} u(x,t)e^{-imx}dx$ (5.16). If $\partial_t u(x,t)$ exists and is continuous on $[-\pi, \pi] \times (0, \infty)$, we can interchange time differentiation and integration, $C_m'(t) = 1/(2\pi) \int_{-\pi}^{\pi} \partial_t u(x,t) e^{-imx} dx = 1/(2\pi) \int_{-\pi}^{\pi} a\partial_x^2 u(x,t) e^{-imx} dx$. We integrate by parts twice; the boundary terms cancel because of the periodic boundary conditions, $C_m'(t) = -a m^2 C_m$. Further $C_m(0) = 1/(2 \pi)\int_{-\pi}^{\pi} f(x) e^{-imx}dx = \hat{f}_m$. We solve the ordinary differential equation, $C_m(t)=\hat{f}_m e^{imx} e^{-am^2t}$. That the series in (5.16) converges uniformly on $[-\pi, \pi] \times [ \epsilon, \infty)$ for any $\epsilon >0$ and solves (PDE) in (5.15) is shown analogously to the proof of Theorem 5.4. Define $u_m(x,t)=\hat{f}_m e^{imx} e^{-am^2t}$. Then $u_m$ is infinitely often differentiable and satisfies the heat equation, $\partial_t u_m(x,t)=-am^2u_m(x,t) =  a \partial_x^2u_m(x,t)$. For all $k, \ell \in \Z_+$ and $m \in \Z, \partial_x^k \partial_t^{\ell} u_m(x,t)=\hat{f}_m(im)^k(-am^2)^{\ell} e^{imx} e^{-am^2 t}$ and $|\partial_x^k \partial_t^{\ell} u_m(x,t)|=|\hat{f}_m| |i|^k |m|^k a^{\ell}m^{2 \ell} |e^{imx}| e^{-am^2 t} \leq |\hat{f}_m|  |m|^{k+2\ell} a^{\ell}  e^{-am^2 t}$. For $m \in \Z, |\hat{f}_m|\leq 1/(2 \pi) \int_{-\pi}^{\pi}|f(x)| |e^{-imx}|dx \leq 1/(2 \pi) \int_{-\pi}^{\pi}|f(x)| dx =: A$. Let $\epsilon > 0$. For $m \in \Z$ and $k, \ell \in \Z_+$ and $t \in [\epsilon, \infty), |\partial_x^k \partial_t^{\ell} u_m(x,t)| \leq A |m|^{k+2 \ell}a^{\ell} e^{-am^2 \epsilon}$. By the ration test, $\sum_{m=1}^{\infty}A m^{k+2\ell}a^{\ell}e^{-am^2 \epsilon}$ converges in $\R$. By the Weierstra$\ss$ test, for each $\ell, k \in \Z_+$, the following series converge uniformly for $x \in [-\pi, \pi], t \in [\epsilon, \infty), \sum_{m=1}^{\infty}\partial_x^k \partial_t^{\ell}u_m(x,t), \sum_{m=1}^{\infty}\partial_x^k \partial_t^{\ell}u_{-m}(x,t), \sum_{m\in \Z} \partial_x^k \partial_t^{\ell}u_m(x,t)$ with the third being the sum of the first and second.  By Theorem 5.3, $u$ is infinitely often partially differentiable on $[-\pi, \pi] \times (0, \infty)$ and $\partial_x^k \partial_t^{\ell}u(x,t) =  \sum_{m\in \Z}\partial_x^k \partial_t^{\ell}u_m(x,t)$. In particular, $u$ satisfies the heat equation on $[-\pi, \pi] \times (0, \infty)$.  Since $e^{imx}$ is $2\pi$-periodic for all $m \in \Z, u(\pi,t)=u(-\pi,t)$ for all $t >0$ follows from the uniform convergence of the series in (5.16) converges uniformly on $[-\pi, \pi] \times [\epsilon, \infty)$ for any $\epsilon > 0$. Further, $\partial_x u(x,t) = \sum_{m\in \Z}C_m(t)mie^{imx}$ with convergence being uniform for $x \in [-\pi, \pi], t \in  [\epsilon, \infty)$ for any $\epsilon > 0$. By the same token as before $\partial_x u(\pi,t)=\partial_x u(-\pi,t), t>0$. (b) This is analogous to Theorem 5.6 and 5.7. We first assume that $f$ is Lipschitz continuous on $[-\pi, \pi]$ and $f(\pi) = f( -\pi)$. By Lemma 4.16, $f$ can be extended to a Lipschitz continuous $2 \pi$-periodic function on $\R$. by Proposition 4.14, the following series converges, $\sum_{m \in \Z}|\hat{f}_m| = |\hat{f}_0|+\sum_{m=1}^{\infty}(|\hat{f}_m|+|\hat{f}_{-m}|)$. For all $m \in \Z, |C_m(t)|\leq |\hat{f}_m| |e^{-am^2t}|\leq |\hat{f}_m|$. Thus $|C_m(t)e^{imx}|= |C_m(t)| |e^{imx}|=|C_m(t)|\leq |\hat{f}_m|, m \in \Z$. For all $m\in \N, |C_m(t)e^{imx} + C_{-m}(t)e^{-imx}|\leq |C_m(t)e^{imx}| + |C_{-m}(t)e^{-imx}| \leq |\hat{f}_m| + |\hat{f}_{-m}|$. By the Weierstra$\ss$ test. $\sum_{m\in \Z}C_m(t) e^{imx}=C_0(t) + \sum_{m=1}^{\infty}(C_m(t)e^{imx} + C_{-m}(t)e^{-imx})$ converges uniformly on $[-\pi, \pi] \times [0, \infty)$. This implies that $u$ is continuous on $[-\pi, \pi] \times [0, \infty)$. We now assume that $f$ is integrable and $\int_{-\pi}^{\pi}|f(x)|^2 dx < \infty$. Let $\langle \phi, \psi \rangle = 1/(2\pi)\int_{-\pi}^{\pi} \phi(x) \overline{\psi(x)} dx$ be the inner product of choice on $L^2([-\pi, \pi], \C)$, the space of square integrable functions. Then $\{v_j; j\in \Z\}$ with $v_j(x)=e^{ijx}$ is an orthonormal basis.  We intend to apply Theorem 4.11. Notice that $g: \N \ra \Z$ with $g(2n) = n$ and $g(2n-1)=-n, n\in \N$, is a bijection.  By the considerations at the beginning of this section, $\langle u(\cdot, t), v_m \rangle = \langle f, v_m \rangle e^{-a\lambda_m^2 t}$ (5.17), which are uniformly continous functions of $t \in \R_+$. Further $|\langle u(\cdot, t), v_m \rangle| \leq  |\langle f, v_m \rangle|$  and, by Parseval's relation (Theorem 4.10), $\sum_{m \in \Z}|\langle f, v_m \rangle|^2 = ||f||^2$. Theorem 4.11 implies that $U: \R_+ \ra L^2([-\pi, \pi], \C)$ with $U(t) = u(\cdot, t)$ is continuous. (c) This is analogous to Theorem 4.18. For fixed $t > 0, u(x,t)$ is the limit (uniformly in $x \in [-\pi, \pi]$) of Fourier sums $\sum_{m=-n}^n \hat{f}_m e^{-am^2t} e^{imx} = \hat{f}_0 + \sum_{m=1}^n (\hat{f}_me^{imx} +\hat{f}_{-m}e^{-imx})e^{-am^2t}$. The same proof as for Theorem 4.18 shows that $\hat{f}_me^{imx} +\hat{f}_{-m}e^{-imx}$ is real if $f$ is real-valued. (d) Let $\langle \phi, \psi \rangle = 1/(2\pi)\int_{-\pi}^{\pi} \phi(x) \overline{\psi(x)} dx$ be the inner product on the Hilbert space $L^2([-\pi, \pi], \C)$. Let $\{v_m; m\in \Z\}$ be the orthonormal basis with $v_m(x)=e^{imx}$. Notice that $v_0(x) = 1$. By orthonormality and part (a), $\langle u(\cdot, t), v_0 \rangle = C_0(t) = \langle f, v_0\rangle = \langle u(\cdot, 0), v_0 \rangle$. This implies the assertion. (e) For all $t \geq 0, |u(x,t)-1/(2\pi)\int_{-\pi}^{\pi}f(x)dx| = | \sum_{0 \neq m\in \Z} \hat{f}_m e^{-am^2 t}e^{imx}| \leq  \sum_{0 \neq m\in \Z} |\hat{f}_m| e^{-am^2 t} \leq A \sum_{m=1}^{\infty}  e^{-am^2 t}$ with $A = 1/\pi \int _{-\pi}^{\pi} | f(x)| dx$. The estimate can be continued by $\leq A \sum_{m=1}^{\infty}  e^{-am t}= \sum_{m=1}^{\infty}  A (e^{-a t})^m = A (e^{-at})/(1-e^{-at}) \ra^{t \ra \infty} 0 \qed$