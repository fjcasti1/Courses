{\bf R 5.5}. The derivatives of $u$ can be found by diff term by term $\partial_x^k\partial_t^{\ell} u(x,t)=\sum_{j=1}^{\infty}B_j(0)(d^k/dx^k)\sin(\lambda_jx)(d^{\ell}/dt^{\ell})e^{-a\lambda_j^2t}$ with the series converging unif on every set $[0,L]\times [\epsilon, \infty), \epsilon > 0$. {\it Proof}. For $m \in \N$, set $u_m(x,t)=B_m(0)\sin(\lambda_mx)e^{-a \lambda_m^2t}$ (5.5). Then $u_m$ is infinitely often diff and $\partial_x^k\partial_t^{\ell} u_m(x,t)=B_m(0)(d^k/dx^k)\sin(\lambda_mx)(d^{\ell}/dt^{\ell})e^{-a\lambda_m^2t}$. Notice that  $|B_m(0)| \leq 2/L \int_0^L |f(x)|dx =: B_0 < \infty$. Let $t \geq \epsilon >0$. Then, by the form of $\lambda_m, |\partial_x^k\partial_t^{\ell} u_m(x,t)| \leq |B_m(0)| \lambda_m^{k+2 \ell} a^{\ell} e^{-a\lambda_m^2\epsilon} \leq B_0 c m^{k+2\ell}\eta^{(m^2)}$ for $\eta = e^{-a\lambda_1^2\epsilon} \in (0,1)$ and some $c >0$ that only depends on $k$ and $\ell$. It follows from the ratio test that the series $\sum_{m=1}^{\infty}B_0cm^{k+2\ell}\eta^{(m^2)}$ converges.  By T 5.1, each series $\sum_{m=1}^{\infty}\partial_x^k\partial_t^{\ell} u_m(x,t)$ (in particular the series for u) converges unif on $[0,L] \times [\epsilon, \infty)$. Let $x \in [0,L] = I_1$ and $t >0$. Choose $I_2=(t_1,t_2)$ with $0<t_1<t<t_2<\infty$. Applying T 5.3 repeatedly (or using induction) $\implies u = \sum_{m=1}^{\infty}u_m$ has partial derivatives of all order and can be diff term by term on $I_1 \times I_2$. So $u$ is infinitely often diff on $[0,L]\times (t_1,t_2)$ and can be differentiated term by term.  In particular, all partial derivatives of $u$ exist at $(x,t)$ and can be found by term by term differentiation.  Since each $u_m$ satisfies (PDE) and (BC), so does $u$ on $[0,L]\times (0,\infty) \qed$  We explore in what sense the initial condition (IC) is satisfied.