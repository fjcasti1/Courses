{\bf T 4.17}. Let $f: [-L,L] \ra \C$ be Lipschitz cont, $f(-L)=f(L)$. Then $f$ is the uniform limit of its Fourier series, $f(x)=\sum_{j\in \Z}\hat{f}_je^{i\lambda_jx}, \hat{f}_j=(1/2L)\int_{-L}^L f(y)e^{-i\lambda_jy}dy, \lambda_j=j\pi/L$. 
%{\it Proof}. We extend $f$ an an $2L$-periodic way.  The extended $f$ is Lipschitz continuous with same Lipschitz constant. We define $g(x)=f(xL/\pi), x \in \R$.  Then $g$ is Lipschitz continuous and $2\pi$-periodic. Let $I$ be the interval $[-\pi,\pi]$ and $X$ the vector space $B(I)$ of bounded complex-valued functions on $I$. We define a norm on $B(I)$ by $||g||_{\infty}=\sup_{x\in I}|g(x)|$. With this norm, $B(I)$ is a Banach space. For $j \in \Z$, define $g_j(x)=\hat{g}_je^{ijx}, x \in I$. Then $g_j\in B(I)$ and $||g_j||_{\infty} \leq |\hat{g}_j|$. By Propostion 4.14 and Lemma 4.16, $\sum_{j\in \Z}||g_j||_{\infty}$ converges in $\R$. By the Weierstra$\ss$ majorant test, the Fourier series of $g, \sum_{j\in \Z} g_j$ converges in $B(I)$, i.e. $\sum_{j\in \Z}\hat{g}_je^{ijx}$ converges in $\C$, unformly for $x \in I$. The Fourier series of $g$ converges to $g$ in $L^2[-\pi,\pi]$ by Theorem 4.10. This implies that $g$ and its Fourier series are equal almost everywhere.  Since both are continuous, they are equal.  In combination, $g$ is the uniform limit of its Fourier series, and so is $f$. The formula for the Fourier coefficients of $f$ follows by substitution. $\qed$