{\bf T 4.23}. Any classical sol $u$ of (4.5) is a gen sol of (4.5). A gen sol is uniquely determined by finding its Fourier sine series, $u(x,t)=\sum_{j=1}^{\infty}B_j(t)\sin(\lambda_jx), \lambda_j =j\pi/L, B_j(t)= (2/L)\int_0^L u(x,t) \sin(\lambda_jx)dx, j \in \N$. (4.7) For each $t \in \R$ the convergence of $\sum_{j=1}^{\infty}B_j(t)\sin(\lambda_j \cdot)$ holds in $L^2[0,L]$. Let $j \in \N$ and choose $\phi(x)=\sin(\lambda_jx). \; \phi$ is a test func and so the following derivatives exist and satisfy $(d^2/dt^2)\int_0^L u(x,t)\sin(\lambda_jx)dx=c^2\int_0^L u(x,t)(d^2/dx^2) \sin(\lambda_jx)dx=-c^2\lambda_j^2\int_0^L u(x,t) \sin(\lambda_jx)dx$ and $(d/dt_{[t=0]})\int_0^Lu(x,t) \sin(\lambda_jx)dx=\int_0^Lg(x)\sin(\lambda_jx)dx, \int_0^Lu(x,0) \sin(\lambda_jx)dx=\int_0^Lf(x)\sin(\lambda_jx)dx$. $\implies$ Fourier sine coef of a gen sol, $B_j(t)$, satisfy the ODEs $B_j''+c^2 \lambda_j^2B_j=0$ and the ICs $B_j(0)=(2/L)\int_0^L f(y)\sin(\lambda_jy)dy, B_j'(0)=(2/L)\int_0^L g(y)\sin(\lambda_jy)dy$. Gen sol, $a_j\cos(c \lambda_jt)+b_j\sin(c\lambda_jt)=B_j(t)$. From IC obtain  $a_j = B_j(0) = (2/L) \int_0^L  f(y)\sin(\lambda_jy)dy$ (4.8) and $c\lambda_jb_j=B_j'(0)=(2/L)\int_0^L g(y) \sin(\lambda_jy)dy$. (4.9) Subst and get Fourier sine series for any gen sol, $u(x,t)=\sum_{j=1}^{\infty}[a_j\cos(c\lambda_jt)+b_j \sin(c \lambda_jt)] \sin(\lambda_jx)$. (4.10) The d'Alembert form provides a gen sol if $f$ and $g$ are cont on $[0,L]$ and $f(0)=0=f(L)$ and $g(0)=0=g(L)$ and $f$ and $g$ are extended in an odd and $2L$-periodic way.  Set $v(x,t)=(1/2)(f(x+ct)+f(x-ct))$. (4.11) After a change of variables, $\int_0^L\phi(x)v(x,t)dx=(1/2)\int_{ct}^{L+ct}\phi(y-ct)f(y)dy+(1/2)\int_{-ct}^{L-ct}\phi(y+ct)f(y)dy$. Since $\phi$ is twice cont diff and $f$ is cont, we can use the Leibnitz rule, $\implies$ expression is diff and $(d/dt)\int_0^L\phi(x)v(x,t)dx=(c/2)[\phi(L)f(L+ct)-\phi(0)f(ct)]-(c/2)\int_{ct}^{L+ct}\phi'(y-ct)f(y)dy-(c/2)[\phi(L)f(L-ct)-\phi(0)f(-ct)]+(c/2)\int_{-ct}^{L-ct}\phi'(y+ct)f(y)dy$.  Since $\phi(0)=0=\phi(L), (d/dt)\int_0^L\phi(x)v(x,t)dx=-(c/2)\int_{ct}^{L+ct}\phi'(y-ct)f(y)dy+(c/2)\int_{-ct}^{L-ct}\phi'(y+ct)f(y)dy$ (4.12). This expression is 0 at $t = 0$. Use Leibnitz rule again, $(d^2/dt^2)\int_0^L\phi(x)v(x,t)dx=-(c^2/2)[\phi'(L) f(L+ct)-\phi'(0)f(ct)] + (c^2/2) \int_{ct}^{L+ct}\phi''(y-ct)f(y)dy-(c^2/2)[\phi'(L)f(L-ct)-\phi'(0)f(-ct)]+(c^2/2)\int_{-ct}^{L-ct} \phi''(y+ct)f(y)dy$.  Since $f$ is odd about 0 and $L$, the boundary terms cancel each other and, after reversing the subst $(d^2/dt^2)\int_0^L\phi(x)v(x,t)dx=\int_0^L c^2\phi''(x)v(x,t)dx$. (4.13)  Set $w(x,t) = 1/(2c) \int_{x-ct}^{x+ct} g(y)dy$. (4.14) Since $g$ is cont, $w$ is diff wrt $t$ and $x$ and $\partial_t w(x,t)=(1/2)[g(x+ct)+g(x-ct)], \partial_x w(x,t)=(1/2c)[g(x+ct)-g(x-ct)]$. (4.15)  At $t =0$, the first expression is $g(x)$. Since $\partial_t w$ is cont, we can diff under the int and obtain $(d/dt)\int_0^L\phi(x)w(x,t)dx=\int_0^L\phi(x)(1/2)[g(x+ct)+g(x-ct)]dx$. The same consideration as before with $g$ replacing $f$ (see (4.12)) yields $(d^2/dt^2)\int_0^L\phi(x)w(x,t)dx=(-c/2) \int_{ct}^{L+ct}\phi'(y-ct)g(y)dy+(c/2)\int_{-ct}^{L-ct}\phi'(y+ct)g(y)dy$. After refersing the subst, $(d^2/dt^2) \int_0^L \phi(x) w(x,t)dx = - \int_0^L \phi'(x) (c/2)[g(x+ct)-g(x-ct)]dx$.  Observe from (4.15) that $(d^2/dt^2) \int_0^L \phi(x) w(x,t)dx = - \int_0^L \phi'(x) c^2 \partial_x w(x,t)dx$. We int by parts, recall $w(0,t)=0=w(L,t)$, and obtain $(d^2/dt^2) \int_0^L\phi(x)w(x,t)dx=\int_0^L\phi''(x)c^2w(x,t)dx$. Since $u(x,t)=v(x,t)+w(x,t)$, so $\int_0^L\phi(x)u(x,t)dx$ is twice diff and $(d^2/dt^2) \int_0^L \phi(x) u(x,t)dx = \int_0^Lc^2\phi''(x)u(x,t)dx$ and $(d/dt)\int_0^L\phi(x)u(x,t)dx=\int_0^L\phi(x)g(x)dx, t=0$. 