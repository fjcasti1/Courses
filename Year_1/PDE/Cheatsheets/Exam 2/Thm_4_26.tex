{\bf 4.26 Theorem}. Any classical solution $u$ of (4.17) is a generalized solution of (4.17).  The point of a generalized solution is that a function can solve a differential equation without being differentiable. (b) The uniqueness of a generalized solution is shown by calculating its Fourier cosine series. Notice that $v_j(x)=\cos(jx, j \in \Z_+$, are test functions. This works analogously to the zero boundary problem. Because of the different boundary conditions, we use the Fourier cosine representation $u(x,t)= \sum_{j=0}^{\infty}A_j(t)\cos(\lambda_jx), \lambda_j=j\pi/L, A_j(t)=(2/L)\int_0^Lu(x,t)\cos(\lambda_jx)dx, j \in \N, A_0(t)=(1/L)\int_0^Lu(x,t)dx$. (4.18) Let $j \in \Z_+$ and choose $\phi(x)=\cos(\lambda_jx). \; \phi$ is a test function and so the following derivatives exist and satisfy $(d^2/dx^2)\int_0^L u(x,t)\cos(\lambda_jx)dx=c^2\int_0^Lu(x,t)(d^2/dx^2)\cos(\lambda_jx)dx=-c^2\lambda_j^2\int_0^Lu(x,t)\cos(\lambda_jx)dx$ and $(d/dt_{[t=0]})\int_0^L u(x,t)\cos(\lambda_jx)dx=\int_0^Lg(x)\cos(\lambda_jx)dx, \int_0^L u(x,0)\cos(\lambda_jx)dx=\int_0^L f(x)\cos(\lambda_jx)dx$.  This implies that the Fourier cosine coefficients of a generalized solution, $A_j(t)$, for $j \in \N$, satisfy the ODEs $A_j''+c^2\lambda_j^2A_j=0$ and the initial contitions $A_j(0)=(2/L)\int_0^Lf(y)\cos(\lambda_jy)dy, A_j'(0)=(2/L)\int_0^L g(y)\cos(\lambda_jy)dy$. The ODEs have the general solutions, $a_j\cos(c\lambda_jt)+b_j\sin(c\lambda_jt)=A_j(t)$. From the initial conditions we obtain that $a_j=A_j(0)=(2/L)\int_0^L f(y) \cos(\lambda_jy)dy$ (4.19) and $c\lambda_jb_j=A_j'(0)=(2/L)\int_0^L g(y) \cos(\lambda_jy)dy$. (4.20) For $j=0, A_j''=0$, and the initial conditions $A_j(0)=(1/L)\int_0^L f(y) \cos(\lambda_jy)dy=a_0, A_j'(0)=(1/L)\int_0^L g(y) \cos(\lambda_jy)dy = b_0$. We integrate twice, $A_j(t)=tb_0+a_0$. We substitute and obtain the Fourier cosine series for any generalized solution, $u(x,t)=\sum_{j=1}^{\infty}[a_j\cos(c\lambda_jt)+b_j\sin(c\lambda_jt)]\cos(\lambda_jx)+b_0t+a_0$. (4.21) (c) Finally we show that the d'Alembert formula provides a generalized solution if $f$ and $g$ are continuous on $[0,L]$ and $f$ and $g$ are extended in an even and $2L$-periodic way.  The extended $f$ and $g$ are continuous. Set $v(x,t)=(1/2)(f(x+ct)+f(x-ct))$. (4.22) After a change of variables, $\int_0^L \phi(x)v(x,t)dx=(1/2)\int_{ct}^{L+ct}\phi(y-ct)f(y)dy+(1/2)\int_{-ct}^{L-ct}\phi(y+ct)f(y)dy$. Since $\phi$ is twice continuously differentiable and $f$ is continuous, we can use the Leibnitz rule that implies that this expression is differentiable and $(d/dt)\int_0^L \phi(x)v(x,t)dx=(c/2)[\phi(L)f(L+ct)-\phi(0)f(ct)]-(c/2)\int_{ct}^{L+ct}\phi'(y-ct)f(y)dy-(c/2)[\phi(L)f(L-ct)-\phi(0)f(-ct)]+(c/2)\int_{-ct}^{L-ct}\phi'(y+ct)f(y)dy$. Since $f$ is even about 0 and $L$, the boundary terms cancel each other and $(d/dt)\int_0^L \phi(x)v(x,t)dx=-(c/2)\int_{ct}^{L+ct}\phi'(y-ct)f(y)dy+(c/2)\int_{-ct}^{L-ct}\phi'(y+ct)f(y)dy$. (4.23) Notice that this expression is 0 at $t=0$. We can use Leibnitz rule another time, $(d^2/dt^2)\int_0^L \phi(x)v(x,t)dx=-(c^2/2)[\phi'(L)f(L+ct)-\phi'(0)f(ct)]+(c^2/2)\int_{ct}^{L+ct}\phi''(y-ct)f(y)dy-(c^2/2)[\phi'(L)f(L-ct)-\phi'(0)f(-ct)]+(c^2/2)\int_{-ct}^{L-ct}\phi''(y+ct)f(y)dy$. Since $\phi'(0)=0=\phi'(L)$, after reversing the substitution, $(d^2/dt^2)\int_0^L \phi(x)v(x,t)dx=\int_0^L c^2\phi''(x)v(x,t)dx$. (4.24) Set $w(x,t)=1/(2c)\int_{x-ct}^{x+ct}g(y)dy$ (4.25) After a substitution $w(x,t)=1/(2c)\int_{-ct}^{ct}g(x+y)dy$. (4.26) Since $g$ is continuous, $w$ is differentiable with respect to $t$ and $\partial_t w(x,t)=(1/2)[g(x+ct)+g(x-ct)]$. (4.27) At $t=0$, this expression is $g(x)$. Since $\partial_tw$ is continuous, we can differentiate under the integral and obtain $(d/dt)\int_0^L\phi(x)w(x,t)dt=\int_0^L\phi(x)(1/2)[g(x+ct)+g(x-ct)]$. The same consideration as before with $g$ replacing $f$ (see 4.12)) yields $(d^2/dt^2)\int_0^L \phi(x)w(x,t)dx=-(c/2)\int_{ct}^{L+ct}\phi'(y-ct)g(y)dy+(c/2)\int_{-ct}^{L-ct}\phi'(y+ct)g(y)dy$. After reversing the substitution, $(d^2/dt^2)\int_0^L \phi(x)w(x,t)dx=-\int_0^L \phi'(x)(c/2)[g(x+ct)-g(x-ct)]dx$. We observe from (4.14) that $(d^2/dt^2)\int_0^L \phi(x)w(x,t)dx=-\int_0^L\phi'(x)c^2\partial_xw(x,t)dx$.  We integrate by parts, recall $\phi'(0)=0=\phi'(L)$, and obtain $(d^2/dt^2)\int_0^L \phi(x)w(x,t)dx=\int_0^L\phi''(x)c^2w(x,t)dx$.  Since $u(x,t)=v(x,t)+w(x,t)$, we have shown that $\int_0^L \phi(x)u(x,t)dx$ is twice differentiable and $(d^2/dt^2)\int_0^L \phi(x)u(x,t)dx=\int_0^Lc^2\phi''(x)u(x,t)dx$ and $(d/dt)\int_0^L \phi(x)u(x,t)dx=\int_0^L \phi(x)g(x)dx, t=0 \qed$