{\bf T 5.10}. Let $T \in (0,\infty)$ and $u: \Omega \times (0,T] \ra \R$. Assume that $u$ is once partially diff wrt $t \in (0,T]$ and twice partially diff wrt $x_k$ at each $x \in \Omega, t \in (0,T], k = 1, \dots, n$, Let $c: \Omega \times (0, T] \ra \R$ be strictly neg. Assume the differential inequality $(\partial_t - L)u \leq c(x,t)u, x \in \Omega, t \in (0,T]$. Then $u$ has no positive max in $\Omega \times (0,T]: \exists$ no $t\in (0,T], x \in \Omega$ such that $u(x,t) \geq u(y,s)$ for all $s \in (0,T], y \in \Omega$, and $u(x,t) > 0$.
% {\it Proof} (by cont). Suppose that such $t \in (0, T], x \in \Omega$ exist. Then, for all $s \in (0, t], u(x,t) \geq u(x,s)$ and $(u(x,s)-u(x,t))/(s-t) \geq 0$ because the numerator is nonpositive and the denominator is neg. So $\partial_t u(x,t) = \lim_{s \ra t} (u(x,s)-u(x,t))/(s-t)  \geq 0$. (It is possible that $t = T$. Then the last limit is only a limit from the left and $\partial_t u(x,t)>0$ could occur.) Define $\phi(r) = u(x_1 + r, x_2, \dots, x_n, t)$ for $r \in (-\delta, \delta)$ and $\delta >0$ small enough s.t. $(x_1 + r, x_2, \dots, x_n) \in \Omega$ whenever $r \in (-\delta, \delta)$. Then $\phi(0) \geq \phi(r)$ for all $r \in (-\delta, \delta)$ and $\phi$ is twice diff on $(-\delta, \delta)$. Thus $0 = \phi'(0)=\partial_{x_1}u(x,t), 0 \geq \phi''(0) = \partial_{x_1}^2 u(x,t)$. Similarly $\partial_{x_k} u(x,t) = 0$ and $0 \geq \partial_{x_k}^2 u(x,t)$ for all $k = 1, \dots, n$. Since $u(x,t)>0$ and $c(x,t)<0, \partial_t u(x,t)-(Lu)(x,t)\geq 0 > c(x,t)u(x,t)$, a contradition to the differential inequality in the assumptions of this theorem$\qed$.