{\bf T 5.3}. Let $(c_j)$ be a seq of non-neg numbers s.t. $\sum_{n=1}^{\infty}c_n < \infty$. Let $I_1$ and $I_2$ be two bounded nondegenerate intervals and $(f_n)$ be a seq of cont funcs $f_n: I_1 \times I_2 \ra \K$. Assume that each $f_n$ has partial derivatives wrt the first var and $|\partial_1 f_n(x,t)|\leq c_n$ for all $n \in \N, x \in I_1, t \in I_2$. Assume  $\sum_{n=1}^{\infty} f_n$ converges pointwise on $I_1 \times I_2$. Then $\sum_{n=1}^{\infty}f_n$ is partially diff wrt the first var and $\partial_1(\sum_{n=1}^{\infty} f_n) = (\sum_{n=1}^{\infty} \partial_1f_n)$, with the second series converging uniformly; this partial derivative is bounded. If each $\partial_1f_n$ is jointly cont on $I_1 \times I_2$, so is $\partial_1 (\sum_{n=1}^{\infty} f_n)$. If each $f_n$ is diff with both partial derivatives being cont and $|\partial_jf_n(x,t)|\leq c_n$ for $j=1,2$, then $\sum_{n=1}^{\infty}f_n$ is cont diff and we can interchange diff and sum (diff term by term). 
%{\it Proof}. Fix $t \in I_2$. Consider the functions $\phi_n(x) = f_n(x,t), x \in I_1$. By Theorem 5.2, $\sum_{n=1}^{\infty}\phi_n$ is differentiable on $I_1$ and differentiation and series commute. Since $t\in I_2$ has been arbitrary, this means that $\sum_{n=1}^{\infty}f_n$ is partially differentiable with repect to the first variable and the partial derivative of the series is the series of the partial derivatives. If the partial derivative of each $f_n$ is continuous on $I_1 \times I_2$, so is the partial derivative of the series by Theorem 5.1. If each $f_n$ has continuous partial derivatives with respect to both variables, so has the series. Thus the series is (totally) differentiable with continuous derivative$\qed$ Armed with these result, we can prove that our formal solutions are solutions indeed. 