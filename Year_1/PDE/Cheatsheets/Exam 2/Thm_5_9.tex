{\bf T 5.9}. Let $f: [0,L] \ra \R$ be integrable and $\int_0^L|f(x)|dx<\infty$ and $u$ the sol of the heat eq from T 5.4.  Then $u(x,t) \ra 0$ as $t \ra \infty$ uniformly in $x \in [0,L]$. {\it Proof}. Recall  $|u(x, t)|=B_0 \sum_{m=1}^{\infty} e^{-a \lambda_m^2 t}, B_0=(2/L) \int_0^L |f(x)|dx$. For $t > 0$, with $\kappa = (\pi/L)^2$, since $\lambda_m^2 \leq \kappa^2m^2$, $|u(x, t)|=B_0 \sum_{m=1}^{\infty} e^{-a \lambda_m^2 t} \leq B_0 \sum_{m=1}^{\infty} (e^{-a \kappa t})^m$. Using the geometric series formula, for $t >0$, $|u(x, t)|=B_0 (e^{-a \kappa t})/(1-e^{-a \kappa t}) \ra 0, t \ra \infty \qed$
%{\bf Large-time behavior of a diffusing population. Critical patch size} 
%Let $L, a >0$.  (PDE) $(\partial_t -a \partial_x^2)w = r(t) w, 0 \leq x \leq L, t > 0$, (IC) $w(x,0)=f(x), 0 \leq x \leq L$, (BC) $w(0,t) = 0 = w(L,t), t > 0$ (5.7).  Assume that $r: \R_+ \ra \R$ is continuous.  Set $u(x,t)=w(x,t)g(t), g(t)=exp(-\int_0^tr(s)ds)$. Notice that $g'(t)=-r(t)g(t)$ and $(\partial_t - a \partial_x^2)u=g(t)(\partial_t - a \partial_x^2)w - r(t)u = g(t) ((\partial_t - a \partial_x^2)w-r(t)w)$. Hence, $w$ is a solution of (5.15) iff $u$ is a solution of the heat equation (5.1). From $u, w$ inherits the Fourier representation $w(x,t)=\sum_{m=1}^{\infty}\langle f, v_m\rangle v_m(x)e^{(\bar{r}(t)-a \lambda_m^2)t}, t>0$ (5.8), where $\bar{r}(t)$ are the time average of the per capita growth rates, $\bar{r}(t)=1/t \int_0^tr(s)ds$. By Parsval's relation, $||w(\cdot, t)||^2 = \sum_{m=1}^{\infty} |\langle f, v_m\rangle|^2e^{2(\bar{r}(t)-a \lambda_m^2)t}, t>0$ (5.9), Assume that $f \in L^2[0,L]$ and that the asymptotic time average $\bar{r}(\infty):=\lim_{t \ra \infty} \bar{r}(t)$ (5.10) exists.  Case 1: Let $\bar{r}(\infty)<a \lambda_1^2$. Then $||w(\cdot, t)||^2 \leq \sum_{m=1}^{\infty}|\langle f, v_m\rangle|^2e^{2(\bar{r}(t)-a \lambda_1^2)t} = ||f||^2e^{2(\bar{r}(t)-a \lambda_1^2)t} \ra 0$ (5.11). Indeed, let $0 < \epsilon < a \lambda_1^2 - \bar{r}(\infty)$. Then there exists some $T>0$ such that, for all $t \geq T, a \lambda_1^2 - \bar{r}(t) > \epsilon$ and  $e^{2(\bar{r}(t)-a \lambda_1^2)t} <e^{-\epsilon t}$. Case 2: Let $\bar{r} (\infty)>a \lambda_1^2$ and $\langle f, v_1 \rangle \neq 0$. Then $||w(\cdot, t)||^2 \geq |\langle f, v_1\rangle|^2e^{2(\bar{r}(t)-a \lambda_1^2)t} \ra \infty, t \ra \infty$ (5.12). Notice that $\langle f, v_1 \rangle >0$ if $f \geq 0$ and not $f = 0$ a.e., because $v_1(x) >0$ for all $x \in (0,L)$. So it depends on the sign of $\bar{r}(\infty)-(a \pi^2/L^2)$ as to whether the population dies out or survives. 
%In our model, where we have neglected any overcrowding, survival means that the population size tends to infinity.  So a large per capita growth rate and a large habitat are beneficial for survival and so is a small diffusion rate.