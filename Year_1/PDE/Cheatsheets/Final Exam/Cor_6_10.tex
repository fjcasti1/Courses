{\bf C6.10}. For given $f: \Omega \ra \R$ and $g: \partial \Omega \ra \R \; \exists$ at most one cont fctn $u: \bar{\Omega} \ra \R$ that is twice diff on $\Omega$ and satisfies $Lu=f$ on $\Omega, u = g$ on $\partial \Omega$. {\it Prf}. Assume $\exists$ two such fctns $u_1$ and $u_2$ s.t. $Lu_j = f$ on $\Omega, u_j = g$ on $\partial \Omega, j = 1,2$. Set $v = u_1 - u_2$.  The $Lv = 0$ on $\Omega, v=0$ on $\partial \Omega$. This implies $\max_{\bar{\Omega}} |v| = \max_{\partial \Omega} |v| = 0$. So $u_1(x) = u_2(x)\; \forall \; x \in \bar{\Omega} \qed$