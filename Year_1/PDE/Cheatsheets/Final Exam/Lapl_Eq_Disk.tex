{\bf The LE on a disk}
Now $\Omega \subseteq \R^2$ is the open disk with center 0 and radius $a > 0$ and $\partial \Omega$ the circle with center 0 and radius $a$. We represent the solution in polar coords, $u(x, y) = v(r,\theta), x = r \cos \theta, y = r \sin \theta, 0 \leq r \leq a, \theta \in \R$, where $v(r,\theta)$ is $2 \pi$-periodic in $\theta$. The BC is easily expressed as $v(a,\theta) = f(\theta), \theta \in \R$ (6.15). Here $f: \R \ra \R$ is $2 \pi$-periodic and cont. We translate the Laplace operator into polar coords.  By the chain rule, $\partial_r v(r,\theta) = \partial_x u(x,y) \cos \theta + \partial_y u(x,y) \sin \theta, \; \partial_r^2 v(r,\theta) = \partial_x^2 u (\cos \theta)^2 + 2\partial_x \partial_y u \cos \theta \sin \theta + \partial_y^2 u (\sin \theta)^2, \; \partial_{\theta} v(r, \theta) = \partial_x u(x,y) (-r \sin \theta) + \partial_y u(x,y) r \cos \theta, \;\partial_{\theta}^2 v(r, \theta) = \partial_x^2 u r^2 (\sin \theta)^2 - \partial_x \partial_y u [r^2 \sin \theta \cos \theta] - \partial_x u r \cos \theta + \partial_y^2 u r^2 (\cos \theta)^2 - \partial_x \partial_y u [r^2 \sin \theta \cos \theta ] - \partial_y u r \sin \theta$. Thus $\partial_r^2 v(r, \theta) + (1/r^2) \partial_{\theta}^2 v(r, \theta) = \partial_x^2 u + \partial_y^2 u - (1/r) \partial_x u \cos \theta - (1/r) \partial_y u \sin \theta = \Delta u - (1/r) \partial_r v(r, \theta)$. The Laplacian of $u$ takes the polar coord form $\Delta u = (\partial_r^2 + (1/r) \partial_r + (1/r^2) \partial_{\theta}^2)v(r, \theta)$. The LE takes the form $(r^2 \partial_r^2 + r \partial_r + \partial_{\theta}^2)v(r, \theta) = 0, 0 \leq r < a, v(a, \theta) = f(\theta), \theta \in \R$ (6.16), with the understanding that $v(r, \theta)$ and $f(\theta)$ are $2 \pi$-periodic in $\theta$ We write $v$ is a complex Fourier series in $\theta, \; v(r, \theta) = \sum_{j=-\infty}^{\infty} \hat{v}_j(r)e^{ij\theta}, \hat{v}_j(r) = (1/(2\pi)) \int_{-\pi}^{\pi} v(r, \theta)e^{-ij \theta} d \theta$. If $v$ is smooth enough, $(r^2 \partial_r^2 + r \partial_r)\hat{v}_j(r) =  (1/(2\pi)) \int_{-\pi}^{\pi} (r^2 \partial_r^2 + r \partial_r) v(r, \theta)e^{-ij \theta} d \theta=  (1/(2\pi)) \int_{-\pi}^{\pi} (-1) \partial_{\theta}^2 v(r, \theta)e^{-ij \theta} d \theta$. Since $v (r, \theta)$ is $2 \pi$-periodic in $\theta, \partial_{\theta}^k v(r, -\pi) = \partial_{\theta}^k v(r, \pi)$ for $r \geq 0, k = 0, 1, \dots$. Since the analogous properties hold for $e^{-ij \theta}$, we IBP twice and obtain $(r^2 \partial_r^2 + r \partial_r-j^2)\hat{v}_j(r)=0, j \in \Z$. If $j = 0, 0 = (r \partial_r^2 + \partial_r)\hat{v}_0(r) = (d/dr)(r \hat{v}_0'(r))$. So $r \hat{v}_0'(r) = \alpha_0$ and $\hat{v}_0(r)=\alpha_0 \ln r + \beta_0$. The continuity of $\hat{v}_0$ at 0 enforces $\alpha_0 = 0$ and $\hat{v}_0$ is const, $\hat{v}_0(r) = \hat{v}_0(a) = (1/(2 \pi)) \int_{-\pi}^{\pi} f(\eta) d \eta = \hat{f}_0$. For $j \neq 0, \hat{v}_j$ satisfies Euler's equation which is solved by the ansatz $\hat{v}_j(r) = r^n$. This yields $0 = (n-1)n + n - j^2 = n^2 - j^2$. So $n = \pm j$ and a general solution is given by $\hat{v}_j(r) = \alpha_j r^{-j}+ \beta_j r^j$. Since $\hat{v}_j$ exists at $r = 0, \hat{v}_j(r) = \gamma_j r^{|j|}, j \in \Z, j \neq 0$. From the BC, (6.15), $\hat{v}_j(a) = (1/(2 \pi)) \int_{-\pi}^{\pi} f(\eta)e^{-i j \eta} d\eta = \hat{f}_j$ (6.17) and $\gamma_j = a^{-|j|}\hat{f}_j$. We sub this into the formula for $ \hat{v}_j, \hat{v}_j(r)=\hat{f}_j(r/a)^{|j|}, j \neq 0$. We sub this result into the Fourier series of $v, v(r,\theta) = \sum_{j \in \Z} \hat{f}_j(r/a)^{|j|}e^{ij\theta}$ (6.18) $ = \hat{f}_0 + \sum_{j=1}^{\infty}\hat{f}_{-j}(r/a)^j e^{-ij \theta} + \sum_{j=1}^{\infty}\hat{f}_j(r/a)^j e^{ij \theta}$ (6.19). The procedure is now alalogous to the one for the HE, $v(r, \theta) = \sum_{j=0}^{\infty} v_j(r, \theta), 0 \leq r < a, \theta \in \R$, with $v_0(r, \theta) = \hat{f}_0$ and  $v_j( r, \theta) = (r/a)^j (\hat{f}_{-j}e^{-ij \theta}+ \hat{f}_j e^{ij \theta}), 0 \leq r \leq a, \theta \in \R, j \in \N$. Notice that $|\hat{f}_j| \leq (1/(2\pi))\int_{-\pi}^{\pi} |f(\theta)|d\theta = c_f$ and $|e^{ij \theta}| = 1$. So $|\partial_r^k \partial_{\theta}^{\ell} v_j (r, \theta)| \leq ((j! j^{\ell})/(j-k)!)  a^{-k} (r/a)^{j-k} 2 c_f, k \leq j, 0 \leq r \leq a, \theta \in \R$, and $\partial_r^k \partial_{\theta}^{\ell} v_j (r, \theta)=0$ if $k > j$. By the ratio test $\sum_{j=k}^{\infty}  ((j! j^{\ell})/(j-k)!)  a^{-k} (r/a)^{j-k} 2 c_f < \infty, 0 \leq r < a$. By T 5.3, $v(r, \theta)$ is infinitely often diff at $0 \leq r < a, \theta \in \R$ and can be diff term by term. We check that each $v_j$ satisfies the LE in polar coord. Since $v_0$ is const, this holds for $v_0$. For $j = 1, \partial_r^2 v_1 = 0, r \partial_r v_1 =  v_1$ and $\partial_{\theta}^2 v_1 = - v_1$.  So $v_1$ satisfies the LE. For $j \geq 2, (r^2 \partial_r^2 + r \partial_r + \partial_{\theta}^2)v_j = (j(j-1)+j - j^2)v_j=0$. Since $v$ can be diff term by term for $r <a, \; v$ also satisfies the LE in the interior of the disk. Notice that $|v_j(r, \theta)| \leq |\hat{f}_{-j}|+|\hat{f}_j|, 0 \leq r \leq a$. Assume that $f$ is Lip cont. By T4.14, $\sum_{j=1}^{\infty} (|\hat{f}_{-j}|+|\hat{f}_j|) < \infty$. By T5.1, $v(r, \theta)$ is cont at $0 \leq r \leq a, \theta \in \R$. We summarize. 