{\bf E5.4.3}. Let $u$ be as in E5.4.2. Assume $u$ is twice partially diff wrt $x$ on $[0,L] \times (0,T)$ and $\partial_xu$ and $\partial_x^2u$ are cont on $[0,L] \times (0,T)$ and $(\partial_t - a \partial_x^2)u = F(x,t), x \in [0,L], t \in (0, T), u(0,t)=0=u(L,t), t \in (0, T)$, where $F: [0, L] \times (0, T) \ra \R$ is cont. Show: For every twice contly diff $\phi: [0,L] \ra \R$ with $\phi(0) = 0 = \phi(L), \int_0^L \phi(x)u(x,t) dx$ is diff in $t \in (0,T)$ and $d/dt \int_0^L \phi(x)u(x,t) dx= \int_0^L a\phi''(x)u(x,t) dx + \int_0^L \phi(x)F(x,t) dx, 0 < t < T$. {\it Prf}. By the previous exercise, $\int_0^L \phi(x)u(x,t) dx$ is diff in $t \in (0, T)$ and $d/dt \int_0^L \phi(x)u(x,t) dx=\int_0^L \phi(x) \partial_t u(x,t) dx=\int_0^L \phi(x) (a \partial_x^2 u(x,t) + F(x,t)) dx= \int_0^L \phi(x) a \partial_x^2 u(x,t) dx + \int_0^L \phi(x) F(x,t) dx$.  Since $\phi$ is twice contly diff, we can IBP twice.  Since $\phi(0)=0=\phi(L)$ and $u(0,t)=0=u(L,t)$, we do not obtain any terms at the int limits and $d/dt \int_0^L \phi(x)u(x,t) dx=  \int_0^L a \phi''(x)  u(x,t) dx + \int_0^L \phi(x) F(x,t) dx \qed$
