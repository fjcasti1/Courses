{\bf E 6.1.3}. Solve the LE with mixed BC, $(\partial_x^2 + \partial_y^2) u(x,y) = 0, 0 \leq x \leq L, 0 \leq y \leq H, u(0,y) = g(y), u(L,y) = 0, 0 \leq y \leq H, \partial_y u(x,0) = 0 = \partial_y u(x, H), 0 \leq x \leq L$. Make an educated guess which conditions $g$ must satisfy for $u$ to be a soln. Explain why you choose these conditions. Is $u$ unique? {\it Sol}. Look for the soln in the form of a Fourier cosine series in y, $u(x,y) = \sum_{m=0}^{\infty} A_m(x) \cos(\lambda_m y),  \lambda_m=m \pi/H, m \in \Z$, $A_m(x)=2/H \int_0^H u(x,y) \cos(\lambda_m y) dy,  m \in \N,  A_0(x) = 1/H \int_0^H u(x,y) dy.$ To determine $A_m$ we derive a diff eq. Start with the special case of $A_0. \; A_0(x)''=0 \implies A_0(x)'=C_1 \implies A_0(x)=C_1 x + C_2$. We look at BCs. $A_0(0) = 1/H \int_0^H g(y) dy = C_2. \; A_0(L) = 0 = C_1 L + C_2 \implies C_1= -1/(H L) \int_0^H g(y) dy.$  Put these together $A_0(x) = -1/(H L) \int_0^H g(y) dy + 1/H \int_0^H g(y) dy =1/H \int_0^H g(y) dy (1-x/L)$. Now we derive a diff eq for the general case $A_m''(x)=2/H \int_0^H \partial_x^2 u(x,y) \cos(\lambda_m y) dy=-2/H \int_0^H \partial_y^2 u(x,y) \cos(\lambda_m y) dy$. IBP $A_m''(x)$ = $-2/H [\partial_y u(x,y) \cos(\lambda_m y) ]_0^H -  2 \lambda_m/H\int_0^H \partial_y u(y,t) \sin(\lambda_m y)dy$. Because $\partial_y u(x,0) = 0 = \partial_y u(x,H)$ we have $A_m''(x)=  -  2 \lambda_m/ H \int_0^H \partial_y u(y,t) \sin(\lambda_m y)dy$. IBP again $A_m''(x)=  -2\lambda_m / H [u(x,y) \sin(\lambda_m y) ]_0^H +  2 \lambda_m^2 / H \int_0^H  u(y,t) \cos(\lambda_m y)dy$. Because $\sin(\lambda_m y) = 0$ for $y = 0$ and $y = H$ we have $A_m''(x)= 2 \lambda_m^2 / H \int_0^H  u(y,t) \cos(\lambda_m y)dy = \lambda_m^2 A_m (x).$ Further $A_m(L) = 0, A_m(0) = 2/H \int_0^H  g(y) \cos(\lambda_m y)dy$ (1). A poss fund set of solns for this ODE is $e^{\lambda_m x}, e^{-\lambda_m x}$, but in view of the condition $A_m(L) = 0$ the fund set $\cosh(\lambda_m(L-x)),  \sinh(\lambda_m(L-x))$ is more practical.  Then $A_m(x) = C_m \cosh\lambda_m(L-x)) + B_m \sinh(\lambda_m(L-x))$. Initial conditions (1) yields $C_m = 0,  B_m = 2/(H \sinh(\lambda_m L)) \int_0^H  g(z) \cos(\lambda_m z)dz$ (2). We combine all this to get $u(x,y) = \sum_{m=0}^{\infty} u_m(x,y), u_m(x,y) = B_m \cos(\lambda_m y) \sinh(\lambda_m(L-x)).$ Then, if $0 \leq x \leq L$ and $0 \leq y \leq H, \partial_y^k u_m(x,y) = \pm \lambda_m^k B_m \sinh(\lambda_m(L-x)) =\cos(\lambda_m y)$ or $ \sin(\lambda_m y)$ and $\partial_x^k u_m(x,y) = \pm \lambda_m^k B_m \cos(\lambda_m y) \sinh(\lambda_m (L-x)), k \in \N, k \text{ even},$ or $\cosh(\lambda_m (L-x)),  k \in \N, k \text{ odd}$. So, $ |\partial_x^k \partial_y^{\ell} u_m(x,y)| \leq \lambda_m^{k+ \ell} |B_m| \{\sinh(\lambda_m (L-x))$ or $        \cosh(\lambda_m (L-x)) \}, 0 \leq x \leq L. $  Since sinh and cosh are increasing on $\R_+,    |\partial_x^k \partial_y^{\ell} u_m(x,y)| \leq \lambda_m^{k+ \ell} |B_m| \{ \sinh(\lambda_m L)$ or $ \cosh(\lambda_m L) \}, 0 \leq x \leq L. $ Recall $\cosh z - \sinh z = (e^z + e^{-z})/2- (e^z - e^{-z})/2 = e^{-z} \leq 1.$  Again, since sinh is increasing, $\cosh(\lambda_m L) \leq \sinh(\lambda_m L) + 1 \leq ( 1 + 1/(\sinh(\lambda_1 L))) \sinh(\lambda_m L),  m \in \N.$ We combine these considerations and find a const $c >0$ s.t. $|\partial_x^k \partial_y^{\ell} u_m(x,y)| \leq \lambda_m^{k+ \ell} |B_m| \sinh(\lambda_m L),  k,\ell \in \N, \; 0 \leq x \leq L.$ By T 5.3 and (2), $u$ is twice partially diff (and satisfies LE by construction) if $ \infty > \sum_{m=1}^{\infty} \lambda_m^2 |\int_0^H g(z) \cos(\lambda_m z)dz|=\sum_{m=1}^{\infty}  |\int_0^H g(z) d^2/dz^2 \cos(\lambda_m z)dz|$.  If $g$ is twice cont diff and $g'(0) = 0 = g'(H)$, by partial integration the last expression equals $\sum_{m=1}^{\infty} |\int_0^H g''(z) \cos(\lambda_m z)dz |$ which is finite if $g''$ is Lip cont and $g''(0) = 0 = g''(H)$ (P 4.14 and E4.3.3). Our solution $u$ is unique because there is only one series and the series is unique. $\qed$
