{\bf E 6.2.1}. Use the properties of the GF to derive T6.4 from T6.3.  Hint:  Approximate cont boundary data by Lip cont boundary data.  Then use ideas from T5.19 and C5.20.  Here is a possible seq of steps. Step 1: For a cont $2\pi$-periodic $f: \R \ra \R$ construct a seq $(f_n)$ of Lip cont $2\pi$-periodic functions $f_n: \R \ra \R$ such that $f_n \ra f$ as $n \ra \infty$ uniformly on $\R$. Step 2:  Define $v_n(r,\theta) = \int_{-\pi}^{\pi} f_n(\eta) G(r,\eta-\theta) d\eta, v(r,\theta) = \int_{-\pi}^{\pi} f(\eta) G(r,\eta-\theta) d\eta$. Apply the cosiderations leading to T6.3 to $v_n$ and $v$ and show that $v_n(r, \theta) \ra v(r, \theta)$ as $n \ra \infty$ unif for $0 \leq r < a$ and $\theta \in \R$. Step 3:  Show that $v(r, \theta) \ra f(\theta), r \nearrow a$ unif in $\theta \in \R.$  Step 4: Show that, if we extend $v$ by $v(a, \theta) = f(\theta), \; v$ becomes cont on $[0,a] \times \R$.  {\it Prf}.  Step 1:   We construct a seq $(f_n)$ of Lip cont $2\pi$-periodic functions $f_n: \R \ra \R$ such that $f_n \ra f$ as $n \ra \infty$ unif on $\R$.  Define $f_n(\theta) = n \int_{\theta}^{\theta + (1/n)} f(\eta) d\eta,  n \in \N$. A simple change of var gives us $f_n (\theta) =  n \int_0^1 f(\theta + \eta/n) d\eta,  n \in \N$. Since $f$ is unif cont, $f_n \ra f$ as $n \ra \infty$ unif on $\R$.  Now we diff to get $f_n'(\theta) = n ( f(\theta + 1/n)-f(\theta)).$ Since  $f_n'$ is a function of $f(\theta)$ and $f$ is bdd, then $f_n'$ is also bdd and thus $f_n$ is Lip cont and since $f$ is $2\pi$-periodic, so is $f_n$. Step 2:  We now define $v_n(r,\theta) = \int_{-\pi}^{\pi} f_n (\eta) G(r,\eta-\theta) d\eta, v(r,\theta) = \int_{-\pi}^{\pi} f(\eta) G(r,\eta-\theta) d\eta$.  It follows from the considerations leading to T6.4 (since $G$ and $f_n$ are cont) that $v_n(r, \theta)$ and $v(r, \theta)$  are cont at $0 \leq r < a, \theta \in \R$.  Since $v_n$ and $v$ are $2\pi$- periodic in $\theta, \; v_n$ and $v$ are unif cont on $[0,a) \times \R$.  Since $G$ is non-neg $|v(r, \theta)-v_n(r,\theta)|=|\int_{-\pi}^{\pi} (f(\eta)-f_n(\eta)) G(r,\eta-\theta) d\eta| \leq \int_{-\pi}^{\pi} |f(\eta)-f_n(\eta)| G(r,\eta-\theta) d\eta \leq \int_{-\pi}^{\pi}  G(r,\eta-\theta)d\eta (\sup_{\eta \in \R}|f(\eta)-f_n(\eta)| ) \leq (\sup_{\eta \in \R}|f(\eta)-f_n(\eta)| ) \ra 0$, as $n \ra \infty.$ So $v_n(r, \theta) \ra v(r, \theta)$ as $n \ra \infty$ unif for $0 \leq r < a$ and $\theta \in \R$. Step 3:  $\forall \; n \in \N, 0 \leq r < a$ we apply the TI, $|v(r, \theta) - f(\theta)| \leq |v(r, \theta) - v_n(r, \theta)| + |v_n(r, \theta)-   f_n(\theta)| + | f_n(\theta)- f(\theta)|.$ Let $\epsilon > 0. \; \exists \; n \in \N$ s.t. $| f_n(\theta)- f(\theta)| \leq \epsilon/4$ and $|v(r, \theta) - v_n(r, \theta)| \leq \epsilon / 4 \; \forall \; r \in [0,a)$ and $\theta \in \R$.  Since $v_n$ is unif cont on $[0,a] \times \R, \exists \; \delta \in (0,a)$ s.t. $|v_n(r, \theta)-   f_n(\theta)| \leq \epsilon /4$ if $a-\delta < r < a$. Now we let $a-\delta < r < a$ and $\theta \in \R$. Then we put this together to get $|v(r, \theta) - f(\theta)| < \epsilon$. So $v(r, \theta) \ra f(\theta), r \nearrow a$ unif in $\theta \in \R.$  Step 4:  We know that $v$ is cont on $[0, a) \times \R$ and we know that $f$ is cont.  Let $\epsilon > 0. \; \exists \; \delta >0$ such that $|f(\eta)-f(\theta)| < \epsilon/2$ if $|\eta - \theta| < \delta$. We can choose $\delta$ s.t. $\delta \in (0, a)$ and $|v(r, \xi) - f(\xi)| < \epsilon/2$ whenever $a-\delta < r< a, \xi \in \R$. We use the TI to get $|v(r, \eta) - f(\theta)| \leq |v(r, \eta) - f(\eta)| + | f(\eta) - f(\theta)| < \epsilon.$  The transition between polar and rect coords is cont in both directions on $\R^2 \setminus \{(0,0)\}$ and so $u$ is cont on $\bar{\Omega} \setminus \{(0,0)\}$.  Equation (6.19), Poisson's formula in rectangluar coords, now shows the cont of $u$ at $(0, 0) \qed$ 