{\bf E6.3.5}. Let $\Omega$ be an open bdd subset of $\R^2$. Let $u: \bar{\Omega}\ra \R$ be cont and twice contly diff on $\Omega$ and satisfy $(\partial_x^2 + \partial_y^2)u - \partial_x u + \partial_y u \geq 0,  (x,y) \in \Omega.$ Prove from scratch that $\max_{\bar{\Omega}} u = \max_{\partial \Omega} u$. {\it Prf}. Case 1:  Assume $(\partial_x^2 + \partial_y^2)u - \partial_x u + \partial_y u > 0, (x,y) \in \Omega$. Since $u$ is cont, $\exists$ a point $(x,y) \in \bar{\Omega}$ s.t. $u(x) = \max_{\bar{\Omega}} u$. If $(x,y) \in \Omega, \partial_x u(x,y) = 0 = \partial_y u(x,y)$ and $\partial_x^2 u(x,y) \leq 0$ and $\partial_y^2 u(x,y) \leq 0$. So $(\partial_x^2 + \partial_y^2)u - \partial_x u + \partial_y u \leq  0,$ a contradiction.  So $(x,y) \in \partial \Omega$ and the assertion follows. Case 2:  Assume 
$(\partial_x^2 + \partial_y^2)u - \partial_x u + \partial_y u \geq 0,  (x,y) \in \Omega.\; \forall \;  \epsilon > 0$ set $u_{\epsilon}(x,y) = u(x,y) + \epsilon(x-y)$. Then $(\partial_x^2 + \partial_y^2)u_{\epsilon} - \partial_x u_{\epsilon} + \partial_y u_{\epsilon} = (\partial_x^2 + \partial_y^2)u - \partial_x u + \partial_y u + 2 \epsilon > 0.$ By case 1, $\max_{\bar{\Omega}} u_{\epsilon} = \max_{\partial \Omega} u_{\epsilon}$. Since $\bar{\Omega}$ is bdd, $\exists \; c >0$ s.t. $|x| + |y| \leq c \; \forall \; (x,y) \in \Omega$. So $\max_{\bar{\Omega}} u \leq \max_{\bar{\Omega}} u_{\epsilon} + \epsilon c \leq \max_{\partial \Omega} u_{\epsilon} + \epsilon c \leq \max_{\partial \Omega} u + 2 \epsilon c.$
Since this holds for any $\epsilon > 0, \max_{\bar{\Omega}} u \leq \max_{\partial \Omega} u$. Since the opposite inequality is trivially true, equality holds $\qed$
% 6.4 The Laplace equation on a rectangle once more
Consider (PDE) $(\partial_x^2 + \partial_y^2)u(x,y) = 0, 0 < x < L, 0 < y < H$, (BC) $u(0,y) = g_1(y), u(L,y)=0, 0 < y < H, u(x,0)=0, u(x,H) = 0, 0 < x < L$ (6.28). This time we only assume that $g_1$ is cont and $g_1(0) = 0 = g_1(H)$. As in the proof of T5.16, we find a sequence of Lip cont fctns which are zero at 0 and $H$ that converges to $g_1$ unif on $[0, H]$. Every Lip cont fctn that is zero at 0 and $H$ can be unif approximated by its Fourier sine series.  This implies that $\exists$ a seq of inf often diff fctns $\tilde{g}_n : [0, H] \ra \R$ such that $\tilde{g}_n \ra g_1$ as $n \ra \infty$ unif on $[0, H], \tilde{g}_n(0) = 0 = \tilde{g}_n (H), \tilde{g}_n''(0) = 0 = \tilde{g}_n''(H)$. Let $u_n$ be the solution of the BVP (6.28) with $\tilde{g}_n$ replacing $g_1, \Omega = (0,L) \times (0, H)$. These solutions exist by T6.1.  By E 6.3.2, $\max_{\bar{\Omega}} |u_n - u_m| = \max_{\partial \Omega} |u_n - u_m| = \max_{[0,L]} |\tilde{g}_n - \tilde{g}_m|$. This implies that, for each $x \in [0, L], y \in [0, H], (u_n(x,y))$ is a (unif) Cauchy sequence.  Let $u(x,y) = \lim_{n \ra \infty} u_n(x,y)$. Then $u_n \ra u$ as $n \ra \infty$ unif on $\bar{\Omega} = [0, L] \times [0, H]$ and $u$ is cont, and satisfies the BC in (6.28).  In order to show that $\Delta u = 0$ on $\Omega$, let $z_0 \in \Omega$. Since $\Omega$ is open, there is an open disk $D$ with center $z_0$ and radius $a$ such that $\bar{D}$ is contained in $\Omega$. Let $D_0$ be the open disk with center $(0, 0)$ and radius $a$. Set $\tilde{u}_n(z) = u_n(z + z_0), \tilde{u}(z) = u(z + z_0), z \in \tilde{D}_0, f_n(\theta) = u_n (z_0 + a (\cos \theta, \sin \theta)), f(\theta) = u(z_0 + a (\cos \theta, \sin \theta)), \theta \in \R$. Then $\Delta \tilde{u}_n = 0$ on $D_0$ and $\tilde{u}_n (a \cos \theta, a \sin \theta) = f_n(\theta)$ for $\theta \in \R$. By (6.24), $\tilde{u}_n(x,y) = 1/(2 \pi) \int_{-\pi}^{\pi} f_n (\eta) (a^2 - x^2 - y^2)/(a^2 - 2 a x \cos \eta - 2 a y \sin \eta + x^2 + y^2)d \eta, x^2 + y^2 < a^2$. Since $\tilde{u}_n \ra \tilde{u}$ unif on $\bar{D}_0$ and $f_n \ra f$ unif on $\R$, we can take the limit as $n \ra \infty, \tilde{u}(x,y) = 1/(2 \pi) \int_{-\pi}^{\pi} f (\eta) (a^2 - x^2 - y^2)/(a^2 - 2 a x \cos \eta - 2 a y \sin \eta + x^2 + y^2)d \eta, x^2 + y^2 < a^2$. Then $\tilde{u}$ is infinitely often diff on $D_0$ and satisfies LE there. So $u$ is infinitely often diff on $D$ and satisfies LE on $D$. Since we can find such a disk around any $z_0 \in \Omega, u$ is infinitely often diff on $\Omega$ and $\Delta u = 0$ on $\Omega$.
{\bf T6.11}. Let $\Omega=(0,L) \times (0,H)$ and $g: \partial \Omega \ra \R$ cont.  Then $\exists$ a cont fctn $u: \bar{\Omega} \ra \R$ that is infinitely often diff on $\Omega$ and satisfies $\Delta u = 0$ on $\Omega$ and $u=g$ on $\partial \Omega$. {\it Prf}. Splitting the BVP into four parts and adding the four solutions yields a solution of the original problem provided that $g$ is zero at the four corners of the rectangle. Suppose that this is not the case.  Let $\phi(x,y) = a_0 + a_1 x + a_2 y + a_3 xy$. Then $\Delta \phi = 0$. We determine the coeff in such a way that $\phi$ equals $g$ at the corners of the rectangle.  For the origin, we get $a_0 = g(0,0)$. For $(0,L), g(L,0) = a_0 + a_1L$. and so $a_1 = (g(L,0)-a_0)/L$. Similarly $a_2 = (g(0,H)-a_0)/H$. Finally $g(L, H)=a_0+a_1 L + a_2 H + a_3 L H$, from which we determine $a_3$. By our previous consideration, $\exists$ a cont fctn $\tilde{u}$ which is infinitely often diff on $\Omega$ and satisfies $\Delta \tilde{u} = 0$ and $\tilde{u} = g - \phi$ on $\partial \Omega$. We set $u = \tilde{u} + \phi$. Then $u$ has all the required properties$\qed$