{ \bf E7.3.3}. Consider the following version of the Neumann boundary problem for the LE. $ \Delta u(x) + c(x) u(x) = f(x), \quad x \in \Omega, \partial_{\nu} u(x) = g(x), \quad x \in \partial \Omega$. Find a sign cond for the fctn $c$ that guarantees that there is at most one soln $u$ even if $\Omega$ is not path-connected (with prf). {\it Prf}.  Let $u_1$ and $u_2$ be two solns and $v = u_1 - u_2$. Then $\Delta v(x) + c(x) v(x) = 0, \quad x \in \Omega, \partial_{\nu} v(x) = 0, \quad x \in \partial \Omega$. By GF1, $0 = \int_{\Omega} v(\Delta v(x) + c(x) v(x))dx = - \int_{\Omega} \Delta v \cdot \Delta v dx +\int_{\Omega} v \partial_{\nu} v \partial \sigma +\int_{\Omega} c(x) v^2(x) dx  \leq \int_{\Omega} c(x) v^2(x) dx.$ Suppose that $c(x) < 0\; \forall \; x \in \Omega$. Then $0 \leq \int_{\Omega} c(x) v^2(x) dx \leq 0 \quad \text{ and } \quad \int_{\Omega} c(x) v^2(x) dx = 0.$
Assume that $c$ is cont. Then $c(x) v(x) = 0\; \forall \; x \in \Omega$ and $v(x) = 0\; \forall \; x \in \Omega \qed$. 
