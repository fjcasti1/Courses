{\bf T7.6}. The Neumann BVP $\Delta u = f$ on $\Omega, \partial_{\nu} u = g$ on $\partial \Omega$ only has a soln if $\int_{\Omega} f dx = \int_{\partial \Omega}  g d \sigma$. If $u$ is a soln, then $\tilde{u}(x) = u(x) + c$ with a const $c$ is also a soln. 
% Boundary value problems for the heat equation
HE in several space dims, $(\partial_t - a \Delta_x - c(x,t))u = f(x,t), x \in \Omega, t \in (0, T), \beta(x,t)\partial_{\nu} u(x,t) + \alpha(x,t)u(x,t)=0, x \in \partial \Omega, t \in (0,T)$ (7.5). $\Omega$ is a normal subset of $\R^n, \; a$ is a pos const, $\Delta_x = \sum_{j=1}^n \partial^2/\partial x_j^2, \alpha$ and $\beta$ are cont non-neg fctns on $\partial \Omega \times (0, T)$ and $\alpha + \beta$ is strictly pos.  Further $c$ and $f$ are cont bdd fctns on $\Omega \times (0, T)$. We derive an energy estimate for $\int_{\Omega}u^2(x,t)dx$. Assume that $u$ is cont on $\bar{\Omega}\times [0,T]$ and that the PDs $\partial_t u$ exist and are cont on $\bar{\Omega} \times (0, T)$. Assume that $u(x,t)$ is twice contly diff in $x$ and that $\nabla_x u$ can be contly extended to $\bar{\Omega} \times (0, T)$. Then the derivative $\partial_t u^2 = 2 u \partial_t u$ exists and is cont on $\bar{\Omega} \times (0,T)$. So $\int_{\Omega} u^2(x,t) dx$ is diff in $t \in (0, T)$ and $d/dt \int_{\Omega} u^2(x,t) dx= \int_{\Omega} \partial_t u^2(x,t) dx = \int_{\Omega} 2 u(x,t) \partial_t u(x,t) dx= \int_{\Omega} 2 u(x,t) ((a \Delta_x + c(x,t))u(x,t)+ f(x,t)) dx= 2a \int_{\Omega}u(x,t) \Delta_xu(x,t)dx+2 \int_{\Omega}c(x,t)u^2(x,t)dx + 2 \int_{\Omega} u(x,t)f(x,t)dx$. By GF1, $\int_{\Omega} u(x,t) \Delta_x u(x,t) dx$ = $- \int_{\Omega} \nabla_x u(x,t) \cdot \nabla_x u(x,t)dx + \int_{\partial \Omega} u(x,t) \partial_{\nu} u(x,t) d \sigma(x)$ = $- \int_{\Omega}|\nabla u(x,t)|^2 dx - \int_{\partial \Omega \cap \{\beta > 0\}} \alpha(x,t)/\beta(x,t) u^2(x,t) d\sigma(x) \leq 0$. So $\partial_t \int_{\Omega} u^2(x,t)$
$dx \leq 2 \int_{\Omega} c(x,t) u^2(x,t) dx + 2 \int_{\Omega} u(x,t)f(x,t)dx$. Let $\bar{c} = \sup_{\Omega \times (0,T)}c$. Choose some $\epsilon > 0$. By the Cauchy-Schwarz ineq, $\partial_t \int_{\Omega}u^2(x,t)dx \leq 2 \bar c \int_{\Omega} u^2(x,t) dx + 2(\int_{\Omega}u^2(x,t)dx)^{1/2}(\int_{\Omega} f^2(x,t)dx)^{1/2} \leq (2 \bar c + \epsilon)\int_{\Omega}u^2(x,t)dx+ 1/\epsilon \int_{\Omega} f^2 (x,t)dx$. Here we have used the ineq $2rs \leq \epsilon s^2 +(1/\epsilon)r^2$. Set $\kappa = 2 \bar c + \epsilon$. Using an integrating factor, we obtain $\int_{\Omega} u^2(x,t) dx \leq e^{\kappa t} \int_{\Omega} u^2 (x,0)dx + 1/\epsilon \int_0^t e^{k(t-s)}(\int_{\Omega}f^2(x,s)dx)ds$. Assume $\bar c < 0$. Then choose $\epsilon > 0$ s.t. $\kappa < 0 \implies e^{\kappa t} \int_{\Omega} u^2 (x,0)dx  \ra 0, t \ra \infty$ and $1/\epsilon \int_0^t e^{k(t-s)}(\int_{\Omega}f^2(x,s)dx)ds \leq \sup_{s \in (0,t)} \int_{\Omega}f^2(s,x) dx 1/(-\kappa \epsilon)$. Notice that $-\kappa \epsilon = -(2\bar c + \epsilon)\epsilon = (2|\bar c|-\epsilon)\epsilon$ takes its maximum at $\epsilon = |\bar c|$ where it is $\bar c^2$. So we pick the estimate, $\int_{\Omega} u^2(x,t)dx \leq e^{\bar c t} \int_{\Omega}u^2(x,0)dx + 1/ \bar c^2 \sup_{0<s<t} \int_{\Omega} f^2$
$(x, s)dx$. 
