{\bf T9.3}. If $u_0$ is measurable and either integrable or bdd, formula (9.6) provides a solution of the HE on $\R \times (0, \infty)$.  If $u_0$ is bdd and cont on $\R$ and $u(x,0):=u_0(x)$ for $x \in \R$, then $u$ is bdd and cont on $\R \times [0, \infty) \qed$. We investigate under which conditions on $u$, a soln $u$ of (9.1) would  necessarily be given by (9.6). To start, we assume that $u: \R \times \R_+ \ra \R$ is cont and bdd on $\R \times [0, T]$ for every $T \in (0, \infty)$. Fix $t > 0$ and $x \in \R$. Define $v(s) = \int_{\R} \Gamma(t-s, x-y) u(y,s)dy, 0 \leq s < t$, and, for each $n \in \N, v_n(s) = \int_{-n}^n \Gamma(t-s, x-y) u(y,s)dy, 0 \leq s < t$.  Choose $\epsilon \in (0, t/2)$ and assume that $u$ and $\partial_t u, \partial_x u$ and $\partial_x^2 u$ are bdd on $[\epsilon, t]$. Then by the properties of $\Gamma, v_n$ is diff on $(\epsilon, t)$ and $v_n'(s) = \int_{-n}^n \partial_s [ \Gamma(t-s, x-y) u(y,s)]dy$. By the product rule, $v_n'(s) = \int_{-n}^n [\partial_s  \Gamma(t-s, x-y)] u(y,s)dy + \int_{-n}^n  \Gamma(t-s, x-y) \partial_s u(y,s)dy $. Using the respective PDEs, $v_n'(s) = \int_{-n}^n [-\partial_y^2  \Gamma(t-s, x-y)] u(y,s)dy + \int_{-n}^n  \Gamma(t-s, x-y) \partial_y^2 u(y,s)dy $.  We IBP, $v_n'(s) = \partial_y  \Gamma(t-s, x+n) u(-n,s) - \partial_y  \Gamma(t-s, x-n) u(n,s) +   \Gamma(t-s, x-n) \partial_y u(n,s)  - \Gamma(t-s, x+n) \partial_y u(-n,s)$. By the props of $\Gamma$ and $u$, the rhs converges to 0 as $n \ra \infty$ unif for $s \in (\epsilon, t-\epsilon)$. Further $v_n(s) \ra v(s)$ as $n \ra \infty, s \in (\epsilon, t-\epsilon)$. This implies that $v$ is diff on $(\epsilon, t-\epsilon)$ and $v'(s) = 0$. So $v$ is const on $[\epsilon, t-\epsilon]$ and $\int_{\R} \Gamma(\epsilon, x-y)u(y, t-\epsilon)dy = \int_{\R} \Gamma(t- \epsilon, x-y)u(y, \epsilon)dy$ (9.8). Let us assume that $u, \partial_t u, \partial_x u, \partial_x^2 u$ are bdd on $\R \times [\delta, 1/\delta]$ for every $\delta \in (0,1)$. Then $u(y, t-\epsilon) \ra u(y,t)$ as $\epsilon \ra 0$ unif in $x \in \R$ and $u(\cdot, t)$ is unif cont.  So $|\int_{\R} \Gamma(\epsilon, x-y)u(y, t-\epsilon)dy - u(x,t)| \leq |\int_{\R} \Gamma(\epsilon, x-y)u(y, t-\epsilon)dy - \int_{\R} \Gamma(\epsilon, x-y)u(y, t)dy| + | \int_{\R} \Gamma(\epsilon, x-y)u(y, t)dy - u(x,t)|$. Since the last term tends to 0 as $\epsilon \ra 0$, we only need to deal with the last but one term estimated by $\int_{\R} \Gamma(\epsilon, x-y)|u(y, t- \epsilon)- u(y,t)|dy \leq \int_{\R} \Gamma(\epsilon, x-y) dy \sup_y|u(y, t- \epsilon)- u(y,t)| \ra 0, \epsilon \ra 0$. Further, by Lebesque' thm of dominated convergence, $\int_{\R} \Gamma(t- \epsilon, x-y) u(y, \epsilon)dy \ra \int_{\R} \Gamma (t, x-y) u(y,0) dy$. Notice that, for $0< \epsilon < t/2, \Gamma( t- \epsilon, x) \leq (2 \pi t)^{-1/2}e^{-x^2(4t)^{-1}} \leq \sqrt{2} \Gamma(x,t)$ (9,9). Take the lim of (9.8) as $\epsilon \ra 0$ and obtain $u(x,t) = \int_{\R} \Gamma(t, x-y) u(y,0)dy$. 