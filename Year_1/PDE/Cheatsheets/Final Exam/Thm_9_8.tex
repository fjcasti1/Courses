{\bf T9.8}. Let $u_0: \R^n \ra \R$ be bdd and unif cont.  Define the fctn $u: \R^n \times (0, \infty) \ra \R$ by $u(x,t) = \int_{\R^n} G(x-y, t) u_0(y) dy, t>0, x \in \R^n$. Then (a) $u(x,t) \ra u(x,0)$ as $t \ra 0$, unif for $x \in \R^n$. (b) For any $t > 0, u(x,t)$ is a bdd unif cont fctn of $x \in \R$. {\it Prf}. By a change of variables $y \mapsto x - z, u(x,t) = \int_{\R^n} G(z, t) u_0(x-z) dz, x \in \R^n, t \in (0, \infty)$ (9.20). (a) By (9.20) and P 9.6 (b) and the nonneg of $G, |u(x,t)| \leq \int_{\R^n} G(z, t) |u_0(x-z) |dz \leq \sup_{\R^n} |u_0|$ (9.21). So $u$ is bdd and $\sup_{x \in \R^n, t > 0} |u(x,t)| \leq \sup_{\R^n} |u_0|$. Again by (9.20), P 9.6 (b), $u(x,t) - u_0(x) = \int_{\R^n} G(z,t)(u_0(x-z)-u_0 (x)) dx$. Since $G$ is non-neg, $|u(x,t) - u_0(x)| = \int_{\R^n} G(z,t)|(u_0(x-z)-u_0 (x)) |dx$. Assume that $u_0$ is bdd and unif cont. Let $\epsilon > 0$. Then $ \exists \; \delta >0$ s.t. $|u_0(x-z)-u_0 (x)| < \epsilon/2$ if $z, x \in \R^n$ and $||z|| < \delta$. We split up the integral accordingly, $|u(x,t) - u_0(x)| \leq  \int_{||z|| \geq \delta} G(z,t)|u_0(x-z)-u_0 (x)|dz +  \int_{||z|| < \delta} G(z,t)|u_0(x-z)-u_0 (x)|dz \leq 2 \sup|u_0|\int_{||z|| \geq \delta} G(z,t) dz + \epsilon/2 \int_{||z|| < \delta} G(z,t)dz \leq 2 \sup|u_0|\int_{||z|| \geq \delta} G(z,t) dz + \epsilon/2$. By P 9.6 (d), $\exists \; \eta > 0$ s.t. $2 \sup|u_0|\int_{||z|| \geq \delta} G(z,t) dz < \epsilon/2, 0 < t < \eta$. So $|u(x,t) - u_0(x)| < \epsilon \; \forall \; x \in \R^n$ if $0<t< \eta$. (b) Choose $\delta > 0$ s.t. $|u_0(x) - u_0(\tilde{x})|  < \epsilon \; \forall \; x, \tilde{x}, \in \R^n$ with $||x - \tilde{x}|| < \delta$. Let $x, \tilde{x} \in \R^n$ with $||x - \tilde x||< \delta$. Then $||(x - z) - (\tilde{x} -z)|| < \delta \; \forall \; z \in \R$ and, by (9.20) and P 9.6 (b),  $|u(x,t) - u(\tilde{x}, t)| \leq \int_{\R^n} G (t,z)|u_0(x - z) - u_0(\tilde{x} - z)| dz \leq \int_{\R^n} G (t,z)\epsilon dz = \epsilon \qed$  
Let $X = BUC(\R^n)$ be the Banach space of unif cont bdd fctns from $\R^n$ to $\R$ with the sup norm.  For $t>0$, define $(S(t)f)(x) = \int_{R^n} G(x-y,t)f(y)dy, f \in X$. Then $S(t)$ is a bdd linear operator on $X$ with $||S(t)||=1, S(t)S(r) = S(t+r), r,t > 0$ (9.22) and $S(t)f \ra f, t \ra 0, f \in X$ (9.23). Families of operators with these properties are called $C_0$-(operator-)semi-groups. That $S(t)$ maps $X$ into $X$ follows from part (b) of the last thm while (9.23) follows from part (a). (9.22) follows from the Chapman-Kolmogorov equations by switching the order of integration.  Let $f \in BUC(\R^n), t,r > 0, x \in \R^n$, and $g=S(r)f, [S(t)S(r)f](x) = [S(t)g](x) = \int_{\R^n}G(x-y,t) g(y)dy = \int_{\R^n}G(x-y,t) (\int_{\R^n}G(y-z,t) f(z)dz)dy$ = $ \int_{\R^n}(\int_{\R^n}G(x-y,t) G(y-z,t) dy) f(z)dz$. We substitute $y = \tilde y + z$ and use the Chapman-Kolmogorov equations, $[S(t)S(r)f](x) =\int_{\R^n}(\int_{\R^n} G(x-\tilde{y}-z,t) G(\tilde{y},t) d\tilde{y}) f(z)dz$  = $\int_{\R^n} G(x-z,t+r) f(z)dz = [S(t+r)f]{x}$. Since this holds $\forall \; x \in \R^n$, we have $S(t)S(r)f = S(t+r)f$. The linearity of $S(t)$ follows from the linearity of the integral.  The boundedness of $S(t)$ and $||S(t)|| = 1$ follows from $\int_{\R^n} G(xt)dx = 1, ||S(t)f||_{\infty} \leq ||f||_{\infty}$, with $||f||_{\infty} = \sup_{x \in \R^n} |f(x)|$. See (9.21). Notice that $BUC(\R^n)$ contains the const functions and $S(t)f = f$ for any const function $f: \R^n \ra \R$. 