\begin{questions}

\question{Solve the wave equation
\begin{align*}
&\partial^2_tu-c^2\partial^2_xu=0,~~~~x,t\in\R,\\
&u(0,t)=f(t),~~~~~~~~~t\in\R,\\
&u(cx,x)=g(x),~~~~~~x\in\R.
\end{align*}
where $f,g:\R\rightarrow\R$. State appropriate assumptions for $f$ and $g$ such that you really have a solution.
}
\begin{solution}
Before starting to solve the PDE, we can obtain one condition that $f$ and $g$ must satisfy. We have that
\begin{align*}
u(0,0)=f(0),
\end{align*}
and also
\begin{align*}
u(0,0)=g(0).
\end{align*}
Thus, $f(0)=g(0)$. We continue with the general solution of the wave equation
\begin{align*}
u(x,t)=F(x+ct)+G(x-ct),
\end{align*}
and imposing the boundary conditions
\begin{align*}
u(0,t)&=F(ct)+G(-ct)=f(t),\\
u(cx,x)&=F(2xc)+G(0)=g(x).
\end{align*}
From the previous equations we obtain the following system
\begin{align*}
F(x)+G(-x)=f\left(\frac{x}{c}\right),\\
F(x)+G(0)=g\left(\frac{x}{2c}\right),\\
\end{align*}
where we have substituted $ct$ for $x$ in the first equation and $2cx$ for $x$ in the second one. Substracting both equations we get
\begin{align*}
G(-x)=G(0)+f\left(\frac{x}{c}\right)-g\left(\frac{x}{2c}\right).
\end{align*}
Thus,
\begin{align*}
G(x-ct)=G(0)+f\left(\frac{ct-x}{c}\right)-g\left(\frac{ct-x}{2c}\right).
\end{align*}
Coming back to $F(x)=-G(0)+g\left(\frac{x}{2c}\right)$ we can obtain
\begin{align*}
F(x+ct)=-G(0)+g\left(\frac{x+ct}{2c}\right).
\end{align*}
We can now write the solution of the PDE
\begin{align*}
u(x,t)=F(x+ct)+G(x-ct)=f\left(\frac{ct-x}{c}\right)-g\left(\frac{ct-x}{2c}\right)+g\left(\frac{x+ct}{2c}\right),
\end{align*}
and we can check that satisfies the boundary conditions
\begin{align*}
u(cx,x)=f(0)-g\left(0\right)+g\left(x\right)=g(x),
\end{align*}
and
\begin{align*}
u(0,t)=f(t)-g\left(\frac{t}{2}\right)+g\left(\frac{t}{2}\right)=f(t).
\end{align*}
In order for the previous solution to be really a soluton we need $f$ and $g$ to be twice differentiable and $f(0)=g(0)$ to satisfy the boundary conditions.
\end{solution}

\end{questions}